\documentclass[]{ctexbook}
\usepackage{lmodern}
\usepackage{amssymb,amsmath}
\usepackage{ifxetex,ifluatex}
\usepackage{fixltx2e} % provides \textsubscript
\ifnum 0\ifxetex 1\fi\ifluatex 1\fi=0 % if pdftex
  \usepackage[T1]{fontenc}
  \usepackage[utf8]{inputenc}
\else % if luatex or xelatex
  \ifxetex
    \usepackage{xltxtra,xunicode}
  \else
    \usepackage{fontspec}
  \fi
  \defaultfontfeatures{Ligatures=TeX,Scale=MatchLowercase}
\fi
% use upquote if available, for straight quotes in verbatim environments
\IfFileExists{upquote.sty}{\usepackage{upquote}}{}
% use microtype if available
\IfFileExists{microtype.sty}{%
\usepackage{microtype}
\UseMicrotypeSet[protrusion]{basicmath} % disable protrusion for tt fonts
}{}
\usepackage[a4paper,tmargin=2.5cm,bmargin=2.5cm,lmargin=3.5cm,rmargin=2.5cm]{geometry}
\usepackage[unicode=true]{hyperref}
\PassOptionsToPackage{usenames,dvipsnames}{color} % color is loaded by hyperref
\hypersetup{
            pdftitle={1121 所得稅法一},
            pdfauthor={柯格鐘},
            colorlinks=true,
            linkcolor=Maroon,
            citecolor=Blue,
            urlcolor=Blue,
            breaklinks=true}
\urlstyle{same}  % don't use monospace font for urls
\usepackage{natbib}
\bibliographystyle{apalike}
\usepackage{longtable,booktabs}
% Fix footnotes in tables (requires footnote package)
\IfFileExists{footnote.sty}{\usepackage{footnote}\makesavenoteenv{long table}}{}
\IfFileExists{parskip.sty}{%
\usepackage{parskip}
}{% else
\setlength{\parindent}{0pt}
\setlength{\parskip}{6pt plus 2pt minus 1pt}
}
\setlength{\emergencystretch}{3em}  % prevent overfull lines
\providecommand{\tightlist}{%
  \setlength{\itemsep}{0pt}\setlength{\parskip}{0pt}}
% \setcounter{secnumdepth}{5}
% Redefines (sub)paragraphs to behave more like sections
\ifx\paragraph\undefined\else
\let\oldparagraph\paragraph
\renewcommand{\paragraph}[1]{\oldparagraph{#1}\mbox{}}
\fi
\ifx\subparagraph\undefined\else
\let\oldsubparagraph\subparagraph
\renewcommand{\subparagraph}[1]{\oldsubparagraph{#1}\mbox{}}
\fi

% set default figure placement to htbp
\makeatletter
\def\fps@figure{htbp}
\makeatother

\usepackage{booktabs}
\usepackage{longtable}

\usepackage{framed,color}
\definecolor{shadecolor}{RGB}{248,248,248}

\renewcommand{\textfraction}{0.05}
\renewcommand{\topfraction}{0.8}
\renewcommand{\bottomfraction}{0.8}
\renewcommand{\floatpagefraction}{0.75}

\renewcommand\contentsname{目錄}
\renewcommand\listfigurename{圖目錄}
\renewcommand\listtablename{表目錄}


\ctexset{chapter ={
name={第,堂},
number=\arabic{chapter},
}
}



% \renewcommand{\cftsecfont}{\stxs} %设置section条目的字体

\let\oldhref\href
\renewcommand{\href}[2]{#2\footnote{\url{#1}}}

\makeatletter
\newenvironment{kframe}{%
\medskip{}
\setlength{\fboxsep}{.8em}
 \def\at@end@of@kframe{}%
 \ifinner\ifhmode%
  \def\at@end@of@kframe{\end{minipage}}%
  \begin{minipage}{\columnwidth}%
 \fi\fi%
 \def\FrameCommand##1{\hskip\@totalleftmargin \hskip-\fboxsep
 \colorbox{shadecolor}{##1}\hskip-\fboxsep
     % There is no \\@totalrightmargin, so:
     \hskip-\linewidth \hskip-\@totalleftmargin \hskip\columnwidth}%
 \MakeFramed {\advance\hsize-\width
   \@totalleftmargin\z@ \linewidth\hsize
   \@setminipage}}%
 {\par\unskip\endMakeFramed%
 \at@end@of@kframe}
\makeatother

\makeatletter
\@ifundefined{Shaded}{
}{\renewenvironment{Shaded}{\begin{kframe}}{\end{kframe}}}
\@ifpackageloaded{fancyvrb}{%
  % https://github.com/CTeX-org/ctex-kit/issues/331
  \RecustomVerbatimEnvironment{Highlighting}{Verbatim}{commandchars=\\\{\},formatcom=\xeCJKVerbAddon}%
}{}
\makeatother

\usepackage{makeidx}
\makeindex

\urlstyle{tt}

\usepackage{amsthm}
\makeatletter
\def\thm@space@setup{%
  \thm@preskip=8pt plus 2pt minus 4pt
  \thm@postskip=\thm@preskip
}
\makeatother

\frontmatter

\title{1121 所得稅法一}
\author{柯格鐘}
\date{2024-01-03}

\begin{document}
\maketitle

% 

{
\setcounter{tocdepth}{2}
\tableofcontents
}



\hypertarget{preface}{%
\chapter*{Preface}\label{preface}}


所得稅法一

112-1 學期《所得稅法一》課堂筆記。采用R的bookdown製作,輸出格式為bookdown::gitbook和bookdown::pdf\_book。

\mainmatter

\hypertarget{section}{%
\chapter{20230904\_01}\label{section}}

\begin{longtable}[]{@{}l@{}}
\toprule()
\endhead
課程:1121所得稅法一 \\
日期:2023/09/04 \\
周次:1 \\
節次:1 \\
\bottomrule()
\end{longtable}

\hypertarget{ux5c0eux8a00}{%
\section{【導言】}\label{ux5c0eux8a00}}

沒有指定課本。

陳清秀稅法總論、稅法各論。

當然還有其他的,稅法的其他的稅務的法律的規定,包括營業稅,包括遺產稅與稅,那麼也有另外一本是黃茂榮老師葛克昌老師跟陳清秀老師3位老師主編,然後由這個我們國內的一些學者去共同寫作的一本稅法各論。那除了這兩本稅法各論以外啊,很可惜,到目前為止。並沒有一本專門針對所得稅法的一個教科書出現。老師有確實有想要把這個部分把他做講義來,至於比較比較特殊的寫作,但,時間上很有限的緣故,老師到目前為止還並沒有辦法把他形諸於文字,當然希望未來能夠透過我們這個課程上的寫作,能夠把他進一步的有文字化的給同學做參考。這個是未來的目標,但目前為止,確實是沒有這樣的一個一個講意識的課程。你可以講所有的課程的內容大概都只能在老師現在的腦袋裡面喔。那,也因此我會希望我在課堂上可以盡可能簡單清楚明瞭的告訴各位所得稅法的意義跟內容,也歡迎各位同學修這一堂課,歡迎各位。如果你的這個課課程之外,或者在課堂上有不清楚的地方,希望你能夠踴躍的提問。因為老師看著各位的臉龐,不一定能夠分辨是不是聽懂了內容,你必須告訴我,我才能夠反應給你,你擔心別的同學的課程會因為上課發問而延遲,那沒有關係,你可以在課後問。老師也有另外的Office hour給各位,如果你有問題,歡迎你提問,歡迎你用寫的也可以,或者請我們助教轉達也可以。歡迎各位提問,因為你不提問,我是無法每一位同學都去詢問你是否了解。

那我會按照每一次上課的進度給各位我們這個所得稅法所想要介紹的內容。
至於每一次的進度內容,我已經寫在ntu cool上面,請各位能夠自行參考,我們每一次上課就是講這些內容的具體化。那也因此,各位可以比照在ntu cool上你想要去了解老師上課的大概進度,那就請各位再來做參考,那當然我會隨時稍微做一些調整,假設如果沒有講完的話,因為這個很難說。老師自己以往的經驗也常常會顯示,我有一些地方講太多呢,導致於沒有辦法講得很完整。那我可能會利用下一次上課前面一段時間稍微做一下補充。如果還有不足的地方,那可能也希望各位同學能夠告訴我,因為我們課堂時間基本上只有100分鐘,我不一定會聽得到那個鈴聲,所以我可能不一定能夠照每一個他課堂中間下課休息時間,各位同學自己有需要,請自行離開,我們要上廁所或是喝茶,或者是覺得有點悶,有一點不太舒服,你自己去行動就可以了不用特別跟老師報告。

然後另外第二件事情有同學問說要不要請假,老師不點名,所以今天可能會是各位最多的一次。啊,大家都當過學生,我也當過啊,我們第一次上課通常會看老師不是好不好欺負,就是他講的內容會不會一樣,那我可以跟各位說不太會一樣,那當然歡迎各位,每一次即使,你沒有點名你自己覺得有這樣的一個需要你來上課跟老師做討論。

那麼老師因為不點名的緣故,所以我沒有辦法以出席率、還有討論來做為評分的標準,老師的評分標準只有兩次,考試的機會一次是期末考,所以我現在要跟各位講我們的期末考的時間。那一次期末考請各位務必要來,因為你不來的話,我會產生一個不太容易在一個平等的基礎上去做評分的,這樣的一個可能,當然如果同學們真的到時候萬一發生了不可預測之天災,像颱風,或者是事故啊,真的是發生了意外,沒有辦法來參加,沒關係,你就照 程序請假,老師一定會給各位補考。我們的臺大期末考週就是在12月18號,務必請各位同學把那個時間空下來。另外一次期中考的時間\ldots\ldots{}

我們有一次是在10月9號,那一次是調整彈性上課,跟這個放假的時間,也就是10月9號,那一天會放假。那麼根據政府表定的時間是把他放到9月23。但我認為9月23對我來講,稍微有點趕了一點點,因為9月23的話大概是在上課,剛開始下下個禮拜就已經開始了,那老師希望把這一個課的時間,我把他改成到11月4號做期中考試。11月4號早上的時間,因為我要跟我另外一堂稅法總論的課配合,所以他應該會是在,早上10點半到12點半之間去做考試期中考,我們有兩次考試,11月4號跟12月18號兩次考試請各位務必到來做考試,至於考試的方式,原則上12月18號我會用,實務案例改編的實例題來考各位,因為那個才是真正的實務。我會用綜所稅的實務案例把他做一下改編變化。這個判決太多了,所以各位大概很難猜測我會從哪裡取出一個題目的方向,那我可以跟各位說,他一定是實務發生的案例,改編了,我會把他稍微簡化,把一些事實做一些篩選。那如果是,期中考的11月4號,這一次,由於我們是在期中,我根據以往的經驗,有些實例題考的比較早了一點點,那這個時候可能對各位同學在解釋上面會有一點小小的困難,那沒關係,我們在期中考的時候,老師會稍微變換一下,我們不要用那麼嚴格的實例題,也就是在實例當中,我們埋一些問答,讓各位能夠有一些基礎觀念的理解來做題目上的回答,也就是說,比如說基本的問題,就是說在這個地方涉及到什麼樣的問題,然後這個問題底下有哪一些,譬如說基本原則或者是稅捐構成要件的爭議,那麼會有一些跟問答有關的題目來給各位做一個測驗,那這個對老師來講也是一個嘗試。因為以往我在所得稅法的科目裡面,我盡可能都是出實例,但有時候實例出太難了。同學們考試起來會不太打擊各位的信心,我自己,我如果是學生,我也會被打擊到啊。如果考試出來,成績不太好的話。但回過頭來來講,這才是真正的現實。因為實務上的到法院去的稅捐的救濟的案件,絕對不會是問答題,他一定都是從個案衍生出來的爭議,歡迎各位同學來所得稅法的課堂上去理解跟了解,認識我們的稅捐爭訟實務的樣子。

老師依序之後會做一些稅各,因為有一些客觀因素的添加,表示以後再稅法的教學上面會盡量往稅課去推動推進,也就是未來我們這個課程除了今年的這個上個學期的綜所稅以外,而是因為下個學期要休假,所以我下個學期沒有開課,但這個之後我會依序的把營所,遺產稅跟贈與稅,一個一個科目來進行。
我會把信託放在跟遺產贈與稅那個地方一起。

所得稅法四是要談房地合一稅跟所得基本稅額條例。我們的所得稅法總共就是4個學期,他是一個比較完整的一個總和性地認識所得稅法,因為這個科目在法律體系上面不是太完整,稍微有點小複雜,但也希望各位不會因為他復雜而退卻,當然以各位要考國家考試,假如各位今年或者明年要考國家考試的話,以這樣的教學的密度來講是蠻長的,但各位不用特別擔心,因為國家考試裡面的稅法,基本上都是法條,演變而已。他不會太複雜。那個都不會真的像訴訟實務上的稅法爭議問題那麼大。國家考試的稅法大部分都是法條演變,當然不能直接考你法條超被這樣,那沒有意義,他都是法條的演變,所以各位你在考試的時候,你不用特別擔心,你就只需要把我們上課裡面所提到的這些基本法條概念弄清楚就好。考試是基本觀念的延伸,好就是他是一個基本觀念,再去做一些實務上的延伸訴訟實務也是訴訟實務。當然有一些東西我們沒有辦法在課堂上去考。我舉個例子來講,像證據評價,這個我們就真的沒有辦法在課堂上考要考,是我恐怕需要去法官學院,請他們給我們卷,大量的一個遺產贈與稅案件全部丟給你,然後你去看這個到底是遺產、還是贈與、還是所得? 讓你去就所有的證據資料去做評價。在學校的考試我只能簡化事實,然後告訴你說這個地方有一筆財產,進了誰的口袋裡面去,我大概都已經直接跟你講那個答案。通常在實務上面,這個答案是不存在,因為你要必須要去像拼圖一樣的。有些人說刑事訴訟是拼圖,他把被告究竟有沒有從事該項行為,慢慢的慢慢的一塊一塊地透過證據去把他拼圖補滿,檢察官要補被告的犯罪的拼圖,納辯護人不需要,只需要讓這個檢察官的拼圖裡面缺少了重要的一塊,就可以了。那其實行政訴訟何嘗又不是如此呢?國稅局也要去拼你的圖,拼上當事人在這個地方究竟有沒有拿到錢,有沒有所得,有沒有遺產,有沒有贈與,有沒有營業的行為,那相對的訴訟人之納稅人這一方他的主要的說明上面,他可能是去破壞那樣子的一個拼圖,說另外一套故事,讓稅捐稽徵機關原先的拼圖拼不起來。

無論如何,我們的稅課的教學會往前去推進,不光只是直接稅而已,老師也希望未來,心有餘力的可能性,我們還有一個非常重要的拼圖,是間接稅就是營業稅。營業稅是非常重要一塊拼圖。另外還有一部分叫關稅,他是國境稅,裡面因為所有的這些直接稅,我們在稅制裡面把他稱之為叫做内地稅。一國之內課徵的稅捐主要分直接稅跟間接稅這兩大類,那麼國跟國之間的貨物的往來的課稅,這個主要就是關稅。我們還有這一塊拼圖,這個算是在國內幾乎各大學,沒有在教授,包括財政系會計系法律系,更不用講,都沒有在教授,當然老師自己個人對關稅關務有一定程度上的興趣,不過他在我的研究的拼圖上面是比較後面的,我需要把内地稅先把他照顧起來。
那麼還有一塊是未來也是各位非常重要,也需要擅長去就這個部分的領域去研究的就是國際租稅。國際租稅未來會在我們,臺大法律學院,陳衍任老師應該會給各位上關於國際租稅的這個部分。我雖然有興趣,但一個人一輩子也只能做好有限的幾件事情。也就是說,如果老師順利,沒有什麼身體健康問題,退休之前還有時間當然可以做,但如果做來不及的話,我們需要分工,需要一同去推進稅捐法在法律人的領域裡面的各方面的研究,沒有研究,不會有更多的我們的從事於這個工作的法律人。

你現在在外面看不到,那是因為,我們以前法律人不研究稅法。那是果,不是因。當然,因果相互循環,也希望未來下個世代由各位所領導主導的法律人的世代能夠有更多的從事於稅捐這一塊領域,不光只是救濟,如同訴訟法一樣,他其實很前端的非訟法,也就是稅捐規劃,能夠有更多的法律人,切入這個領域。這個是作為法律背景的老師終生終生的期盼,希望在下個世代\ldots\ldots 老師這個世代來不及了,因為在這樣一個有限的時間裡面,能做的很有限,希望未來在各位的是在裡面有比較多的可能性,我們能夠有更多的從事稅捐規劃,對稅捐這件事情,去預測當事人的稅捐負擔,從而在他的私法契約裡面去做一些對應。這個稅捐負擔往往有時候會影響到,當事人契約價格的決定。不要小看這件事情,沒有考慮到稅捐,往往你在稅捐重要的價格因素決定上面不會注意到哦,原來這個地方還多出來一個稅負,你本來想要淨賺的數額會因為稅負而有所減少。一個思慮周密的法律人應該在契約裡面將相關的稅負把他考慮進去。

一個思慮周密的法律人,也應該在各式的組織行為乃至於身份上面去考慮。各位可能沒有辦法想像,遺產到跟贈與跟我們現在目前離婚的契約,很多人從事於這一類的工作,但從來沒有去想過離婚之後,小孩子的扶養費誰出誰報稅的問題。這個是幾乎在實務上面,包括法院到實務從事於這一類工作的法律人比較容易忽略的,往往之後就產生諸多諸多的爭議,希望未來下一代的各位不至於有這麼多的困擾或問題存在。這個是我想未來在規劃上面,希望提供給各位同學上這個課。

除了剛剛我所提到的要注意到,考試時間以外,另外一件事情是可能希望各位,我會希望各位至少要有一個工具書,就是你要把所得稅法下載下來,你不買稅務小六法沒關係,你要拿那個簡明小六法。不過因為簡明小六法,一般而言不一定會有稅。你如果沒有法條老師上課很難推得動,而且在考試上面你會對法律本身就不熟悉。我再講一次,國考的稅法其實不難啊,大部分都是法條延伸而已。那你因此不是要求各位買,而是要求各位,你至少要有那個工具,你沒有工具的話,你一天聽老師在課堂上面講說,哦,這個第幾條第幾項規定,你可能做一個筆記,也許你回去還會看。但不看的話,其實說實在話,很可惜你就不知道到底老師上課裡面講的那些觀念,或是那些問題到底在哪裡。

另外一件事情是因為所得稅法的法條本身。所得稅法算是稅法裡面規範密度高一點的。但很多重要的構成要件規定稅捐構成要件規定其實連法律本身都沒規定,所以我們必要的時候會有一些參考解釋函令,因為實務的操作就是這些解釋案例去建構的稅捐稽徵實務。在法律學院之外臺大的會計系會計系的課堂上面他一樣會上稅務法規。稅務法規在會計系的教學裡面,他們非常著重在這些解釋函令。以法律系的觀點解釋函令,幾乎是整個規範的末端,根本不能牴觸依法課稅原則,或者是根據法律規定沒有明確規定的。原則上你總不能透過行政機關的解釋函令去做這個樣子的延伸或者是補充。但機身實務上完全不考慮這件事情,他就是沒規定。沒規定,但要每天要操作要怎麼辦?我們在實務上有大量的解釋函令,也因此有一部分重要的解釋函令,我會在這一個所有小六法裡面把他補上去。他本來應該是法律規定,但沒有那也沒有辦法,我也只能跟各位說,那你要把那個解釋函令找出來,就這樣而已,那其實,所以我想說法只是方便各位去找出比較重要的法規,命令行政規則就這樣而已,各位也可以自行上網去下載考試的時候,各位可以帶六法,包括你認為重要的解釋函令你自己帶。因為我沒有辦法應給你,你就自己帶。當然國家考試的做法是,他會把比較重要的法規範附在後面,附在後面寫,那我一直覺得這是在暗示:這個是重要,這個是會用到的。這個不太不太符合真實的狀況。原則上面來講,你如果法律適用上需要用的法條跟解釋函令,那麼原則上應該在法律適用的過程當中,裡面不管是納稅人方的律師或會計師,或者是國稅局,他們都會去找非常對個案來講重要的解釋函令,那也因此。如果我在出實例題的時候,需要用解釋函令,我會補給各位啊,為了要稍微增加一點難度,會稍微魚目混珠一下啊,不一定真的用得到,不一定真的用得到,只是拿來魚目混珠而已,看起來有點關聯,就這樣而已,也因此請各位在考試的時候做一些斟酌。

還有一件事情,原則上,如果各位,初選沒選到加選的話,我們這邊已經寫給各位,原則上老師會盡可能的給各位在課堂這個容量許可的範圍內給各位做加選,所以各位不用擔心。那要不要每一次請假,你自己決定啊,因為其實不用跟我講,我也不會點名,然後你就兩次考試到而已。有問題請各位來提問,沒問題的話,你覺得你很清楚,那我也沒意見,不過我只是跟各位講,因為上課老師都是在課堂上面去講重點的部分,請各位再斟酌。

我們這一堂課,所得稅法,老師並不要求特別一定要以稅法總論作為前提。也因此,今天我們講的一些內容,如果你是同時修稅總跟所得稅法的話,因為這個理論上可以該當構成,因為老師之前也有碰過同學們,他直接跟老師表示說,老師,我很認真,我願意自己去上課,或是我自己去買教科書來看好,那你就自己承擔風險好,如果老師上課講的一些基本的東西不清楚沒關係,請各位你在。是你再去找那個稅法總論裡面的教科書或是稅法總論裡面的一些基礎觀念,待會我會講內容的時候也一樣會有一些基礎的部分,各位如果是初次聽到沒有太多概念的話,建議各位,你先把關鍵字寫下來,然後你再自己去找那個稅法總論的教科書或是稅法總論的這個課程去把他補上去。老師在學校的臺大開放性課程也有錄一整個學年的稅法總論的課程。各位,你可以自己上去看。各位,你也可以去買稅法總論的教科書來補起來。
你自己要修這個課程,我歡迎各位好,我歡迎各位。因為跟稅法總論有關聯性的當然會有,但也有一些是所得稅法中所稅自己要處理的一些問題,如果你剛好同時兼修總論跟所得稅法,那可能各位稍微要多花一點點時間,同時間把那個基礎觀念把他補起來就可以。我們的整個行政規則,我就跟各位介紹到這裡。

關於考試、關於加選、關於修課,關於我們課程進行的順序跟這個內容,請問各位同學有沒有問題? 假設各位沒問題,那我們接下來就開始要上課囉。好,我會盡可能照下課的時間,不過因為有時候我沒有聽到,各位如果有事情的話,或者是你需要先行離開的話,就請自便。

\hypertarget{ux7a05ux76ee}{%
\section{【稅目】}\label{ux7a05ux76ee}}

我在今天的課堂上,我就必須要先跟各位先說一下,所得稅法裡面不是只有一個稅目。我們出現第一個專有名詞:稅目。稅法,所得稅法當然是一個由立法者制定的法律,可是在我們的所得稅法的規範機構裡面,我們的所得稅法並不是一稅目而已。我們的立法的背景比較特殊,所得稅法是從民國4年,1915年開始。當時我們的所得稅法,他就兩個稅目,就有兩個,也就是後來演進成,我們現在的個人綜所稅,以及現在的營利事業所得稅。

所謂的稅目是一個簡單的名詞的簡稱,叫稅捐名目,課稅名目的意思。他的名稱,他的名目。我們的所得稅法,包含了個人綜所稅這個稅稅目,跟營利事業所得稅,我們這種立法體系其實是蠻罕見的。一般而言,一般而言通常是一稅目一稅法。分散式立法的結構特色就是一稅目一立法。但臺灣的稅法從來都不是這樣,他比較沒有規則,就是有時候立法者會一稅目一稅法,比如說印花稅,他就一稅目一稅法,契稅有一稅目一稅法,可是我們也有一些說,他本身一開始就兩個以上的稅目。像遺產贈與稅,他是從民國42年開始制定,當時只有一個稅目,叫遺產稅。後來就發現,大家不會把遺產留到最後面,在生前給,就不要被課到遺產稅,所以後來到民國62年的時候,我們立法者趕快把他補進去,就變成是現在的遺產贈與稅法。所以就變成兩個稅目,因為課一個稅目很容易被規避掉。

那我們的所得稅法又是另外一個故事。他是一開始我們就跟別人不太一樣,我們就一開始就兩個稅目一個稅法。那也因為這樣一個立法結構,其實大多數國家通常是這個稅目,都是一個個別的稅法,這個叫做個人綜所稅,如果用英文翻譯叫individual,或者是personal income tax就是individual,就是自然人,個人的意思。但如果是講營利事業所得稅,各國大部分都會以公司組織作為他的課稅的名目,可能稱之為叫公司稅,也可能稱之為繳法人稅,名稱不一而足,但原則上一定是以一個營利作為指向目標課徵的一個主體稅,這一點倒是沒有什麼不同。大多數國家通常是individual or personal income tax或是Corporation tax,或者是作為叫做所謂的法人稅,因為德國就稱之為叫法人稅,當然法人不是指公法人,一定是指私法人。不管是哪一種稱呼,大部分都是兩個稅法兩個稅目,臺灣自始就是用兩個稅目一個稅法,然後我們到105年的時候。要加進來另外一個稅目叫房地合一稅。因此我們現在在這一部所得稅法裡面,我們其實總共有三個稅目。國家考試在講說,我要考所得稅法,其實他就跟你講,我要考試這些。所得稅法包含了三個稅目。每一個稅目都有每一個稅本身的課稅構成要件。他的稽徵程序的規定。所以我們在一部所得稅法,他一開始本身就是有兩個,後來變三個,他原則上會是三個兩個以上的課稅構成要件、法律效果、稅捐稽徵程序跟制裁法的規範。其實是別人兩部稅法或三部稅法的加總而已,但我們的法條規範結構基本上是比較簡略的而已。

所以我們今天第一節課,我們除了跟各位大概介紹所得稅法這一部法典,但我們的上課,其實這個學期我們大概只會把重心放在其中的這個稅目,個人綜所稅。這個是這個學期之後,老師休完假以後。接下來是營利事業所得稅。接下來當然就是遺產稅及贈與稅。這個在實務上面最多稅捐訴訟案例的稅目。他也是由兩個稅目所構成。這兩個稅目基本上是一種特種所得,是一種特種類型的所得。簡單來講是從市場上賺來的,這一種財產上的增加,我們課所得稅的所得,再按照自然人或是營利事業去做分類。自然人,或者是,法人組織的營利事業去做這個分類而已。當然,我們民國105年又加進來房地合一稅,納房地合一稅是把這兩類所得裡面,把房地,拉出去做分離課稅。也就是,他的,稅捐的規範結構是把他從本來綜所跟營所,本來是要把他全部包含起來,他把他拉出去,做一個獨立的稅目,做一個個別分離在原先稅基之外的課稅。

遺產跟贈與原則上不來自於市場經濟活動。而是來自於主要是兩種關係,第一種是來自於血緣。你有好爸爸。有好爸爸,他死的時候給你。這個就叫遺產。你的好爸爸死之前就已經把財產給你,這個就叫贈與。基於血緣關係的遺產取得,或者是基於任意性的給予,無償增益他人財產,不一定來自於血緣,而是來自於任意性,喜歡OK。像小三小王贈與,不一定有血緣關係。只是為了某一種特定的關係,而任意性給予的,這個就是所謂的遺產跟贈與稅裡面的特種所得。除了這幾個稅目以外,我們還可以看到本質上,應該跟所得是一樣的類型,但在我們的法律規範裡面,我們另外把他拉出去做分離課稅。那個就是土地增值稅,土地稅法裡面的,一個稅目。土地稅法又是另外一套。同樣我們這個稅法裡面總共有三個稅目,土地稅法總共有:地價稅、田賦、土地增值稅。以土地,為集合,規定在土地稅法。但土地增值在本質上面是一種被稱之為叫資本利得。在英文裡面很好玩的,我們中文其實看不出來這個意思,你看。這個概念在英文裡面並不包含capital gain。Capital gain這個的英文字眼,照中英文的字面來翻,叫做資本利得。因為英文裡面的爲income那個概念,他稱這個不是指那個capital本身,而是透過capital去轉嫁去轉售的價差,這個才叫做Capital gain。你把他想像成一件事情是樹會長果實,只有果實才叫income,只有果實,而且果實還要掉下來哦,這是他長出來的,多出來的。
德文稱之爲孳息,各位在民法裡面學到的孳息。
但如果是本體的這個部分的話,在英文,income的概念是不包含所謂的資產本體的價值增加。這個是英文的字彙,在中文字彙裡面我們不太能夠去感受到這個差別,因為我們從語言裡面看不到這個差別。那這個差別主要是來自於income這個概念,一般他是指所謂的,長出來的孳息。那長出來的孳息,這個才叫income。如果是還連接在財產本體裡面,透過財產的本體的轉售而取得的轉售價差,這個在英文裡面就被稱之為叫capital gain。他是一個獨立稅目。比如說在印度income就不包含capital gain這個概念,他是另外獨立分開來的。香港也是一樣,凡是受到英國影響的,大部分,對income的概念基本上跟capital gain是不同的概念的想法。

但回到中文的世界裡面,其實我們不太有這個差別。總而言之,你今天賣果樹,只要上面長果實價值增加,這也是一種所得的概念。你本來是用100塊,結果你把他透過勞動力的施作,讓他變成長出果實,他上面可以賣到120塊,這個轉售價差,這個在中文的概念裡面來講,其實我們也認為那是所得。那也因此capital gain這個概念在我們的法制裡面,把土地增值這個部分另外拉出去,另外去做分離課稅。除了土地以外,其他基本上都是回到所謂的所得稅裡面有一種類型叫做財產交易的損益。我們後面的課程會去講到財產交易損益:我們的所得類型裡面有一種類型叫財產交易損益。所以除了不動產的土地拉出去以土地增值稅課稅以外,其他的財產交易的轉售價差,原則上我們的綜所稅是放在財產交易損益裡面去做課稅,做相關的課稅的規定。也因此除了綜所營所遺產稅贈與稅土地增值稅,還有後來加進來的房地合一稅。

當然你現在看到這個地方的圖形的說明的時候,很快地就會注意到,都是針對不動產,土地跟房地,有重疊。所以我們105年。引進的房地合一稅,他特別把土地增值稅的稅基扣掉,避免重複課稅,所以土地增值稅,目前為止還在。從105年以後,我們加進來房地合一稅。為了避免這兩個稅目打架------重複課稅的意思。我們在稅基計算上,我們特別把土地增值稅的稅基把他扣掉,以避免重複課稅。那麼相關條文,我們到所得稅法四的時候,再跟各位去說明。關於所得基本稅額跟土地增值,還有房地合一稅,我們到所得稅法四,也就是第四個學期,再跟各位去做說明。那其實也是法條而已,你去看法條大概就可以去了解,在那個地方,把稅基扣除掉。

最後一個我們就跟各位講,所得基本稅額條例的所得基本稅額。個人跟營利事業還有一個所得基本稅額。好,怎麼會有這種東西?這個是臺灣法制上一個奇蹟,一個很奇怪的結構。因為臺灣的所得稅,例如綜所稅,我們原則上只課中華民國來源所得的所得稅,我們不對境外來源所得去課所得稅。因此在民國95年的時候,2006年林全當財政部長任內的時候,我們就做了一個立法,叫所得基本稅額條例的最低稅負制的立法。把個人在境外取得的,個人在境外取得本來在綜所稅裡面不課的所得稅的所得------他其實是所得,但是不課所得稅,由於不課所得稅,所以我們在民國95年的時候立法把他改課所得基本稅額。這個名稱有點拗口,有點怪異。他其實是所謂的我們一般稱呼叫做最低稅負制,就是你如果都賺到海外所得,至少讓你負一個最低的稅捐負擔,他的概念是這樣。
所以,我們臺灣的稅法裡面把一部分本來在綜所稅裡面不課的所得;在營利事業所得稅,因為臺灣有很多營利事業所得稅的稅捐優惠規定,所以我們現在把在原先所得稅裡面可能是被免稅的,可能是被稅捐優惠的,我們透過2006年的所得基本稅額條例,再把他拉回來,課一個最低稅負,這個叫做個人所得基本稅額,或叫營利事業所得基本稅額。有一點拗口的一個稅目名稱,那這個部分我們到所得稅法四這個時候才去跟各位做進一步說明。但我要跟各位說,其實所得基本稅額條例,因為太仰賴太吃稅法法條規定,在考試上面幾乎很難考,因為給法條,大概就給答案了。但是實務操作上他卻很重要。
因為實務操作上大部分都是看著法條,然後去跟國稅局去吵說欸,我有稅捐優惠適用,然後國稅局說哦,沒有你的租稅優惠是我們再用所得基本稅額條例又再把你拉進來,課稅。大概在實務上面吃法條很重的,考試上面不太容易去考,但操作實務上他真的還蠻重要的,我們到時候的稅法四,我們把後面這幾個撿回來,一併去做一個說明。一個是資本利得稅的土地增值稅,房地合一稅,然後另外一個就是所得基本稅額的這個最低稅負制,這是我們所得稅法四。

所以你看到我們大致上所得稅法,個人綜所稅,針對個人原則上是綜合所得的課稅。如果你是一個營利事業就改課營利事業所得稅。如果你是從爸爸那裡拿來的錢,就課遺產稅,課贈與稅。如果你是炒地皮賺的,就是土地增值稅,105年以後再加徵房地合一稅。在海外賺來的錢就課徵個人所得基本稅額條例裡面的最低稅負制。

這個就是我們現在的所得稅,以一個個人作為中心。透過這一個不同稅目的拼湊,你大致上就會瞭解\ldots\ldots 先不批判,先不批判,你要先認識客觀現況的立法。你在臺北賺到了錢,課綜所稅所得。你在美國賺到的錢,課所得基本稅額條例的最低稅負制。你炒地皮賺的錢,課土地增值稅房地合一稅。你叫爸爸,爸爸很開心給你的錢,課遺產稅。爸爸生前給的這個就叫贈與稅。每一種不同的財產上的增益,課徵不同的稅目。

這個就是我們立法者所做的,我們把他稱之為叫價值決定。立法者在不同的稅法裡面,埋不同的稅目,用不同的稅目去切割,本質上都是財產上增益的所得。這一種叫價值決定。立法者,由於他是一個可以在民主國家裡面負起政治上抉擇的一個團體,在權力分立底下,享有這樣一個,我們在公法上憲法學上稱之為叫立法形成空間。在這個地方具體做出來的決定就叫價值決定。各位聯繫上憲法,立法形成空間,權力分立原則,對於這一件事情,司法跟行政只能看立法者,因為只有立法者能做這樣一個決定。

先請各位不要批判他,因為你要先認識現況長什麼樣子,有沒有道理,當然這個是我們法律人最特殊,也是最厲害的本事,這也是各位在市場上你會跟會計師背景可能會有所差別,因為會計師比較知道現況長這個樣子,但他不會採取批判的角度。只有法律人,在教育訓練上,因為我們對現況的這一種價值決定。固然認為立法者有他的立法形成空間,但這個形成空間是否違反憲法上的原則?特別是平等原則。在這個地方,法律人就會發揮他去對立法者這樣一個價值決定的批判,乃至於認為他有違憲的疑慮。這個也正是各位未來在實務現場裡面,可能跟會計師不管是合作或競爭的關係,原則上我們的背景思考,大致上就是給各位這樣一個能力。但我要請求各位。你一定要先了解現況大概長什麼樣? 不要不了解現況,就一味地去做批判。這個地方立法者的價值決定就是做了不同的稅基切割。同一筆財產上增益,顧客賺來,老闆賺來的,跟爸爸拿來的錢,做不同的稅目的課徵做不同的稅基的計算,這一種叫價值決定。我們之後有機會,當然在稅法總論的課程裡面,會去進一步去討論,平等原則應該是怎麼樣一個情況。

那我們先休息一下,我大致上用第一個小時的後半段的時間跟各位介紹,我們接下來一個學期要上的綜所稅。跟其他稅目同樣是屬於財產上增益的課稅,跟這些各稅目之間的關係,
我們休息之後,我們第二個小時還要跟各位繼續介紹一下,跟其他稅目還有稅法所得稅法的規範結構,讓各位能夠清楚。先休息。

\hypertarget{section-1}{%
\chapter{20230904\_02}\label{section-1}}

\begin{longtable}[]{@{}l@{}}
\toprule()
\endhead
課程:1121所得稅法一 \\
日期:2023/09/04 \\
周次:1 \\
節次:2 \\
\bottomrule()
\end{longtable}

\hypertarget{ux7d9cux6240ux7a05ux5728ux6211ux570bux7a05ux5236ux7684ux5730ux4f4d}{%
\section{【綜所稅在我國稅制的地位】}\label{ux7d9cux6240ux7a05ux5728ux6211ux570bux7a05ux5236ux7684ux5730ux4f4d}}

我們跟各位說一些。所得稅個人綜合所得稅大概在整個我們國家的稅務裡面的位置。因為,我們大概能夠整體地看的時候,你才會比較清楚說,哦,原來所得稅法是針對我們哪一些的個人的財產和增益去做課稅。透過剛剛的說明,大致上建構在,我們國家的課稅法制被分成四種類型的所得的這種前提,以下我們把他分成叫做所得、財產、交易,跟消費這四種不同的稅類。

那麼,綜所稅是在這個所得稅底下的一種稅目,針對個人的市場經濟活動所獲得的財產上增益,課徵所得稅。營利事業的市場經濟活動獲得的財產增益就課營所稅。如果是從爸爸那邊拿來的。生前拿的叫贈與,死後拿的叫遺產。那為什麼他也是所得?他是不名為所得的所得,因為對你來講,對拿到錢的繼承人或受贈人而言,他其實也會增加你的稅捐負擔能力,只是他不稱為所得稅的所得而已。因為對你來講,那個錢就是錢。1塊錢就是1塊錢,不管是老闆給你的錢,還是是爸爸給你的錢,對你來講,那個錢永遠都表彰出一個稅捐負擔能力。同樣的道理,炒地皮也是錢。炒地皮跟辛苦花費你的勞動你去賺錢? 錢永遠是錢。

因此, 我們在這個地方會稱之為叫量能課稅。依照每一個人的稅捐負擔能力。稅捐負擔能力是透過金錢去計算的。稅捐負擔能力不是看天賦能力。不是你說你多厲害,你IQ 150啊,說,欸這個工作如果我去做我一定下去做好了啊,你才賺100萬,不夠,像我這麼IQ這麼好的,我也可以賺到200萬可以賺到一億。不用吹牛,你的IQ天賦能力不是我們稅法要考察的重點。我們考察的重點是你有沒有換成以金錢為單位計算的稅捐負擔能力。

稅捐負擔能力作為課稅標準的這個原則,被稱之為叫量能課稅原則。他從憲法上的平等原則而來。簡單來講,就是具體化憲法平等原則的稅法原則,就稱之為叫量能課稅原則。這個部分。沒有上過稅總,請各位再自行去看憲法平等原則跟稅法量能課稅原則的關係。這個部分,我只跟各位講解到這裡。根據這個量能課稅原則,在不同的經濟活動的階段。不同的經濟活動的階段,意味著國家在不同的地方看到了稅捐負擔的能力。稅捐負擔能力是用金錢的流動去展現的,所以你看到這個地方,在這裡面背後是一種金錢上的流動。

第一個是當你賺到錢。你賺錢可能是透過勞動力投入,可能是透過資本投入。總而言之,你賺到錢第一個階段,這裡面流出來那個錢,表彰出稅捐負擔能力。最後一個階段叫消費,就是你去花錢的時候,國家也看到哦。你買得起瑪莎拉蒂,買得起海神,哦,代表你有實力。因為沒實力的人,就是買不起,客觀現實就是這樣。好OK,你賺錢的時候就被課一次所得稅,你花錢的時候又被課一次消費稅。國家在兩個不同端點兩個不同端點,你第一次賺到錢的時候,誒,他課你一次所得稅,然後另外一個端點就是你去花錢的時候,又課一次消費稅。然後賣你海神賣你瑪莎拉蒂那個人,他賺到所得,所以他被課所得稅。各位看得到那個經濟流動嗎? 你賺錢被課一次的稅,你出去買東西消費吃海鮮餐廳。你每次消費行為都被課一次稅,然後賣你東西那個人,他被課所得稅,為什麼?因為他是獲得財產上的增益。所以這個地方我們雖然把他切割,但其實他背後是一個經濟上的流動。也因此,經濟學特別把他稱,這個是一個流量。流量,經濟上的流動,表彰出金錢的流動,這個就是國家掌握稅捐負擔能力。看不到是因為你法眼不具啊。等下看錢的人喔,那裡面錢在動\textsubscript{錢在動}錢在動,錢就代表著稅捐負擔能力好。你賺到錢,花錢這兩個端點,都被課稅。

那你說那我不花總可以吧,好,你存下來。國家才不傻呢? 你存下來,他也可以課財產稅。所以被稱之為針對存量。因為有流動,全部都被課說,你賺到錢,每天把他花完,又被課。那我可不可以不花,我不花,總不會被課稅吧,沒有,國家才不傻哩,你存下來,國家也想辦法給你要一點財產,說這就是我們的財產稅。

那至於什麼叫交易稅?交易是這樣,交易是消費的前端。消費是交易的成果。你去買東西過來的時候,買來把他用掉,這個叫消費稅。原則上,可以消費的就不課交易稅,不能消費的國家也不會放過,就對你交易行為課稅。

所以在這個地方,透過經濟活動,觀察稅捐負擔能力。國家在4個不同的經濟活動階段切進去,他看到你有錢,他看到了稅捐負擔能力,要求你在這個階段付出來的稅捐,這個就是我們透過經濟活動所觀察出來的稅捐負擔能力。也因此,我在編這個稅務小六法的時候,要做一個整體的考察,就要把所有的稅目把他做歸類。

歸類是為了做什麼用? 歸類就是為了做平等原則的對待。因為等者等之不等者不等之,同樣一種類型應該原則上要長得一模一樣。你要做應然的批判基礎,歸類是非常重要的功夫,不同類的東西比不起來,不同類的東西本來就沒有平等原則的問題,因為他沒有什麼好比的。

所以我舉例而言。各位在老師的課堂上,老師給各位打分數,這個就有可比較性。可是如果你拿我的課去跟謝煜偉老師的課比,請問有什麼比較性? 你不能說謝煜偉老師,他每一位同學都打90分吶,柯老師,你為什麼只有打80分,不平等? 因為,大概我要好好檢查我是不是應該要打更高,更低一點才對。不是這樣,所以可比較性是指你們兩個在這個前提基礎上面來講,你們有其相似性,根據你的規範的目的而去決定有沒有可比較性。我如果今天要決定颱風天,誰要出去幫忙幫忙把書本或是把東西抬進來,因為下雨了,怕書本會濕掉,那可能我有一個分類標準,那個分類標準是說,因為下雨下很大風很大,那這個時候也許我們戴眼鏡的同學就不要出去。戴眼鏡的同學留在室內,不戴眼鏡的同學就出去幫忙搬進來,讓戴眼鏡的同學在室內去接著他。這個是一個根據你的規範目的去做的分類。但如果我在課堂上用戴眼鏡跟不戴眼鏡來做為我給各位的分數標準,這個標準就不好,這個分類就不好。這個分類可以適當地建構在以各位的考試評價作為基礎。也可以用出席率,當然你如果用出席率,可能他不會是太好,因為也可能你人在心不在。對吧? 所以什麼是最好的評價,基本上就是各位考試考卷上面所表現出來的這些基礎的知識。也因此回過頭來,在稅法上去評價,為什麼我要對你課稅? 他看到了金錢流動。哦,你賺到錢了,所以你比較有稅捐負擔能力。第二個你去吃高檔的餐廳,一餐花42萬,別人只能花4千二的,那你當然是相對比較有稅捐負擔能力。

當然你可以辯解說,哇,那國家真的是到處要稅,所以才會有一句話叫做,中華民國萬萬稅? 沒有一個國家,不是萬萬稅。每個國家都會在不同的經濟活動的不同端點想辦法去要到,國家要的稅捐收入,因為國家本質上就是一個不從事生產,本身就只是專門從事分配工作的一個組織。
國家如果從事生產,你才要害怕。國家如果從事生產,有很多國營事業,那我看你才要真的害怕,為什麼?因為國家不務正業。本來就是一個不從事生產,所以他要維持功能的必要存在,他當然就只能從人民的經濟活動。或者是經濟活動的成果,裡面的所得或財產去做課稅,這就是現代稅制。這一種不同的稅目,不同的經濟活動可以不同稅目,我們在學理上稱之為叫多元稅制。我們稱之為叫多元稅制。相反的,叫單一稅制。一個國家只仰賴一個單一稅目就能取得需要的財政收入,這個是烏托邦。從來沒有在現實上存在過。所以那個說中華民國萬萬稅,他其實是調侃,只是一個,不符合現實狀況的調侃說明而已。就是說國家到處要錢。對國家還真的要到處要錢。當然很多人會說,那為什麼我賺到錢的時候,要被課一次所的時候,花錢的時候又被課一次消費稅,不公平? 那大概是這樣的一個邏輯跟理解,如果你的稅負不平均,分配到不同的稅目上面去,他很容易會帶來(問題)。譬如說我只課所得稅,我不課消費稅。你知道徵課所得稅不課消費稅,那誰會最開心呢? 從事不法經濟活動的地下經濟人會最開心。因為你可所得稅不課消費稅,換言之,他們今天本來從事不法經濟活動,比如說做詐騙,比如說,販賣人口販賣毒品走私槍支,他們本來應該要報所得稅,但他從來不報所得稅。如果你在法制上面不課到所得稅,那麼大致上消費稅是一個最低限度,這些不被課到所得稅的,好歹他去買瑪莎拉蒂,跟買豪宅的時候,會被課到一層的消費稅。簡單來講,國家基於現實的實然,讓就算從事不法經濟活動的人,原則上沒有報繳所得稅的人,他們至少對國家的稅收還有一定程度上的貢獻,就是從他的消費行為裡面去做課稅。

國家很現實,先不講倫理上的非難,也就是國家在不同的端點不同的地方都要課稅,你把他形容成一隻牛扒3次皮。啊,可能這一個的形容不是很貼切,應該是說一隻鵝拔3次毛4次毛,然後不同的地方拔一點,不同的地方,拔一點。額頭,拔一點,腋下拔一點,屁股再拔一點,臨走的時候再給你拔一點。那個鵝,不會回過頭來去啄你,不會有稅捐抵抗。所以任何一個國家都是多元稅制,他只是分佈端點,不太一樣而已。

一般而言,比較注重分配正義的國家通常,前面這個部分課比較多,這個就是學理上稱之為叫直接稅的稅目。所得稅跟財產稅比較著重分配正義,我講的是那一些比較偏社會福利國家國家的這個財政支出比較高的一般而言,會對前面這兩種所得課比較高稅負的稅捐負擔。因為賺越多,財產越多,在他們這些國家的觀念上來看,就是比較有稅捐負擔能力,應該對國家的團結一致,負起更大的責任,因為國家本身不事生產,我只能從你們身上去要。相反的。一個國家,如果稅捐稽徵效率不高,人民逃漏稅捐乃至於從事地下經濟非法活動非常猖獗的話。維持國家運轉的稅捐大部分都是間接稅。也就是後面這兩者交易稅跟消費稅,大致上是反映出來,一個國家不一定說他不重視分配正義。不過,受限於稽徵效率,往往他直接說課不到。所得大部分移動到海外財產可能都有價值的財產都跑到國外,像高額的價值的藝術品。好,那國外去,那你國內課不到,那怎麼辦呢?留下來大概課交易稅跟消費稅,也就是我們所講的間接稅類。

那也因此,一個國家的稅捐負擔,從直接稅或是間接稅的稅收的收入,大致上也可以反映出這個國家本身在意識形態上面以及他的稅捐稽徵效率上面,大致上來講,可以反映出這個事實上的現況。我們國家大致上而言,這兩個所得稅類裡面的。綜所營所大約,通常情況底下大約佔三成左右,另外的三成是後面的消費稅。也就是說,我們國家的稅捐的負擔比例在直接稅裡面,一般而言有一定程度上的直接稅的稅捐負擔比例。相對於某些比較地下經濟活動猖獗的這一些國家,南歐的國家有一些和東歐的國家,由於稽徵效率以及,人民的那種地下經濟活動比較猖獗,從而國家的財政收入不足的話,通常會課比較高的交易稅和消費稅。

臺灣在所得稅的稅率裡面,比如說綜合所得稅,我們最高到40\%。營利事業所得稅大約現在目前是20\%。我們的消費稅,臺灣沒有名稱被稱之為叫消費稅的稅目,臺灣的消費稅是改成營業稅。我們的營業稅稅率是全世界範圍內少有的低。日本本來跟我們一樣,低都是5\%,但在安倍時代逐步調整成從5\%到吧。到現在的10\%。大多數OECD國家的消費稅平均起跳都是15\%以上。以德國為例,德國的消費稅分兩個級距。民生必需品7\%。一般生活品的消費19\%。法國到22\%。好希臘南歐國家有許多是到25\%,等於是你去買一個,那個飯的啊,或是買一個grocery(?)這一些東西,他全部都在價格上面把你加上去。當地的消費稅20\%或25\%。臺灣的消費稅相對而言在全世界範圍內算很少。在某程度上來講啊,當然你可以覺得說,哇,我們才5\%。那我們是不是可以把他調整呢? 每一個消費稅的調整就是營業稅的增加,從5\%+3\%,到8\%,他就代表著物價的上揚。為什麼?因為這一類的稅負一定要轉嫁給消費者負擔。也就是這一類的營業稅,我們稅目叫營業稅,他其實是要消費者付稅。所以只要調消費稅,營業稅的稅率是比物價就會調漲。也因此不可能不轉嫁,因為這個本身的稅目本身就是希望消費者去負擔節制消費行為。當然啦,這個節制消費行為是稅負增加的效果,其實從稅法的量能課稅來講,是因為你消費的起,因為你花得起。你花得起,你就有這個能力。我們把這整個所得稅所得稅的稅內總共有這七類。綜所、營所、遺產贈與、土地增值,房地合一稅再加個人所得基本稅額總共有七類。這一類,他的共同特色就是他們原則上是直接稅裡面針對財產上的增加,只是差別在市場經濟活動與否。

從爸爸媽媽那邊來的叫遺產贈與,從市場經濟活動,我們再把他拆分成是土地或房屋,土地不動產或是來自於所謂的境外來源所得。是這個差別而已。但他都是一個市場上你可以認識的一個財產上的增益。那這個財產稅,我們國家總共只針對以下這三或四種類型,課財產稅。第一個是針對土地,我們有課農地的田賦,跟非農地的地價稅。這裡面有兩個稅,田賦跟地價稅好,我就不寫在這裡。一樣,他是屬於稅法總論,我們會在課堂上面去做一些介紹的。第二個是針對房屋課的房屋稅。第3個是針對機動車輛船舶課的牌照稅。所以你可以說他是3個,也可以說他是4個,因為土地課兩種不同稅目,農地農用課田賦,非農地作農用,或者是農地作非農用,或者是非農地非農用,本來就是課地價稅。田賦或地價稅。房屋課房屋稅。機動車輛跟船舶,原則上我們課牌照稅,掛牌了就課牌照。除此以外,我們不課總體財產稅。

臺灣不課總體財產,所以所謂的臺灣的總體財產就是課列舉式的。個別的。只針對不動產,跟,動產的機動車輛,所以臺灣你只要持有非財產稅標地的財產,原則上不會課到任何稅賦。例如,臺積電股票,不會被課稅。例如持有畢卡索的名畫,高價藝術品不用被課稅。臺灣不課總體財產,我們把這一類把一個人擁有的一切財產全部算進來了,這個稅字叫總體財產稅。歐洲一些國家有一些是課總體財產。臺灣是課個別財產說,我們只針對不動產跟機動車輛的動產課稅喔,這個是我們的財產稅。

交易稅,有四類:證交、期交、契稅跟印花。證交、期交、契稅、印花,你把他想像成不能吃的不能咬的不能消費的。針對交易行為,或者是表彰交易行為的契據,課徵的這個就是交易稅。

最後一個叫消費稅。消費稅分成一般消費跟特種消費。一般消費是課營業稅。 我們的一般消費行為,所謂的一般就是針對原則上所有的商品跟勞務的消費行為,原則上都被課稅的這個叫一般消費稅,臺灣課營業稅。我們還有特種消費,也就是特種物品的消費行為、特種勞務的消費行為,總共有以下這四種不同的特種消費稅。包括貨物稅、菸酒稅、娛樂稅、以及特銷稅,他本身就叫特種貨物及勞務銷售稅。

所以這一種不同的稅捐名目的分類,他放在這個經濟活動跟量能課稅原則的指標底下去,共同去做理解。你很快就可以明白稅法是一個整體。去計算人民的稅捐負擔和不同的經濟活動跟不同的成果去做課稅。我在編輯這一本所得稅務小六法的時候,就是照這個邏輯下去去編排。簡單來講就是,原則上屬於所得稅類的,他會有一個該有的大致上的規範結構,譬如說他們都會有免稅額跟扣。他們可能都會有一個累進稅率的適用。因為他們是同一個家庭出來的小孩子,長得理論上應該要一樣才對,當然也會有同中有異。財產稅跟所得稅一樣,都是針對直接的持有,被稱之為叫稅捐負擔能力的這個部分去課稅。但財產跟所得的差別在,所得是實然所得,財產是應然,應該------你擁有財產,你應該會有這麼多所得。所以,財產稅又被稱之為叫,針對應然收益,應該產生應該出現的收益,所做的稅捐負擔的預估,跟計算應然收益。

交易跟消費,我們看到他是對一個經濟活動比較偏末端的課稅的活動,這個課稅的活動原則上有消費稅就不課交易稅。哦,原則上如此,當然,在我們的法制一堆,有可能是構成例外。相反的不消費的東西原則上是課交易稅。交易稅在本質上是一個,透過交易行為所產生出來的費用的增加。所以交易稅的本質是費用稅,也就是說,他增加了一個維護交易秩序所需要的成本。透過,交易稅的課稅,要求這些參與交易行為人去負擔,所以他在稅捐本質上接近規費。接近規費的本質。這個地方,我們在稅法總論裡面會有進一步去說,稅捐跟規費差在哪裡。
交易本身是對交易行為產生出來的費用的一個回饋的機制,而這種交易稅本質上面喔應該是反映出他增加出來的行政上的費用。

最後一個叫消費,本身也是一個稅捐負擔能力,剛剛提到一般消費行為跟特種消費特種消費行為,由於加重稅負,所以必須有特別的立法上理由,往往都是來自於節制能源的使用,也就是對環境不友善的生產製造的行為所課徵的稅賦,譬如貨物稅;或者是對國民健康有害之物品的消費,比如說菸酒稅;乃至于娛樂稅的時代背景底下,認為對一些娛樂活動娛樂活動所加增的娛樂稅的稅賦;跟對奢侈品課徵的特銷稅\ldots\ldots 都具有特別的社會目的在這背後,從而正當化稅捐負擔加重的。

透過這個圖形,我們大致上讓各位看到綜所稅跟營所,他們的稅基的主體就是不一樣。稅基主體不重疊。綜所跟營業稅。他的課稅的稅捐負擔能力是不一樣的。不要混在一起。分類的目的除了辨識誰跟誰應該長一樣,也在辨識你跟他哪裡不一樣? 他營業稅跟所得稅不會重複課稅。我不希望看到有同學上完這個課以後。還在繼續講一個奇怪的錯誤的觀點:所得稅跟營業稅是重複課稅。好,那請問你,你的所得稅跟營業稅重複重複在哪? 賣你錢,賣你勞務,賣你東西的課所得稅。消費這個東西,課營業稅,他們會認為這個是重複課稅,是因為他覺得賣你東西那個人,他除了要繳所得稅以外,他又透過------我們現在目前的消費稅都是營業稅------營業人要負擔了,也就是營業人同時要付營業稅跟營利事業所得稅,所以他覺得這樣叫重複課稅。這個觀念是錯的。

營業人負擔營業稅? 他的本質是消費稅。稅法上容許,而且也要求你一定要轉嫁出去給消費者負擔。當然這個要求有時候營業人受迫於市場經濟活動,他如果抬高價格,他可能絕對會失去顧客。這一個沒有現實上沒有轉嫁,確實會因此減少他的利潤。比如說我賣100塊的東西,我標價原則上賣出去就是要加5塊錢的營業稅。假如我為了要爭取客戶,我就只賣100塊,我確實會因此實質上,因為我負擔那個營業稅,因為我是營業人,我去賣東西的時候,我要報營業稅。會減少我的營利事業所得的利潤。但這個不是稅法想要的結果。這個是你沒有把他轉嫁出去,因此會回過頭來影響減少你在營利事業所得的收入啊,收入減少就是說你的費用會因此增加,你沒有把那個消費稅把他轉嫁出去,你會回過頭來內傷,讓你的營利事業的利潤減少。從而他確實在實質上減少的你的營利事業所得額。但他在法制上面並不構成營業稅跟所得稅的重複課稅。

我們講重複課稅。我們講重複課稅必須是在同一種稅內,同一個稅類,不同的主體,因為同一個經濟上的活動,而被課徵同一種類型的稅目。所謂的雙重課稅,他的前提是同一種類型的稅捐,同一個稅捐主體,在同一段時間內,被課徵同種類型的稅。這種情況才叫重複課稅。重複課稅最容易表現在國際租稅法的領域,因為。他可能同一個經濟上的活動的所得會被甲課稅主權國跟乙課稅主權國同樣課稅,因為一個認為他是你的稅籍居民國,另外一個認為他是你的所得來源地國,所以他會同時對你施展一個課稅權限的行使,這個時候同一個主體同一個客體同一段時間備課程同一種類型的稅捐,這種情況才叫重複課稅。不是這種情況,不要混在一起。

一國課稅主權內,自國的立法者當然要避免重複課稅。所以土地增值稅跟房地合一稅要自己避免重複課稅。這個除了土地以外。營利事業跟個人綜合所得也是一樣的道理啊,原則上是要避免一個同樣一個獨立的稅捐主體,被課徵兩個不同的稅目。這個我們在不同的時間點裡面被課徵同一個稅目,其實不構成重複課稅。不同時間點。也因此重複課稅,他本身是有一個定義上的要求,並不能夠隨便去做使用的。

回到我們的綜所稅,在這個架構底下。綜合所得稅是專門針對個人市場經濟活動,原則上綜合在一起被課徵的一種主體稅。這樣我們就把他,基本上先定義下來。不然的話,以後我們再討論個人綜所稅的時候會有,互相沒有交集的問題。那麼回過頭來,透過這個圖形,這四種稅都是一國之內課徵的稅捐,一國課稅主權範圍內課徵的稅捐。

相對於一國之內課徵的稅捐,此外的稅賦,這個我們就稱之為叫:國境稅、關稅。是通過國境被課徵的稅捐。概念上。關稅可以針對通過國際的人、物,課徵。可以針對進口、出口、轉口課徵。概念上,概念上可以針對人,也可以針對物。可以針對進口,可以針對出口,可以針對轉口。但我們各國的關稅在臺灣,我們只針對物。我們只針對進口,才課徵關稅。所以日本人韓國人來臺灣旅遊,我們不課徵關稅。

那麼什麼東西會被課徵關稅?稅法裡面所列舉的,原則上是一切貨物,除非有免稅的規定。只針對進口,我們原則上不針對出口。但如果要保護本國的稀有資源,是可以課徵出口關稅的。比如說保護木材,保護,石灰。稀土,有些國家針對一些稀有的能源去課徵的出口關稅。

轉口關稅一般而言,世界各國除非地處要害。像新加坡處於麻六甲海峽和巴拿馬處於巴拿馬運河。那你對來往的。不是要以你們國家作為進出口目的的這個課徵。轉口稅,轉口關稅,除非你們家剛好就位處在地理要衝,不然一般而言,如果你課徵轉口說一般商貿就不會從你們家過,就這樣而已。這就是關稅關稅,我們原則上在我們國家只針對進口貨物課徵關稅關稅裡面再進一步分成一般關稅,跟特種關稅。哦,他還有分一些所謂的特種關稅,那麼關稅被認為是阻礙自由貿易的,一個過時的產物。所以WTO成立的目的就是為了消滅關稅及一切非關稅之貿易自由的阻礙跟障礙的事物。

有些國家會透過比如說比較高的農業檢驗標準啊,或者是透過用關稅進口配額的方式,好像我們臺灣早期為了保護自己本身的紡織產業,對國外進口的紡織產品,我們給予一個叫做進口關稅的配額,也就是一年的扣打就這麼多。好,你要進來的話,大致上也就適用這種進口關稅的方式去控制你的國外進口的紡織品的數量,這一個都是保護本國內國的貿易行為的,一種非關稅或關稅的一切貿易障礙是WTO組織成立的一個首要目標,關稅是WTO要成立,要設立要消滅的目標,只是到目前為止沒有被消滅而已,反倒越來越興盛。

因為各國的自由貿易的這一種理想終究不敵現實。目前為止實現各國之間關稅。大家組成一個共同關稅聯盟,大概只有在歐盟境內歐盟境內彼此之間不課徵關稅,但歐盟針對第三國,包括美國加拿大日本或是臺灣,他本身還是會課徵所謂的進口的關稅啊,相反的,現在包括東南亞在內,包括美國加拿大大部分都是簽署雙邊的租稅貿易協定,或者是貿易協定來免除互免關稅。到目前為止並沒有完全被消滅。

WTO成立的目的只是要消滅關稅,他並沒有消滅内地稅哦,所以内地稅裡面的營業稅内地稅裡面的各種貨物,特種消費稅的貨物稅,跟關稅,不構成被認為重複課稅的態樣,因為內地稅本身要課徵的稅務,往往是針對消費或者是特種消費行為所課徵的稅負。

我們第二個小時。那前半段就跟各位介紹到這裡,綜所稅在整個我們的稅制裡面的地位,在經濟成果經濟活動的前端,屬於直接說,直接透過市場經濟活動所取得的。財產上增加,他是針對主體所課徵的稅捐,也就是自然人個人課徵的稅捐。

\hypertarget{ux6240ux5f97ux7a05ux6cd5ux7684ux898fux7bc4ux67b6ux69cb}{%
\section{【所得稅法的規範架構】}\label{ux6240ux5f97ux7a05ux6cd5ux7684ux898fux7bc4ux67b6ux69cb}}

接下來請各位。翻開來所得稅法,因為我們要跟各位介紹一下我們所得稅法的規範架構。這個介紹,極為重要,因此各位如果手頭上沒有法條,還是請各位稍微把他從網路上面稍微看一下我們的所得稅法。因為,不是只有一個稅目。不是只有今天我們要上的綜所稅,還包含營利事業所得稅,所以各位同學,你可能要稍微分開來一下,要稍微把那個所得稅法拉出來。

我們的稅法的規範結構,從他的性質來看,總共區分成3個大區塊,一個我們非我們稱之為叫稅捐實體法,也就是稅捐債務法。稅法的規範結構。我們的稅法的規範結構在幾乎大多數的稅捐都長的非常相像,都非常類似。

第1個大區塊被稱之為叫稅捐債務法,或者稱之為叫實體法。他主要是規範,稅捐的構成要件,跟法律效果。所得稅法的規範結構。他的第一個最重要的區塊,這也是個稅法規範的最重要的任務,就是規定他的構成要件。課稅構成要件免稅構成要件跟他的法律效果稅額怎麼算,稅額怎麼算?
第二個區塊叫做稅捐稽徵法。規範他的申報,跟繳納的義務。
第3個我們稱之為叫稅捐制裁法。也就是規範納稅義務人或者是申報協力義務人的申報義務的違反的制裁。這個地方只有行政法的制裁,我們的刑事法制裁放在稅稽法。

所得稅法的規範結構亦如是。大致上分成稅捐債務法、稅捐稽徵跟稅捐制裁法。所以稅捐債務法裡面又分成,根據課稅的構成要件,我們再把他分成了稅捐的主體,第二個稅捐的客體,第3個稱之為叫稅基,第4個稱之為叫稅率的規定。依法課稅原則,展現在,原則上,稅法自己會定這些主體、客體、稅基、稅率。

各位,今天下課回去以後,我想要請各位做一個homework好,當然不需要交,就是你把法條翻出來,把構成要件,稅捐主體,就是講納稅義務人,是誰? 所得稅法上的稅捐主體,個人是指誰? 好接下來,客體,所得稅法上的所得就是他的客體。所得稅法上的稅基就是應稅所得額。所得稅法上所課稅的稅率是他的課稅的累進稅率。所得稅法上的法律效果就是他的應納跟結算稅額。所得稅法上的稅捐稽徵程序就是申報的規定,申報程序之外的繳納的規定。所得稅法上裡面的所得的稽徵程序,除了申報繳納程序以外,我們還有另外一個就源扣繳程序。所得稅法上的制裁,就是關於你違反申報繳納違反扣繳義務的制裁規定。

今天各位的一個任務就是回去先認識一下你要上課的對象,好好多看一下他長什麼樣子。因為我們的稅法,所得稅法,他沒有按照老師給各位的這個順序一條一條地規定。理論上比較好的規範結構就是,一個條文規範一個概念。但是由於我們的所得稅法,他包含了,不同的兄弟在這裡面:包含了營利事業所得稅,包含了房地合一稅。所以你就只好把他們挑出來,好把他挑出來。
然後稅捐主體,稅捐客體有些沒有規定哦,沒有寫。好,那這個地方客體的類型,因為本來是課綜合所得稅,但是我們還是有很多分類所得的規定。然後稅基就是應稅所得額的計算。啊稅率累計稅率好,我們的綜合所得稅,是累計稅率,申報繳納,就源扣繳,跟違反協力義務的制裁的規定。相關連的一些基礎概念,有一些可能我們在稅總裡面,再請各位,按照需要再去把他稍微做一些背景的了解。

我們這一個學期的課,原則上就是逐次展開,只是講得深入或淺出差別的不同而已。但各位要參加考試,你好歹法條要稍微熟一點吧,因為是各位要考試啊哦是吧?

我們上課,原則上就是透過一整套的構成要件、法律效果、稽徵程序、制裁來跟各位講,當然我們下個禮拜會先從原理原則開始,因為這個是講應然的一個非常重要的部分,很多稅法的規範構成要件,並不符合應然的這些課稅的基本原則。我們下個禮拜再跟各位去談這些課稅基本原則,今天先到這裡。

\hypertarget{section-2}{%
\chapter{20230911\_01}\label{section-2}}

\begin{longtable}[]{@{}l@{}}
\toprule()
\endhead
課程:1121所得稅法一 \\
日期:2023/09/11 \\
周次:2 \\
節次:1 \\
\bottomrule()
\end{longtable}

我上個禮拜,是希望各位同學把所得稅法的法條找出來,我要你去做歸類。哪一些是屬於稅捐債務法,哪一些是稅捐稽徵法?或者是第幾條到第幾條是稅捐債務法的規定,第幾條到第幾條是稅捐制裁法的規定?
各位同學務必在接下來的課程裡面都必須要去做這個工作,因為我考試會考這個。請你按照順序重新排列一下,我們的綜所稅究竟在我們整個法條規範的順序,法條的内容,就是------稅捐債務法,稅捐稽徵法,稅捐制裁法。我們的每一部稅法的法典,基本上都是這3塊組成。那你必須要把他分辨出來,哪一個部分是哪一個區塊的。

\hypertarget{ux7a05ux6cd5ux898fux7bc4ux9ad4ux7cfb}{%
\section{【稅法規範體系】}\label{ux7a05ux6cd5ux898fux7bc4ux9ad4ux7cfb}}

在稅捐債務法裡面,我們可以依照依法課稅原則。
我先給各位一個實然,就是現在的狀態,現在就長這個樣。所謂的法律規範的實然狀態,我們在法學方法上,把他稱之為叫外在體系,就是法條跟法條之間的規範結構。
體系做最好的立法,就是民法。因爲歐陸法系就從民法開始,所以歐陸法系的民法,本身體系分明。整個民法建構在,包括所謂的身份,所謂的契約,那麼這兩個都共同適用的法條規定,我們把他放在民法總則。接下來,在契約裡面,或者是在我們財產法裡面把他再進一步分成了債法跟物權法。原則上我想各位都很清楚這樣一個規範的結構。
其次做最好的規範體系安排的,像刑法,刑法有總則編,有個別的犯罪規定的,個編的,也就是刑法分則。我們會按照國家、跟社會或個人之間的法律的侵害,或者是超個人法益的侵害,就是包括國家跟社會。
不管怎麼分類,他基本上的思考邏輯,反映在立法者的規範體系結構上面。刑法總則是分則的適用上的諸多抽象化的原理原則,具體化成為法條規範內容。

這個就叫做體系,立法者表現出來的立法上的體系,我們這個把他稱之為叫外在體系。根據立法者所表現出來的外在體系,往往他未必能夠表現出我們在思考上面來講,從抽象到具體,從一般到個別,這樣的一個規範上的結構。理論上來講,從抽象到具體,所以在個別法律適用的時候,原則上是具體優先於抽象,特別優先於一般,這個是我們各位,幾乎你在其他法領域裡面學過的基本的常識,基本的這種規範結構。

稅法本來也應該是如此,理論上也應該要有一個總論的東西。

但實際上,臺灣由於一開始本身就做分散式的立法,所以臺灣的稅法,結構上沒有很完整的外在體系,他是分散式的。
如果我們有一個稅法總則的法條規定的話,那麼關於稅捐稽徵程序跟稅捐制裁的相關規範,其實他就收攏過去,讓稅法總論去做規範就好。

以德國為例,大致上就是這樣一個規範結構,也就是說,各稅裡面,原則上他只有稅捐債務法,進一步去區別。各稅,原則上只有規範了稅捐構成要件跟法律效果。如果有必要,做個別稽徵程序的個別規定,那他就在那個稅法裡面,再做一個特別的規定,因為你要有別於原先在稅捐總則法裡面的稅捐稽徵程序。那樣的法條規範結構會讓法律的學習本身變得體系上比較完整。

很可惜的是,我們從稅法一誕生開始,我們一直就不是這樣立法,我們基本上分散式立法,也就是講好聽一點,叫做我們看到哪裡有問題,哪裡有需要,因此我們就立法。講難聽一點,就是我想要什麽稅,想要什麼東西的時候,我就才個別去做立法。因此,這種立法體系結構很紊亂,也很容易產生重複規範,甚至是相互矛盾而不自覺。也就是,一般法跟特別法之間的規定,他可能互相在這裡面兜不起來。而且也常常會有這樣一個重複規範的結構。那么我們現在就已經長這個樣子,沒辦法。再怎麼醜,也還是自己的法律。我們不是要學德國法,我只是告訴各位一個實然狀態,跟相對的叫應然。所謂的應然,就是我們透過一個一套基本的價值來去重整出來的法條規範的結構。

我們的法條現在的狀態,我們把他稱之為叫外在體系。

我上個禮拜跟各位講過,透過量能課稅原則,我們可以把所有的稅法規範,稅法的各稅全部都把他排列成一個經濟活動的不同端點,去做課稅。這個在某種程度上是透過所謂的量能這樣一個價值。内在體系就是透過價值而去做的規範排列安排。體系如果一致,就代表著這個法領域的發展相對是比較成熟的。人們,即使是沒有看法條規定,他大概可以猜想法條應該長什麼樣。這個部分,尤其各位在學民法,比較容易有深刻的感受。因為民法體系是整個歐陸法系發展最早的。越早的東西,他就越成熟,越完整。

拿破崙統一整個法國以後,他首先要證明自己不是只有打仗,不是只是從一個科西嘉島,農村鄉下來的孩子,他認為自己是有這樣一個文治能力的,就是通過1804年拿破崙民法。晚一點的德意志邦聯國家,到19世紀末20世紀初期,施行了德國民法。這兩部法典其實都是歐陸法系繼受了羅馬法的精神以後,把他成文化的一個最典型的代表。法律本身如果反映出內在價值體系,他就會讓你的外在體系規範結構,接近我們剛剛所講的那樣一個價值規範。也就是說,如果今天我想要把一個法條做一個規範結構上的重新排列組織,在我們的體系裡面,稅捐債務法,其實就是透過構成要件、法律效果\ldots\ldots 什麼叫依法課稅,就是法定債權債務關係。

\hypertarget{ux7a05ux6cd5ux4e0aux69cbux6210ux8981ux4ef6}{%
\section{【稅法上構成要件】}\label{ux7a05ux6cd5ux4e0aux69cbux6210ux8981ux4ef6}}

那至於哪一些是構成要件裡面的要素?我們在稅法總則的課堂上面我們花了許多時間去跟各位去講,基本的構成要件,叫稅捐主體,稅捐客體,稅基,稅率。這4個,就被稱之為課稅的構成要件。
各位如果上所得稅法,沒有這個基本觀念,我會強烈地建議,你一定要回過頭去看稅法總論,總論的東西就是這一堂課的基礎。當然我們上課裡面有許多基礎觀念的部分,如果你上過還是不太清楚,請你務必再回過頭去看。因為我們在這個課堂上沒有辦法,就這些基礎觀念,再一次地去跟你講,那個是什麼東西?要做自己的功課。

\hypertarget{ux4e3bux9ad4}{%
\subsection{【主體】}\label{ux4e3bux9ad4}}

何謂主體? 權利義務的歸屬對象。只有人,不管是自然人或法人都沒關係,但只有人,才可以是一個權利義務之歸屬主體。就跟民法第6條一樣,人才享有權利才能負擔義務,這一種享有權利負擔義務的能力,資格,身份,地位,ability,這個就叫做權利能力。訴訟法亦如是,相對應於實體法裡面的權利能力概念,叫當事人能力,死人是不能享受權利,也沒有辦法負擔義務的。歐陸法系的整體結構,就是建構在一個體系井然分明的:人,才能享受權利負擔義務。

以所得稅為例,稅捐主體(基本權主體)是人,所以,取得所得的人,那個人就叫稅捐主體。

課稅主體(課稅權限行使之高權主體)在現行稅法裡面不規定。課稅主體,我們是放到財政收支劃分法裡面,簡單來講就是如果你比照民法上的債權債務關係,債務人,我們在稅法規定;債權人,我們則放在財政收支劃分法。這個也是在稅法總論裡面會談的問題。稅捐債權人,在各稅法原則上是不規定。我們規定稅捐債務人。

至於你的權利歸屬的,有收稅權限的,在我們國家放在財政收支劃分法,在德國放在憲法。所以他們稱之為叫財政憲法。財政憲法不是隨便的,因為他們在基本法的規範上面直接明定這個收稅的人是誰。是國家聯邦,還是是各邦,還是各地方自治團體,他們是直接放在憲法層級,保障地方自治團體的稅捐收入,不會被代表國家的聯邦給侵奪。
每個階段,每個層級的政治團體,他受到了所謂的制度保障,那個概念,他不是空口說白話,因為他是放在憲法裡面,這就作為一個聯邦國家的那個體制,他本身呈現出來。
我們如果放到法律的規定,像我們是財政收支劃分法的話,他的保障層級相對就比較低。因為,既然放在立法,那立法者自己就可以形成,自己就有形成的空間跟可能性。

\hypertarget{ux5ba2ux9ad4}{%
\subsection{【客體】}\label{ux5ba2ux9ad4}}

至於對象的這個,就稱之為叫客體,也就是在所得稅法裡面,稅法裡面所要課稅的那個對象,是物是行為,那個我們把他稱之為叫客體。

\hypertarget{ux7a05ux57fa}{%
\subsection{【稅基】}\label{ux7a05ux57fa}}

什麼叫稅基?稅基是客體的數量化。
舉例,民法上的同樣法定之債,各位想到這裡法定之債是什麽?在學民法債編的時候,有契約之債,有法定之債。法定之債是什麼?最典型的法定之債,侵權行為啊。所以,侵權行為,一定要每個侵權行為的構成要件,民法184條第一項前段因故意或過失不法侵害他人權利者,應負損害賠償責任。因故意,或過失,不法,侵害,他人,權利者。這個叫構成要件的各項要素。我們把構成要件,把他每一個排開來。

\hypertarget{ux7a05ux7387}{%
\subsection{【稅率】}\label{ux7a05ux7387}}

稅率就是課稅的比例。

\hypertarget{ux7a05ux6cd5ux4e0aux6cd5ux5f8bux6548ux679c}{%
\section{【稅法上法律效果】}\label{ux7a05ux6cd5ux4e0aux6cd5ux5f8bux6548ux679c}}

法律效果其實只有一個,就是稅基乘稅率,再得到稅額。所得稅法的規範結構裡面,稍微有一點點的不太一樣,就是我們這個稅額在所得稅裡面,他又分兩個概念,一個叫應納稅額,一個叫結算稅額。因為算出來稅基乘稅率的稅額以後,原則上並不是最終實際要繳納的那個稅額。這是法律上算出來,第一層裡面稅基乘稅率算出來,應該法律上應該要付的稅捐債權債務關係。

但由於我們有許多先前繳納的稅捐,所以你可以去做扣抵。有一些透過稅捐優惠,我們讓你可以去做扣抵,所以他會有第二個層次的,結算稅額。結算意思就是settle,就是最終你要繳給國家,繳給稅捐稽徵機關的那個數額。所以,所得稅法稍微對稅額再分兩個層次。一般的其他稅目不會有這麼細,一般來講都是只有應納稅額,應納稅額通常就是結算。但是所得稅法自己本身分得比較細一點。

\hypertarget{ux518dux8ac7ux7a05ux6cd5ux898fux7bc4ux9ad4ux7cfb}{%
\section{【再談稅法規範體系】}\label{ux518dux8ac7ux7a05ux6cd5ux898fux7bc4ux9ad4ux7cfb}}

在我們的規範結構裡面,我上個禮拜,是希望各位同學把所得稅法的法條找出來,我要你去做歸類。
這件工作,因爲我們的立法者,外在體系沒有做很好,所以就變成是各位的工作,是我們在教學上面要跟各位去强調。因為這個其實本來應該放在稅稽法。你看哦,我們臺灣自己就有自己的稅稽法,可是稅稽法沒有規定稅稽應該要有的相關規定,所以他又放到所得稅法裡面去規定。

我們的幾乎所有的稅都長這個樣子,稅稽法沒有稅捐總則,稽徵法,這樣該有的規範結構,所以很抱歉,不好意思就只好請你個別去看各稅的稽徵程序,因為我們的稅捐稽徵法,講白一點就是該規定都沒規定。
大概就是這樣一個外在體系殘缺不全的狀態,所以他也讓稅法難學,這個也是他的一個背景因素。稅法本身,從內在體系,其實他是會有一個整體規範結構,就像你在學民法跟刑法一樣,他其實並沒有什麼特殊性。

只是由於立法者自始從我們這一整套的法律\ldots\ldots 其實我們的各稅往往歷史都還比稅捐稽徵法還要更早。許多稅,以所得稅而言,我們上一次跟各位講過,他民國4年開始誕生的時候就出來了。所以當時沒有稅稽法的規定,因此當時的立法者沒有實際徵稅。民國4年誕生所得稅條例的時候,其實講白一點,那時候只是法律,根本不是實際執行的法律規範,為什麼?因為他掛在中央政府。中央政府在哪裡?在北京那又怎樣?北京,他的課稅權限,除了在北京城附近可以行使以外,到了南方誰理你啊?課稅權限,很大程度上是跟這個國家的統一跟分治的狀態有很大的實際的關聯性。
所有的稅的出現都是當時候多數都是國家需要錢,以稅為名跟人民要錢。法律制訂出來了,就看國家的能力能不能徹底去挖到人民的口袋,把錢掏出來。基本上都是這樣一個規範結構。當然我們,在法律裡面,想要把這樣一套的國家對人民徵收財產金錢為內容的,給付義務,希望把他做體系化,希望依靠原理原則,一方面,固然許可了,國家根據這樣一個原理原則,跟人民徵收稅捐金錢的可能,但回過頭來,其實他也是保障國家不得對人民進行違法徵收跟過度的徵收、過度干預。他一體兩面。他許可了國家對人民的徵稅行為,在同一個時間,他其實也在保障,國家不可以做法律之外,對人民財產權跟工作權的過度干預。

相關的課程也是回到稅法總則,因為我們在各論裡面,我們只就個別的法條規定,個別的規範結構之間的體系性的關係去做進一步的深化,不然我們會永遠一直都還在稅法總論裡面去打轉。好像各位學完民法總則,你總是要進一步推進到債編總論,要推進到債各,你的那個民法,才會具體化,才會活起來。

法律運用的時候,一定是抽象具體,一般個別之間的交互地去做提示。其實稅法亦如是,只是我們規範的外在體系跟內在體系,這個鴻溝不是只有一條,我這樣畫的白線而已,他是差距很大。我們今天就是要跟各位去講,從內在體系的價值去看,我們現在目前的法條規範結構,外在體系,內在跟外在。內在就是講價值,外在就是講立法者客觀的呈現。當然哲學上有一句話講,存在有他當為的基礎,為什麼長這個樣子,為什麼長這麼醜?他當然有他現在為什麼長這麼醜的原因,但沒有辦法,我們還是必須跟各位講,他其實本來應該可以長得漂亮一點,意思就是說,讓各位比較容易像民法像刑法一樣學習。不過很可惜,現在不是長這個樣。因此,請各位如果你沒做,沒關係,我現在帶著各位一起做。但是你要有法條,你要看著那個法條,你才有辦法把他一個一個把他給扶(?)起來。

\hypertarget{ux958bux59cbux770bux6cd5ux689d}{%
\section{【開始看法條】}\label{ux958bux59cbux770bux6cd5ux689d}}

我們來看一下所得稅法的條文,我們從第1條開始。第1條:所得稅,分為綜合所得稅及營利事業所得稅。這個條文告訴你說,我們所得稅法包含了幾個稅目。但這個條文很快你就可以知道,從民國105年以後不合時宜,爲什麽,有房地合一稅啊,他也不是綜所也不是營所。法條自己都立刻打臉,我們的立法者就是這種分散式立法,他後面改,前面不會改。他只會改他看到的那個地方,這個叫頭痛醫頭脚痛醫脚。第一個條文就先打臉自己。以前,105年以前我們的所得稅確實分成了綜所跟營所這兩個,這一部法典主要是規定這兩個稅目,但實際上,105就多出來一個無以為名的稅目。其實你如果待會去看那些法條規定的時候,其實房地合一稅,還只是財政部在他的行政規則裡面去,稱他叫房地合一稅,法條本身的名字還不一定真的這麼清楚地,給他一個稅目的正式名稱,因為我們在那個地方連那個稅目名稱都沒有在我們法條的規範結構裡面出現。

\hypertarget{section-3}{%
\subsection{【1\textasciitilde3-3】}\label{section-3}}

我們來看一下,第1條,所得稅分綜所跟營所,我都用簡稱就好了。

第2條是綜所的規定,第2條就是典型的綜所稅規定,就是課稅構成要件的規定。當然也包含法律效果。就像民法184條前段,因故意或過失不法侵害他人權利者,應負損害賠償責任。故意跟過失,什麼叫不法侵害,什麼叫應負損害賠償責任?其實他還是要由後面一大串的法條規範來把他補滿。

第2條本身就大概是這樣一個規範結構。有中華民國來源所得之個人,稅捐主體就是個人。取得中華民國來源所得,這個就是你的客體,他這裡面包含你的客體的實現地點。
這個就是我們之後每一堂課會講一個課稅構成要件。個人就是主體。客體則是所得,在中華民國的來源所得,我們會有所得的實現地點。

然後,應就中華民國來源之所得,這個地方沒有稅基的規定,他沒有講很清楚,真正要講,是說,應依本法規定計算所得額。所得額才是稅基。

客體是所得,把他數量化,所得額或者比較好的說法叫應稅所得額。剛剛提到的侵權行為,因故意或過失不法侵害他人權利應負損害賠償責任。損害賠償責任,就是在講,損害賠償額怎麼算。重點不在客體,要負損害賠償的重點,其實輸贏上下就是那個數額怎麼算?

比如說我不小心把你撞傷了,我願意賠你,但對不起,因為我月薪收入只有3萬,我一個月1\%賠給你3000塊,可不可以? 一個月只賠3000塊,要賠多久?如果各位將來進入律師這個職務的工作場合,侵權行為損害賠償。那個人說我很有誠意,我願意賠你。很抱歉,對不起,我不小心撞傷你了。哦,但是談賠償的時候,他說我願意賠你100萬,可不可以幫我簽個和解?那請問你怎麽支付那100萬?他説,對不起,我工作收入很少,一個月只有3萬塊,那我一個月賠你3000。你想想看,一個月賠你3000,一年才三萬六。100萬要賠多少年?你認為,那樣的一個侵權行爲人,有沒有賠償的誠意呀?他説我願意一輩子賠你。對啦,先給我看看你的錢在哪裡,你的誠意在哪裡?一個月賠3000塊,對100萬來講,這根本不叫做有誠意的損害賠償。

回過頭來,損害賠償是整個損害賠償法裡面,最根本關鍵的。各位都學過在民法裡面究竟如何計算損害賠償?有民法196有民法213,第一項或第三項的規定。不管是回復原狀或者是,對於減少物之市場上的價格計算,他都是一種approach,如何算出你損害賠償額度的辦法。

那麼,同樣的,在我們的稅法裡面也是一樣,稅基最重要。老師以前早期在講稅法的時候,尤其是早期在民國104年的時候,臺灣曾經想過要課證所稅。稅基最重要,看你所得怎麼算,這個才是根本關鍵。學財稅的、學會計的人不會有這個觀念,他們認為稅率最重要。稅率最重要?不對。任何正確的稅捐負擔能力的掌握是看你的稅基。稅基怎麼算?你今天跟我講要課所得稅,但是所得額亂算。這個不叫做量能課稅,不叫做稅捐負擔的公平正義。

我就舉臺灣最常見的,為什麼一堆人去炒土地?在房地合一稅沒出來之前。 因為臺灣長期以來,土地就沒有被實價課稅過。我要課土地增值稅,土地增值稅啊,結果你的增值怎麼個算法? 從來不是按照實際轉售價值,從來就是按一個所謂的政府公告現值差。政府公告現值差就是一年調一次。所以你只要年初買年底賣,請問你怎麼會有公告現值差? 就算房地產你買進來是100萬,年底賣出去是200萬,白花花,我就是賺了100萬。結果在政府或是在法律的眼中裡面說這個沒有公告現值差。所以我們土地增值稅,說要課稅,結果呢?他並沒有真的課,所以有土斯有財不是原因,他是結果。

簡單來講,就是臺灣長期以來對土地,根本沒有正式去面對他,然後我們一天到晚在那邊講說,土地增值稅有憲法上的基礎,我都還不知道你憲法基礎是怎麽來的。胡説八道,真是太誇張。臺灣長期以來沒有真正的土地增值稅。不需要做太多事情,就是你真正買多少? 賣出去多少?這個不是很簡單的基本價值嗎?對臺灣就是做不到,很抱歉,我們真的到105年為止,我們就從來沒做。從105年,我們才終於開始。但,有日出條款。

所以現在,臺灣絕大多數的房屋,原則上你只要在那個時間以前取得的,原則上都沒有房地合一稅,都是按照舊制。所謂的舊制,就是他不是實際的轉售價差,他是一個根據想象出來的稅捐負擔能力。所以回過頭來還是跟各位講,想像的,不是真正的量能課稅。量能課稅是實際上你的應稅所得的額怎麼算。

稅率就是課稅的比例。在直接稅裡面,基於19世紀進來社會國的思想,所以通常會有累進稅率的設計。

我們的所得稅算是在那個時代裡面的產物,做不做得到是其次,確實在我們的直接稅裡面,最典型的所得稅,是有累進稅率。遺產贈與稅也是一種直接稅的類型,他是一種特種所得的類型,所以我們現在表彰出來的,也是一種累計稅率的設計。不過遺產贈與稅也曾經在我們的某一個時期,是只有固定比例的課稅。

這個就是我現在要透過法條規範的結構來慢慢地在實然跟應然之間去做比較。這也是我們今天課程的主要內容。

第2條第一項是課稅的基礎構成要件,跟他的法律效果的一個籠統性規定,因為實際上他並沒有把各項構成要件的要素都完整描述。從規範結構上看起來,基本上只有主體和客體,但怎麼算稅基,應該是用哪一條的稅率,他是沒有講得很清楚,只是在「應依本法的規定」裡面,做籠統性的規定,就這樣而已。所以你說他是完全性法條? 不太是。真正很典型的完全性法條是像,剛剛我念到的那個民法184條第一項前段。所謂的完全性法條就是,這個條款裡面,構成要件跟法律效果,他都已經把他寫完了。

刑法很多,刑分裡面大部分都是完整性的法條規定。比如說,因故意致人於死者,他的法律上效果,他就直接在這裡面構成要件,法律效果全部在一個法條完成,這個叫完全性。但大多數的法條都是還是要透過其他的法律規範去補充他,所以我們稱這一些補充性的叫不完全性的法條,因為他本身不是獨立存在,他是要去補一個完全性法條,讓那個構成要件跟法律效果能夠完整。

那也因此,完全性法條其實是第2條的規定。第2條第一項跟第2條第二項。因為第2條第一項是境內居住者的課稅構成要件的規定跟法律效果規定;第2條第二項叫非境內居住者的課稅構成要件,跟法律效果的規定。

第3條,就不是綜所稅的規定,就是營所。回到上個禮拜跟各位講,我們的法條規範結構是混亂的,因為你每一個法條你都要去辨識一下,他到底是綜所還是營所?他適用的主體對象,稅捐主體是誰?因為這個地方稅目是有別的,稅目有差。

所以接下來我們來看一下,3-1條被刪掉的。

3-2條,看得出來規範的稅捐主體是誰嗎?
3-2條第一項規定,委託人為營利事業之信託契約,信託成立時,明定信託利益之全部或一部之受益人為非委託人者,該受益人應將享有信託利益之權利價值,併入成立年度之所得額,依本法規定課徵所得稅。

我們的法條的立法者,透過這個條款規定,你看得出來,他想要針對他的那個稅捐主體是自然人的個人還是在講的是營利事業?一定看不太出來對吧。
大多數的情況是適用在自然人,不過概念上不排除,不排斥,也有可能是適用在營利事業,如果他信託契約裡面的受益人是營利事業的話,確實是有可能,因為這個樣子,讓營利事業變成是信託這個收益的受益人,併入營利事業所得稅。
但這個條款,其實在實務上大多數就是營利事業,透過信託的方式讓自然人受益。大多數的情形是適用在自然人的情況。

但關於信託的課稅制度,我們會到所得稅法三才講。關於信託的所得,或者是透過信託去做遺產贈與稅規劃的,我們到所得稅法三,一併去講。因此,這個條文,我們跳過。

接下來3-3條。配合第3條的規定,信託財產的設定原則上不課徵任何稅捐,所以3-2條、3-3條,我們全部跳過。

\hypertarget{ux6240ux5f97ux7a05ux6cd5ux7b2c4ux689d}{%
\subsection{【所得稅法第4條】}\label{ux6240ux5f97ux7a05ux6cd5ux7b2c4ux689d}}

接下來第4條,第一項的規定:下列各種所得,免納所得稅。接下來,請各位一一分辨,他主體適用的對象是自然人還是法人?比如說第三款規定,傷害或死亡之損害賠償。誰能夠被傷害死亡?當然是自然人,法人不會有傷害死亡。所以4條第一項第三款前段基本上是在講綜所稅裡面的,損害賠償金免納所得稅。

第4條就是很典型的雞兔同籠,就是綜所營所都適用,那你要個別分辨。我們在這個地方我們確實會有一堂課來跟各位去討論這裡面的免稅所得的規定,因為他的法條雖然寫免納所得稅,可是就像我們公司法上的決議,他的決議瑕疵狀態包括了不成立,無效,跟得撤銷。是哪一種類型?你在學公司法的時候,要去辨識。那我們在這裡,都叫做免納所得稅,但實際上他可能是非所得。非所得,本身概念上就不是所得。所以規定免納所得稅其實不是,稅捐優惠,他其實只是去反映出,本質上不是所得,譬如傷害的損害賠償。去年,不小心在路上,被後面一個冒失鬼撞傷了,有人開車,不小心把你撞傷,撞斷了一條腿,賠你100萬,要不要去報所得稅?請問這個算不算所得?學過法律的各位,毫不猶豫,你不用什麼特殊會計學或是財務學的觀念。什麼叫損害賠償? 填補固有利益的損害,沒有財產上的增加。所以我們剛剛看到那個法條第4條第一項第三款傷害或死亡之損害賠償,請問放在這是稅捐優惠嗎?當然不是稅捐優惠。他是一個非所得,本來就不是所得,我今天放在這裡,只是告訴你,宣示一個既明的道理。就是,只要是填補固有利益之損害的損害賠償,原則上不課所得稅。
我們再請各位來看另外一個條第4條。這裡面一堆雞兔同籠的條文規定,包括了,比如,第十六款的規定。第4條第十六款,個人及營利事業出售土地。第十七款的規定,因繼承、遺贈或贈與而取得之財產。第4條的第二十三款的規定,個人稿費、版稅、樂譜、作曲、編劇、漫畫及講演之鐘點費之收入。這些全部都鷄兔同籠,叫做免納所得稅。

我們把第三款的因死亡或傷害的損害賠償,跟十六、十七款的個人出售土地跟個人取得遺產遺贈,跟個人取得稿費、版稅等收入。各位辨識一下來看看,哪些是真優惠? 哪些其實不是真優惠?他並不是優惠,他只是一個在闡明所得跟非所得概念的區別。
剛剛我們講過第三款的,因死亡或傷害的損害賠償,那個就是一個典型的他是非所得,所以他不是稅就優惠。我想各位學過民法,應該很清楚。
我們來看一下,二十三款的規定個人稿費、版稅、樂譜、作曲、編劇、漫畫及講演之鐘點費之收入。這些活動,爲什麽可以18萬免稅?因為立法者鼓勵你做文化活動。稿費、版稅是錢嗎?是。是你透過你的文化的創作,透過你的智慧的產出,所產生出來的收入。所以他是典型的所得。只是立法者鼓勵文化,這一種智慧財產的創新,或者是創設智慧財產權的文化活動,而給予的一個基於文化政策的稅捐優惠。也因此,這個條文規定,他本身就存在著立法者給予優惠的意思。但他跟第三款的規定,他都同樣放在這個地方裡面去做一致性的規範,看起來好像是同等對待,實際上是鷄兔同籠,因為他們性質完全不同。

回到十六、十七款,這是一個很典型,在臺灣實務上一直被誤認的規定,因為看起來叫免納所得稅,所以以為是稅捐優惠,實際上不是。十六、十七款,實際上是因為我們有另課土地增值稅、另課遺產贈與稅。所以他是一個分離課稅,避免重複課稅的規定。分離課稅,分離在綜所稅的稅基之外,併課別的稅目。為了避免重複課稅,所以規定免納所得稅。請問這個叫優惠嗎?其實不是稅捐優惠,其實是一個避免重複課稅,讓別的稅目去擔負起量能課稅的功能任務的一個條款規定。

光只有所得稅法第4條規定,你就可以看得到,同樣都叫免納所得稅,結果這裡面,是有所得跟非所得,有分離課稅的所得,跟真的是受到優惠獎勵的所得。你就知道我們的實然跟應然差多大。不是各位的問題,是我們立法者,在法律規範上,很大程度上沒有去區別出他該有的樣子。如果在這樣的規範底下,要能夠去理解整個規範的價值,說實在,沒有經過一個體系的呈現,有時候確實是不容易清楚地理解。

\begin{center}\rule{0.5\linewidth}{0.5pt}\end{center}

我們先休息一下,我只是用這個例子,但接下來我可能要很快的,把我們這一次上課要跟各位去做說明的,一些基本的原理原則跟各位介紹。

因為透過量能課稅的原理原則,我們大致上會把一些比較重要的,綜所稅的稅捐構成要件,立法上本來應該長什麼樣子,但我們現在實然的狀態並不是長這個樣子,我們要把情況把他做一個比較完整的呈現,我們先休息一下。

\hypertarget{section-4}{%
\chapter{20230911\_02}\label{section-4}}

\begin{longtable}[]{@{}l@{}}
\toprule()
\endhead
課程:1121所得稅法一 \\
日期:2023/09/11 \\
周次:2 \\
節次:2 \\
\bottomrule()
\end{longtable}

\hypertarget{ux9019ux5b78ux671fux6388ux8ab2ux7bc4ux570dux9010ux689dux7c21ux6790}{%
\section{【這學期授課範圍,逐條簡析】}\label{ux9019ux5b78ux671fux6388ux8ab2ux7bc4ux570dux9010ux689dux7c21ux6790}}

我們來看一下,第幾條是跟綜合所得稅有關的稅捐債務法的規定。

所得稅法2到17條是稅捐債務法的規定,可是這裡面2到17條,並不是這裡面每一個條文都跟綜所稅有關。他在概念上面來講,他可能是別的稅目,或者是分離課稅的稅捐客體。因此,真正屬於綜所稅的部分,第2條的規定是,課稅基礎構成要件跟法律效果的規定,但他沒有很完整。接下來的3-2條,一直到3-4條,都不是綜所稅這個課程要介紹的範圍。第4條是免除他稅捐負擔或者是非所得的規定,這個我們也要個別去做討論。那麼接下來的第4-1、第4-2條一樣,他是一個免稅規定,所以我們會併到第4條裡面去做說明。個人買賣證券跟個人買賣期貨免稅,這個是一個典型的稅捐優惠規定,我們會在免稅的稅捐優惠那個地方一併去做說明。

第4-3又是信託的規定,只是涉及到公益信託而已,那我們會留待到所得稅法三再來去做遺產贈與信託,包括私益跟公益信託,一併做統一的說明。
第4-4,就是105年我們增訂的房地合一稅。他的規範結構,沒有獨立稅目,但實際上他也並不併計在綜合所得稅的稅基裡面一併去做計算。所以4-4的規定,他是另外一個稅目,就是我們剛剛講的房地合一稅,但法條的外在體系看不出來是一個稅目的規定。

4-5,一樣,因為他是4-4,課稅規範的,給予例外免稅。他的規範結構就跟我們立法者向來第2條做課稅規定,然後接下來到第4條的時候就做免稅規定,也就是我們立法者的立法技術。他通常就是說,我要課什麼,哦,他就會在前面做一個課稅的規定,然後之後隨即就哪些東西他不課的,就做一個免稅規定。那我們的立法者也是一樣,第4-4條是房地合一稅課稅對象的規定,4-5條是房地合一稅的例外,給予稅捐優惠的規定。

第5條規定呢? 第5條規定,原則上在第5條的這個第一項是關於,後面第17條免稅額跟扣除額的相關規定的調價規定,也就是說,按照物價指數去做一些調整的規定。

第5條第一項,本來這個他的意義就是,隨著物價的調整,維繫生存需求的額度會隨著物價調整而調整。這個調整規定,本來他也是一個很正確反應量能課稅原則的規定,只是他的立法體例其實是放到17條那個地方去做一併規定就好,課予立法者一個所謂的作為的義務,就是與時俱進,反映當代維繫生存需求,調整維繫生存需求的額度。這個地方是課予立法者,與法律適用機關,他必須要去做一個按時與時俱進調整的規定。

但這個第5條的第一項的規定,在我們納保法的對人民的基本生活費的不課稅原則裡面,很大程度上被取代掉。這個條文規定,由於有一個體制上問題,就是我們免稅額當初規定的稅基太低。所以他每一年這樣調,從6萬開始調整調到最後面,基本上也只能調到8萬多塊。

我們後來在納保法裡面制定了,基本生活費不課稅原則,所以現在目前在實務上幾乎,基本生活費不課稅的納保法第4條規定,很大程度上取代了所得稅裡面的免稅額跟標準扣除額的規範適用、根據第5條第一項去做調整的規定。

這個部分比較稍微稍微有點小複雜,但老師其實在那個稅法總論裡面會去談這個問題,因為維繫生存需求,我們本來應該是在所得稅法裡面去反應做規定,可是我們現在,又以納稅者保護為名,另外訂了一部納稅者保護法,然後在那個地方用,最低基本生活費的這個概念去替代掉了實務上面,真正在這個地方,本來應該反映最低生活費的免稅額跟扣除額的規定。

第5條第二項的規定,就是綜所稅稅率的規定,我們稱之為叫累進稅率。法條規範結構,反映出立法者重視這個構成要件要素的地位,所以我們這一套法律基本上是比較偏財稅領域的,因為他們正是最重視稅率。他們在規範上面,他們會注意到課稅的對象,主體客體,然後哪一些是免稅的對象\ldots\ldots 在他的規範結構上,課稅對象通常是第1條,然後免稅對象是第2條,接下來他就會規定稅率,把稅基規定都是放到後面去。大多數財稅領域的學者,大部分都是這種思考方式,他認為稅率最重要。

稅率不是最重要? 稅基比稅率更重要。因為如果特別是在累進稅率以下,稅基不對,會扭曲那個負擔分配的效果。而且累計稅率會加速擴大,扭曲所得或遺產的稅負分配,如果沒有做正確計算的話。所以臺灣的稅改,一天到晚都在講稅率,各位,你可以去注意,臺灣在實務上在談稅改這個字眼的,通常財政學者他們都在談稅率。這個領域從來不問法律老師的見解。

我們希望未來能夠改變,真正的稅改是稅基改革。稅基要算得正確,稅率加上去才不會扭。沒有把握,你就回到比例課稅,你今天要做累計課稅的話,稅基算不對,會讓本來不應該繳那麼多稅的人,他會因為累計稅率,那個稅負會放大。

我們之後會談到,我們稅負不當被放大,因為他有一些錯誤的觀念,比如說你只是適用累進稅率20\%,這個其實在我們臺灣人一般情況來看啊,大概都中產階級而已。但他的幼兒扣除額,不能扣除,他的長照特別扣除額,也不能扣。在累進稅率的適用以下,他的稅負會被扭曲、被放大。

臺灣,稅改的話,財稅學者比較偏重在稅率改革。但法律應該強調,在稅基的改革。只有正確的稅率才有正確的稅額。

雖然這兩個乘起來確實是稅額的前提構成要件,但正確的稅基才會有正確的稅額,才會有正確的稅負分配的效果。不是稅率。我只是跟各位強調這一件事情,因為這個是我們法律學者跟財稅學者最大的不同,因為他們比較忽視稅基的計算的

我們再來看,第5條第二項的規定,是綜所稅的稅率的規定。接下來第三項第四項,這一些都是一個對主管機關,應依職務做調整的規定。這個其實都不是所得稅法要規定的,像這種東西就作業規則去規範就好了,就授權到那個施行細則去規定。真正在稅法本身要出現的是立法者要自己決定的稅捐構成要件這些要素。

來看第5條的第五項,這個又是營所稅的稅率。所以第5條基本上是稅率的規定,第5條的第二項綜所第5條第五項是營所。

然後5-1條,一樣,我個人認為都是可以放在施行細則去規定的。從5-1條以下到第6-2條的這些規定,都是屬於比較技術性、細節性、非重要基礎的構成要件要素,這一些原則上都在施行細則規定就好。

我們接下來的第7條的規定,就是最典型的稅捐主體的規定,納稅義務人的規定。第7條第一項的納稅義務人的規定是,我們下個禮拜上課的重點。我們講的境內居住者跟非境內居住者。

第7條的第五項,扣繳義務人的規定。這個是稽徵程序裡面的行為義務人。在整部綜所稅的稅捐稽徵程序裡面,我們分成兩個程序。一個是,結算申報,71條規定;一個是就源扣繳,88+89條的規定。我們整個綜所稅的稅捐稽徵程序的規定,就是以71條跟88+89條所建構起來的兩道稅捐稽徵程序。

在就源扣繳程序裡面出現了為別人去完成公法上的義務的第三者,那個人叫扣繳義務人,在第7條的第五項規定做了一個定義型的規定。

第8條是關於客體的規定,客體的所得實現地點的規定。這個條款規定非常重要,特別在實務上面。這個也是綜所營所混在一起做規定的。各位可以看其中第九款的規定。在中華民國境內經營工商、農林、漁牧、礦冶等業之盈餘。第8條的第九款,原則上是適用在營利事業,因為他有經營的字眼,原則上是指在中華民國境內經營事業,經營行為,因此所獲得的盈餘。他的意思是這樣,但由於我們這個條文,綜所營所混在一起,也因此營利事業所獲得的經營的內容,其實也可能去適用,譬如說第三款的規定,中華民國境內提供勞務之報酬。

如果你經營的營利事業涉及到勞務的提供的話,那麼他也可能會同時去適用第8條第三款的規定。那當然第8條的第九款也有可能是在個人經營事業,也就是我們常講的獨資商號或合夥事業。從而也有可能在第8條第九款裡面,同時間有個人跟營利事業適用之可能。

倒過頭來本來應該第8條第三款,本來是個人適用的,個人在中華民國境內提供勞務,那麼由於我們現行實務上的看法,他也可能適用在營利事業。因此每一個條文,要個別去判斷他要適用的主體對象。

第9條是一個法律條文定義的規定,也就是對所得稅裡面的一個類型。

所得稅法第14條,總共有十類的類型。第9條是對於第七類裡面的房屋財產交易所得,所做的定義規定。
所以第9條的法條定義規定,其實也可以放在第14條的第七類,去做定義上的說明。

第10條的規定又回到了營利事業所得稅。這不是我們綜所稅的範圍。

第11條,這裡面交雜著一些是屬於綜所稅也,也包括了營利事業所得稅可能適用的規定。11條第一項,顯然是在講綜所稅裡面的執行業務者,因此11條第一項是我們這個學期要講的綜所的內容。第11條的第二項,本法所稱之營利事業,看起來是只是用在下個學期,所得稅法二裡面的營利事業主體的定義規定。但由於個人獨資商號,我們也是適用個人綜合所得稅的規定,所以11條的第二項也會是在第14條第一項,我們所講的營利事業裡面,營利所得,還是有適用在個人綜合所得稅的可能。

第11條的第三項、第四項、第五項,原則上他都是在營利事業所得稅裡面才有適用餘地的相關條文規定。

第13條到第17條之間,原則上都是綜所稅的規定,因此也會是我們這個學期要上課講解的範圍。第14條,特別是涉及到各類型所得的條文規定。

第14-1條的規定。個人持有公債、公司債及金融債券之利息所得,應依第八十八條規定扣繳稅款,不併計綜合所得總額。
這個地方適用的主體對象,當然也是個人,只是把他從14條的綜合所得稅的稅基以裡面去做分離課稅的規定。

也就是對於14-1條的利息所得,我們把他,從綜合所得稅的稅基切割開來,做了分離課稅的法條規定,一個立法上的價值。從而14-1是把他從所得稅法第14條裡面的第四類利息所得裡面切出來,個別去做就源扣繳,然後比例計算稅負的一個條文規定。

14-2被刪除掉。

14-3。看一下他的法條規範,包括個人跟營利事業。個人跟營利事業都有適用的可能性。所以14-3條是一個最典型的紊亂法條規範體系結構。因為在這裡,從頭到尾,理論上來講,這一個區塊都是綜所稅的稅捐債務法的規定,但突然夾雜了一個14-3,又適用個人也適用營利事業所得,這樣一個稅捐條文的規定。

14-4到14-8,是承繼前面的4-4跟4-5的,房地合一稅的規定。簡單來講,就是我們的立法者採用一個前所未見的立法技術,就是沒有名目,沒有稱呼。他本來應該要獨立出來,有一個稅目名稱,叫房地合一稅。課稅的客體放在4-4,免稅的客體放在4-5,稅基計算放在14-4到14-8。這個是房地合一稅的稅基計算規定,我們放到所得稅法四再來做介紹,這學期我們不談房地合一稅的客體、稅基、跟稅捐稽徵程序的規定。

15條就又回來綜所稅的規定,合併申報的規定。

16條也是一個合併申報、合併計算盈餘虧損的規定,盈餘虧損的跨主體合併計算規定。

第17條是最重要的稅基規定。在17條的規定裡面,反映出納稅義務人,維持本人、配偶及受扶養親屬的維繫生存最低需求。17條的規定,配合前面的第5條第一項的規定,本來立法者課予行政機關要與時俱進檢討,對於各項維繫生存所需要的費用,是否應該要與時俱進地調整,但由於稅基的規定太低,也就是說我們這個法條的規定,免稅額是從6萬塊錢開始起算,過低的免稅額規定,導致他每年調整,其實都不足以反映出維繫生存的最低需求。

後來透過了納保法的第4條的所得基本生活費用的不課稅原則,才大致上將我們目前維繫生存最低需求以適當的方式去做扣除,否則在纍進稅率底下,我們的綜合所得稅的稅捐負擔,是被扭曲負擔分配的。也就是維繫生存的費用,沒有被適足、適當反應。

17-1條的規定,關於死亡或離境時的數額的計算,比例計算的規定。

17-2則是出售自用住宅,也就是在前面第14條裡面的,關於財產交易所得裡面的自住房屋,稅捐優惠的規定。要放稅捐優惠,我個人是比較反對,但法律體系上你也應該跟第4條的規定放在一起。
4條,4-1,4-2,是我們前面立法者體系上面他放免稅或稅捐優惠規定的地方。

在這個地方你就可以看得出我們的立法者是多麽恣意,不尊重自己所決定出來的立法上的體系。

17-3條的規定,關於投資儲蓄特別扣除,是要去配合第17條裡面的特別扣除額的儲蓄投資特別扣除額的項目。

第17-4,關於非現金財產捐贈,一樣,這個條款規定,是在配合第17條裡面的列舉扣除額,捐贈列舉扣除額。

每一個條文的體系上的位置,本來應該對應的位置,是直接對應到以各類所得裡面,去做相對應的規定。
這個是我們現在目前的規範,實然跟應然狀態之間的差異。

接下來的第三章我們都跳過去,因為第三章營利事業所得稅的部分,我們這個學期不講。

接下來我們到第四章的稽徵程序。其中第一節,暫繳程序,在個人裡面沒有。個人原則上不暫繳,因為他只是用在營利事業。

因此,我們是從第71條第二節的結算申報制度,開始進入稅捐稽徵法的規定。

71條、71-1、72、73條,非境內居住之個人,都有適用的可能性。

73-1條則只適用在營利事業所得稅的國際金融業務分行。
74條一樣只是用在營利事業所得稅。
75條,這一些都是營利事業所得稅適用的規定。
76條,結算申報時所應具備的文件,這個放到施行細則規定就好。
77條,申報書的規定,也是一樣,放到施行細則去規定就好。

78條的催報,這一個,甚至在所得稅法裡面不規定也沒有太大的差別,這個都是屬於稅捐稽徵機關在職務上應作為的部分的規定。這個作為的部分的規定,不是在對納稅人的權利義務形成上面,具有關鍵重要的要素的事項。

第79條的規定,納稅人未依限申報所做的處理,也就是對於納稅義務人沒有做帳簿憑證,或者是做協力義務的,推計課稅的規定啊。

第80條,調查核定的規定。以下相關的,81條,跟83條83-1條的規定,通常應該是在稅捐稽徵法裡面做規定,但他適用的對象主要是營利事業。因為在營利事業有記帳的需要,跟記帳的法律上的要求。個人的部分,原則上不記帳,除非你是一個個人執行業務者,才會有所謂的執行業務者所需要的記帳的協力義務的規範。

因此在整個第3節調查的規定裡面,要不是適用在營利事業所得稅裡面,不然就是應該在稅捐稽徵法裡面去做統一的規定即可。

接下來到第4節扣繳程序。

88跟89條,被稱作為叫就源扣繳程序。這個是在我們所得稅,特別是在綜所稅裡面,除了第71條的結算申報程序以外的第二個最重要的一個稅捐稽徵程序。而這個稅捐稽徵程序將原來在結算申報程序裡面的,納稅義務人的結算申報行為義務,轉嫁給第三人負擔由第三人來完成這個公法上的任務,也就是稅捐稽徵程序的任務。
我們在這個學期,會針對就源扣繳程序,跟各位做的討論跟說明。儘管這個部分其實在稅法總論的稅捐稽徵程序裡面,我們也會去討論到。

接下來是到89-1是扣繳稅款的規定。這個是跟扣繳有關聯性的規定,所以當然就會連接在88到89條。

那么接下來91條又是營利事業的記帳規定跟報告稽查配合調查的規定,其實就放在前面的調查那個地方。因為調查的規定,理論上來講,都是直接在稅捐稽徵法做相關規定即可。

92條是扣繳憑單的規定,從而扣繳憑單的規定聯繫著89-1條的規定,關於扣繳稅款的程序的規定,在施行細則規定即可。

92-1條是信託,所以我們這個學期不會去討論這個部分。

第93條的規定是關於扣繳的審核跟配合調查的義務,跟主管機關得派員調查的義務,其實這一些也都是屬於職權調查的範圍,並沒有特別要做另外單獨規定的必要性。當然這個地方是因為是扣繳義務人,是第三人的關係。

94條的規定,扣繳義務的法律上的效果,這個是跟89條裡面,關於誰來負擔扣繳行為義務有密切關聯性,所以我們也一樣會在綜所稅的申報程序裡面跟扣繳程序裡面一併去做說明。

94-1條,免填發扣繳憑單。我們到時候會談到扣繳程序,除了扣取、申報、繳納義務以外,我們還要再填發一個扣繳憑單,這也是我們在稅捐實務上面一個非常擾人的,非常重複而沒有必要的稅捐的稽徵程序。

那么第95條跟97條的規定,扣繳稅款的準備用規定,我們在扣繳程序裡面再一併去做說明。

第五節的自繳規定,其實是結合在結算申報程序裡面,一般來講就是申報的結算申報制度,我們是先繳納,然後完成稅款申報程序。也就是先繳後完成申報,我們是報繳合一制度。所以,第五節的自繳規定,其實是放在申報程序裡面,因為我們是報繳合一。

因為我們有一些稅目,是報繳分立,也就是先報不用繳,等核定再繳納。像土地增值稅稅,遺產贈與稅,我們都是先報後繳,不是報繳合一制度。

我們在綜所稅跟營所稅,都是報繳合一制度,換言之,申報跟繳納在程序上面,是連接在一起。我們的制度上面,其實這個都應該放到稅捐稽徵法裡面去做統一性規定。

報繳要分立或報繳要合一,這個是立法上的政策抉擇,沒有必然絕對好壞的問題,但你總是要做一個統一性的規定。

第六節,盈餘申報,很明顯,並不是綜所稅適用的範圍,因為盈餘申報這個部分理論上來講是營利事業對營業盈餘的申報的相關程序的規定。從而這個部分就留待到營利事業所得稅。這個地方,未分配盈餘也是由營利事業來進行稅捐的申報,因此如果沒有做這個正確的未分配盈餘申報的話,營利事業也會受到事後的制裁的規定。

從第五章103條之後,就是所謂的稅捐制裁法的規定,也就是行為人違反協力義務。行為人包括了納稅義務人跟扣繳義務人,都有被制裁的可能。

因此我們請各位去注意到,納稅義務人違反義務的制裁規定,是在第110條的規定,110條的第一項跟第二項。

扣繳義務人的違反行為義務的這個處罰規定,主要是在第114條的規定。協力義務的違反分成了納稅人的協力,違反跟扣繳義務人的協力義務違反。

我們相對應的法條規定,大致上就是110條跟114條的規定,當然這裡面夾雜著非常多的行為罰的規範。有一些是針對稅捐稽徵機關公務員的職務的規範,這個條文的規範的內容,再請各位回去,去做一個整體的鳥瞰。

簡單來講,第2條到第17條,這個是所謂的稅捐債務法規定。理論上,他應該就構成要件,算出稅額前提的各項要素,要依序去做規定。接下來在稅捐稽徵法裡面就是結算申報跟扣繳程序的規定。違反申報跟扣繳義務的規定,則是第110條跟的114條。
因此,我們這個學期要跟各位上課的內容,就是這個部分。2到17,71,88+89,110到114條的規定。其他當然會夾雜著一些有的沒有的一些條文的規定,但理論上來講,這些,比較是屬於在所得稅法學習裡面,各位一併去參照就可以。

這個是我們這個學期要跟各位談的範圍。法條規範的結構,其實應該要,依序就構成要件去做展開。

\hypertarget{ux91cfux80fdux8ab2ux7a05ux539fux5247ux7684ux5404ux9805ux5b50ux539fux5247}{%
\section{【量能課稅原則的各項子原則】}\label{ux91cfux80fdux8ab2ux7a05ux539fux5247ux7684ux5404ux9805ux5b50ux539fux5247}}

我們這個學期的第二個小時的課程,本來就要跟各位去談,從量能課稅原則衍生出來的各項子原則。我在這個地方做一個條列的排列的說明。包括了,普遍原則、全部、實現、實價、客觀净值、主觀净值、個人、終生、綜合、比例。總共10個從量能課稅原則,所進一步具體化下來的子原則。

我們在稅法總則課堂上面,跟各位做了一些介紹。如果沒有,也請各位再去參考老師寫的稅捐法秩序的文章,因為那裡面我把他寫得(我個人認為)非常地仔細,也完整。

我是看到陳敏老師對量能課稅原則的質疑,我看到這一段的時候,我才能夠明白,為什麼量能課稅,除了釋字745號,曾經短暫曇花一現地,說了量能課稅原則的客觀净值原則以外,之後之前都沒有提到量能課稅原則作為稅法上的原則。

陳敏老師認為,量能很抽象。量能不具體,這個抽象而不具體的內容,無法當作稅法上的基本原則。看到這一段,我才特別要寫清楚,不是這樣。

這個量能再怎麼抽象,我們的工作任務就是把他具體化。不是倒過頭來說,量能不是稅法基本原則。如果是這樣,那稅法除了依法課稅,就沒有任何倫理價值的判斷標準,來判斷何謂等者等之何謂不等者不等。依法課稅沒辦法給你這個等者相等,不等者不等的這個倫理價值的標準,量能可以。

量能課稅原則的各項子原則:

\begin{enumerate}
\def\labelenumi{\arabic{enumi}.}
\tightlist
\item
  第一個原則普遍原則。
\end{enumerate}

什麼叫普遍原則? 中華民國人民,你只要賺到錢,有所得都要課稅,這個就叫普遍原則。他是哪一個稅捐構成要件要素的原則?稅捐主體啊。

這些量能課稅原則的子原則,都是為了把這些課稅構成要件的各項要素,把他進一步具體化。

所謂的普遍原則就是你只要賺到錢,不管你的身份階級地位都要繳稅,這個會有困難嗎? 民國100年以前,我們真的有困難。因爲民國100年以前,我們有軍教人員免稅。沒有透過這個,你不能知道立法者多麼樣的恣意而為。他説軍教人員對國家有特殊貢獻,我承認啦,我也算教育人員。但我沒有被優惠過,這是我很不平的地方。義務教育的人可以受到優惠,但為什麼我高等教育不能呢?為什麼不能?你告訴我。這是立法者的立法形成。所以我從來沒有受過優惠,我現在講起這個話就會比較有點立場。

那你知道為什麼從來軍教人員這些免稅都沒有被挑戰?很簡單啊。你自己被優惠,你當然沒有課稅處分啦。我沒有被優惠的高校教師,難道我能夠去做稅捐抵抗,請國家對他們也課稅,不然我拒絕繳納自己的稅捐。你可以主張這樣一個方式,去要求國家必須先對那個基礎義務教育的教育人員課稅,否則我拒絕繳納自己的稅捐,拒絕我作為高教教師的稅捐負擔的義務嗎?可以嗎?你有這樣一個主張平等對待的平等權利嗎?你可不可以主張,我們還需要好好的研究平等權的意義和内涵。

不會有司法訴訟。

類似的這種案例裡面,我們在稅法總則裡面還提到另外一個例子,就是從事性工作的不用繳稅。在軍教人員的免稅被廢止之後,我們目前為止還沒有法律明文規定,但確實在實務上是不課稅。

我們也許還有一個問題,請問非法活動所獲所得要不要課稅?答案是,當然要課。但我們現在有一個活動,基本上他從我們的實務上面來講,他是不課稅,就是從事於性工作服務的所得,這個是不用課。沒有法律明文規定為基礎,純粹透過解釋函令。你可以看到我們在實務上面,一些基本原理原則,不管是透過立法或者是行政,這種被踐踏的狀態。如果不從量能課稅原則的子原則,請問你如何去檢視立法者自己做的決定,他怎麼從立法課稅原則裡面說自己不對呢?

\begin{enumerate}
\def\labelenumi{\arabic{enumi}.}
\setcounter{enumi}{1}
\tightlist
\item
  全部原則
\end{enumerate}

什麼叫全部(原則)?境內境外的來源所得,錢就是錢,1塊錢就是一個稅捐負擔能力,境內境外都是錢。所以。只要你是境內居住者,你不管是賺到中華民國來源所得,跟非中華民國來源所得,錢就是錢,稅捐負擔能力是一樣的。這個就叫全部所得課稅。

所以剛剛我念了第2條第一項。
境內居住者只就中華民國來源所得課稅。請問這個條文抵觸了那一個原則?全部(原則)啊。你沒有透過這些應然的基本原則去解釋,你怎麼知道立法者可以這麼恣意?

臺灣現況是這樣,你從美加賺到的,都不課所得稅所得,他現在是改課所得基本稅額條例的所得,免稅額度670萬。你如果是賺到中華民國來源所得,你看看你賺670萬,你要繳多少稅? 送你40\%,200多萬。立法者可以這麼恣意,怎麼可以,到目前為止都沒有人去挑戰他說,這樣不可以。你都不會覺得臺灣這樣的法律這樣規定,你不覺得荒謬嗎?對,所以今天老師來不及當立法委。你們有一天當立法委員,今天就公開在財政委員會裡面去質詢財政部長,部長,為什麼臺灣賺670萬要繳40\%的稅,為什麼我今天在海外賺了,在美國賺670萬,我1毛錢都不用繳,所的基本稅額條例的稅。告訴我,這公平正義在哪?

學稅法讓各位在立法院很有利,真的會讓各位很清楚地明白,稅捐是負擔分配的一環。稅捐負擔分配不義,我們的分配正義早就不義了,不用講那麼多關於住房分配的事情。因為你就輸在一開始的分配上面。

你今天在臺灣中華民國境內努力工作,你能賺670萬,你知道你是什麼階級才可以賺到670萬?結果你被課稅是40\%最0高邊際稅率。但如果你今天有本事在海外賺670萬誒,好,免稅。

我再跟各位講一下我們多荒謬的情況,因為如果不到這種程度,你不能知道是這個的地方這麽荒謬。

炒股票的要不要繳稅?剛剛我就念了4-1條。你努力讀書考上律師執業,你賺了670萬,在臺灣賺了670萬,你要繳到40\%。可是如果你有本事,你在美加賺到,同樣那麽多,或者是你今天是去股票市場裡面去炒股票。

你如果真的從K線裡面,你可以去看得出來,你,你也很厲害。我們稅裡面的最基本原則就是這樣,你有賺錢?請你繳稅。就這樣而已,這個很難嗎? 很素樸又簡單的正義啊。你有賺錢嗎?要繳稅。

你不能誆我啊。你真正賺多少錢?那就這樣而已啊。臺灣之所以分配不義,就是土地從來沒有實質破課稅。證券交易沒有做實價課稅。這很難嗎?所以他告訴你一件事情,有本事大家去炒地皮,不要考律師。考律師、當律師一點用都沒有。
你有本事就是從銀行挖錢出來去炒地皮。臺灣民國60、70年代的時候一堆農會漁會的總幹事,從農會漁會裡面搬錢出來,幹嘛?都市重劃的時候都是這樣,因為他知道都是怎麼個重劃,知道這一條路要蓋\ldots\ldots 重劃的時候,剛好都是政經弱勢的被劃到公共設施吧?政經比較強勢的,都是剛好你們家是建地,一劃過去,公共設施保留地,立刻價格跌落成1/10,如果你改成建地,立刻價值,就漲10倍,一來一回之間差100倍。

炒地皮的錢不用我自己的錢,一堆人去擔任農會漁會的總幹事,從金融機構搬錢出來炒地皮,炒輸了,就倒給你,就跑路啊。6、70年代,因為那是我當時學生的年代,我印象好深刻。這個情況到現在沒有太大的改變,因此到我們105年才終於真正地叫做房地合一課稅。但我們有日出條款。

所以各位,你關心分配正義這件事的話,我要很明白又誠實地講,不太容易。但因為稅制負擔分配這些事情,如果我們持續地面對這種情況,沒有做任何改變\ldots\ldots 我還是要提醒各位,因為任何事情都是在立法院裡面去做。立法院怎麼做出來的決定?我們在民國102年到104年,號稱要課證所稅,結果就部長就下臺。這是我們現在目前的現況。

各位把每一個量能課稅的原則,都把他好好弄清楚就可以知道,我們的應然跟實然差多大。

我再重複一遍,所有人賺到錢,應該要課徵所得稅,這個叫普遍。

境內境外的錢,都是錢。這個叫全部。

\begin{enumerate}
\def\labelenumi{\arabic{enumi}.}
\setcounter{enumi}{2}
\tightlist
\item
  實現原則
\end{enumerate}

實現,就是你賺到,實現的時候,實際取得的時候才有稅捐負擔能力。

\begin{enumerate}
\def\labelenumi{\arabic{enumi}.}
\setcounter{enumi}{3}
\tightlist
\item
  實價(實值)原則
\end{enumerate}

實價,應該依照真正的市場上的價格,售價,來做實價課稅原則。

\begin{enumerate}
\def\labelenumi{\arabic{enumi}.}
\setcounter{enumi}{4}
\tightlist
\item
  客觀净值原則
\end{enumerate}

客觀净值,你的任何活動都會有收入以及成本費用的支出。客觀净值,因此包括準許減除成本費用,還有盈虧互抵。

\begin{enumerate}
\def\labelenumi{\arabic{enumi}.}
\setcounter{enumi}{5}
\tightlist
\item
  主觀淨值原則
\end{enumerate}

主觀淨值,這個是我們法律學跟經濟學最大的區別。經濟學者一般認為,只有客觀净值沒有主觀净值。法律學者,一般認為,我們在計算這些負擔能力的時候,除了扣掉成本費用,除了扣掉以虧損以外,我賺錢不是為了國家,我賺錢是要先養活自己。從而要扣掉維繫個人生存需求。還有,我還有家庭,我還有婚姻。所以我的配偶跟受扶養親屬的維繫生存需求,這個叫做主觀净值,這個叫生存權保障。

經濟學者一般不認為所得稅包含主觀净值的計算,法律學者反而認為這個比客觀净值更重要。

\begin{enumerate}
\def\labelenumi{\arabic{enumi}.}
\setcounter{enumi}{6}
\tightlist
\item
  個別主體課稅原則
\end{enumerate}

每一個人就是一個獨立的主體,每個人是按照個別各自的稅捐負擔能力計算,這個叫個別主體課稅。

\begin{enumerate}
\def\labelenumi{\arabic{enumi}.}
\setcounter{enumi}{7}
\tightlist
\item
  綜合所得課稅原則
\end{enumerate}

綜合所得,意思是說不分類,綜合在同一個稅基底下,利息賺來的錢跟薪資賺來的錢,或者是炒地皮來的錢,錢就是錢,不管你多辛苦。錢就是錢,跟你多辛苦,跟什麼類型所得是沒有關聯性,這個就叫綜合所得。相反地就叫分類所得。

\begin{enumerate}
\def\labelenumi{\arabic{enumi}.}
\setcounter{enumi}{8}
\tightlist
\item
  終生所得課稅原則
\end{enumerate}

終生所得,原則上所得可能會有跨年度的問題,因此我們,以終生作為一個理想,但現實上我們做不到,所以我們改用期間課稅。所得稅是一個年度稅,是一段期間課稅。

\begin{enumerate}
\def\labelenumi{\arabic{enumi}.}
\setcounter{enumi}{9}
\tightlist
\item
  比例稅率原則
\end{enumerate}

比例稅率,原則上你的稅基適用同一組稅率,累進稅率,確實是現在目前我們很多直接稅裡面會採用的,但這個是摻進來社會國原則的思想。

透過這一堂課,有一點快地介紹,跟各位去說明我們的稅捐外在體系,我們稅法所得稅法外在體,某種程度上是比較凌亂。因為稅目還有構成要件的這個部分,並沒有按照秩序去排。那我也請各位回去,務必做這個工作大致上就是2到17,71,88,89,跟110,114,把他按照歸類,自己去排一個表出來,因為我在考試的時候,應該會問到這個問題。各位只是先做一下準備而已。你把這個東西好好地排一下,其實你就把所得稅把他帶在身上,帶在你的腦袋裡面。我的用意是這樣。其實我們接下來上課就是每一個條文去解析他的構成要件,然後去看客體、所得實現的地點,或者是稅基稅率。這一套方法就是典型的法學方法。學法律是方法論,不是在背誦,根本不是在法條之間去做拼湊而已。

那我們今天先到這裡。

\hypertarget{section-5}{%
\chapter{20230918\_01}\label{section-5}}

\begin{longtable}[]{@{}l@{}}
\toprule()
\endhead
課程:1121所得稅法一 \\
日期:2023/09/18 \\
周次:3 \\
節次:1 \\
\bottomrule()
\end{longtable}

\hypertarget{ux91cfux80fdux8ab2ux7a05ux539fux5247ux7684ux61b2ux6cd5ux4e0aux610fux7fa9}{%
\section{【量能課稅原則的憲法上意義】}\label{ux91cfux80fdux8ab2ux7a05ux539fux5247ux7684ux61b2ux6cd5ux4e0aux610fux7fa9}}

綜所稅,適用各項從憲法平等原則具體化的量能課稅原則,然後再進一步具體化的各項子原則。綜所稅是體現所有量能課稅之原則最完整的法律的領域。
其他的包括營所,營業稅都是部分反應而已。量能課稅原則會在綜所稅裡面做最完整的呈現。

我們課程沒有要求,各位非得要上過稅法總論,但對於量能課稅原則的各項子原則,還是希望各位無論如何去把他補起來。
因為你只有在這些子原則的指導底下,你才能知道我們現行的立法跟法律解釋方向。
我再一次強調,作為憲法位階具體化的原則,他是具有行政立法司法的上位的效力。簡單來講,就是立法者應該往那個方向去立法;法律,在沒有變更之前,不管是行政或司法,在可能的文字範圍內,他要盡可能往那個方向去解釋,這是作為一個法律為憲法位階的基本原則的功能跟作用。如果這個原則不能夠起到對立法指引跟法律解釋適用的指引的效力,那我們講這些東西是沒有太大意義的。

當然個別原則他會有一些取捨,我們會在接下來的課程裡面去看到各種不同原則,會有做不同程度的取捨。所謂的取捨的意思,就是他有一些這些子原則在綜所最裡面做了一部分的退步退讓,但他不是全然放棄。打個比方,各位一定學過的私法自治,在民法裡面擁有無可置疑的最重要的主要原則的適用地位,可是私法自治仍然在,比如說定型化契約條款,比如說對未成年人的保護,裡面他要退讓一步。不是因此他就放棄了作為私法自治作為私法裡面最高指導原則的地位,而是在那樣一個,如果透過定型化契約條款,透過經濟上的強弱勢,將以契約自由之名行剝奪自由之實的這一種的契約條款的話,我們仍然可能會透過立法乃至於法律的解釋適用,去做部分的調整退讓。如同這樣一個私法自治,跟弱勢消費者的保護,或是對未成年人保護,是同樣的概念跟原理。也就是說,在那個地方,私法自治作為私法最重要的基本原則,他並沒有因此折損他的地位,只是在這樣的一個對消費者的保護,對未成年人的保護這樣一個前提底下,他能夠更完整的展現。其實私法自治,就是要展現一個人作為一個人的,可以就自己的事情做充分自主完全決定的這樣一個概念。未成年人,因為他可能在他的意思表示形成過程當中,他還沒有辦法,完全會自己做最佳最正確的考慮,因此才會有一個代理人,法定代理人來協助他為了未成年人的利益而來做最有利,保護他利益的一個解釋。

從而他其實回到我們憲法裡面。我們憲法雖然沒有寫得這麼清楚,但各位一定都會聽過人性尊嚴這個概念。因為我們從比較法來看,比如說德國基本法第一條,人性尊嚴是不可以與貶損的這樣子的一個基本價值跟原則,所以從而私法自治也同樣在體現人作為人,對於自己本身所有的事物,享有自己自我做決定的最終的權限。在這樣一個前提底下,如果透過契約上的強弱勢的經濟地位的安排,要求相對人去做退讓,這個毋寧才是一個假自由之名,而去剝奪對方的作為一個人的人性尊嚴的一種做法。這也是我們在私法裡面各位都學到的基本的概念。

來到了稅法亦如是,量能是在稅捐法律體系裡面作為,我們可以講,他跟私法自治在民法一樣的最高位階的地位,簡單來講就是說,雖然量能本身並沒有直接被稱之為叫做等同於所謂的平等原則,可是正因為平等是一個蠻抽象,某種程度也是空洞化的一個原理原則,因為他必須要去比較兩個群體之間的所謂等跟不等。基於這樣子的一個概念,所以他透過一個所謂的分類標準去確定A跟BA群體跟B群體之間為什麼他是不等,為什麼他們是相等? 那稅法的話,就是去憑這個所謂的分類標準,就是透過稅法裡面的量能原則來去鑑別,為什麼兩個人在這個地方做不同的稅捐負擔的不同的待遇,跟為什麼做相同的稅捐負擔的待遇。這也是我們在稅法總則的課程裡面,我們會特別去交代,為什麼可以從憲法平等原則到稅法的量能課稅原則。
如果沒有這樣一個分類標準的話,平等原則,是空洞的,因為他講不出更多的內涵,什麼叫等者等之,什麼叫不等不等之,又究竟又要多麼大的不等,才叫做不等而違反平等原則。
以打分數來講的話,就是一個很典型的例子,我不可以說,坐在左邊的一律及格,坐在右邊的一律不及格來作爲一個分類標準,因為這明顯的是老師的恣意。我不可以有戴眼鏡的沒戴眼鏡的來作為分類標準,這個也很明顯的是一個恣意。那我究竟要用什麼標準來評價各位的考試分數?很簡單,你來考試。符合於事務之本質的分類標準,就是一個最為適當的分類標準,因此等者等之這一件概念,在稅法裡面就是用你有你能你可以。你能,你可以你賺到錢,你有錢,你有辦法消費。就這樣一個基本的觀念,去建構起我們整個稅法裡面所謂的憲法平等原則。

因此量能跟平等是結合在一起,除卻量能的平等是空洞化的原則,沒有辦法去做具體的操作。這也是我強烈的要去說,我們國內如果還不能夠一直學憲法的學者一直在講平等原則,那量能就是讓他落地化。沒有這樣一個原則,你根本就無法去評價,為什麼,賺同樣100萬的人可以有不同的稅捐負擔,或者是為什麼賺100萬人跟賺1000萬人,為什麼稅捐負擔差異可以這麼大或差異這麼小?他的標準就建構在我們剛剛所講的平等原則裡面的具體化的量能。

體現量能最完整的各項子原則,就在我們今天跟各位講的綜所稅。上個禮拜,我很快地交代了十項子原則,其實各位如果有興趣,你當然可以再參考老師寫的,《稅捐法秩序》第541頁以下,全部都是以所得稅法為例子。總論跟各論,其實本來就相互結合,就像各位在學刑法總論跟刑分,各位在學民總、債總跟債各之間的關係。

我們教學上固然把他體系化切割成每一個細分的部分,去做體系化的教學,但請各位同學,務必時時回過頭來來想整體。整體跟個體之間兩者之間的梭巡來回,是各位對這個法領域裡面充分掌握適用的一個前提構成要件。我們不是在背法條,而是在總則分則、原理原則到個別具體法條之間的解釋適用,又回過頭來來去看他到底有沒有符合原理原則的,這一個過程的充分的一個學習。

各項子原則,包括了,普遍原則、全部、實現、實價、客觀净值、主觀净值、個人、終生、綜合、比例。還有另外一個,跟這幾個原則會混雜在一起去做討論,就是所謂的實值課稅原則,即實價課稅原則。實值,就是真正的價值的意思。我們後來還有一個叫實質課稅原則,這個地方也請各位稍微分辨一下。

因為,法律的精細是在使用標準正確的文字去描述你要描述的對象,如果這個詞雖然中文發音聽起來一模一樣,但如果在考試的時候,我看不到正確的字眼,我就會懷疑到底是我沒講清楚,還是你沒聽清楚。
如果我現在都已經特別講了的話,那我當然就會評價你可能沒有聽清楚,也沒有搞清楚,這就是我們在考試裡面所想要做的檢驗,希望各位透過考試,可以檢驗自己,這些基本原理原則,能不能很清楚而充分的純熟地去做運用。

\hypertarget{ux91cfux80fdux8ab2ux7a05ux539fux5247ux7684ux5404ux9805ux5b50ux539fux5247ux5728ux7d9cux6240ux7a05ux4e0aux7684ux9055ux53cd}{%
\section{【量能課稅原則的各項子原則在綜所稅上的違反】}\label{ux91cfux80fdux8ab2ux7a05ux539fux5247ux7684ux5404ux9805ux5b50ux539fux5247ux5728ux7d9cux6240ux7a05ux4e0aux7684ux9055ux53cd}}

接下來談量能課稅原則的各項子原則。

\hypertarget{ux7b2cux4e00ux9805ux666eux904dux539fux5247}{%
\subsection{【第一項普遍原則】}\label{ux7b2cux4e00ux9805ux666eux904dux539fux5247}}

第一個我們稱之為叫普遍原則,也就是中華民國人民基本上不分男女性別黨派居住區域、身份階級。就像我們的憲法第7條的規定一樣,一模一樣。
只要有所得,你賺到錢就要課到稅,這個就所謂的普遍原則。這個看起來是一個似乎普世皆然的一個原理原則,但其實包括臺灣本身在內,在民國100年以前,其實我們並沒有實踐。因為我們有一群在所得稅法第4條第一款跟第二款規定(現在已經被廢止,所以現在你看不到),我們在100年之前的軍教人員免稅,就是因為階級、職業別而產生的區別對待。待立法者在他的立法明文裡面去做了這樣一個區別對待。理由呢?因為認為軍教人員待遇低、辛苦,所以透過這種免稅的待遇來給他實質上的補貼。當時候的大多數的正當化,也都是建構在對國家有特殊貢獻,以這個為前提或理由來做的違反普遍原則的規範。

我要強調,違反平等原則,其實不是因為這樣違反,所以就被認為違憲,而是違反在這個地方立法者有沒有足以正當化這樣違反或這樣偏離的一個正當化理由。在這裡,平等原則跟比例原則的操作上面有一點最大的不同。比例原則,你也可以講就是過當就是一個違反比例的一個國家過度干預行為。平等原則的操作,是我們先辨識了這個地方,確實立法者做了區別對的,也就是說,我們先辨識了立法者在這個地方做法律上的不同區別對待。因為平等原則並沒有要改變事實上的不平等,比如說男女的性別不一樣,比如說年齡的不一樣,立法者再怎麼立法,也不可能改變這種客觀現實的不同。這個就是一個客觀上面、事實上的不平不是立法者要改變的任務,立法者也做不到這件事情。可是法律上的區別對待這一件事情,卻是我們在對國家的行為檢驗他是否合於平等原則的時候,第一步先辨識在這個地方,法律上做了一個不同的區別對待。

第二道步驟,就是在檢驗為什麼要做這樣一個不同的區別對待,也就是第二道,我們稱之為叫基於什麼樣的正當目的,而做不同的區別對待。所謂平等原則,我個人認為就是階段理論,就一個階段一個步驟。平等原則,不會因為他違反量能,就直接違反憲法平等原則,所以就宣告違憲。而是在一步一步的檢驗過程當中,你們我們去檢視立法者做這樣一個區別對待的正當性,跟這樣一個區別對待的理由背後是不是足以說服、說明這樣一個區別、不同的對待,具有足夠的正當性存在。

因此這裡面確實存在著所謂的價值判斷。我們只是先第一階段區辨了這個地方做了法律上的不同對待。第二個,他是來自於立法者自己所說明的追求某一個目的,而那個目的接下來才是去講變辨別,關於他偏離平等原則的這種不同對待,跟這個正當化理由之間,他是不是能夠符合我們在價值判斷裡面的正當性。

因此我們進一步的檢驗,以剛剛所講的軍教人員免稅爲例,透過所謂的對軍人或教師(當時候的教師,並不包括高等教師,而是包括公私立屬於義務教育裡面的教師,也就是說中小學,基礎的義務教育這個部分)是有稅捐優惠。高中之後顯然就沒有。所以包括我自己在內,高等教育的老師,基本上是沒有在軍教人員的免稅這個階段,透過這個所得稅法第4條第一款第二款的規定被優惠到。

那他的正當性就如同我剛剛所提到一般而言,我們的立法者會在他的立法理由沒有理由裡面去說明,說,哎,為什麼要給他一個特殊的待遇?但我也要強調跟各位去強調,在我們的立法者往往會懶惰,甚至是恣意到連交代立法理由都不交代,這種情況在稅法是很可能發生,而且會經常發生的就是立法理由裡面寫按黨團協商條文通過。我要是大法官,我就直接宣告違憲。那個就是明顯的恣意。我知道政治是妥協的,所以有時候必須要發給。連理由都不寫,這個在法律上絕對過不了平等原則講的第三階段,就是你要有一個正當理由嗎?你連理由都不給。就更不用講,在稅法的條有一些是直接寫,按照某黨建議修正條文通過。我不一定有那個機會,但各位將來有有一天有機會當大法官,你一定要勇敢宣告違憲,這個就很明顯的恣意,就是無正當理由而做區別對待。你連立法理由都不寫的,那當然是一個不正當理由。有膽量你就寫出來,就是說軍教人員對國家特殊貢獻,薪水又少,所以我們給他特別優惠,當然接下來呢,就會涉及到很多更複雜的就是這個地方的區別對待,有沒有足以反映出這樣一個價值的判斷?當然這個地方我承認會有見仁見智的看法。

說軍教人員對國家有特殊貢獻的人,他也許並沒有強調,不是這個領域的人對國家就沒有特別貢獻。比如說我作為一個教育的從業人員來講,當然不會說,因此我們高等教育就完全不重要,基礎義務教育很重要,但為什麼在這個地方做了對同樣從事於教育,做基礎跟義務的區別對待,也許是在這裡面國家認為他的薪水或者是他的職務,或者是他的階級更為辛苦,也許是這樣。好,那就看,如果個案裡面就看你怎麼去說理,你認為軍人跟教師對國家有特殊貢獻,那公務員沒有貢獻嗎?農漁民各行各業難道不是在自己的職業跟崗位上面去對國家所謂的做出貢獻嗎?做區別對待的理由又在哪?如果說他的薪水少,時代變遷了以後,難道沒有與時俱進檢討的空間跟可能性嗎? 在這個過程當中,無疑是立法者的怠惰。
縱使某個時代背景裡面有其正當性跟必要性,但並不代表著每一個時間每一個時代都認為這樣做是有正當性的,特別是在當今,勞動環境急速惡化,特別在二戰之後的勞動環境急速惡化,因為全球化帶來全球勞動力,或者工作機會的移轉,其實很大程度上會把高所得國家的勞動機會帶往低所得國家的勞動市場,這個是二戰之後的全球化帶來必然造成的一個趨勢。
所以勞動環境的惡化,薪資長期的無法上升,在這一種環境大變化下,所謂的教師、軍人是否有足夠的正當性去獲得這樣的特殊對待?

當然,軍人是一個比較特殊的職業,這一點我是承認。因為軍人是一個極為特殊的行業,在一般時期裡面,軍人這個行業,如果你自己脫逸軍隊的控制,他就是殺人,他就是強盜的行為,他是一個構成一個犯罪的一個行為。很多人基於良心或宗教的因素,根本不願意進入軍隊去。

但我們還是要回過頭來去講一件事,就在這個地方裡面,教師跟軍人究竟有沒有正當化的理由的區別對待,也許我們在稅法總則的課堂上裡面做進一步的討論。

回到我們的所得稅法裡面,我們這個地方,因為目前為止我們軍教人員的免稅已經被廢止掉啊,所以目前為止不太會有這個問題。

唯一剩下來留下來的是上個禮拜有跟各位提到,直接稅處26.4.21.處字第203號訓令,民國26年4月21號,你光看那個民國26年,大概就知道,這個直接稅處是因為當時候還沒有財政部,他是,直接稅處。

訓令就直接提到關於從事性工作的人是不用課所得稅。這是一個很典型的因為工作而做的區別對待。曾經在我們在宣告,關於從事性服務這一件事情是不是要把他合法化,特別是在罰娼不罰嫖的時候,我們在大法官解釋的時候,那時候一個非常大的爭議就是關於這樣子是不是強迫了更為弱勢經濟上更為弱勢的人,也就是從事與性服務工作的女性,因此在地位上面有更不利的待遇。那么這個訓令的背後邏輯是認為從事與信徒工作這個是被禁止的,也就是不法的行為,優先被取締,而不是透過課稅的方式予以正當化或者合法化。直接稅處的邏輯基本上是建構在所謂的課稅就帶來正當化的這樣子,或者是合法化,這個邏輯基本上是錯誤的。

最近也看到有一個政治上的新聞爭議,違章建築要不要課房屋稅?倒過頭來,房屋稅的課稅可不可以證明他不是違章建築? 答案是,建築法歸建築法,稅法歸稅法。我們並沒有比建築法更低位階,我們也沒有比建築法更高位階,因為我們是各自管轄。在建築法裡面沒有取得建築執照、使用執照而蓋成的房屋,我們稱之為叫違章建築。因此,依照法律上的規定,原則上他必須要被拆除,否則他就是要依照法律去做相關執照的申請或者是做改建,那個是建築法的規範。對稅法而言,一個東西,只要他被蓋成四有周壁上有頂蓋,作為一個房屋,我們就可以課稅。對稅法來講,就是他是不是構成房屋,而不是他是不是有合法使用執照建築執照的房屋,所以違法的房屋仍然有課房屋稅的可能。回過頭來,如果違法的房屋不課房屋稅,只有合法的房屋才要課房屋稅,他反而會在稅制上形成一種不中立,反而是讓多數人會取向,未取得合法建築執照使用執照而去建住房。

也就是說,在稅制上面來講,我們不管原則上不管前端的他到底行為的合違法性,或者是那個財產的取得本身得很違法性。我們只看後面我們看稅法的價值,這一點也是要跟各位要特別強調稅法的價值是在量能而不在前階段,他是什麼手段去除的?所以殺人放火賺到的錢也是錢。

在羅馬帝國時代的時候,曾經有課過,上去上公共廁所要尿尿稅。尿尿會臭,但錢不臭啊,就是你去使用公共廁所的時候他他可以徵稅。因為對那個這個皇帝來講的話,認為說,哎,你去使用公共設施,你是有能力的,所以從而他要跟你課一個稅捐負擔。我們在這裡面主要是去講,不管你前面的行政法的規範,認定他是合違法性如何,我們在稅裡面的標準就是當你取得所得,當你有財產,當你有消費行為的時候,原則上根據稅法的標準跟講,你有稅捐負擔能力,原則上你應該要去課稅。

回過頭來去談到這一個從事於性服務工作這一件事情,不管我們國家的行政法的規範究竟開放或禁止到什麼程度。原則上只要有所得,就應該被課稅。所以即使是從事性服務工作,我們稅法還是毫不留情,也毫不猶豫地認為,你有賺到錢嗎?有,就請你繳稅。性工作本身的合違法性或合法化跟稅法是沒有關聯性,因此到過頭來,同樣的道理就是課稅這件事情無法去回頭證明這個行為本身在該國家裡面是不是取得合法地位這件事情。

我們提到的普遍原則,剛好會跟我們今天講的稅捐主體構成要件是有關聯性的。量能課稅的這10個子原則,基本上都是對應某一些稅捐構成要件,那這裡面對應著稅捐主體構成要件的就是普遍原則。目前為止,我們的所得稅法第4條第一款第二款違反普遍原則的條文規定,在現行立法上面已經被廢止掉,從而我們現行立法上並不存在著因為職業身份性別階級而做的區別對待。但我們實務上確實仍然存在著對性工作服務者因此取得的報酬。在財政部的前身直接稅處26.4.21.處字第203號訓令底下,這個還存在財政部編的每一年的所得稅法令彙編,他就放在第4條之後的附錄,也就是他沒有辦法被涵攝到所得稅法第4條裡面。所以他是透過我們財政部長期從大陸時期的直接稅處就把他既受過來,然後一直沿用至今,認為是仍然有效的一個解釋函令。很明顯,這個解釋函令,既無法律明文規定作為其依據,又不符合量能課稅的平等原則,更是違反稅捐中立的一個違法解釋案例。因此老師個人認為,像這一種違法解釋函令的存在,應該要加以改變。至於要怎麼改變?我上個禮拜,我跟各位提到過,有些學者認為這個是構成行政慣例。很抱歉,違法的解釋函令,不會讓你變成是一個具有法律確信的行政慣例,也就是這個地方,根本不構成行政上的習慣法。因為在行政法裡面,認為習慣法是一個法源。但稅法強調的依法課稅原則,並不承認習慣法具有這樣的一個法律地位,我們只承認制定法。制定法才有能夠確認國家跟人民之間稅捐法上權利義務關係的法律上的地位。習慣法,沒有辦法透過反覆之慣習人民之確信,而因此證立稅捐負擔義務。同樣的道理,人民也不因此證立不存在繳稅的義務,因為依法課稅,依法才能免稅。

我們不是一個單純依法免稅,不是一個單純的給付行政,他是一個依法課稅原則的反面構成要件。因此只有制定法才能夠成立發生人民的稅捐負擔義務,相反的,也只有透過制定,最低限度就是他授權的法規命令來做稅捐負擔義務的免除,而不可以透過行政上的慣例。行政機關無權去變更、創設人民的稅捐負擔義務的免除,即使他在結論上是對人民有利的。所以老師個人認為從來沒有所謂的,因為對人民有利,所以就可以,這樣子的一個所謂的行政慣例。從而這個行政慣例不成為法律,他就是一個很單純的違法行政上的慣例,違法的行政規則。

違法的行政規則,第一個,他不可能有稅務訴訟,因為被優惠的人不會去接到稅單。不被優惠的人,可能從事於同樣類型服務的工作,他並沒有辦法去主張說,哎,為什麼他從事性服務工作可以免稅,我不是做這個,可是我從事類似的工作,但為什麼我沒有稅捐優惠。如同那個道理,就是我從事高等教育的我沒有辦法去主張說今天國家對我徵稅,我拒絕繳稅,因為你應該要先對那些基礎義務教育人也同樣徵稅,我們大家都做教育的,為什麼他可以免稅,為什麼他可以享受稅捐優惠,我卻沒有享受稅捐優惠,以此為由來主張要求要平等原則的對待。因為當我主張這件事情的時候,他可能就涉及到,這根本不是權利,而是一個不法的主張,也就是說,我要求國家應給我一個免稅的優惠待遇,以讓我跟我認為是相同地位的人處在相同的地位上面。因為我不存在著這一種對國家主張應給我免稅優惠待遇的所謂權利,從而無法正當化,我透過拒絕繳納稅捐的方式去抵抗國家的稅捐徵收的行為。同樣的道理也存在,對從事於非性服務工作,但他主張我可以同樣滿足跟性服務一樣的,比如說身心靈的一樣的滿足的工作。你沒有辦法去主張,因為我們任何這一類的勞務或服務這一個的投入的行為,其實他都會被課薪資所或者是執行業務所得,他都會有相同的這些稅捐負擔的存在。

因此,第一個我們談到普遍課稅原則,尤其在所得稅的稅捐主體上,理論上來講,就是中華民國人民不分職業身份階級黨派原則上都只要有所得就應該課稅,這就是普遍性。那在普遍性原則底下,我們過去曾經有軍教人員的免稅原則上不符合這樣一個普遍性原則,他不構成行政慣例,他因為沒有個案的司法的判決出現,所以從來沒有被挑戰過。那也因此,學者也有一些建議,說我們可以立法的方式去改變他。因為延續著前面剛剛講就是行政習慣法。所以大家通過制定法的方式去加以改變,因為這個地方我個人是非常懷疑。我在想說,立法者怎麼會定一個法律條文說從事性服務者應繳稅並與敘明。這個是誰捅出來的簍子?是行政的問題啊,你怎麼會叫立法者去補你的漏洞呢?所以換言之,很簡單,這件事情就是財政部自己把他刪除掉就好了。好簡單的一個方法,結果到現在還刪除不了。其實刪掉的用意只是告訴你賺到錢要繳稅跟性服務工作的合法化是沒有關聯性。不能拿這個來換這個。有一些看法是這麼說,他可以用繳稅來代替行業的合法化,其實我也認為這個行業應該合法化,但是跟繳稅真的是兩回事。當然,接下來的困難就是從事於性服務工作,他究竟是執行業務者還是哪一種受僱的薪資所得,還是他是一個獨資商號,會有成本費用扣除的問題? 這些是進一步去加以面對的問題。

所得稅不會只有在應課稅跟非課稅這個單純問題,因為接下來就是一連串的問題,就是如果你要課稅,我們要怎麼課稅、我們要怎麼樣的手段,是後面連續的問題。

\hypertarget{ux7b2cux4e8cux9805ux5168ux90e8ux539fux5247ux53caux5176ux5f8cux5404ux9805}{%
\subsection{【第二項全部原則及其後各項】}\label{ux7b2cux4e8cux9805ux5168ux90e8ux539fux5247ux53caux5176ux5f8cux5404ux9805}}

十項子原則裡面第二個叫全部原則,就是所得不分境內外,原則上你只要是所謂的稅籍居民,原則上,中華民國來源所得跟非中華民國來源所得錢就是錢,我們紛紛全部所得都應該課稅。

第三個叫實現原則,他要現實上實現,因為錢他這一個作為以通貨表彰出來的稅捐負擔能力,是一種現金支付能力,原則上是以拿到錢為基準。你設想預想他將來會拿到,那你那個預想跟設想將來可能會產生預估跟現實的差距。然後另外第二個就是時間,你太過早實現這個稅捐負擔,這個會是一個可以在實現原則裡面進一步去討論問題。

第四,實價(實值)原則。實價,應該依照真正的市場上的價格,售價,來做實價課稅原則。

第五,客觀淨值。這個我們在後面的法條裡面會一再地跟各位去談到。

第六,主觀淨值,是我們這個學期上課的一個重心所在,也就是個人賺的所得,原則上是要先扣掉維持自己跟扣掉維持配偶及受扶養親屬的費用。因為配偶實踐了婚姻保障家庭的制度,受扶養親屬是實踐家庭制度保障的一環,從而維繫自己個人生存配偶配偶的生存,以及維繫家庭的親屬的這些受扶養親屬的生存,這個會是主觀淨值原則。這也是我們這個課程裡面,在相對處理上面來講比較複雜,因為他要涉及到不是只有所得稅法要涉及到我們後來105年修法制訂的納保法,然後他又跟社會法裡面的最低生活費有密切的關聯性,因為他那個地方反映出來一個社會的最低生活標準,最低生活的所需的財產究竟有多少。

第七,個人個別主體課稅,就是individual,每一個人是每一個獨立的主體。但我們很快在我們綜所稅裡面會發現,我們個人到第15條規定的時候,變成家庭家戶所得課稅之在那個地方配偶跟受扶養親屬,從法律的意義角度來看,他不是納稅,艾薇兒是附屬在納稅人身上的另外一個個別的足以他是一個被扶養的對象而已,這個法律的體系,我個人認為是一個違反現代法律體系建構的基本精神就是個別主體,個別原則。但我們的大法官解釋裡面有許多大法官稱這個是見仁見智的立法價值抉擇,這個我看到這裡真是全身冒冷汗,這個怎麼會是這樣?現在法律就建構在每個人是自己的行為主體,私法自治,在講個人啊;刑法呢,個人罪責原則啊,這個最典型的就是每一個人就自己的行為去負責啊,我為了別人負責這種情況,怎麼能夠在現代法律裡面去發生了這種情況?什麼轉嫁罰大法?這是什麼概念啊? 轉嫁把你做不對的事情,結果把那個處罰罰到我身上。其實在我們的文化裡面,往往有這一種連坐的概念。連坐處罰跟這個是不符合現代法律基本的精神,我們是個別權利。原則上。為自己的行為做起最終的決定,私法自治就這樣。那如果自己的行為啊,有可歸責的事由,那要去對這個行為去負起責任,這是個人罪責。我們的民法,我們的刑法是這樣一個基礎,而典型的以個人為主體。當然個人在憲法裡面被上升提升到所謂人性尊嚴的層次,這樣一個層次,這是整體現代法律制度的精神。我不否認,在進入近代化之前,確實有許多不同的國家區域,會有一些不太一樣的概念,特別是像部落。部落,不是一個個別所有權,他是一個集體所有權的概念,這種集體也沒有強調個別的權利,而是在強調這整個部落這整個族群本身對某一個特定事務或是特定區域的這樣一個享有特定事務的權利,這一個確實是跟近代的法律權利思想有所區別而有所差別的。這一點我承認,這也是我們現在目前在建構原住民族或者是少數族裔,特別是在,現代社會裡面,由於在經濟跟許多教育資源的弱勢的情況底下,現代法律制度裡面的精神,特別是所有權跟所有財產權的保護,也讓他們不太能夠維護這樣一個自我認同的一個價值。這個是一個現代法律裡面,我們個別還需要再進一步去做處理的問題。
但我還是回到,如果以我們現在法律基本上的精神是個別主體的話,課稅也是個別主體算。原則上特別注意算我跟另外一個人結婚,我要養他,那我會因此我付錢給他,他也是一個主體,他並不付出於我。他並不附屬於我,所以我只付給他的是他來維繫他的生存需求,而你容許我去扣除掉我支付給另外一半的這一個生活費用是維護了婚姻。他其實在說理上並沒有困難之處。不要把特別主體責任個別主體課稅原則在這個地方動搖,說我們採家戶課稅制,這個是我們現在目前的一些法律的部分的困難理論說理上的困難,跟一些法律建置上的困難

第八,綜合所得課稅原則。原則上所有的所得放在同一個稅基,我們不分離開來,當然這個其實是有很大程度的立法價值選擇,包括以英國為主的各個所得稅,或受英國法律體系影響的,譬如說香港新加坡,他們就是分離所得課稅制。在香港是個別所得的來源,他沒有放在一起去做出綜合所得稅的結算申報,那也因此在香港新加坡或是受英國法律體系所影響的。他們的稅制跟歐陸法系跟美國的法法的法律體系所表現出來類似。我們是綜合,我們是放在同一個稅基底下,這個就叫綜合不分類。我們講到第14條的時候,會去跟各位去談這個綜合所得課稅原則。

第九,終生所得原則。終生,意思是跨期間的意思,跨一個課稅期間。我們會到跨年度盈虧互抵跨期間盈虧互抵的地方跟各位去進一步去做說明。

第十,比例課稅原則。所得稅法我們在綜合所得稅是用纍進稅率,但在營所,我們卻是採比例稅率。就大概是這樣子的一個,各種不同態樣的實然跟應然之間的討論。實然也就是現實狀況的時候,這個地方從量的課稅原則的角度來講,他本來應該是這個樣子一個規定,但我們現在實務上的規定並不是如此。那是不是因此就因為違反量能、就違反平等原則,所以被宣告違憲?那也不盡然是如此,還是要看他在那個地方個別的價值判斷,就是立法者有沒有說出一個正當化的理由,而足以去正當化偏離量能課稅原則的這樣子的一個正當性

我們先休息一下之後我們來開展,今天我們想到的第一個,所得稅的構成要件,就是稅捐主體,也就是納稅義務概念,境內居住者、非境內居住者之納稅義務人。
所有的法律都有主體概念,民法有權利主體、訴訟法有當事人能力這種主體概念的意涵。我們先休息一下。

\hypertarget{section-6}{%
\chapter{20230918\_02}\label{section-6}}

\begin{longtable}[]{@{}l@{}}
\toprule()
\endhead
課程:1121所得稅法一 \\
日期:2023/09/18 \\
周次:3 \\
節次:2 \\
\bottomrule()
\end{longtable}

\hypertarget{ux7a05ux6350ux4e3bux9ad4ux69cbux6210ux8981ux4ef6}{%
\section{【稅捐主體構成要件】}\label{ux7a05ux6350ux4e3bux9ad4ux69cbux6210ux8981ux4ef6}}

關於境內居住者之納稅義務人跟非境內居住者之納稅義務人,在法條的規範構成要件上面,我們的所得稅法在構成要件的分段落上面並不是非常清楚,所以必須要我們法律的適用者,乃至於是法學者在這個地方去進一步去解析,在這整部的所得稅法的法條裡面,哪些是關於稅捐主體的構成要件中所說,這個稅目的稅捐主體構成要件。

我們請各位來看到所得稅法第2條第2條的規定,這個就是。我們在第2次上課的時候跟各位講過,他是最基礎的構成要件的規定,因為在這裡面,原則上他把課稅的各項構成要件的要素跟法律上的效果,他都在這裡面做了一個規定。雖然他不是很完整,但他還是整部綜所稅課稅基礎構成要件的規定。如果你問我課稅處分應該要理由要怎麼去寫?其實在我來看課稅理由就是要去寫你的課稅構成要件,哪一些是被滿足。

因此我們來看一下所得稅法第2條的第一項跟第二項的規定。凡有中華民國來源所得之個人,應就其中華民國來源之所得,依本法規定課徵綜合所得稅第二項規定,凡中華民國境內居住之個人而有中華民國來源所得者,除本法另有規定外,其應納稅額分別就來源扣繳。就這樣一個條文的規定,第2條的第一項,一個是關於主體的規定。個人就這個個人。可是這個個人看起來並不是一個很完整的描述,理由在於第一項跟第二項做了進一步比對,以後個人毋寧是比較是含糊其詞的,也就是他是比較上位的。我們同時間請各位來看第7條第一項第二句的規定。所得稅法第2條第一項的個人,加上所得稅法第7條第一項第二句的規定.

稅法沒有用我們在民法學界或者是刑法比較慣用的幾段幾段,是因為到所得稅法的條文的時候,你會發現我們的立法者的段不是很清楚,例如之後會講到所得稅法第14條跟第17條這兩個條文,我自己在教學研究上,光要指稱他非常困難。例如說所得稅法第14條的執行業務者。他在法條規範結構上是14條第一項啊,但是他不用我們常用的款,他用類所得這把14條第一項的第二類好,接下來下面,你看那個第二類的規範的文字,總共有3個大段落,所以,比如說我要去講記帳義務的時候,在稽徵協力義務裡面的時候,我要去描述執行業務者,至少要設置日記帳,這個時候我在法條文字上,我要去精確描述他,我只能說14條第一項第二類,第二段第一句。法律重在精確引用你所講的規範文字。就好像我們在民法184條第一項前段後段第二項是不同的構成要件,是一樣的道理。法律重在精細,你不能跟我講,根據所得稅法規定。所得稅法是哪一個規定?你不能,說哦,有這樣一個行為,又有這樣一個繳納稅款的效果。你要明確的指稱,14條的第一項第二類,第二段的第一句裡面的第1分句句。

因為我們的稅法就長這麼醜,我也沒辦法,但我們為了要精確指涉。民法條文比較精練,他是抄過來的,所以抄得很完美。這個稅法是我們自己訂的,我們的文字是既不精煉又不完整。各位以後會看到很多。為了要精確去指涉,就只好參考德國式的稱述的方式,就按句來編號就好,第幾句有分號的話就分句。這個也是我希望各位同學未來在這個領域裡面做精確指涉的一點表現方法,當然你不這樣做,我也不能說你完全不對,但法律要求要相互有一個討論的共識基礎,我們需要把法條規範做一個比較精確的指涉出來到底是哪一個條文是這樣規定的?我在這個地方就建議各位用分句,句,這幾個的描述的方式,當然不強制,只是一種表述方法而已,那因為在德文裡面,他大概都是用第幾句,因為德國不管是憲法或是稅法或是行政法大部分都是用這種方式去做表現,那這個也是參考他們的一種表現方法而已,就這樣而已。

第2條的第一項、第2條的第二項都有提到個人,而這個個人連結上第7條的第一項的第二句的規定,這個個人是指自然人。那回過頭來,我們去講納稅義務人的定義呢,在第7條第四項規定,也有進一步去把納稅人做了一個定義規定,本法稱納稅義務人,係指依本法規定,應申報或繳納所得稅之人。在我來看這個條文規定有規定跟沒規定差不多。就是第2條再講的條文的內容。因為第2條本來就是在講有中華民國來源所得之個人,應就其中華民國來源之所得,依本法規定課徵綜合所得稅,他講的就是你有中華民國來源所得之個人的人作為納稅人依本法的規定來課徵綜合所得稅。第二項的規定一樣,他也是在講自然人的個人,也是在講納稅義務人,可是從第一項的規定裡面,你不太看得出來,到底一項跟二項都是在講個人,那究竟有什麼差別?這也是我們的立法文字不精練的一種表徵。在這個地方,我們因此在學理上實務上多數稱第一項,這個人把他稱之為叫境內居住之個人。境內居住之個人第二項叫非境內居住之個人。或者你把他帶換成境內居住者。非境內居住者。這樣子的一個稱呼,才比較能夠去區辨出來第一項跟第二項之間的主體差別。因為這個主體差別本來會是在全部所得課稅裡面做區別對待。這個主體差別就是在大多數分辨稅籍居民的國家裡面,所謂的稅籍居民原則上是就全球來源所得都課稅。

如果你是非設籍居民,你只就屬地的來源,也就是從那個國家去取得來源地,就這個限度範圍屬地課稅,這個就是各位在所有的教課書裡面幾乎都會提到的所謂的主人跟屬地主義。原則上稅籍居民的個人,是就其全球來源之所得課徵綜合所得稅。如果你叫綜合所得稅,那就是課徵綜合所得稅,如果你叫所得稅就課徵所得稅,原則上是這樣,這個就叫屬人主義。如果你不是我們的稅籍居民,相反的,非稅籍居民只就你俗的取得的來源所得被課徵當地的所得稅,這個就叫屬地原則。所以我們在學理上面,由於我們的法條文字本身不是很精練,因為第一項本身只有講個人,到底是在跟第二項的個人有什麼差別呢?你透過2條一項跟第二項做對照,因為第二項的文字裡面有一個非中華民國境內居住,所以我們就把他精煉成境內居住者這個概念或境內居住之個人啊。是透過一項跟第二項對照,反襯出來第一項的稅捐主體概念,這一種方法叫做體系解釋法,因為文字本身不好,我們的稅法的立法文字本身不精練,喜歡加一些有的沒有的東西,這個地方不涉及到意識形態,你會看到我們的所得稅法,很喜歡加中華民國這幾字。我沒有加意識形態,我只是告訴你,這個是不必要的文字廢話。我們下一次去講到第8條,你扣掉那個中華民國,你都發現就幾乎是第14條就這樣而已。我不太知道為什麼我們在稅法的領域裡面,中華民國這幾個字出現的頻率非常高。德國的所得稅法規定,從德國取得的、從非德國取得、德國的稅籍居民的\ldots\ldots 不會一天到晚在重複文字,因為那個不是法律上面的重點。

在這裡面,每個課稅主體原則上就是以法律能夠執行的這個限度範圍內裡面去講。所以我們在這裡面其實真正比較精練的文字是凡境內居住之個人(境內居住者)取得境內來源之所得,應依本法第14條到第17條之規定,按照所得稅法第71條的規定來結算申報繳納所得稅。這個才是我們這樣一個完整的法條結構,在這裡面,原則上應該要呈現的。不過我們的法條文字第2條的第一項,他不是很清楚完整的呈現。第2條的第一項。他的法條規範,他說的個人是指自然人,是指納稅義務人,而這個個人跟納稅義務人,他其實是叫境內居住者。我們實務上習慣稱為境內居住之個人,學理上則有不同的稱呼,叫境內居住者。德國把他用括弧的方式稱之為叫無限制納稅義務人。相反地,他們的相對概念就叫做有限制的納稅義務人。所謂的不限制跟有限制,就是對應到我們的境內居住者跟非境內居住者,也就是原則上境內居住者是全球來源所得,所以在德國的法境以下,這個叫無限制納稅義務。你一個德國人,你是德國的稅籍居民,原則上你拿到德國瑞士盧森堡,你有所得,全部都歸算到你在德國的所得稅裡面去課德國的所得稅,美國的所得稅也是一樣的概念。他原則上是一個所謂的屬人,也就是無限制納稅義務人,因此我們在學理上面稱無限制納稅義務人,或是我們的境內居住者,我們把他稱之為稅籍居民。

接下來就要跟各位去談稅籍,跟國籍,戶籍,是不同的法律不同的管制規定。稅法中,稅籍居民,除了稱之為叫境內居住者,無限制納稅義務人以外,我們有時候有一些也會用稅務居民。所以我個人來講,反正這個叫做稅籍稅務居民或者境內居住者啊,只是我們的法條文字看不出來境內居住了這個字眼而已。在這裡面稅籍跟國籍跟戶籍是不同的法律規定。一般而言,不同的法律依不同的法律去判斷,你是不是他們的屬人的歸屬者,也就是稅籍居民或者是國籍的國民。戶籍,也就是在中華民國臺灣地區,依照數值法所登記的設立戶籍的戶口的這些人,這個就叫做戶籍。戶籍法規定,國籍法依國籍法規定,稅法呢稅法則依稅法來判斷稅籍居民的概念或稅務居民的概念,我們稱之為叫境內居住者,我們也可以稱之為叫無限制納稅人,也可以稱之為假稅務居民。稅務居民的自然人,我們到下一個學期或下一次上課,營利事業所得稅的時候一樣,會有一個總機構在中華民國境內,各位你看一下第3條的規定。第3條的第二項。

為了要描述那個稅捐主體,我們立法者用了非常多的文字,叫做營利事業之總機構在中華民國境內者,這個叫總機構在中華民國境內之營利事業。稅籍居民前面那個地方是自然人,後面這個地方就是歲吃。營利事業稅的居民企業,我們用一個比較簡單的字眼,這個也是立法文字上面,本來透過立法者本來應該要去做的,但因為立法者沒這樣子做,每一次我們為了要講那個稅捐主體,我們都要花很多文字去描述這個稅捐主體。在綜所稅裡面叫境內居住者,在營所稅裡面就是我們的文字叫做總機構,在中華民國境內之營利事業,那這個就是稅籍居民企業,就是他是屬於我國籍營利事業這個概念。
這個是立法文字上面,我們首先去辨識稅籍國籍或者是戶籍。

登記戶籍,本來就基本上是個別依照個別的法律的規定。可是我們對稅籍的要件要求又是如何呢?則是透過第7條第二項的規定。稅籍或稅務居民的概念在這裡面,第7條的第二項,做了兩個要件的規定,也就是這兩個要件就是擇一要件,你只要滿足其中一個要件,原則上就會是我們的所謂稅籍居民,也就是本法稱中華民國境內居住之個人。其實第7條第二項本法稱中華民國境內居住之個人,我們整部法條根本都沒有寫中華民國境內居住之個人。在這裡,第7條第二項境內居住者境內居住之個人是指兩種。

第一個,第7條第二項的第一款叫在中華民國境內有住所,並經常居住在中華民國境內者。這兩個要件是第七條第二項第一款規定的第一個類型。

第二個類型。七條二項第二款,在中華民國境內無住所,而於一課稅年度內在中華民國境內居留合計滿一百八十三天者。因為我們的課稅年度不是按會計年度,我們是用日曆年,也就是每一年的一月一號到每一年的十二月三十一號。如果沒有特別說明,所有的課稅年度都是按日曆年計算。每個日曆年原則上365天 所以就是超過日曆年的一半以上,這個時候你就被稱之為叫境內居住之個人,也就是我國的稅籍居民。

第2條的第一項連結第7條的第一項的第二續的規定,再加上第7條的第二項的第一款第二款的規定,這個就構成了我們綜所稅裡面的稅籍居民稅務居民或稱無限制納稅義務人或稱境內居住者之個人的稅捐主體。原則上只要他符合住所,經常居住中華民國境內。第二個是沒有住所。在境內停留超過歷年的一半以上,這個是一個對於境內居住者我們所得稅法的法律條款規定。這個就叫設籍居民或稅務居民,跟國籍是依照國籍法裡面的規定不一樣。戶籍法裡面規定,在中華民國臺灣地區設立戶籍的,這個是戶籍的概念。每一個法律有不同的法律規定。

第7條第三項規定,就是所謂的非境內居住者,就是前面的邏輯上的相反,所以7條第三項的規定,非稅籍居民就是指前條規定以外之個人,這個條款規定基本上也是廢話一句,因為你只限定了第7條第二項,這個相反原則上就是7條第三項的規定連結到的前面的第2條第二項,也就是稅籍居民的要件,核心構成要件的要素就在7條二項第一款或第二款的構成要件滿足上面。至於什麼叫住所、什麼叫經常居住,這才是真正的關鍵所在。法條本身則沒有任何說明。也就是稅籍居民的概念,你如果只看法條,他大概可以指出來的就是2條一項7條一項第二句,第二句的規定7條第二項的規定。7條第二項裡面又分成第一款跟第二款所謂的中華民國境內境內有住所跟經常居住在中華民國境內。如果你沒有滿足這個構成要件,那就是中華民國境內沒有住所,經常就停留超過一個歷年183天以上,這樣就可以是我們的稅籍居民,所以我們很快的,我先用這個作為基礎。外籍移工在臺灣停留超過一年的一半以上,不是我們的國民,不一定在這個地方設戶籍,但他是我們的稅籍居民。所以外籍移工有沒有繳稅?答案是,理論上他們都是中華民國的稅籍居民。他們沒有國籍,他們不一定設戶籍,但他停留超過一年的一半以上。所以外國人來臺灣工作,他就算不是我們的國籍的國民,不一定在本國設有戶籍,他同樣是我們的稅籍。這個是我們稅籍的概念。

那我們接下來我們的問題是,我們的法條本身所講的經常居住,跟住所又如何去解讀,這也是我們自己本身的所得稅法,你如果講法律明確性,那我就跟各位講,法律就是不明確,那就是長這個樣子,因此在實務上都是透過某一好財政部所公布的解釋令。101年9月27號,財政部臺財稅字第10104610410號,中華民國境內居住之個人認定原則。我把他放到稅務小六法。你終於知道我們實務上面操作,實務上沒有人在看所得稅法第7條第二項第一款或的。因為那個地方只有講什麼叫有住所,什麼叫經常停留在中華民國境內那個地方,本身並沒有很清楚的。透過實務上101年9月27號的這個解釋令,所得稅法第7條第二項第一款所稱中華民國境內居住之個人認定原則,其認定原則如下。在一個課稅年度內,如果在境內設有戶籍,而且有以下情形之一者。所以。以我們稽徵實務的看法,這個地方設有戶籍,他對應的就是在中華民國境內設有住所的概念。這是實務上的邏輯,我們先不講有沒有符合依法課稅原則。如果你要去批評他不符合依法課稅,我也認為確實他是不符合。因為法律並沒有這樣定。法律,只有講第7條第二項只有講在中華民國境內有住所,那至於什麼叫中華民國境內有住所?根據這一個解釋並的看法是說,在中華民國境內設有戶籍。所以他把住所這件事情是按戶籍法裡面地設有戶籍的概念去套用。這個在稅捐機身實務上很常見喔。所謂的設有住所,不適用民法的,事實上就是以一定作為你的生活中心所在。我們的民法本身是以實質的生活重心所在地作為住所。但稅捐稽徵實務的看法,我們先不看,先不談他有沒有違反依法課稅原則。以稅捐稽徵實務的看法,他是以設有戶籍來認定你有所謂的住所,中華民國境內設有住所。接下來第二個要件,也就是第7條第二項第一款裡面的第二個他所提到的,經常居住中華民國境內,這個又怎麼認定呢?他的認定原則是這樣。有以下情形之一者,是在一個課稅年度內在中華民國境內合居住合計滿31天。也就是你戶籍設置在中華民國,你依照戶籍法規定有設戶籍,然後再一個課稅年度內,境內居住合計超過31天,這個時候就是經常居住在中華民國境內。第二個,他說,如果你是在一個課稅年度內在中華民國境內居住,是超過一天以上未滿31天,其生活及經濟重心在中華民國境內,這個也認為你是所謂的境內居住之個人,也就是設有戶籍在中華民國境內,並且經常居住在中華民國境內這個要件。

第7條第二項第一款的構成要件的要素裡面是境內有住所,經常居住在境內。由於法條規範,文字並沒有很清楚的理解,他到底什麼叫境內有住所跟經常居住在中華民國境內?根據實務上財政部的101年9月27號的解釋令。他認為,所謂的境內有住所,不是按民法上的標準,而是按在戶籍法裡面依照戶籍法的規定去設立戶籍。而有以下情形。第一個是一個課稅年度,也就是歷年在中華民國境內停留超過31天。第二種情形是超過一天以上。未滿31天,但其生活及經濟重心在中華民國境內。這樣的一個標準正是在稽徵實務上面認為,符合境內居住者第7條第二項第一款裡面的這兩項構成要件要素標準。用設戶籍的跟用你客觀上有沒有在歷年內在我國境內停留31天以上、或者是1到31天,但生活及經濟重心在我國,這樣一個看起來是極為形式上的一個認定標準來替代原先在所得稅法第7條第二項第一款裡面所提到的境內有住所跟經常居住在中華民國境內。這一套標準各位可能認為違反依法課稅或至少是違反稅法的明確性,因為幾乎關於稅籍居民的判斷標準在實務上。所得稅法本身沒有規定實物,幾乎都是透過解釋這個解釋領來去做判斷標準,因為他有很形式而明確的標準就是戶籍法,戶籍登記,你停留在一段時間內,在中華民國境內,這個時候就該當。在這樣的一個情況底下司法實務從來沒有認為這個標準違反依法課稅跟法律明確。換言之,這一個是經由目前的至少在司法機關實務上面並沒有認為違反法律依法課稅原則跟法律明確性,他反而認為這個正是在解釋所得稅法所講的境內有住所跟經常居住在中華民國境內。已設有戶籍,作為境內有住所跟以停留在中華民國境內31天,跟1天到31天未滿,但生活及經濟重心在我國境內作為他的一個判斷標準,然後這一個解釋令有在第2點裡面,第二就是在這個地方,其認定原則,因為他總共有2點,總共有2點,第2點他就提到所謂的生活及經濟重心在中華民國境內。應衡酌個人之家庭與社會關係政治文化及其他活動參與情形職業營業所在地管理財產所在地等因素,參考下列原則,綜合認定:
(1)享有全民健康保險、勞工保險、國民年金保險或農民健康保險等社會福利。
(2)配偶或未成年子女居住在中華民國境內。
(3)在中華民國境內經營事業、執行業務、管理財產、受僱提供勞務或擔任董事、監察人或經理人。
(4)其他生活情況及經濟利益足資認定生活及經濟重心在中華民國境內。

這個才是實務上真正判斷,稅所得稅法裡面所稱之為叫稅籍居民的標準。雖然還是很多抽象的不確定的法律文字,至少他已經告訴你在實務上面認為形式上的一個判斷標準,設立戶籍31天作為標準,如果是1到31天,則綜合判斷他的個人家庭生活社會關係政治文化及其他活動參與情形職業營業所在地管理財產所在地等因素,參考下列原則,綜合認定。說實在,你也真的很佩服稽徵實務。這個文字基本上到底想要講什麼?

我們其實在這裡面總結跟你講幾個要素,他判斷要素,第一個你的保險在這裡,各項的勞健保、全民健保、國民年金保險、農民健康保險等社會福利。第二個你的配偶未成年子女在臺灣。第三個你在本地經營事業管理財產、執行董監經理人業務。也就是經濟生活經濟生活重心在中華民國境內的一種表徵。前面的兩款是你的生活重心所在。後面就是中華民國境內經營事業執行業務管理財產,你在本地擔任董監事或經理人的職務。這幾個要件,在告訴你,生活跟經濟重心在我國境內,那你就是我們的稅籍居民。這些都只是在描述所謂的生活及其經濟重心之所在。我們這一長串的描述在講兩個基本價值,一個就是你在本地生活,第二個你錢是從這裡賺。這個標準,實質上是對的,但是立法上的手段是可以討論的。因為這個標準本質上並沒有在法律構成要件的層次裡面去展現出來。而是透過法律本身是一個極為抽象的住所及經常居住在境內這樣一個文字去進一步具體化,透過財政部的解釋令生活及經濟重心是去表現出你住所在哪裡。經常居住在中華民國境內,則是要去說,因為你是從本地取得你的經濟上的所得,從而使經濟重心所在的這個標準基本上在。國際之間來講也是符合國際之間判斷稅務居民的一個最常使用的標準,就是生活及經濟重心之所在。不是按國籍。不算搏擊國籍不足以判斷你的生活經濟重心所在。各國不一定會有戶籍的制度。以德國而言,德國怎麼去判斷你是他們的稅務居民也是一樣,看你在德國境內停留的天數跟你是不是從德國去取得你主要的經濟來源。因為稅捐建立在你的經濟上的負擔能力至上。

在這裡面,德國有一個案例,很值得各位去做參考。也是根據這個理由,在德國,經濟重心在德國可以成為德國的稅籍居民,也就是無限制納稅義務。有一個叫舒馬克Schumaker的比利時人,他是德語的比利時人。他是住比利時靠近德國邊境的區域。

你看那個名字的拼音喔,應該是之前他的祖先是Schumacher,做鞋子的人,那個車神就叫Schumacher。
其實在比利時的邊境比利時境內,雖然大家認為的比利時是一個講法文的國家可是比利時自己本身其實是有很多他們的官方語言,至少就有所謂的法語跟荷蘭語。再靠近德國邊境其實有很多講德語的本身是德國裔的,但他當然在國籍上面是比利時人。

舒馬克是一個住在比利時靠近德國邊境,每天穿過比利時跟德國邊境到德國阿肯上班的一個人。也因為他的多數所得都是從德國取得,但依照德國在1960年代的所得稅法的規定是不可以扣除,因為他非德國人,他不生活重心,並沒有在德國。他是生活重心在比利時。但他是在德國阿肯上班的外國人,他是歐盟境內的外國人。這個案例裡面,由於蘇馬克在德國沒有辦法享有德國所得稅法提供給德國稅籍居民的關於他的扶養費用,扶養配偶的費用的扣除、各式各樣的生活費用的扣除。作為一個外國人,他取得的所得依照當時的德國的所得稅法的規定,只能夠做就源扣繳,而且是按照毛額的方式去做扣繳,不足以反映出來對他賺取所得,回去去養他的家庭成員的相關的費用,從而舒馬克就在德國去起訴以違反歐盟的平行移動平等原則。我作為一個外國人,我卻在德國受到歧視。因為我賺的錢我跟其他德國同事一樣,但他們可以扣除,我卻不能扣除,只因為我是一個德國的外國人。

在這個地方舒馬克的案例判決,歐盟判決德國為違反歐盟的平等原則,從而之後德國的所得稅法就加進來,只要你從德國取得一定比例以上的經濟所得,是來自於德國,可以改申請以德國的稅籍居民,也就是無限制納稅義務人的要件來申報德國的所得稅。簡單來講,跟你的國籍沒有關聯性,而你主要的所得來源是來自於德國,那這個時候你享有跟德國人一樣在境內取得所得來源的時候的,做比如說扣除掉維繫生存所需要的,或者是維繫配偶,或者是維繫未成年子女生存所需要的相關費用的,可以這樣的一個屬人的課稅的各項減除的項目。正是從這個案例裡面,我們可以清楚地見到,也就是說,以所得稅而言,固然是以生活重心所在的區域去判斷,因為生活重心所在,所以你跟屬地具有比較當地具有比較高度的連結。另外一方面是,當你的經濟主要是從某一個地方而獲得的,從經濟上產生一個稅捐負擔能力的貢獻能力,因為本身稅捐是一種金錢給付,從而以生活跟經濟重心作為考量,判斷是否為該國的稅務居民,這個在國際法的角度來看是一個正當連結。只是我們的立法者,用一個比較看起來不太看得出來,因為你從中文怎麼讀,都是讀不到生活跟經濟重心這兩個字眼。

我們的法條文字是寫,境內有住所。他其實是在講,你是在以境內這裡面為我們的生活重心所在,經常居住他是在講一個工作,你的保險都是在本地,你在本地去執行相關的經濟跟這個財產管理上的業務上的行為。從而這個中華民國境內居住之個人認定原則,我個人包括跟盛老師,我們也討論過這個。雖然法律上並不是立法文字本身不是很妥當,而且他也不是在所得稅法裡面做明文規定,但他傳遞出來的價值是生活經濟中心所在地作為稅務居民自然人的判斷標準這個標準。我們的立法手段有所欠缺。當然你可以講,這個地方以我們的法律文字而言,其實是極其不當的啊,因為並沒有很明白的表現,特別是他把稅籍連結到戶籍這件事情。

我們是釋字415號就已經強調過一次了。釋字415號說稅籍不是以戶籍為標準,而是以生活重心實質的生活重心作為判斷標準。釋字415號,當時候就是針對所得稅法的施行細則的規定,施行細則21之2條。關於家庭成員是按照所謂的實質上生活在一起來認定。家裡怎麼不是按照戶籍?當時的所得稅法施行細則21至2的規定,把所謂的家是用戶籍法設戶籍在一起,作為一個形式上判斷標準。釋字415號說這個是不對的,違反平等原則,各位回去再參考。

戶籍不是住所,這幾乎是法律學界的共識。可是正是因為如此,戶籍當然不會是生活重心所在,戶籍可以在某程度上推定為是生活重心所在,這個是另外在釋字558號解釋裡面,戶籍非不得推定具有久住之意思。但正是因為戶籍跟住所之間的連結,在法律上面是有所欠缺的,我們都很清楚的知道戶籍具有行政戶政管理的用意,包括你的社會連結,包括你的教育連結,常常都是以戶籍作為根據地也許你生活地點都在臺北,可是你戶籍可能還留在雲林的老家。在這種情況底下,這種連結往往跟你生活重心所在的產生了歪離的這種現象。所得稅法本身只有講境內設有住所,那他稅法上的連結是透過財政部的解釋函令去連結到戶籍,到目前為止沒有像釋字415號被挑戰。但你問我說,如果撇除掉這一層的關係,其實毋寧他的理解是說這是你的生活重心所在,當你的生活重心在這裡,你的經濟重心在這裡,這個時候就產生了稅務居民的正當的連結,這也是在全球範圍內裡面原則上是一個比較符合屬人主義的連結的標準。正是因為如此,所以到目前為止,司法實務上似乎到目前並沒有挑戰他的正當性跟他的合法性,也因為這樣一個對稅務居民的這樣一個理解是如此,所以第二個要件7條二項第二款的要件,一個更簡單的連結,叫做歷年超過一半以上,往往在實務上這個一半以上,如果是對外籍移工,那大概就會是用這個標準。那如果你是設有戶籍,那麼只要超過一天以上,生活經濟重心在中華民國境內就被認為是稅籍居民。

最後面,我們補充一下,在疫情那3年期間,就有人連一天都回不來。他產生了一種反向的問題的效果,就是因為疫情期間連一天都沒有辦法回來的人,甚至因此喪失我們的稅籍居民的這樣子的一個認定的標準。你會想說,嗯?喪失稅籍居民有什麼差?有。因為稅籍居民才可以做結算申報做相關的申報的那些所得稅法第14跟17條的扣除。非稅籍居民,就直接就源扣繳,用毛額的方式去做扣繳,所以對這些人來講,在疫情2020到2022之間,反而產生一種反向的問題。本來他是說,你只要進入中華民國,今天一天,你在這裡面社會保險,這個全民健保在中華民國境內,這樣,你就是我們的稅籍居民。可是剛好在疫情那一段時間裡面,可能有人3年內都沒辦法回到臺灣,在這種情況底下,財政部曾經有寬認解釋,但沒有行諸於於文字,而是個案認定。如果你因為疫情緣故,所以3年都沒辦法回到臺灣的話,則寬認解釋,從寬認定,可以認定為我們的社區居民,可以透過結算申報程序去申報他的所得稅,因為有許多人都是透過網路申報,並不妨害他作為稅籍居民的身份。這個這是我們實務操作,有許多是我個人認為是合情合理,但是法律依然規範密度極低。他從來都沒有名字,但我們的操作實務在這個限度範圍內這樣操作。我個人跟盛老師在私底下討論過,但沒有寫在文章上面,但我可以以德國作為一個例子來跟各位,就是說,這個標準其實是對的,這個是合理的一個標準。

今天關於所得稅的稅主體的部分就跟各位談到這裡。

\hypertarget{section-7}{%
\chapter{20230925\_01}\label{section-7}}

\begin{longtable}[]{@{}l@{}}
\toprule()
\endhead
課程:1121所得稅法一 \\
日期:2023/09/25 \\
周次:4 \\
節次:1 \\
\bottomrule()
\end{longtable}

\hypertarget{ux7a05ux7c4dux570bux7c4dux6236ux7c4d}{%
\section{【稅籍、國籍、戶籍】}\label{ux7a05ux7c4dux570bux7c4dux6236ux7c4d}}

接著上個禮拜跟各位提到的關於稅籍居民的或者是境內居住者,因為我們的法條其實並沒有把他當作是一個專有名詞。儘管在所得稅法第7條的第二項規定好,他有提到本法稱中華民國境內居住之個人,境內居住之個人或境內居住者這個名詞,但其實,包括第2條第一項本身那個地方也就只有寫個人而已,你如果用所得稅法的條款下去查詢,其實根本除了第7條第二項自己有說境內居住者這個名詞以外,基本上我們沒有把他當作一個專有名詞來對待,來做定義的規定,所以變成是第7條第二項,除了這個地方有提到以外,我們就沒有一個統一的說法。

這也是我們國家在相關法律制訂上一個蠻大的問題點。我們沒有辦法做一個簡單的專有名詞的描述,你大概就可以知道我們在講他是所謂的稅籍居民或者稅籍自然人的這個概念。德國有,德國,他們在所得稅法裡面後面就括弧就寫,無限制納稅義務人,也就是稅籍居民的概念。所以德國人,他們比較在立法的技術上面來講,相對我們會有更好一點,因為這個之後,他只要講到同一個概念,就用無限制納稅義務人這個名詞去說明就好,但臺灣就沒有辦法這樣子。這個變成是大家相約成俗,因為他不是一個法律的強行定義的一個名詞,這是第一個。

然後第二個稅籍不等同於國籍跟戶籍,這我們上個禮拜跟各位講。稅籍,原則上是依照個別稅法的規定去判斷,他跟我國的課稅主權是有比較密切的聯繫上的關係。根據7條二項第一款跟第二款的規定,原則上是以住所跟經常居住在中華民國境內。「住所跟經常居住在中華民國境內」,其實法律規範文字並不是很清楚,而且這個也常常會跟我們特別再講住所那個概念會混在一起。實務上,因此就透過一個所謂的中華民國境內居住個人的認定標準,在法規範的層次上,他顯然就不屬於法律的層次,因為法律本身沒有這樣寫,所謂的住所或經常居住在中華民國境內。因此你如果很強調依法課稅原則,那我們很明白的告訴你,他根本就沒有法律位階的規範,他甚至連授權規定也沒有。所以他變成是一個在立法技術上是一個透過行政規則的方式去解釋,所得稅法第7條第二項所講的境內居住者。這個時候無論如何,他一定會產生一個被認為他是不是增加,對稅籍居民這個重要基本概念的一個不必要的法律之外的限制,或者是對他去做進一步的具體化。那當然其實到目前為止,我們的司法實務也沒有回過頭去挑戰這樣子的一個稅籍居民的認定標準,也就是剛剛我們所在上個禮拜跟各位提到的關於中華民國境內居住之個人認定原則,這個的行政規則的,合法性跟正當性。

但我在上個禮拜也跟各位去提到我們的稅籍居民的概念,其實建構在,第一個,戶籍登記在哪裡。這個是從我們的這個境內居住者的這個認定原則出來的。在那個地方,我們進一步把稅籍這個概念跟戶籍連結在一起。也就是說,依照認定原則的標準,他必須在中華民國境內設有戶籍者,就符合他7條第二項第一款裡面所講的「境內有住所」這個概念。但任何學法律的人都注意到,我們的住所(民法有住所概念),他不等同於戶籍裡面所講的設有戶籍。因此這件事情如果你去講他的正當性的時候,他一定會必然帶來一個困難,就是行政機關透過你的認定原則,說他必須要戶籍,憑什麼?

這個地方,我們上個禮拜跟各位講過,釋字415號,就是用這個觀點來跟你講,所謂的住所,民法上所講的住所,原則上是以事實上有這樣一個共同居住的生活上的事實,然後他主觀上有長期居住的意思,我們用民法13條的概念去理解所謂的住所概念,然後他也是透過民法的概念,去否定了當時候在所得稅法裡面,當時所得稅法施行細則21-2的規定,把所得稅法17條所講的同居一家的親屬限定在必須設戶籍統一這個要件。釋字415號就說,這個設置是增加了法律所無之限制。如果用同樣的標準,你很快就可以套用到我們現在所講的個人認定原則,你憑什麼在這個地方加上一個其實在所得稅法自己本身並沒有這樣子的所謂設有戶籍的規定的要件。我們所講的戶籍並不直接認為他就是住所。

可是釋字558號,他卻也同樣說,哎,如果你今天設戶籍的話,其實並不妨害我們透過此來推定你有在當地設有住所的意思。558號的這個偶然出現的一個文字,其實是在他的理由裡面。但其實大多數的法律人是沒有在實務操作上真的這樣認為。因為我們學法律人很多很清楚的知道戶籍是一種行政管理措施,可能是作為認定你的就學、教育相關事項的標準。你戶籍設哪裡,原則上就近就學。或者是社會救助裡面的,你今天是不是屬於該地方自治團體的這個社會救助之對象,通常會以戶籍來做為一個認定標準。戶籍跟住所之間的關聯性,因此在法律上面來講,他不會有產生直接上的必然存在的連結。釋字558號這個偶然的這樣一個說明,其實在法律的強制上來講,並沒有太大真正被嚴肅以對,認為他真的會在實務操作上,就以登記的戶籍作為住所的一個判斷的一個標準。我們在法學界裡面的共識是如此。

回過頭來,還是跟各位談,認定原則,這個標準裡面,他其實除了設定戶籍,至少要有一定的天數,這個居住,顯然又比所得稅法第7條第二項第二款的那個183天的標準有更為寬鬆。因為他只要求31天。如果是1到31天內,只要具有生活跟經濟重心在我國境內,這個時候他就已經符合該當這個境內居住者認定標準。坦白說,我個人也不太明白為什麼是用31天這個作為區別標準。我去請問那個財政部賦稅署了,也沒有人告訴我一個比較明確的答案,為什麼是31天?我們法律講183天,歷年的一半,這個很明確,大概他就是跟你講,你如果在我國境內停留歷年的超過一半以上,可以推認你大概應該就是中華民國境內生活的稅籍的居民。這個概念可以理解。但為什么透過一個戶籍的設定,因此就變成用31天作為基準,如果是1到30天,則要加上經濟跟生活重心。目前為止,我也沒有辦法給各位很明確的答案說,這個正當性到底在哪裡。

但回過頭來講,生活跟經濟重心這件事情,倒是真的有他的正當性。理由在於大多數歐陸法系的國家在認定稅籍居民這個概念的時候,他基本上的連結點就是,你經常生活的地方,那個地方就具有課稅管轄的正當性。歐陸法系,包括像德國、愛爾蘭、法國,原則上你在當地生活,那你就讓當地國家對你這個人因此所取得的所得,享有一個屬人為中心的課稅權限。歐陸法系確實是以住所地為中心,因此生活重心所在,這個是有正當性的。舒馬克這個案例,1995年歐洲法院的判決說,這個人雖然他並沒有在德國境內生活,他是一個邊境的穿梭工作者,他住比利時,但是每一週都跑到德國來上班,因此他的主要所得來源是來自於德國,但他卻沒有辦法享受跟德國人一樣的,可以在申報所得的時候,透過結算申報制度去扣除掉他維繫家庭跟個人生存需求的這些額度。也就是德國所得稅法上面所提供給個人,作為屬人稅的各項減除的項目,這個舒馬克一個都不能用,是因為他並不是德國的稅籍居民,因為生活重心不在德國境內。但他的所得基本上都是來自於德國。

所以1995年的這個歐盟法院的判決說,這個是對歐盟境內的外國人產生一個歧視,也就是不利於跨境工作的歐盟的居民,從而宣告德國當時的所得稅法是違憲的。你並沒有給他機會,讓他可以適用你德國所得稅法裡面的規定,讓他可以去做相關的生活費用的扣除。所以1996年開始的德國的所得稅法,就讓,你如果生活不在德國境內,但是你的所得到達一定比例以上,譬如說以立法者設定一個標準,九成都來自於德國境內,用我們的話講,叫做經濟重心在德國,這個時候讓你可以向德國的稅務機關去申請,成為德國的稅籍居民。所得表彰出經濟上的負擔能力,而連結上稅捐負擔能力,因此用經濟重心來做為一個判斷標準,這個是有正當性基礎的。

從而不管是從生活和經濟重心這個角度來看,財政部這個認定原則在某程度上來講,他還真的有一定的正當性,也許也正因為如此,目前為止,在司法實務上,他並沒有遭受到所謂的「違法依法課稅原則,增加法律所無之限制」,這樣的挑戰的原因所在。但我們還是必須要跟各位講,如果可以依據重要性理論,其實他是最捐主體,是稅捐構成要件裡面非常重要的構成要件的要素,應該要立法者自己來明文規定。儘管我們認為生活經濟中心確實是一個正當連結稅籍居民跟課稅主權國的一個正當的連結的基礎,但你透過行政規則的方式,依然是法治國家裡面不妥當,甚至會有違反依法課稅原則疑慮的一個做法。

\hypertarget{ux907aux8d08ux7a05ux6cd5ux7684ux7a05ux7c4d}{%
\section{【遺贈稅法的稅籍】}\label{ux907aux8d08ux7a05ux6cd5ux7684ux7a05ux7c4d}}

今天再跟各位談另外一個層次的問題。

我們在上個禮拜跟各位談到稅籍不等同於國籍,稅籍不等同於戶籍。首先稅籍不等同於國籍,這件事情是在歐陸法系的概念,因為國籍是依照國籍法,稅籍則是依照稅法的規定而去做,跟屬地、跟所屬國家的之間的課稅主權的連結的這個聯繫因素。

但因為主權是獨立的,某些國家確實會把國籍跟稅籍透過他們的立法權限,把國籍跟稅籍連結在一起。

譬如說美國,美國就是一個很典型的例子,就是國籍是他們的稅籍基礎。也就是說,你是美國人,拿到美國國籍,不管你多久不生活在美國境內都沒有關係,你是拿美國護照的,就是美國的稅籍居民。所以IRC,也就是他們的內地稅法,原則上連結的聯繫因素的標準,是以國籍作為稅籍居民的認定標準。而且跟美國簽署雙邊租稅協定的時候,美國人也會希望透過雙邊租稅協定,把他們的國籍作為稅籍認定標準,推廣到全世界去。假設臺灣將來未來跟美國要去簽雙邊租稅協定的話,那未來可能美國他就會用這個標準來認定所謂的美國稅籍居民。那當然美國稅籍居民不是只有用國籍,他也用拿美國綠卡,輔助以住所地來做為一個認定標準。這個是美國對稅籍國籍之間的連結的觀點,可能會產生跟歐陸法系不太一樣的觀點。

那實際上這個觀點也影響到了我們在遺產贈與稅法裡面的稅籍居民的觀點。各位可以做一個對照,在我們的遺產贈與稅法裡面所謂的稅籍居民的概念。我們遺產贈與稅法,跟所得稅法一樣,也沒有對稅捐主體做一個比較完整而定統一的稱呼。各位可以翻開遺產贈與稅法的第1條的第一項規定:「凡經常居住中華民國境內之中華民國國民死亡時遺有財產者」。

我們的遺產稅法,聯繫了兩個因素,一個叫經常居住我國境內,一個是我國的國民。換言之,在遺產贈與稅法裡面所採取的稅籍居民的認定,其中的的聯繫因素,一方面有經常居住這個構成要件,另外一方面又連接上國籍,也就是我國國民才會有遺產贈與稅法的第1條第一項,所謂無限制納稅義務人,或者是稅局居民的適用範圍。是否稅籍居民,最大的區別實益就在我國的稅籍居民,要就全球範圍內之遺產去課徵遺產稅。相反的,如果你是第1條二項的非稅籍居民,經常居住中華民國境外之中華民國國民及非中華民國國民,這個就是非稅籍居民的概念。你不是經常居住在我國境內的中華民國國民(也就是你是國籍上的國民,但你沒有經常居住),以及你如果非是中華民國國民的話,就是非稅籍居民。

換言之,這兩個聯繫因素加總起來以後,如果你是非稅籍居民,只有在你在中華民國境內有遺產的時候,才會去課徵我們的遺產稅。也就是在我們稅籍居民這個概念上面來講,臺灣的立法表現出他的價值上的未必完全一致。所謂的未必完全一致,我當然很希望說他是相互矛盾,但我如果先中性講,就是沒有完全一致。

而且,所謂的未必完全一致,還表現在所謂的經常居住這個概念。遺產贈與稅法的第4條第三項的規定:本法稱經常居住中華民國境內係指被繼承人或贈與人有下列情形之一,第一款,贈與行為或是死亡事實發生前兩年內在中華民國境內有住所,第二款的規定是,發生前兩年內在中華民國境內居留時間合計逾365天。

你把遺產贈與稅法第4條第三項的稅籍居民的構成要件,跟所得稅法第7條第二項的構成要件對照來看。都是在講經常居住中華民國境內,結果你就發現了所得稅法7條二項的經常居住,原則上是以一個歷年去做判斷標準。不管你是從認定原則,還是是從7條二項第二款,是用歷年的一半。到了遺產贈與稅法裡面的經常居住的認定標準,是用兩年的一半,也就是超過365天。請問一件事情,為什麼立法者要做這樣一個不同的價值上的決定? 這個是我到目前為止,我無法理解的。你今天去連結沒關係,你的連結這個是立法的形成空間,我可以理解,但我不能理解的就是為什么是同樣的,跟財產、所得有關的課稅,可是我們在連結要素上面,一個一下子是連結所謂的經常居住,也就是生活重心所在地,一下子,再加上所謂的國籍,作為認定標準,然後就算回到生活重心所在地,所得稅法跟屬於性質上特種所得稅的遺產贈與稅也有不同的連結標準。

你告訴我為什麼會有這種?如果你不能正當化這樣一個差別對待,說實在話,這就是對於平等原則,未來如果有可能做違憲審查的一個非常重要的一個立論,或者是討論空間。一切都交給所謂的立法形成空間,並不足以正當化說明,為什麼在這個地方是兩年的一半跟一年的一半就這個差別? 這個是一個法律立法價值上的不相一致不協調的狀態。如果你沒有透過對照,你完全不能夠理解為什麼這個地方產生矛盾。

所以一個好的內在體系會幫助各位在法律解釋適用,甚至是規範違憲審查時,有一個比較好的清楚的視角去切入。不然的話,你永遠不會知道為什麼這個地方會產生矛盾,為什麼我們稅法會這麼支離破碎。因為下一次你告訴我境內居住,我們沒有統一法條規定,我們只能看個別法律規範,個別法律規範裡面訂出來的標準又不太一致。

我們是一個整體的不清楚的狀態。就是我們沒有辦法給你各位一個統一的訊息。這也是稅法,講白講好聽一點,叫難學講難聽一點就是操縱而無正當理由的區別對待。很簡單的一個恣意、違法,也就是立法者在這個地方恣意地去形成這樣一個所謂的立法價值決定。不要每一次都說你立法者可以怎麼做,可以怎麼做,這個地方你統一一個標準嘛。你要連結住所,我沒意見啊,你要apply一個所謂的經濟重心所在,很好啊,很棒啊。原則上你只要法律明文規定清楚就好了啦,但一下子又連結到國籍,好OK好那你要國籍,我也沒意見啊,你在住所跟國籍,在這樣的地方,你只要法律遺產贈與稅法跟所得稅法一致就好。

德國人就是他因為一致,所以他可以全部放到德國的稅捐通則直接作統一規定,因此他就不需要在所得稅法跟遺產稅稅法裡面,再重複做不一樣的規定。臺灣則是各法管各治。老師以前有指導一個在財政部工作的學生,他就跟我說,因為他們各稅管各稅。什麼叫各稅款各稅?所得稅的管所得稅的,遺產贈與稅的管遺產贈與稅,所以送出去的法案,大家都標準不一。噢,你怎麼不內部協調一下? 內部協調不再要聽誰的。每一個法條全部都是透過立法,那你如果尊重立法的價值選擇,當然就沒有違憲的問題,但問題都是立法者,也不能只是單純地告訴我為什麼在這個地方,一個是兩年的一半,一個是一年的一半的這個標準。

\hypertarget{ux7a05ux7c4dux9023ux7d50ux6236ux7c4d}{%
\section{【稅籍連結戶籍】}\label{ux7a05ux7c4dux9023ux7d50ux6236ux7c4d}}

回過頭來,稅籍跟戶籍有沒有連結?一樣有,稅籍跟戶籍的連結,毋寧是目前財稅行政實務最基本的觀念。除了剛剛講認定原則以外,我們接下來再來談幾個法律層次的規定,他真的是直接把稅籍連結到戶籍。

首先第一個就是兩岸人民關係條例,正式名稱是臺灣地區與大陸地區人民關係條例。你可能不會特別去注意,因為講稅法怎麼會跟兩岸人民關係條例有關?有,很抱歉。關於關於兩岸人民關係條例,特別是你是我們設戶籍在中華民國臺灣地區的,也就是在本地設戶籍的,原則上根據兩岸人民關係條例的第24條第一項第一句的規定:臺灣地區人民法人團體或其他機構有大陸地區來源所得者,應併同臺灣地區來源所得課徵所得稅。你只要是臺灣地區人民的話,不管你取得的是臺灣地區的來源所得,或者大陸地區來源所得,根據24條第一項第一句的規定,就要合併計算課徵我們的綜合所得稅的所得稅。什麼是所謂的臺灣地區的人民呢?根據兩岸人民關係條例第2條第三款的規定。臺灣地區人民係指在臺灣地區設有戶籍之人民。這就是一個以戶籍連結稅籍的規定,也就是兩岸人民關係條例的2條三款加24條第一項第一句的規定。

這個條文在立法結構上,以特別法優先一般法的適用的方式,規範的兩岸裡面飛來飛去的這些臺灣籍的,特別是設戶籍在臺灣的,那么他就要兩邊的所得合併計算中華民國課徵所得稅。

同樣的一個情形,看在香港澳門關係條例第28條的規定。港澳關係條例28條第一項的規定,臺灣地區人民有香港或澳門來源所得者,其香港或澳門來源所得免納所得稅。所以,我們現在從法規範的層次來看。一個經常飛兩岸三地(我們講兩岸三地就是臺灣人,飛香港和中國大陸),我們是這樣的一個課稅的結構:

第一個,你拿到的是全部臺灣地區來源所得啊,當然這個是課徵所得稅法上的所得,適用的就是所得稅法第2條第一項,因為你拿到的就是臺灣地區的居民,臺灣地區來源的所得,這個在所得稅法裡面,第2條底下就是中華民國來源所。當你的戶籍登記是在臺灣的,你也是同時在兩岸人民關係條例裡面所稱的臺灣地區人民,因此根據兩岸人民關係條例第24條第一項第一句的規定,臺灣跟大陸地區來源所得全部都課所得稅所得。所以兩岸人民關係條例第24條第一項第一句的規定,再加上所得稅法的規定,構成了如果一個臺籍幹部,在兩岸都有所得,那么對他課稅的稅法的法律上的基礎,就是兩岸人民關係條例再加所得稅法。兩個都併計臺灣的所得稅。

但如果這中間又有一個香港來源所得,那也因此香港來源所得就依據港澳條例的第28條的第一項規定,則免納所得稅。因此,一個臺籍幹部,假設三個地方都有所得,根據我們的所得稅法的規範,結構是所得稅法適用臺灣地區,兩岸人民關係條例適用大陸地區,把他合併計算,如果是香港澳門地區來源所得,那我們就再用港澳關係條例28條的第一項規定,把排斥在境內所得的範圍。

\hypertarget{ux6240ux5f97ux57faux672cux7a05ux984dux689dux4f8b}{%
\subsection{【所得基本稅額條例】}\label{ux6240ux5f97ux57faux672cux7a05ux984dux689dux4f8b}}

接下來我們再加上所得基本稅額條例,因為看法條不能只看單一一個法條。所得基本稅額條例第12條的第一項第一款的規定:個人的基本所得額為依所得稅法規定計算之,綜合所得淨額加計下列各款金額之合計數。第一款的規定裡面就是未計入綜合所得總額之非中華民國來源所得,依港澳關係條例28條第一項規定,免納所得稅之所得。這個地方後面有一個但書,有一個門檻條款的規定就是,但必須是超過100萬的才計入,100萬以內的話,有一個起徵點的規定,這個就是100萬門檻。原則上低於百萬,立法者認為,這個課稅所得我看不見。這個叫起徵點的規定。

法條的規定裡面,這個叫起徵點。不太適當,把他稱之為叫免稅額,我們的實務都是用免稅額概念。但我要給各位一個好的習慣。免稅額這個字眼比較適當的用法,是在反映維繫生存所需額度的概念。起征點這個概念則是另外一種概念,原則上他是數額太低,考慮到稅捐稽徵機關的稽徵成本,考慮到納稅義務人的依從成本。因為所有的稅負遵循都是有成本的,這個是一個很簡單的概念,就所謂的成本會計的概念。納稅人方,因為你要照法律規定你要申報,你要拿憑證,這個就叫做依從成本。納稅捐稽徵機關是稽徵成本?

由於數額太低,叫你來做這麼多準備,我課到稅也不多,就設一個起徵點。這個叫做實用性原則。這一種門檻的設定,我們稱之為叫起徵額。法律重在精確表達背後的憲法意涵。

所謂的免稅額的概念最常用的所得稅法,因為我們之後會跟各位講這個概念。所得稅裡面的免稅額,表彰出稅法對生存權保障憲法生存權保障,因為人維繫生存,我賺錢,不是為了要繳稅,是為了要先養活我自己,除了我自己以外,我還有親愛的配偶跟受扶養的家庭的親屬,我要先養活他們,我才有餘力,我才繳稅給國家。所以捐負擔能力的起點是在我剪掉主觀的維繫生存需求,這一些家庭成員都是我的家庭的一環,那這個時候我要把他扣掉,以後我有能力稅捐負擔能力的起點,在這個時候才開始,這個時候才是計算稅捐負擔能力的起點。在這樣一個前提底下,我們所提到的那個概念叫免稅額。

當然,我必須要承認,我們的法秩序裡面沒有很嚴格的去區別這兩個不同的概念,一個是維繫生存的免稅額,一個是所謂的純粹基於實用性原則,也就是稽徵上面考量到納稅人的依從跟稅捐機關的稽徵成本,當這個成本如果大於他的稅捐收入的話,那還不如不要徵,因為對社會整體實益是不高的。因為稅捐機關要查帳,納稅人要遵循他要記帳保存憑證,都是成本費用,那這個我們把稱之為叫起徵點。

因此我們到之後講營利事業所得稅的時候,法條規定叫免稅額,實際上他是起徵點概念。因為法人沒有維繫生存的問題。法人本來就是一個人的集合體,透過錢產生出來的集合體。自然人才有維繫生存的需求。所以維繫生存這件事情不能推導出來,營利事業有所謂的賺取盈餘後除掉主觀淨值的問題,營利事業從來不存在這個問題。但營利事業如果賺太少的話,確實會有稽徵成本跟依從成本的考量,所以我們在所得稅法裡面,在營利事業所得稅會有一個稽徵點一年12萬的規定。這個叫起徵點,法制上不太適宜用一個相互混淆的免稅額這個概念去混在一起去談。

同樣一個道理,來到個人所得基本稅額條例的基本稅額的計算,這裡面有一個100萬免稅額的規定,實務都說是免稅,實際上他應該正確的表彰為起徵額。因為在這個地方不涉及到個人維繫生存需求的額度,他只是一個涉及到境外的所得查核保存憑證,有一定程度上的困難,這個是一個實用性原則的表現,而不是維繫生存的扣除。

所得基本稅額條例12條第一項第一款的境外所得,原則上要加計進來的,包括了未計入綜合所得總額之非中華民國來源所得。譬如說你在美國跟加拿大或日本所取得的所得,這個叫非中華民國來源所得,以及依港澳條例28條第一項規定,免納所得稅的所得。所以我們現在透過這幾個條文拼湊出來的圖像,就給各位一個簡單的整理。一個臺灣的稅籍居民,原則上是設戶籍在臺灣的人,取得臺灣來源所得,這個叫所得稅法2條一項中華民國來源所得,課綜合所得稅的所得。第二個,如果他同時取得大陸地區來源所得,這個地方在法律適用上是兩岸人民關係條例,再加上所得稅法上的規定,這個時候會並計入臺灣的綜合所得稅裡面,去課徵綜合所得稅的所得。如果是港澳地區來源地所得,那這個時候就是由港澳關係條例的28條的規定,原則上他不計入綜合所得稅的所得,而是改依所得基本稅額條例的12條第一項第一款第一句的規定,該計入基本所得額的計算範圍。第四個,如果他取得了是美加日本的所得,那麼原則上他就直接依據所得基本稅額條例第12條第一項的第一款第一句裡面的非屬中華民國來源所得,計算所得基本稅額條例的基本稅額。這就是我們現在目前法律上對我國設籍在臺灣的稅籍居民,依據所得的不同的分佈而做課稅的規範。

同樣還是那一句話,立法者縱然有形成的空間,為什麼做這樣不同的區別對待?簡單來講。我們的稅籍居民拿到臺灣跟大陸兩地的來源所得要併計綜合所得稅所得,適用的是最高到40\%的邊際稅率。但如果你取得的是港澳來源所得,以及美加日本的來源所得,那么原則上是適用所得基本稅額條例100萬的起徵點的規定,適用的稅率原則上是20\%。為什麼要這個差別對待?是怎樣?我們取得本地的錢,比較有稅捐負擔能力嗎。請問你的除了合法性以外,你的正當性在哪?法律系學的稅法是正當性稅捐負擔分配正義的稅法,不是枝節的條文的跟數字拼湊計算的稅法,這才是法律人的稅法。當然,不同的角度去看稅法稅務的時候會有不同的觀點,因為我們除了強調實然以外,我們也強調應然、正當性。依法課稅原則是一個層次的問題,另外一個,毋寧還是稅捐負擔分配正義、量能課稅原則的遵守的問題。立法者這幾不遵守這一套標準,那請問,各位可以想一想,我們的法制,到什麼時候才可以改變這樣一個不合理的現狀?為什麼在臺灣的人賺到臺灣的錢,免稅額給你大概十幾萬,再加上薪資特別扣除額度,也許一整年,大概給你30萬左右的免稅額,但所得基本稅額條例的12條,第一項第一款的但書規定的是100萬門檻的起徵額。稅率就又更做差別對待,簡單來講,有本事在港澳跟國外賺到錢,稅捐負擔相對至少是綜所稅的一半。所以誰叫你沒辦法去賺港澳跟美加日本的所得?你只賺到臺灣跟大陸地區來源所得,很抱歉,你的稅捐負擔最高就到40\%。這就是我們現在的稅法的規範的實務的狀態。

\hypertarget{ux571fux5730ux7a05ux6cd5ux7b2c9ux689d}{%
\subsection{【土地稅法第9條】}\label{ux571fux5730ux7a05ux6cd5ux7b2c9ux689d}}

戶籍跟稅籍之間的連結也表現在土地稅法第9條的規定。土地稅法裡面對所謂的自用住宅中,所謂的居住這個概念,在土地稅法裡面是以,設戶籍的主體不限於是納稅義務人本人,也還包括了本人或配偶跟直系親屬在該地辦竣戶籍登記。因為土地稅法裡面涉及到居住的土地。這一塊,他要求的標準就是你在上面的房屋要設立戶籍登記,所以在住所這個概念的時候,在土地稅法,他是跟戶籍連結在一起。

同樣的,我們還是必須去詢問,為什麼立法者的價值依然在跳來跳去,在這裡面做一個不太能夠一致的價值上的決定。要詢問的是立法者。如果我告訴你原因,我只能跟各位講,依照老師對這個部分的了解,就是這個部分個別由財政部內不同的單位負責,提出來的標準就是不一樣。如果透過立法者之手,就取得正當性,我也敗給你,輸給你了,你這麼寬容地對待立法者,好棒棒啊。

因此,你可以可以講,我們是沒有什麼太明顯的一個區別的一個標準。必須要看各稅法的規定,才能夠去跟你講那個稅籍居民的概念。我們沒有辦法像德國的這個租稅通則,他就有一個統一的概念,他跟你講,這個住所,你在中在德國境內啊,如果你有一定的所得是來自於境內的話,你可以通過申請的方式去變成是德國的稅籍居民。我們沒有辦法用像德國人這種方式。這也是我們稅法支離破碎,難以統一學習,難以透過一個法律原理原則來共同推導的,很重要的背景的因素。

關於稅籍居民的部分,就跟各位說到這裡。

\hypertarget{ux5c0fux7d50}{%
\section{【小結】}\label{ux5c0fux7d50}}

我們先休息一下,接下來之後跟各位講稅捐客體。

每一個構成要件都是構成所得稅課稅時在判斷上的依據,所以不管是稅務律師或者是法官的判決,乃至於同學們要學依法課稅原則,依法課稅就是每個構成要件該當,這個時候就會產生他法律所規定的法律效果,這個就叫做依法課稅原則。在德國的國家考試,考稅法就是每個構成要件檢驗,檢驗完以後得到全部該當,因此就依照所得稅法的規定,你要做結算申報程序。考試就是這樣操作。因為到後來到德國的稅務財務法院,他也是一樣的,對於稅務律師提出來的書狀法官判決,每個構成要件去做檢驗而已。這一套本來就是從學校學習開始,一路進步到實務。法律的實務操作,從他在學校裡面的教學實務開始,那這個就體現出依法課稅原則。那當然正當性是一路過來,他必須要去對這個法規範背後的道理原理原則,在課堂上去解釋。在實務的操作上,則是透過一路去主張這個地方法律規範本身有違反正當性的問題,最後面當然是透過德國聯邦憲法法院去做規範的憲法審查,來做違憲上的宣告。因此他整個的法秩序的建構,或者是他的教學,他基本上能夠去配合實務上的需要。

臺灣,這件事情還需要非常非常的久。因為其實臺灣法律本身構成要件的規範結構就不是很清楚,因為他構成要件跳來跳去,而且價值又不一致,所以每一次講不同的法律,你要把那個法條展開來,然後自己依照構成要件去做解析,這就是我們在第2次上課的時候,跟各位提到應然實然的差別,因為我們的法律就長這個樣子,沒辦法,很醜,我也知道。但我也只能跟各位說,希望未來有一天能夠重新調整整個稅法的規範結構,但不太容易,因為認識的人太少,政治上的聲音也比較小。

我們先休息一下。

\hypertarget{section-8}{%
\chapter{20230925\_02}\label{section-8}}

\begin{longtable}[]{@{}l@{}}
\toprule()
\endhead
課程:1121所得稅法一 \\
日期:2023/09/25 \\
周次:4 \\
節次:2 \\
\bottomrule()
\end{longtable}

\hypertarget{ux6240ux5f97ux7a05ux7a05ux6350ux5ba2ux9ad4ux4e4bux5224ux65b7ux6d41ux7a0b}{%
\section{【所得稅稅捐客體之判斷流程】}\label{ux6240ux5f97ux7a05ux7a05ux6350ux5ba2ux9ad4ux4e4bux5224ux65b7ux6d41ux7a0b}}

稅捐構成要件的第二個要素就是稅捐客體,他在討論上面來講,其實是蠻重要的一個問題,也就是說,稅捐主體相對於稅捐客體而言啊,他當然是開啟了我們國家對這個稅籍居民課稅的可能性的一個構成要件,但如果沒有第二個構成要件,就是他取得所得,在這個地方,其實你整個課稅構成要件傷還是沒有辦法去滿足。

所謂的稅捐客體,就是你取得一個所得,然後他是應課稅所得,然後這個應課稅所得是中華民國來源所得。

我分三個層次去講。第一個,他必須是所得,第二個,是應課稅的所得,第三個,必須是中華民國來源所得。因為我們所得稅法第2條的第一項跟第二項都加上了一個中華民國來源,換言之,所得的實現地點在中華民國,才會被課我們的所得稅。所以其實在講稅捐客體的時候,其實分了好幾個層次上的問題。第一個,他必須是所得,非所得當然就不進來。我們待會要跟各位去舉幾個說明什麼是非所得。所得,當然可以被正面定義,我自己也寫過一篇文章去定義所得,討論何謂所得,這是我們今天上課的重點所在。

但第二個問題其實蠻重要,就是他必須是應稅所得。因為在我們的稅法裡面有許多稅捐優惠的免稅所得。只有是所得稅的所得,他才會進入第二個層次叫應稅還是免稅所得,這個是第二個層次的問題。

第三個。我們還要再談實現的地點,實現的地點必須要是中華民國來源。因為按照一般而言,稅籍居民的所得其實是不分是否跟屬地有所連結,也就是境內境外來源都要被課所得稅,一般而言是這樣。因為我們的所得稅法和比較特殊,他要求的稅籍居民也還是必須要是中華民國來源所得啊。

我們從第一節課,也了解到,所謂的中華民國來源,根據兩岸人民關係條例,就是你是我們的稅籍居民,原則上臺灣跟大陸兩個地方的所得都是中華民國來源所得,都是所得稅的所得,而不是課所得基本稅額條例裡面的最低稅負制。港澳則不是,港澳,依照港澳關係條例,則是課所謂的所得基本稅額條例裡面的最低稅負制,而不是課所得稅裡面的綜合所得。那這個地方因此產生稅捐負擔上的區別對待。

回到我們這個地方講的,所得的實現地點。這個地方我們必須要跟各位再來去談所得稅法第8條的規定,因為那個地方才是真正涉及到所得來源地的判斷。

那其實接下來還有一個很重要的問題,就是要分類型,所得是哪種類型?因為不同的類型的所得,在計算稅基的時候,他是有區別對待的,不同的類型所得。譬如說,執行業務所得,跟薪資所得,同樣是勞務的付出。獨立的執行業務的所得,他有一個計算成收入的所得的一個標準去,反映出他的稅捐法上的負擔待遇,但你如果是薪資所得,他有另外一套標準去做這個你的薪資所得的計算。

從而在這裡面,我們總共應該分4個層次的問題。

\hypertarget{ux6240ux5f97ux7684ux6982ux5ff5ux6f14ux8b8a}{%
\section{【所得的概念演變】}\label{ux6240ux5f97ux7684ux6982ux5ff5ux6f14ux8b8a}}

第一個問題就是所得跟非所得,要跟所得去做正面的區別,這個其實是不太容易的,因為這個要看,所得怎麼定義。但從歷史的角度來看,所得其實是有不同的定義的方法。最早的對所得定義,來自於財政學者,而這個定義的概念也形塑了最早引進所得稅的英國對所得的想法,就 income這個概念。來自於源泉而產生出來的孳息增長,增益,這個被稱之為叫所得。我們就用樹果來做描述,樹是一個所得的源泉,所得所由生的源泉,從這個上面長出來的果實掉下來以後,這個才叫所得。所得不在講那個財產本體,而在講這個源泉所產生出來的增益、孳息。這個理論就被稱之為叫源泉理論,他是最早的所得概念,從財政學界一路過來,1799年英國的所得稅法,他對於所得的定義。他的源泉來源可能是來自於勞動力投入,可能是來自於資本投入,當然有許多是勞動力加資本的投入,但無論如何是以他長出來的成果。來做為所得的 income 的概念。

那么打完拿破侖戰爭以後,英國獲勝,很快的1804年英國的所得稅法就沒有再實行,因為打完仗,財政需求就沒有了。但這個概念卻很深地影響到後面的各國的所得稅的概念。到19世紀末的德國的所得稅,在所得稅的概念裡面,也一直沿用源泉理論的看法,也就是說,要不是資本投入,要不就是勞動力投入,那至於這兩種投入長出來的果實,就變成了課所得稅的課稅的對象。這個就是所謂的源泉理論。

到了20世紀啊,德國人的講法是1921年在德國有一個財政學者叫 Georg von Schanz。

這個意思是他是一個領地的領主的後代,他們是 Schanz這個地方的地主的後代。德國的名字跟姓氏,姓氏通常是按職業或者是地名。像我有一個朋友,他叫【?】,那他是從【?】這個地方出來的地主或者是領主的後代。德國以前的那個國安部長,現在是歐盟執委會主席主席,叫馮德萊恩,Ursula von der Leyen,就是從那個地方的領地的後代啊。

我在看那個德國文獻的時候,他們就是說,這個概念其實是我們德國人先發明出來的概念,叫所謂的純資產增加說。所謂的純資產增加說是以兩個期間裡面的資產淨值做比較,全然的資產淨值的增加,包括已做的消費行為跟投資儲蓄,全部都要算回來,這個是你的所得。這個概念就叫純資產增加理論。雖然資產淨值為零,但這一段時間裡面,你所做的消費行為其實是你的凈所得的增加。這個純資產增加說的概念是用資產凈值的比較,再加上此段時間裡面所做的消費及投資儲蓄,因為你投資在外,他是用一個財產的形態產生出來的經濟成果。從而1920年代出現的純資產增加說這個概念,跟前面18世紀末19世紀的源泉理論,從一個所得源泉長出來的孳息這個概念,是全然不同的概念。

這個看法到了1930年代變成了美國的Haig跟Simons,這兩位美國財政學者的對所得的定義跟概念。你現在去看會計系的教科書,講所得都只有講Haig跟Simons,可是如果是看德國人的講法,會説,第一個提出這個所得概念的是德國的Schanz。德國的Schanz,美國的Haig跟Simons這三個人所共同認為的所得概念叫純資產淨值的增加。純資產淨值的增加,特別表現在營利事業。因為營利事業是一個數人的組合,但他沒有屬人的像你我一樣的自然人實體的存在,所以營利事業所得稅在計算所得的時候,基本上就是用純資產淨值增加說的概念。下一次上在講營利事業所得稅的時候,會跟各位在稍微簡單的去說一下他的法條規定。就是兩段時間裡面,資產淨值的做比較,然後再加上你如果有消費,你有儲蓄投資就再算回來。營利事業沒有消費,但是你拿來自用就還是要算所得,因為你的自用就構成了一個經濟成果呢的實現。

1920年代開始到1930年代,20世紀初期的所得的概念,這個叫純資產增加說。

這兩種學說看法不同,結果都表現在1789的1891年的德國的統一之後,德國統一是指普魯士3次戰爭以後統一了小德意志,排除掉奧匈帝國的奧地利以後,德國各邦連國家統一起來,

這個19世紀末的德國的所得稅法,把這兩種不同的概念,因為當時候其實對。營利事業所得要怎麼課稅,並沒有一個很統一的想法,1920年代所出現的這一個所謂的純資產增加說在19世紀末的時候,基本上還在討論的階段,但德國的立法者就把他放進去所得稅法裡面。當時的所得稅裡面的所得概念,用勞動力付出的,原則上是採取源泉理論的看法,因為他是影響最久,也是最深遠的一個對所得的概念。但當時在統一之後的德國的稅法,他所對營利事業所做的課稅所得,則採取純資產淨值說,這是當時候在財政學界裡面比較流行的看法。因此到1920年代的德國的所得稅法,就用兩種所得理論同時並存在德國的個人所得稅法裡面。

經營事業的用純資產增加說,勞動力投入的,不管你是薪資或獨立的執行業務,原則上就是採取所謂的源泉理論。

因此到了1979年的時候,到了20世紀末,1980年代開始,德國的稅法學界為了要去統一說明,這兩種不同所得理論,一個是源泉理論,一個是純資產增加說,才產生一個概念,叫做透過市場而實現的經濟成果,首次提出市場所得的概念。所以市場所得概念是1980年代提出的。而且第一個提出來的其實不是德國人,他是奧地利人,是一個奧地利的法律學者,他提到,所謂的所得是透過市場經濟活動所實現的經濟上的成果。不管你是講了源泉理論,或者是純資產增加說,其實要掌握的所得,無非就是市場經濟活動的成果。這個1980年代才出現的市場所得理論提出之後,就變成了德國稅法學界的共識,學界認為市場的概念才能適切表達出,在德國所得稅法原先立法裡面其實並存著源泉理論跟純資產增加說這兩種不同所得的概念。

相對於源泉理論是一個非常狹隘的所得理論。為什麼講到狹隘呢?因為只要果實不落下來,都沒有所得的概念,這就是源泉理論的問題。我們用樹果理論來做說明,果實,只要不落下來,你只要把樹一起賣出去,這樣就不算有所得。所以在英國的income的概念跟capital gain是有所區別的,爲income是指那個掉下來的果實,可是如果你是把果實財產本體把他賣出去,連同他上面的增值就一起賣出去,這個轉售價差,英國人的概念字眼叫做capital gain,那個是一個資本利得,不是income。如果你去問一個英國人,income,他可能理解到了,就說哦,income講的是我們相當於我們所講的孳息的概念,因為capital gain這個概念在英國是另外立法,另外課稅,不在他們的income tax裡面去做課稅。這也是不同的與法系的國家裡面對這個概念可能會產生歧義。

但各位很快就可以幫發現,如果在我們的社會裡面只對孳息課所得稅,卻不對資本的增益課所得稅的話,那麼很快就會發生,有部分的所得沒有被掌握的情況,也就是說能夠表彰出稅捐負擔能力的所得沒有被掌握。換言之,在這裡面對於源泉理論過於狹隘的這種概念,1920年代才因此產生純資產增加說。但純資產增加說最大的問題就是連維繫生存所需要的消費,以及各式各樣的投資跟儲蓄全部都要算回來。還有一個問題,就是純資產增加說,說連隱含所得,也就是你本來在家範圍內的這些消費活動,只要能夠被數量化,就應該計算所得課稅。比如說住自己的家,幫自己的小孩補習。如果我作為一個中學的老師,我在外面當補習班的老師,我可以賣錢,但我幫自己的孩子補習,我孩子因此節省費用支出,在純資產增加說裡面,也算所,也就是他要設算你節省的費用。純資產增加說無血無淚啊,為什麼?因為人就是勞務的付出,你只要賺到錢,我不管你是消費投資儲蓄,就算看到那個經濟成果,我一樣給你設算進來。所以你住自己的房子,要設算租金收入,開自己的車要設算計程車費用,然後幫自己的孩子補習,要設算你的這個節省的補習費的支出,全部都算進來。純資產增加說因此被認為對於所得的課稅範圍過廣。除了在營利事業,我們可以這樣子去計算以外,對個人綜合所得稅用這種方式去理解的話,他的範圍太過廣泛。

從而1979年,1980年代開始的市場所得理論,適切地從一個比較正面的方式去做一個比較適當的定義,也就是說,何謂所得。其實家内的經濟活動是被排除的。因為家不是市場。只有透過市場的活動,因為是一個本來是市場實現活動的經濟成果,才能被認爲是所得。

比如說你這個其實本來是市場經濟活動,你本來可以去透過你的勞務付出去賺進來你的勞務付出的所得。但是由於各種的因素,你沒有勞務上的付出,那你因此去取得替代經濟成果的那個財產上增益,這個本來是透過市場經濟活動會實現的所得,這個時候就應該要課所得稅的所得。我們舉例而言,你本來勞動付出,你會換得所得的就是你的薪資所得,可是你們現在去罷工。德國罷工,他們的工會是發起人的哦,你知道嗎?因為罷工的時候領不到薪水啊。假設雇主沒有根據法律上規定的義務,要付給你一定的薪水。那麼德國工會,因為你平常就交錢給我,所以德國工會就是在你這一段罷工期間裡面,因為我們要跟資方去做團體協商,那你領不到錢的時候,工會會發給你錢,讓你至少維繫這一段時間裡面的經濟上需求,不然罷工幹不下去的。大家要罷工,不是今天上街頭而已,他有這段時間裡面的維繫生存的經濟上的需求。德國在工會的城市裡面就是對參與罷工的人,他提供一定的經濟上的援助,讓他可以去實踐這一種團體協商裡面每一個工人都可以領得到一定罷工金。因此,罷工金在某程度上是替代經濟成果的一種另類的所得。從而在這個角度來講,替代經濟成果的所得是透過市場,也可以去計算出市場所得的範圍。

這個市場所得的概念的提出,在1980年代之後變成的是德國法學界裡面的共識,儘管同一個時間,經濟學界還是比較傾向於淨資產所得的理論。其實講市場這個概念,經濟學者一定不陌生,但市場所得這個概念又會碰到一個困難。到了2000年的時候碰到一個困難,那個困難就是非傳統上被認為是市場概念的這一些非市場,比如說,成為公務員或者是選舉市長或者議員,或是成為立法委員國會議員。那麼,這一些非傳統上被認為是市場概念的範圍,但是在這一種非市場裡面的勞動力付出,因此所取得國家不管是以賠償的名義,或者是以薪資的名義所給予的,或是給予報酬的這個部分非傳統上市場的概念。那請問,這個部分沒有進入市場,算不算所得? 簡單來講,就是總統市長立法委員民意代表領的錢,這個算不算是市場經濟活動底下要掌握的所得範圍?在2000年以後,我自己的論文指導的老師,他就說這個地方應該要把所謂的市場擴張到所謂的以取利益、獲取利潤作為前提而參與市場跟非市場經濟活動,也就是這一些傳統上被認為不在市場活動範圍內裡面的,特別是公領域裡面的,包括這些性質上被認爲其實是參政權的行使的領域。很好玩的是,在德國的公法領域裡面,透過選舉成為民意代表跟成為公務員,他都是一種參政權行使的方式,不是市場經濟活動的範圍,但是他們所因此獲得的報酬,德國認為這個是這一段時間裡面國家提供給你的補償,非市場經濟活動的所得,但這個仍然是屬於在德國概念底下,是透過去獲取收益的方式而取得的所得。

所以為了要包括市場跟非市場,在2000年之後,德國看法,至少在科隆,是認為,所得的概念是以收入超過成本費用而投入市場,不管你是通過職業或者是營業上的活動,也包括了你在非市場領域裡面,可能傳統上被認為是高權領域裡面的參政行為,因此所取得的經濟上的這些成果都被認為是所得的概念。這個概念被稱之為叫營利所得的概念,以獲得收入大過於這個成本費用盈餘作為他的主觀上的構成要件而去透過他的經濟活動的參與行為而獲得的經濟成果。

我們透過一個簡單敘述,介紹關於所得的概念的整個的進展,從源泉理論18世紀末期到19世紀,到20世紀初期的純資產增加說,到20世紀的後半葉所提出來的市場所得理論,到21世紀,我們所提到的營利所得的概念。所得的概念依然還在持續不斷地演變,如何去做適切的掌握這一件事情並沒有改變。

\hypertarget{ux6211ux570bux7684ux6240ux5f97ux6982ux5ff5}{%
\section{【我國的所得概念】}\label{ux6211ux570bux7684ux6240ux5f97ux6982ux5ff5}}

那么,以我們自己的所得稅法,那我們到底是採哪一種概念?這個才是我們今天要跟各位關注的重點。理論,有各種不同的說明。當然這種不同的理論,源泉理論太狹隘了,他顯然不能掌握capital gain,可是純資產增加說,因為太過寬廣,他連隱含所得的這種自家內的消費行為也仍然被課稅,市場所得理論是一個比較適切的,能夠去掌握經濟成果的理論,但又要包含了傳統上被非稱之為叫市場裡面的,特別是在高權領域裡面的參政權行為所產生出來的勞務上的付出,因此所獲得的對價,這個依然還是所得稅要掌握的所得範圍。沒有道理市長總統民意代表所取得的報酬,就不是課所得稅的所得,公務員取得的所得不是所得稅的所得,只有民營機構,他的所得才是所得。他是市場經濟活動,他才是所得,這個沒有道理。應該公務員的薪水跟民營機構裡面的職員的薪水都是一樣的,都應該要課稅。這個概念叫營利所得的概念。

我們來看一下我們的所得稅法的規定。所得稅法的第14條並沒有正面定義所得。就像稅一樣,所得稅法的第14條,我們從頭到尾,沒有定義。所得的概念,我們也似乎是理所當然,以這十種類型來表彰出立法者認為應課稅的所得,所以總共有十類。在這個地方,看起來,我們並不限於源泉,理由在於,第七類,我們的財產交易所得是被包括在所得稅的類型當中。因為有第七類,所以你很快知道我們的所得稅不是源泉理論。那我們是採什麼呢?看起來我們有第七類的財產交易所得,第十類我們有一個其他所得:以上只要不屬於特定類型的話,則以其收入而減除成本及必要費用後之餘額為所得額。

還有另外一個,就是第八類,競技競賽及機會中獎。什麼叫機會中獎?講難聽叫賭博,就是你去買樂透這一種機會中獎機會,中獎的所得,一時性所獲得的這一個財產上的增益,在我們的第八類裡面,也是所得稅的所得類型。雖然在實務課稅上面,其實機會中獎,像剛剛講的樂透或是以前的愛國獎券,或是現在的統一發票給獎,我們都把機會中獎單獨拉出來,來做分離課稅。第八類,雖然把他放在綜合所得稅裡面,當作一種類型,但實際實務上面,我們則幾乎各類的機會中獎都有各類機會中獎自己本身的法規範,拉出去做分離課稅。從財產交易所得的七類或第八類,以及第十類的規範結構來看,我們的立法者原則上是建構在以跟營利事業所得稅一樣的前提基礎上的純資產增加說的看法。也就是說,我們對一時性跟繼續性,沒有區別。錢就是錢,就算你今天是路上撿到的,你沒有歸還,也是錢?頂多只是不算入前面的九種類型,是第十類,其他所得。包括賭博收入。賭博收入最典型的第八類的機會中獎,賭博的收入也是所得。所以,在這裡面,不管你是一時性繼續性合法性非法性都一樣,你只要是錢,表彰出來的稅捐負擔能一樣。

14條沒有做正面定義,但透過類型的適用,可以看到,接近純資產增加說。這也是老師之前在寫一篇文章,關於所得理論,在講的就是這件事情。我們的法律,在立法上面是接近純資產增加說。雖然我個人也認為市場所得理論比較合理,因為純資產增加說,他很明顯的問題就在於範圍過於廣泛,而且他其實是有些進來,有些沒進來,像我們剛剛跟各位提到的隱含所得。自家的居住跟自家的使用勞務,這個在我們的現行法制裡面都不算任何一類所得,也不是第十類的其他所得。有些國家是真的採純資產增加說,他是真的會設算利息跟設算租金,去算你的所得,加進去。像比利時就有這個規定。你今天去租別人家房子,當然有租金的支付,住自己的家的話也要再加進來,設算你本來的消費,把他計算入你的所得,來做這樣一個稅捐負擔能力的掌握。但我們並沒有,我們雖然看起來比較接近純資產增加說,可是其實在我們實務的操作上面並沒有把各種隱含所得,包括你的各式各樣的消費行為都把他包攝進去。這也是我們在實務上面來講,看似純資產增加,實際上比較偏向市場所得理論。我個人也比較認同市場所得理論,儘管他很晚才出來,到1980年代才在德國的法學文獻上面,認為這才是德國所得稅裡面所要想要掌握的所得。到了2000年,雖然有做微調的營利所得的概念的修正,但基本架構,原則上就是經濟成果,不管你是不是市場經濟活動還是非市場,你有勞動力投入或者資本投入,或這兩者相加的投入的經濟成果,才能夠被認為是所得,這個概念是不變的。我個人也比較傾向于用市場所得和營利所得的概念去解讀我們現在目前的所得稅法,儘管他是以純資產增加說的立法結構去表現出來的所得的類型。這是關於所得的概念。

\hypertarget{ux662fux5426ux6240ux5f97ux4e4bux5224ux65b7}{%
\section{【是否所得之判斷】}\label{ux662fux5426ux6240ux5f97ux4e4bux5224ux65b7}}

在這樣一個前提底下,以下這些都不算所得。

第一個損害賠償。損害賠償無論是從哪種所得理論來看,都不是所得。特別是填補固有利益損害之損害賠償,原則上都不是所得。因為填補固有利益損害之損害賠償是因爲固有利益之存續,固有利益的固有狀態的被破壞,於是加害人對被害人添補固有利益,因此所產生出來的賠償。在這個地方,不管是源泉、純資產增加說,都不認爲這是所得。如果以所純資產增加說,他有沒有財產禁止的增加呢?答案是沒有。所以你撞斷一條腿,人家賠100萬,小心被人家打死,賠500萬,在這裡面都沒有資產淨值增加,所以這個數額,不管怎麼給付,原則上都不是所得。不是只有人身固有利益,財產固有利益之損害賠償一樣,沒有所得的概念。財產固有利益的損害賠償,譬如說你今天車子停在路邊,被人家不小心開車撞上去,賠你10萬塊錢,儘管可能看起來哎,明明就不需要賠那麼多。原則上我們在概念上叫固有利益之損害賠償,不構成所得。

相反的,替代履行利益之損害賠償,就有可能構成了所謂經濟成果替代的損害賠償,他就有可能會是所得稅的所得。看履行利益的損害賠償。何謂履行利益?我們各位在學民法裡面的履行利益,特別是預期透過契約之履行,可以獲得的這一個經濟上的成果,譬如說轉售價差。履行利益的損害賠償是一個因為契約無法照預期履行而可歸責於債務人之事由而產生出來的損害賠償。從而履行利益之損害賠償,原則上,特別是在給予預期可以獲得的這樣一個經濟成果的替代,這個部分,基本上就是所得稅的所得。困難就出現在我們的履行利益的損害賠償裡面有包括了,所謂的關於加害給付而侵害了其原先固有利益財產狀態的,或者人身固有利益的這種損害賠償。也就是在加害給付積極侵害債權的類型裡面,因為履行利益只是一個概括的概念,包括轉售價差,但也包括了因為給付而產生對債權人的人身或者是財產固有利益的損害,這一種加害給付也算在我們的履行利益的給付範圍內。嚴格而言,當然要個別區別,做不同的區別對待。屬於固有利益添補只給付不構成所得,只有經濟成果的替代才構成所得,這是稅法上對稅捐負擔能力對所得的掌握。也因此不能夠明白固有利益跟履行利益之差別,自然就無法對所得的概念、應課稅所得的概念,有辦法做正確的瞭解。其實整個法秩序就應該在這個上面去做必要的區別對待。

我們來看一下所得稅法第4條的規定。所得稅法第4條第三款,傷害或死亡之損害賠償。法條只有規定傷害或死亡之損害賠償,為什麼他免納所得?答案是因為損害賠償。因傷害或死亡之損害賠償是固有利益之損害的填補,根本非所得,所以沒有第4條第三款的前段規定,解讀上也是一樣,不是所得稅的所得,他是非所得概念,所以免納所得稅只是表彰出非所得。但是條文並沒有完全涵攝所有的固有利益損害,譬如說財產的損害的損害賠償。這個第4條第一項第三款規定,從法條文字裡面完全不包括財產固有利益的損害賠償,所以我們沒有辦法直接透過解釋的方式,讓第4條第三款的規定去適用到財產固有利益的損害賠償。財產固有利益的損害賠償,毋寧只能透過所得的概念來把他排除在所得的適用的範圍以外,也就是他不是所得稅的所得。傷害或死亡的損害賠償,就算不是透過損害賠償的方式,以和解之方式而給的傷害跟死亡的和解金,一樣,具備非所得的性質。所以個案的情形,請各位個別去做區別判斷。傷害或死亡可以透過損害賠償法去請求損害賠償,也可以透過契約法去做和解、調解出的傷害或死亡的損害賠償金,這個是固有利益的填補,而不是資產淨值的增加,他不是所得稅的所得。法條上沒有涵蓋的財產固有利益的損害填補亦如是。

相反的,屬於經濟成果替代的履行利益的損害賠償則有個屬於所得稅的所得。傷害案件,如果在概念上被認為是因為他減少勞動能力,他本來可以從雇主去獲得的那個勞務付出的報酬,經由損害賠償法制,由加害人對被害人所做的損害賠償,理論上,他可以是經濟成果的替代,可以是課徵所得稅的所得。不過在這個地方,要不要透過第4條第三款的規定,傷害或死亡的賠償金,把他全然都免納所得稅?若如此,在經濟成果替代這個範圍內,他確實就存在著稅收優惠的意義。因為這個地方當然不是讓他有中樂透的感覺,而是說因為在法律解釋上面來講,有時候對這一個減少勞動能力,特別是從雇主那個地方,本來他可以獲得的報酬,在概念上現在是由加害人,因為我加害於你,導致你無法去上班,因此可以去請求加害人對你所做的減少勞動能力的這個的損害賠償,概念上其實是經濟成本的替代,可以去做課稅,但是不容易區別。因為這個部分仍然在實務操作上面仰賴於個案裡面的關於賠償的計算,究竟是固有利益的填補,還是是經濟成果的替代,而做個別不同的區別對待。

比較明顯,而清楚的差別是在關於侵權行為是侵害無體財產的懲罰性損害賠償。比如說對於智慧財產權,盜用他人的專利商標或者是著作,在民事法院裡面所做的懲罰性損害賠償,也就是於依照預估的方式去計算這個著作在市場上發行,因此說可以獲得的經濟上的成果,以此作為前提,而計算出來的損害賠償。損害賠償在這裡無疑是經濟成果的替代。因為無體財產本身並沒有所謂的固有利益狀態而要維護的問題。有體財產有這個問題,因為有體財產,你撞凹一個摩托車,他就是被撞凹了,他的那個固有狀態就受到損害了。但無體財產沒有這個問題啊,你盜印他人的著作,盜用他人的商標,未經許可而去使用他人的專利權。因此,法院在判決損害賠償時,對依照應該取得的市場經濟成果去乘三倍,對所謂的加害人所判處的三倍的這一種,懲罰性的損害賠償,無疑是一種經濟成果預估計算的方式,從而拿到這一筆懲罰性損害賠償的取得者,理論上應該要申報所得稅。所以你的著作被盜用,有人去印,假設預估你會有1萬本的銷售,那這個地方乘三倍所產生出來的這個損害,當然就應該是應課稅的所得。

所以我們講損害賠償,其實是要看他的損害賠償的性質,是否在填補固有狀態,不管是人身生命或者是財產的固有狀態,或者是他是一個履行利益經濟成果的替代而做不同的區別對待。法律上面的理論上是這樣建構,但實際上往往在區別上是有困難的,這一點完全可以理解。但不會因為他是損害賠償法,還是用透過契約用和解或調解的方式因此而改變其本質。

當然,接下來我們要講很多邊際案例的類型。像醫生因為醫療糾紛而給被害人的慰問金。他不承認自己的損害賠償的責任存在,而是說,欸,我這個是基於道德上我給你的一筆慰問金,10萬塊錢。像這一類的給付,到底性質上算哪一類型?這個確實在實務上並沒有辦法很清楚的。如果你有法律上義務的給付,很清楚,這個是損害賠償,原則上他是非所得。但正因為實務的操作的上面,像這一種非基於法律上的義務而做的這一種基於倫理或道德上面你所做的這一種給付,既然不具有法律上的義務,那難不成構成財產上的贈與?所以要申報財產,所以納贈與稅嗎?比如說像現在在屏東發生那個大火,假設雇主自己自認為自己並沒有任何責任,我都已經做到了該有的相關的設施,是你不小心的點燃火把,或者是不小心你人掉進去。有些不良的雇主可能會說,誒,其實這個我都規範很清楚啊,你是你自己操作不當,才掉進去熔爐。在苗栗有一個勞災的情況,就是這樣,掉進去那個1000度高溫,立刻只剩下骨骸,很可怕的勞動現場的損害啊。我們假設,雇主自己自認為自己沒有過失,所以我就給你20萬塊錢慰問金啊,不承認自己的損害賠償。像這一筆錢到底在法律上怎麼定性?這一種沒有法律上的義務,他當然自然不會落入到所得稅中替代經濟成果的這樣子的一個概念的範圍。但法律性質上不是所得,難不成是贈與嗎?這也是邊際案例類型,我們有一些在實務上不太容易去做一個比較嚴格的區別標準。

另外,第二件事情是投資。投入資本的投入,跟投資股款的取回,這不算是所得。一個自然人去投入公司營利事業的資本投入跟投資股款的取回,這都不算是所得。這個地方既然不是他的成本費用,這個地方取回來也不是他的所得,所以投資股款的取回不是所得。

第三個是信託財產的移動。在設定信託時的財產上移動取得都不構成取得者的所得,因為這是信託財產的信託設定行為。信託的設定行為,為了信託而做的財產上移動,對取得者而言不構成所得。我們先跟各位談到這裡。當然其實在這個地方對於所得的概念,只要你是非財產上的增益,非透過市場經濟活動,原則上他就不是所得。

我們第一個層次過了,才會有第二個層次的應稅所得,或者是免稅所得,或者是其他分離課稅的所得,因為在這個地方,第二則往往是比較複雜的應稅免稅或者其他分離課稅的所得,他其實不是免稅,是一個分離課稅的所得,這是我們之前跟各位講過土地增值稅的所得,以及遺產贈與稅的所得,他其實都是分離課稅,所以他不是所得稅裡面的應稅所得,而是另外分離課稅的所得。境外取得的所得也是一樣的道理,境外取得的所得不是課所得稅的所得,而是課所得基本稅額條例裡面的最低稅負制,因為在我們現行的法制裡面,是把所得做這樣一個不同的區別對待,這個是到第二個層次裡面的問題,我們再來去做判斷。

我們今天先講到這裡,下個禮拜,再從那個應稅免稅分離課稅的所得這個角度去做說明。

\hypertarget{section-9}{%
\chapter{20231002\_01}\label{section-9}}

\begin{longtable}[]{@{}l@{}}
\toprule()
\endhead
課程:1121所得稅法一 \\
日期:2023/10/02 \\
周次:5 \\
節次:1 \\
\bottomrule()
\end{longtable}

上周,我們談到,所得稅法4條一項三款,關於傷害或死亡之損害賠償金,及依國家賠償法規定取得之賠償金,免納所得稅。這個條文,本質上是在説明生命或身體所受之損害,而獲得的賠償金,并非所得稅法上的所得。
生命或身體所受之損害的固有利益的損害賠償,即使是通過和解去做約定賠償的數額,雖然會從法定之債轉成契約之債,也就是說,在民事法上面,金錢的給付會從侵權行為損害賠償變成的事契約之債務不履行的損害賠償,但他本質上仍然還是來自於因為生命或身體所受之損害產生出來損害賠償,他並不具有所謂的所得的適格的地位。相反的,替代履行利益之損害賠償,也就是我們稱之為叫履行利益的損害賠償。由於他是經濟成果的替代,而經濟成果作為所得,是所得稅所要掌握的稅捐客體,替代經濟成果的這樣的一個損害賠償,沒有道理排除在課稅所得的計算範圍外。從而,替代經濟成果之損害賠償,還是所得,仍然應該要有課稅所得的適用,因此沒有法律的明文規定,原則上要計入所得的範圍,他會跟4條一項三款,包括身體跟生命,但不包括財產。財產的固有利益的損害賠償不計入所得,原則上是透過解釋的方法,因為他非所得,而免納所得稅,但不會在4條一項三款內。

相反的,替代履行利益的損害賠償,會是一個非常重要的所謂的經濟成果,原則上還是要課稅所得。特別是我們在上個禮拜跟各位講過,在實務上,特別是在無體財產權這一類的專利或商標的侵權行為裡面,往往存在著所謂的懲罰性損害賠償。論其本質,是按照他的經濟成果去做估算的一種損害賠償。無體財產,並沒有因為有體財產的固有狀態被損害產生出來的填補損害的這一種損害賠償的問題。無體財產,顧名思義,他本身就是一個無體的存在,而未經他人許可而去侵害他人的無體財產所產生出來的這一種計算損害賠償,往往都是經濟成果的替代。正因如此,經濟成果替代的懲罰性損害賠償應被計算入個人或營利事業,因為經濟活動所獲得的所得,是應稅所得的適用範圍。

我們在上個禮拜大致上跟各位提到損害賠償,從這個地方我們開始展開一連串的討論。

接下來這幾個問題點,各位都可以去思考。第一個,損失補償,要不要算所得?個人因為國家的公用徵收行為,因此被國家以單方的公權力所受的損失,我們把他稱之為叫損失的補償,特別是目前在公用徵收法裡面,原則上是用市價補償,究竟這一種補償是接近損害賠償的性質,或者是契約上的價金?特別是,我們在徵收法裡面用接近市價的方式來去對被徵收者的特別犧牲給予接近市價的補償費,我們會說是損失補償,這種情況所領的補償金算不算所得?這是第一個問題。

第二個問題,我們要跟各位去講保險給付。保險金的給付保險來自於一個市場經濟的活動,也就是你去投保保險,投保保險之後,保險事故發生的時候,保險的依照保險契約對受益人所做的給付。並不是加害者對被害者所做的損害賠償給付。他是來自於一個保險契約而產生出來的約定的給付,也就是當保險事故發生的時候,我要去對你去填補,由於加害人可能沒有辦法去對被害人去做足額的賠償,為了填補這個風險產生出來的市場機制,我們把他稱之為叫保險。保險給付算不算所得?這個是我們要跟各位去討論的第二個問題。

第三個問題,是信託財產的財產上的移動。在設定信託行為的時候,原則上會有一個財產上的移動,比如說我是委託人,那麼昱如是受託人,那麼我把財產移動到她的名下,這個設定信託財產的這樣一個設定信託的行為,這個算不算是取得財產者的所得?我想各位很快就會説,不是。那變更受託人呢?也許我變更受託人,我今天從昱如變成逸帆,那么最終信託關係終止的時候,他又要再把這個財產移動回來給我受託人或者是我所指定的第三人。在這一種依據信託關係而做的財產移動,到底算不算所得?

第四個問題,投資取款。投資股款的投入跟取回,就是資本投入跟資本取回,這件事情算不算所得?因為曾經在某一個時期,某些公司溢價發行股份,比如說一張股票面額是10塊錢,由於這一家公司的夥伴非常值錢,大家都想要投資進去,所以會形成,在市場上面額是10塊錢,可是因為大家都搶著買,他可能就變成要20塊才能買到一張面額是10塊錢的股票,這個是一個股本投入的行為。相反的叫投資股本的取回,取回資本的行為,算不算所得?

這四個問題都跟他是不是所得的性質的問題有某一種程度上的關聯,都是在第一個階段的問題要去處理的。我們今天第一個小時前半段的部分跟各位很快的說明這一件問題。

\hypertarget{ux640dux5931ux88dcux511f}{%
\section{【損失補償】}\label{ux640dux5931ux88dcux511f}}

我們首先來談一下損失補償。損失補償跟損害賠償最大的區別不是在名詞,而是,不法的侵害行為所造成的損害,我們把他稱之為叫損害賠償;補償行為則是來自於一個合於法律規定對他人的財產權的干預的行為,也就是在徵收補償費的發放的情境裡面,他是一個合法的作為,國家高權公權力的作為。這種國家高權公權力的作為剝奪了相對人的財產權。在這樣一個前提底下,第一個層次,是對公用徵收的合法性跟正當性的檢驗。那么第二個層次才是進入到即使肯認了國家基於高權行政對人民所擁有的財產權,得以在合法跟具有正當性,也就是具有公用徵收之必要的前提底下去干預人民的財產權,仍然還要在第二個層次裡面,必須要國家提供給受到特別犧牲之相對人,以所謂的接近市價的補償的方式,來做市價的補償費的發放,這個就是所謂的損失補償。

兩個不同階段,各自依照各自不同的法理,基本上都在比例原則的這個思考底下,去對相對人因此所受到的特別犧牲,去做合於法治國的考量,損失補償,接近損害賠償的性質嗎?特別是我們現在在公用徵收裡面,有提供要先提供一個讓相對人可以去參與關於該公用徵收的所謂的「協議價購程序」,也就是在國家發動公權力去做多用徵收之前,在法定程序上,必須要給予當事人有思考的可能,是不是可以透過一個買賣的方式來達成所謂的用契約的價購的方式去替代原先所想要進行的國家公權力行使的第一層次的公用徵收跟第二層次的所謂的市價補償程序。

協議價購,是私法契約的性質,還是是公法契約的性質?學說上有高度的爭議,我們在這裡面一併地去說明協議價購的補償金的性質。

第二,我們再講,如果協議價購不成,國家仍然發動公用徵收程序,因此在第二道程序裡面去做損失補償費的發放。這究竟是屬於所得性質的經濟成果的替代,還是是屬於非所得性質的損害賠償,在學理上有討論跟說明的可能性。

特別是某一些認為以接近市價這樣一個性質的補償,特別在前階段有一個協議價購程序。協議價購程序,如果把他當作是一個私法契約,換言之就是視人民把他的土地賣給需地機關(需要土地的這一些國家機關,包括中央包括地方機關),因此認為這是一個私法契約,在這樣一個前提底下,這樣一個程序產生出來的履約的價金,當然是經濟成果,當然是所得稅所應掌握的課稅所得的範圍。

換言之,這種觀點認爲,在國家開啟公用徵收的必要性之下,國家仍然還是要實施單方的公權力來剝奪人民的財產權,但為了避免國家違反比例原則而過度徵用干預人民的基本權,因此在第二個層次裡面提供給予人民接近於市價的補償(現行實務上是以公告現值加四成的方式去計算的),這個損失補償費這個法律上的性質因此被認為是一個經濟成果的替代,而認為應該是課稅所得的範圍。

這個是一個看法。但我個人認為這個看法是不對的。我們先暫且不談協議價購,因為後面再過來談這個事情。損失補償在前述的說明底下,他其實是在國家發動公權力徵收的,這種情況底下對被徵收之相對人,因此所遭受到的特別犧牲的一種對價回饋。簡單而言,就是因為你特別犧牲,從而國家提供一個市價的對價的回饋。這種方式,非基於自願性的參與市場經濟活動,從而不是市場經濟成果的替代。我再一次強調,非自願性的參與市場經濟活動,縱使在結果上,客觀上,是接近市價的補償費發放,在這種情形底下,他仍然不是自由意願之下的參與經濟活動所產生經濟成果的替代。

這個問題,恐怕會跟前面那個無體財產權的損害賠償之性質相混肴。專利跟商標未經過授權者的合法授權而被使用,就是,使用該商標跟專利之人,一樣,沒有經由權利擁有者的意願,而直接就非法去或者是逾越授權的範圍而去擴張使用,產生出來的損害賠償,是不是被認為是基於自由意願產生出來的經濟成果,在學理上有進一步討論的空間。
我個人認為,市場經濟活動中,非基於自由意願而產生出來的經濟成果,跟經濟成果的替代,他依然是所得的一種類型。可是透過國家公權力的作為,由於非基於市場經濟活動而產生的對人民的干預,對人民財產權的干預而產生出來的,相對的補償費的發放,並非市場經濟活動所要掌握的對象。老師個人認為,他只是一個對因為高權行政行為產生出來的損害、損失而做的填補,即使接近於市價,仍然不構成所得,不應該成爲課稅的對象。

同樣的道理,有許多在公用徵收裡面,對,比如說土地上的農作物搬遷,在實務上會有一些搬遷補償費。比如說,你住在這裡,我這裡要做公用徵收,我請你搬離開,那往往會存在著許多搬遷的相關的費用,比如說獎勵及早搬離的獎勵拆遷補償費,或者是農作物補償費。如果你願意在一定時間以前能夠搬離開,騰空房屋的話,實務上往往也會發給獎勵的補償金。那麼,以特別犧牲作為前提的話,我自己個人比較傾向認為,在特別犧牲可以涵蓋的最大可能範圍內,盡量給予非所得的屬性,也就是認為這是一個國家公權力行使底下透過各式拆遷補償費的發放,而來完成公法上的任務,也就是爲了滿足公用地的需要,爲了達成公共目的,而對特別犧牲所做的一個單方的補助的行為,因而,這不構成經濟成果的替代。這也是我們目前在實務上基本的看法,也就是透過拆遷給予的相對的補償,或者是徵收給予的損失補償費,原則上免計所得稅的所得。

因此我們再回過頭來,來看一下第4條第三款還有一個後段的規定,各位也許就更清楚背後他的本意。在所得稅法第4條第三款還有一個後段,「依國家賠償法規定取得之賠償金」。國家的不法侵害行為固然是免所得稅的所得,理由不在給予稅捐優惠,因為國家賠償本身是國家的不法行為,從而國家的不法行為,他本質上一樣在填補被害者的不法的受害而產生固有利益的損害賠償,所以第4條的第三款的後段規定,他的本質仍然是損害賠償,作為損害賠償,跟前面的第三款的損害賠償,都是不法侵害行為產生出來的損害賠償。第4條第三款規定,本質上並沒有明文的去提到關於合法的侵害行為產生出來的補償費的發放。這個合法的侵害行為產生出來的損失補償,理論上,實務上,我們在目前的公用徵收的稽徵實務上,原則上是透過解釋函令的方式來免除掉他的稅捐負擔。就此而言,我個人認為沒有解釋函令,在理解上也應該是一致的。

只是這樣子的一個一致性的理解,

\hypertarget{ux8b8aux52d5ux6240ux5f97}{%
\section{【變動所得】}\label{ux8b8aux52d5ux6240ux5f97}}

我們來看一下在第14條的第三項。這裡面有四種類型的所得,被稱之為叫變動所得。

第一個叫自力經營林業的所得。也就是自己在山上種樹。我有認識一位老師啊,就在山上自己種樹,種一些高經濟價值的這個樹木。人稱十年樹木,他說,他種這個,三十年後可以給他兒子收成。這個叫自力經營林業的所得。

第二種叫受僱從事遠洋漁業,最近八尺門的辯護人很夯,這個議題也大概是這樣。受雇從事遠洋漁業遠洋漁船,出去一次,可能都是兩三年在海上打一次漁,再回來。受僱從事遠洋漁業於每次出海後一次分配之報酬。不管你是船長還是漁工,原則上你一次出去回來的一次的所得,這個是所謂的變動所得。

第三種類型,一次給付之撫恤金或死亡補償。

第四類型,「超過第四條第一項第四款規定之部分及因耕地出租人收回耕地,而依平均地權條例第七十七條規定,給予之補償」。給誰呢?給承租人,也就是地主依照平均地權條例收回耕地的時候,給土地的承租人的1/3的補償費,那個是平均地權條例第77條的規定。

這四種類型的所得,我們在14條第三項裡面,把他用一個名稱叫「變動所得」來加以涵蓋。

這四類型的所得,有一個共同的類型特徵,叫,累積多年所得一次大量實現。由於綜所稅是累進稅率。所以這種累積多年所得一次大量實現的所得,會在累進稅率底下產生稅負急劇增高的效應。比如說你一次自力經營林業,你可能是十年才換到一次經營林業之所得,因為是十年樹木嘛,你現在種樹不可能立刻就有經營成果,你可能十年才獲得一次收入,那你一次就賺個幾百萬,在累進稅率的前提底下,他會產生稅負急劇增高的效果。其實如果綜所稅不是採累進稅率,是像營所稅採比例稅率,這個效果是不會出現的。所以,在累進稅率底下,這種累積多年所得一次大量實現的所得,被認為在這個地方是變動所得,以一半課稅一半免稅來抵消掉這種稅負增加的現象。

那麼我們來看一下自力經營林業所得,其實就是在講自力農林所得,其中也就是他是屬於我們所得稅第14條第一項第六類裡面的自力耕作、漁、牧、林、礦之所得。遠洋漁船一次打於出來回來的一次的所得,本質其實是薪資所得,特別是對漁工而言啊。遠洋漁船的漁工,他本質上是受僱,只是他工作地點不在路上,在海上而已,那麼這個所得當然本質上也是一種薪資所得的態樣,累積多年所得一次大量實現。一次給與之撫恤金,在此之前,必然存在著僱傭關係,因此一次給與之撫恤金其實是薪資所得的後付,也就是你過世之後,再由你的遺屬來取得撫恤金。這個我們在講那個退休所得的性質的時候,會跟各位在進一步去做說明。退休是你的僱傭關係終止之後,雇主或是保險單位繼續根據先前的雇傭關係所做的給付,所以他是一段期間的僱傭關係終止之後,根據保險、根據契約而產生出來的,後面再行給付的金錢給付的法律上的關係。因此這一種撫恤金的給付性質本質上,跟薪資所得一樣。最後一個,地主依據平均地權條例,對土地承租人佃農所做的補償金的給予,我個人在法律性質上認為是一個經濟行政上的補貼,也就是在當時的法秩序底下透過對國家公權力的行使對地主的徵收,然後補貼給承租土地的佃農而產生的一個經濟上的補貼行為。那么,在這樣一個經濟上的補貼行為底下,其實在概念上面,如果是從地主徵收來,然後再發放給承租人,那其實他也是經濟上補助的一種類型,不管是剝奪或給予,理論上都不在所得的範圍。

我們今天跟各位去討論,其實損失補償費,凡涉及國家高權行政公權力之剝奪,或給予,本質上都不涉及到經濟成果或經濟成果的替代。但很明顯的,14條第三項把他列入變動所得的範圍,就會變成,他是所得,只是累積多年所得一次大量實現,所以給予一個計算上的優惠方式去抵消掉這樣的一個稅負急劇增加的效果而已。

在累計稅率底下,累積多年所得一次大量實現,仍然還是一個市場經濟活動的結果,只是他是累積多年一次實現,因為稅負急劇增加,所以我們在變動所得這一類的這個條文規定裡面,給予一個特殊計算方式,就是半數課稅半數免稅。透過這種方式來抵消掉這一種稅負急劇增加的問題跟現象。但他的前提必然是,把這一類的補償費的發放、透過國家對地主的徵收、對佃農的發給,這樣一個經濟行政上的行為,認為他是所得。在這樣一個前提底下,14條的第三項的規定,我個人認為是一個跟現行法秩序互相牴觸的作為。

簡單來講,就是高權行政裡面的作為,涉及國家公權力的行使,不管是干預剝奪或者是給付給予,不在市場經濟活動的經濟成果,或者是經濟成果替代的適用所得的範圍,不是所得。也就是,經濟上的補助補貼不是所得,是屬於高權行政的結果,他是屬於干預行政或者是給付行政,無論是哪一者,都不是市場經濟活動底下的經濟成果或者經濟成果的替代。

因此我們最後面再跟各位去講,協議價購程序確實在法律上的性質存在著公法或私法的爭議。為了讓公用徵收能夠被順利推動履行,協議價購程序先行,讓人民願意配合國家的需用計劃而願意自行以市場的方式把他賣給需地機關,就算他是一個私法上的契約,由於後面的公用徵收程序在現實上,仍然依照我們現行實務的一般的看法,這個部分的損失補償費,並不是課稅所得的計算範圍。

我們先不看14條第三項變動所得裡面的對佃農的經濟補貼,這個例外地被當作是一個所得,只是用變動所得的方式去計算他的所得的數額而已。將這種例外情形排除掉,凡是國家高權行政的作為,產生出來的經濟財產上的移動,原則上他並不構成所得的適格的地位。也因此,協議價購程序,在這個前提底下,為了促進公用徵收的適當地、適合地被履行,我個人比較傾向認為,協議價購,在法律有明文規定的前提底下,應給予免稅的優惠,否則協議價購,同意價購的相對人反而要被課稅,如果不同意架構進入公用徵收程序,以損失補償費的名義取得的補償金,就變成反而他可以免稅。在這個地方,在價值上是不衡平的。你本來是鼓勵他,希望他去做,結果他做的效果是更不利,因為他還要繳所得稅。在這種情況底下,這個反而會讓協議價購程序的目的無法達成。從而,依據特別犧牲理論,往前延伸,關於協議價購程序,老師個人還是傾向,這個協議價購的協議價購款項,不具有所得的適格的地位,應該不要課稅。但為了避免爭議,最好是有法律明文規定。因為他確實在法律性質上被認為具有私法契約的色彩,你自己願意賣的。這個時候如果從邏輯解釋上來講,價購款項會是經濟成果,既然作為經濟成果,法律沒有明文規定,那就應該要課稅,那這個反而會讓協議價購作為公用徵收的前置程序,目的無法順利達成。從而,從整體來看,關於協議價購的價購款項,老師個人認為,以法律有明文規定,而去對相關的款項給予免稅的待遇為宜。

這個是關於損失補償。現行法律在沒有明文規定的操作實務上,任何關於徵收補償費,包括拆遷、農作物補償,甚至是你提早搬離開,減少行政機關的公務人力、徵收上的費用的相關的獎勵,基本上全部都在免稅的所得適用的範圍。沒有法律明文規定,因為4條一項三款並沒有提到損失補償,只有講國家賠償的損害賠償不計入所得的範圍、不適用所得課稅的範圍。也因此在這個地方,我們以特別犧牲理論作為他的正當性的基礎,因為他是填補犧牲所做出來的補償費,性質上面非屬於經濟成果替代的所得。

我們再回到14條第三項所講的變動所得。變動所得這一種累積多年所得一次大量實現的情況,基本上也是建立在實用性原則的基礎上,而提供給予變動所得的納稅義務人一個簡便計算方式,來取得這一類所得來抵消,或者是避免所得稅因為採用累計稅率所產生出來的稅負急劇增高的不利現象的一種方法。因此他是一個在實用性原則底下提供稅捐優惠的一種方式。這一種累積多年所得,我們直接給予半數課稅,半數免稅的這樣一個做法是便利計算的做法。簡單來講,就是我不採用比較複雜的方式,譬如說立法者把你個別所得歸算個回去各年度,再回推以各年度的繳稅所得重新計算你的綜所稅的稅負,這個也許比較符合各年度依據稅捐負擔能力個別計算的課稅所得,但比較麻煩,因為你必須要各年度去做歸屬計算,而且要變更原先的所得。那么,在這樣的前提底下,變動所得是一個比較變通的便利性的計算方法。基於這樣子的一個角度而給予稅捐優惠。從而變動所得的性質,理論上可以適用到各種,如果存在有客觀上累積多年所得一次大量實現的類型特徵上面,應該可以類推適用。

但我國實務上,採取嚴謹的看法認爲稅捐優惠只是例外,從而僅限法律明定的四種類型,才稱之為叫變動所得。凡是不屬於14條第三項所稱的這種變動所得的列舉類型底下,縱使有累積多年所得一次大量實現,仍然要依據所得的取得年度來計算,義務人應該負擔的稅捐負擔。實務上最常出現的案例是,勞工被雇主非法解雇,那么他就請求確認僱傭關係存續,並且要求雇主依然還是要繼續給付薪資。對於這一段他實際上沒有上工的這一段時間裡面,他除了確認僱傭關係存在以外,還要求雇主必須要去做依照原先僱傭契約的條件所做的僱傭的薪資所得的給付。當他最終獲得勝訴的時候,可能會有累積多年所得一次大量實現的客觀上的情況。也就是假如他打了五年的訴訟,最終終於獲得勝訴,雇主也依照判決的內容對他做一次性給付,那麼他在當年度就會一次大量實現五年度的所得。累積多年所得一次大量實現,在適用綜合所得稅的前提底下,一定會產生稅負急劇增高的不利的效果。因為本來每一年的薪資會分散的在累進稅率底下,可能是用比較低的稅率,但是五年一次大量實現一次,大量給付,就必然會產生這種稅負急劇增加的效果。

實務上的看法,認為變動所得的計算方式是一種稅捐優惠,作為稅捐優惠要例外而嚴格的解釋,從而非在法律所規定的範圍內的變動所得,縱使具有累積多年所得一次大量實現的特性,也不適用第14條第三項,半數課稅半數免稅的稅捐優惠待遇。也就是目前為止我們的累積多年所得一次大量實現,特別是剛剛所提到的非法解雇而因此所取得的一次報酬的給付,往往都還是會用累積在你那一次受領的時候就一次去課徵他的所得稅的所得,而在這種情況底下,他會從薪資所得的類型變成其他所得,也就是在這裡面會用其他所得來歸類,認為他是一個綜所稅裡面的一種所得類型,然後全部去計算課稅。

那么如果你是人死了,領撫恤金,就可以認定爲變動所得。所以立法者是告訴你說不要活著啊,你最好是去跳樓自殺,你的一次領取的撫恤金,可以用變動所得的方式,來計算你的課稅所得。稅法上的立法者是這樣告訴你的。我隨便說說啦。這個,我只是讓各位看看立法者背後的荒謬的價值決定。立法者不是永遠是對的,也不在立法形成空間裡面的合法適當的一個展開的方式。

14條第三項目前被認為是稅捐優惠,那麼因此比較嚴格地去做解釋。

\hypertarget{ux4fddux96aaux7d66ux4ed8}{%
\section{【保險給付】}\label{ux4fddux96aaux7d66ux4ed8}}

基於這樣的一個說明,我們接下來進入保險給付。

保險給付正如同剛剛所講,他是一個市場經濟活動。除了法定的保險以外,商業保險的市場經濟活動的特性有更為清楚明白。你不投保保險,保險事故發生的時候,你就不會從保險公司保險人獲得保險金的給付。保險給付是接近,如果無法受加害人的完全給付時,最能填補損害的機制。也就是對於加害人對被害人無法做完全損害賠償的風險,保險是一個風險的共同體集合。簡單來講,大家都有同樣的風險,風險發生在一個人身上的時候,他是100\%。沒發生,就沒發生,發生在你身上,那就是100\%。那么,由於這種風險發生在個別人身上的時候,他是100\%的損害,他可能因此陷於經濟上的無能力去應對,從而,即使他是一個市場上的經濟活動,在填補損害的這一個方面來講,保險金給付也是一個最接近填補損害賠償的一種風險的分散機制。也就是說,他是一個市場經濟活動,我們承認,但是他是一個替代損害賠償的一個風險分散機制,這個特別是表現在財產保險,他原則上沒有有所謂的代位求償的機制,也就是說,被保險人在保險事故發生的時候,保險人對受益人所做的給付,可以在他給付範圍內去代位被保險人去對導致保險事故發生之加害人去做求償。也就是代位權的行使在保險法裡面避免了所謂的道德上的風險,也就是人們透過過度投保的方式來將投保行為當作是一個中樂透的行為。那么同樣的道理,也就是說,在保險裡面,特別是在財產保險裡面有代位的請求,他是最接近損害賠償的市場機制。因此在這樣一個前提底下,保險金給付作為一種損害賠償的替代,老師個人傾向認爲,他是確實是具有非所得的屬性。財產保險本身作為一個損害賠償的替代,而不是經濟成果的替代。儘管保險本身確實是一個市場經濟活動。

但接下來我們要講的是沒有代位權行使的,包括人壽保險跟年金保險,這一類保險法裡面認為沒有過度投保的問題的這一類保險給付。年金保險其實是人壽保險的一種生存險的類型,就是被保險人在時間到了以後基本上還繼續存活,他可以可以繼續透過年金保險的領取來照顧他老後的生活。那麼從而人壽保險裡面特別是生存險跟年金保險的性質,在某程度上面來講,這一類的保險金給付是為了照顧自己老後的生活而產生出來的市場機制的一種為照顧自己的老後生活的風險分散行為。沒有所謂的損害,沒有所謂的損害賠償的問題,因為老不是損害,他是一個自然現象。

正因為如此,人壽保險跟年金保險,是不是要給予像財產保險這樣的非所得的市場定位?老師個人是採取比較傾向,他是一個鼓勵人們投保的稅捐優惠做法
,而不是替代損害賠償的保險金給付而非所得。

人壽保險因為分很多,有這個所謂的身體健康險,這個由於實支實付定額給付,我們原則上如果不區分其類型,沒有過度投保,仍然是屬於損害賠償的風險的承擔,從而這個部分不是所得的類型。可是屬於生存的,特別是照顧老後生活跟經濟活動所需要的這一種年金保險,他比較接近,是一種市場上的經濟活動的成果,也就是你投保多少,你將來就可以從透過年金保險而去取回多少,那也因此這一類型的保險的給付則比較具有經濟成果替代的性質,從而具有所得的適格的地位,因此需要透過法律的明定的方式給予適當的鼓勵。也就是定額免稅的方式,鼓勵人們投保保險,照顧自己老後生活,而不要等風險事故發生的時候,在你身上100\%造成了其他家人的負擔,或者是造成國家社會行政上的負擔,因為沒有家人照顧的個別的國民必定會進入社會安全網,給社會安全網去照顧。這也是在稅捐優惠政策上面,鼓勵人們行有餘力的時候,照顧好自己,透過保險的方式去照顧。所以,定額給予免稅是適當的。至於定額為多少?原則上是以當代社會人民的平均餘命,以及通常的生活水準,特別是在某些職業類別上面,有一定程度上的生活水準,不是每一個人都是最低生活需求。

年金保險不是社會保險,不是社會救助。社會救助是以一個當地社會最低生活費的基準,但年金保險是一個職業上的保險,這種職業保險是有一個對應他職業老後生活的一個該有的生活基準。也就是如果他是一個某一個職業類型,他老後的生活,他之前可以領4萬,退休之後也不能差距太遠的,這種情況,在行有餘力的可能範圍內,應該要給予鼓勵人們照顧好自己老後的生活,因此會根據當地社會的平均餘命、職業收入標準,跟生活費用,去做一個合理的估算。以這個定額作為估算的基準。超過定額,就是過度優惠,會產生人們的規避動機。也就是如果所有的保險金給付一律免稅的話,就會讓人們透過用投保的方式,來享受那個優惠,這就是稅捐規避。

稅捐規避就是一種由於過度優惠措施而產生出來,人們透過私法形成來享受稅捐優惠的一種狀態。是立法者給的,不然他不會去起心動念想要獲得那個利益。

這個正是我們在保險金給付的性質裡面,我們從財產保險給付開始,其實是損害賠償的替代,特別是一個風險的承擔;那麼來到了人壽保險裡面,關於身體健康這種意外險性質,由於不管實支實付還是定額給付,原則上都是替代身體健康的這一種固有利益受損,但可能被害人無法從加害人那裡完全地求償,因此產生出來的風險的承擔,從而老師個人傾向認為這一整塊都是屬於損害賠償的替代,他是一個風險承擔的概念,他不是所得,就算法律沒有明文規定,也適當地給予免稅的待遇。但進入到保險給付裡面屬於照顧自己老後生活的這一種生存險,特別是年金保險,本質上是透過市場經濟活動裡面產生出來的一種經濟成果。那當然,年金保險,由於他都是透過法律的明定而產生出來的強制投保的態樣,因此就這個地方在法律性質上,根據我們第4條的第七款規定,你只要在法律所規定的人身保險、勞工保險及軍公教保險之保險金給付,原則上你只要是法律所定的強制投保的保險範圍,由於法律明定,在此限度範圍內裡面受到強制的行為,本質上就把他排除在所得的適用範圍之外,也就是不是所得。這是來自於法律的明文規定,這個限度範圍內(不是所得的範圍),解讀為給付行政。可是我如果是投保商業保險產生出來的人身保險的給付,理論上而言,他是一個照顧自己老後生活的保險給付的話,這個是因為你投保,所以你有獲得保險公司對你的保險給付,是一個經濟成果的類型。在這裡面,因為第七款的規定沒有給予一個定額的保險的額度,從而就會產生過度優惠的狀態,人們就會利用人身保險,來做比較高額的投保,這就是為什麼實務上投保保險可以節稅的一個很重要的原因。人身保險在保險金給付多少額度範圍內,沒有一個上限規定,正是因為沒有上限規定,從而人們會願意只要行有餘力,有可能,我會把我的儲蓄投資改成用保險的形式,我不要儲蓄投資,我不要拿去買臺積電股票,我改去買一張保單,透過保險的方式來把他轉換成人身保險的保險金,這個地方根據第4條第七款的規定全額免稅。這就是保險節稅規劃秘籍,來自於一個沒有上限的一個保險金給付全部免納所得稅的規定。財產保險不會有這個問題,死亡保險原則上也沒有這個問題。問題通常都是來自於人身保險裡面的生存險,特別是年金保險,非法律強制規定的這一種人身保險的性質,會存在著過度優惠的問題。這是保險金給付的部分。

我們先休息一下。

\hypertarget{section-10}{%
\chapter{20231002\_02}\label{section-10}}

\begin{longtable}[]{@{}l@{}}
\toprule()
\endhead
課程:1121所得稅法一 \\
日期:2023/10/02 \\
周次:5 \\
節次:2 \\
\bottomrule()
\end{longtable}

跟各位提到關於保險之後,我們繼續進一步跟各位說一下,關於信託財產的移動,是不是課稅所得。

\hypertarget{ux4fe1ux8a17ux8ca1ux7522ux7684ux79fbux52d5}{%
\section{【信託財產的移動】}\label{ux4fe1ux8a17ux8ca1ux7522ux7684ux79fbux52d5}}

所得稅法3-3條的規定,特別就關於信託財產的設定,還有存續關係存續中裡面的這個受託人的變更,以及信託關係消滅後,委託人跟受託人之間的財產的取回。

請各位看到所得稅法3-3條,因為信託行為設立成立委託人跟受託人之間\ldots\ldots 我自己認為,這個法條規定這樣不是很清楚,他應該要很清楚地規範,因信託行為成立委託人,將財產移交給受託人,這個對受託人而言,不是他的所得。

3-3條,第一項第二款規定,信託關係存續中受託人變更,從原先原受託人變更成新受託人,你從A公司變成B公司作為受託人,信託關係不變,在這裡面,財產上的移動從原受託人移動給新受託人,這個不是市場經濟活動,這個是在設定信託關係裡面的受託人關係,從而對新受託人而言,不是他的所得。

第三款規定,「信託關係存續中,受託人依信託本旨交付信託財產,受託人與受益人間」。
這個地方在講什麼啊?!你如果在信託關係存續中受託人跟受益人之間有財產上的移動,這個正是所得稅法要掌握的所得範圍,不是設定信託的行為。你現在受託人拿著管理信託財產的利益,如果要交給受益人的話,這個正是整個信託法制要掌握的所得的範圍。

第四款的規定,又回來了,因信託關係消滅,委託人跟受託人之間,或受託人與受益人之間。這個後半段是沒道理的,但前半段是有道理的信託關係消滅的時候,受託人要把他名下的信託財產移轉回來給受委託人,這個是信託關係消滅時的一個標準做法,因為這個本來就不是受託人的,本來就是委託人的,那信託關係消滅的時候,你要回來給委託人,這個是本來信託關係消滅時該要做的事情。所以只有這個部分有道理。但對受益人之間,也就是受託人與受益人之間,這個地方,他想要去規範什麼情況?委託人當初在信託本旨裡面有講說,當信託關係消滅時,請把信託財產交給受益人。也就是受託人受到信託本旨的拘束,在信託關係消滅的時候,受託人要把這個信託財產交給受益人。他是本來是委託人的東西,信託關係存續中變成是受託人名下的財產,信託關係存續結束以後,本來應該要回去,結果現在依照信託之本旨,也就是當初設定信託的時候,委託人告訴受託人說,如果接下來信託關係存續結束的時候,請你受託人,就直接把東西給受益人。請問這個叫什麼?這叫委託人對受益人的贈與呀!有問題嗎?

還回去是本來的道理,你現在透過信託關係,就直接送給受益人,法律是這樣規定的。胡搞瞎搞,莫此為甚,胡說八道。

信託關係的設定存續跟消滅他有一個邏輯軌跡,就是為了要設定信託,委託人把名下財產從自己身上移動到受託人,好,這個對受託人而言,不是他的所得,這應該沒問題。待會再跟各位講另外一面的思考哦,不是理所當然都是如此哦。

好,信託關係設定的時候,從委託人到受託人。變更受託人的時候,從舊受託人到新受託人,這個當然都不是取得者的所得。

信託關係消滅的時候,你本來就應該要回過頭去,如果委託人已經死亡,那就會變成是他的遺產啊,你怎麼可以直接就透過信託本旨就直接從受託人就移動到受益人那個地方?如果直接就移動過去,那也要被認為是委託人對於受益人的一種財產上的贈與,或者是遺產的給予嘛。這才符合一個法律正常邏輯底下該有的法律規定,就這麼簡單一件事情。所以信託是一個好的財產規劃工具,尤其是對有錢人。

法律胡亂規定的時候,你就會看得出來,立法者究竟想要什麼樣的一個價值。最後面,第五款,「信託行為不成立無效或解除或撤銷的時候,委託人跟受託人之間」,那當然啦,因為你的信託關係,因為不成立無效解除或贈與或撤銷,要回復原狀啊,所以你本來從委託人到受託人,要回復原狀,從受託人又回到委託人身上。這之間的財產上的移動,原則上都不是取得者的所得,這個當然沒問題,因為是在回復原狀。

好,我們信託財產上的移動,根據3-3條,摒除掉少數,我個人認為很奇怪、立法價值不明的稅捐優惠規定以外,3-3條,基本上在體現一件事情,就是信託財產的移動,原則上不計入所得課稅的範圍,也就是非經濟成果。那個是純粹為了形成一個法律關係產生出來的財產上移動。

但,這個信託關係其實也很容易被利用作為一種實際上是作為財產上的移動,讓受託人實際上取得管理信託財產的所有權,讓他實質上享受該項利益的一種工具跟手段。信託也可以被利用作為實際上是一種贈與,不管是對受託人,或者是受益人。也就是信託行為,在這一個信託關係存續當中,不在委託人名下,變成是移動到受託人名下,受託人再經由信託本旨,再將管理信託之利益交付給受益人,請問這一個法秩序裡面難道不應該評價為,實際上是委託人透過信託的方式去對受託人或者是受益人,所為的贈與行為嗎?也就是看起來是信託,當然實際上就是以信託之名,讓他可以實際上管理財產,並且交付管理財產所獲得的利益給受益人的一種型態。

簡單而言,對於信託在法秩序底下的正確評價,我固然認為信託設定財產,這一種財產上的移動不是所得,但理論上他可以被評價為是一個將管理財產所獲得的利益做贈與的行為,也就是會落入贈與稅課稅的範圍,而不是所得稅課稅的範圍。也就是說在這個地方,適當的應對的法制是遺產稅跟贈與稅。所以我們把信託財產的課稅法制,我們到遺產贈與稅裡面再一併去討論。

信託關係存續當中,受託人管理信託財產一定會產生信託財產的利益,也就是所得,這個所得究竟要怎麼個計算?這個固然是所得稅法本來應該要關注的,但為了讓信託設定行為跟信託關係存續當中的這種管理信託財產所產生出來的增益一併去做規範,我們會到所得稅法三的遺產贈與稅法再來去談,關於信託行為的課稅。因為實際上信託行為正是拿來作為遺產跟贈與規劃的最佳的財富傳承的工具。

根據3-3條的規定,信託財產的移動,對取得者不是所得,他的背後原理在,他是一個設定法律關係的行為。因此,這個財產上移動,不是市場經濟活動。

\hypertarget{ux80a1ux6b3eux4e4bux6295ux5165ux8ddfux53d6ux56de}{%
\section{【股款之投入跟取回】}\label{ux80a1ux6b3eux4e4bux6295ux5165ux8ddfux53d6ux56de}}

最後就是投資的股款的投入跟股款的取回。股款投入跟股款取回,本質上是一個財產上的移動,沒有錯,但他沒有增益他人財產的意思,也就是他投資某一家公司或投資某一個自然人股東,都是一個投資上的行為,因此投資上的財產上移動,對於取得者而言,他是一個股本的取得,或者是股本的退回,都不是經濟活動的成果,因此投資海外而又再從海外匯回來,這個不是所得稅的所得,而是一個投資股款的取回。因此在境外資金匯回管理運用及課稅條例的規定裡面,這種匯到海外的錢轉了一圈又回來的情況,只要取得相關的證據資料,足以證明是當初匯出去的股本,那麼原則上就不是所得,就不應該課所得稅。可是如果你匯出去100萬,現在存在海外的有1000萬要匯回來,這一種超出原先股本的,就會被認為是屬於你納稅人所取得的境外來源的所得,因為這個境外來源所得,不是課綜所稅的所得,是課所得基本稅額條例的所得,有一個課稅門檻的100萬的規定,有一個670萬的免稅額的規定。這一些都是所得基本稅額條例裡面的規範,我們到所得稅法四的時候,再來跟各位做進一步的說明。因此在這裡面,關於所得跟非所得的概念,就請各位注意到,損害賠償、損失補償、保險金給付、信託財產的移動跟投資股款的投入跟取回,這幾個相關的議題。這個議題的重心,主要是在於有沒有法律明文規定,原則上不影響他應否課稅的本質。

如果他本質上是一個所得,但在法律沒有明文規定底下,是一個經濟成果或經濟成果替代的話,那我們會就進入到第二個層次的問題,也就是應稅跟免稅所得的議題,這就是我們今天要跟各位繼續去談屬於所得的範圍。

\hypertarget{ux61c9ux7a05ux6240ux5f97ux6216ux514dux7a05ux6240ux5f97}{%
\section{【應稅所得或免稅所得】}\label{ux61c9ux7a05ux6240ux5f97ux6216ux514dux7a05ux6240ux5f97}}

那么進入第二個階段,我們就要判斷他是應稅所得或免稅所得。應稅跟免稅的差異在於,他同樣是經濟成果或經濟成果的替代,免稅所得是基於立法,在不同的正當化理由底下,從而給予不同的對待,也就是免稅所得背後有一些正當化的理由,透過這些正當化的理由,沒有法律的明文規定,免稅所得不得給予法外的施恩、給予法外的稅捐優惠。也就是說,在這個地方,所有的所得原則上是應課所得稅的所得,那麼你要免稅所得,必須要有法律位階的反面構成要件的規定。所以,沒有法律明文規定,原則上是不成其為免稅所得的一個稅捐優惠的地位。這是第一個。我們在談完所得跟非所得概念之後,進入到所得這個概念第二個層次的問題,我們要探討應稅跟免稅所得的概念。在現行立法底下,會基於以下幾點理由對於所得稅的所得,給予免稅的優惠。

\hypertarget{ux5be6ux7528ux6027ux539fux5247}{%
\subsection{【實用性原則】}\label{ux5be6ux7528ux6027ux539fux5247}}

第一種,我們把他稱之為叫實用性原則,也就是說,這一類的所得,他的數額由於他數額微小,他是一個小額福利,那麼因為依從成本跟稽徵成本過高,換言之,微利,那我就算了,就不給予算入。這一種稱之為叫小額福利性所得。這一種小額福利表彰在我們所得稅法第4條第一項第五款:「公、教、軍、警人員及勞工所領政府發給之特支費、實物配給或其代金及房租津貼。公營機構服務人員所領單一薪俸中,包括相當於實物配給及房租津貼部分。」

譬如說,在公務員的薪水裡面,公務員有領受租金上的補貼。租金補貼其實理論上是因為來自於勞務的付出所給予的,雇主基於勞務上受僱人的雇主的這個付出,所以給予一個在外做租金補貼的經濟的成果。那麼由於這一類的成果本質上在當年最初最早的時候被認為是一個小額福利,也就是住宿上是一個小額福利,從而他是基於實用性原則,因為太小了,所以我們在課稅的時候就不算。這裡面包括了像交通津貼、住宿津貼還有你因為就職的緣故所產生出來的一些附帶的利益,比如給予一些夜點的給付。這個在實務上往往被以他是一個微小的福利。在英文裡面他叫做Fringe benefit,微小的福利,小額福利,就算了,因為要去做收入的核實計算有困難。
再例如,就職場所裡面所給予的住宿補貼。我舉個例子來講,我現在不住臺北,那我可以領國立臺灣大學給的住宿津貼,一個月700塊。這個叫小額福利,就算了。所以在課稅的時候,不會計算進入到我的所得的範圍。那為什麼會是700塊?因為從以前計算出來的標準,到目前為止都是這個樣。啊,我們當然後面會去跟各位去談,但現在的住宿,用700塊,你租得到,就輸給你。不可能是這樣,所以假設雇主提供住宿的地方,也就是有一個住宿的地方,特別像公務員,因為任職在公家機關單位提供宿舍。這個還可以叫做Fringe benefit的小額福利性所得嗎?這個會有問題喔。好像現在大部分的市長或者是大部分的首長,基本上都有配職務宿舍。職務宿舍就是因為一直都被認為說這個叫小額福利,就比如說我剛剛跟各位講,我沒有住學校宿舍的時候,我就領700塊,就這樣。意思就是你如果住學校宿舍,你也受益700塊。各位如果覺得有道理,我也是輸給你啊。這個700塊根本在實務上發揮不了作用,他的正當性理由就是這是一個小額福利性措施。
除了剛剛講的房租津貼,早期有交通車接送,這個也都是屬於一個小額的福利。

其實相對應的哦,這個在德國像他們在耶誕節由公司所提供給員工的耶誕禮物,不算入所得的範圍,這是同樣一個概念。剛好對應的,我們最近剛過完中秋,假設公司發給你中秋禮節的禮金或者是禮物,這個都是屬於小額福利性所得。所以以第4條第一項第五款的房租津貼或者是實物配給或其代金及房租津貼,這個部分最早的本意是認為他是小額福利,從而給予免稅的稅捐待遇。這是基於實用性原則。

\hypertarget{ux529fux7e3eux539fux5247}{%
\subsection{【功績原則】}\label{ux529fux7e3eux539fux5247}}

第二個,我們來看,基於獎勵給予社會捐優惠,也就是對於在經濟行政上被鼓勵的一個市場經濟活動,表現出獎勵也就是功績原則,因此給予稅捐優惠。譬如說,第4條第一項二十三款:「個人稿費、版稅、樂譜、作曲、編劇、漫畫及講演之鐘點費之收入。但全年合計數以不超過十八萬元為限。」這個是獎勵文化的這一種創造性活動,因此對稿費版稅樂譜作曲編劇漫畫及講演之鐘點費收入給予免稅優惠。就是說,對於這一種文化性的創作活動,以18萬元額度為限,給予獎勵,他是一個典型的依獎勵原則,和我們稱之為叫功績原則。在這種文化性的創作活動裡面,立法者鼓勵你創作,鼓勵你從事於文化性的創作活動,所以給你一個免稅的優惠。這個是功績原則。

\hypertarget{ux9700ux8981ux539fux5247}{%
\subsection{【需要原則】}\label{ux9700ux8981ux539fux5247}}

第三個,立法者會基於什麼樣的理由給予稅捐上的優惠?基於需要。也就是在社會國原則底下,對急、難、救助,對於處於生存有危害的這種情況,或者是老弱病殘,這種需要給予社會扶助或是社會法給予協助救助者這一類的情形,而給予的稅捐上的優惠,這個是應稅所得,但基於需要原則而給予免稅的優惠。那么,在我們的所得稅法裡面,沒有比較典型的需要原則的稅捐優惠,但在嚴重特殊傳染性肺炎防治及紓困振興特別條例第9-1條,受嚴重特殊傳染性肺炎影響,而依本條例、傳染病防治法第53條或其他規定,自政府領取之補貼補助津貼獎勵及補償,免納所得稅。從這個法律規範的角度來看,其實他也可以被認為是給付行政。也就是說,這個是一個政府在covid-19的非常時期對受covid-19影響而導致他需要國家提供給予協助或救急的補助,作為給付行政,因此不在市場經濟活動的範圍。從這個角度理解的嚴重特殊傳染性肺炎防治及紓困振興特別條例9-1條第一項之規定,他不是所得稅的所得啊。這個是第一個角度的理解。第二個理解,即使認為這樣子的所得,是一個市場上的經濟活動的所得,他也是在一個基於急難的緊急狀態,對於陷於需要國家提供協助的企業或個人,在這裡面依據需要原則,因此而具有正當的免稅理由的一個稅捐優惠的所得。也就是說,從他是給付行政的非所得,以及他就算是所得,但也具有依據需要原則,而給予稅捐優惠的正當性基礎的免稅所得,而來正當化對於應稅跟免稅所得的區別對待。原則上基於以上這三種理由而來,對一個本質上屬於經濟成果,但我們在稅捐法制上面給予一個稅捐優惠待遇的這一種做法,這個就是一個免稅所得的正當性。

\hypertarget{ux907fux514dux91cdux8907ux8ab2ux7a05}{%
\subsection{【避免重複課稅】}\label{ux907fux514dux91cdux8907ux8ab2ux7a05}}

還有第4個,為了避免重複課稅,也就是免稅所得是因為我們在法秩序上已經有另外一個有效課稅的稅目,去對該等所得去做課稅,從而我們在所得稅法裡面把他納入免納所得稅的範圍,是為了避免重複課稅,所以背後的本意是分離課稅。也就是在這裡面,基於避免重複課稅,從而把他從應稅所得規定成免稅所得,不課所得稅所得,但改課其他稅目的所得。基於這樣分離課稅的類型的本意,從而在現行法秩序底下才做這樣一個規定。體現這樣子的分離所得課稅的本意的有4-1、4-2,然後4-4搭配第4條第一項的第十六款土地、證交所得、期交所得,還有第17款規定的遺產遺贈跟贈與,因為我們有課遺產稅跟贈與稅。第4條第一項的第十六款的規定,是跟4-4條搭配,我們有另外課土地增值稅,那麼105年以後4-4條的規定,因爲有房地合一稅,所以這一類的立法本意在分離課稅,避免重複課稅,而不是一個真正意義底下的所謂的稅捐優惠。理由在於,因為立法的價值決定,決定採用另外一個稅目來替代所得稅。當然從另外一個角度,我們看所得基本稅額條例的境外所得,也是同樣一個道理,也就是依照我們現在的法秩序的價值的選擇,是採用所得基本稅額條例的最低稅負制來課所得稅,而不是計綜所稅的所得,課徵所得稅。這樣一個立法價值,背後的立法上的原意是在分離課稅,而不在稅捐優惠。當然這樣的分離課稅有沒有道理啊?這個又是另外一個層次的問題,也就是說,立法者的價值決定是分離,但立法者的價值決定卻不一定有他的道理,也就是在一個合理的立法形成空間範圍內

\hypertarget{ux8b49ux4ea4ux7a05ux4ee3ux66ffux8b49ux6240ux7a05}{%
\subsection{【證交稅代替證所稅?】}\label{ux8b49ux4ea4ux7a05ux4ee3ux66ffux8b49ux6240ux7a05}}

我認為最沒有道理的就是4-1,跟4-2的證交替代證所,證交稅、期交稅替代證所稅。這個是明文寫在4-1、4-2條立法理由裡面的規定,尤其是4-1條。我們大法官第693號解釋,也明白地認為證交稅是替代證所稅,以稅代稅。我建議,看要不要把他翻譯成英文,我們這個國家告訴你,交易稅可以替代所得稅。我學稅這麼久,我還不知道,原來這兩個稅可以互相替代。一個在經濟活動的末端,一個在經濟活動的前端。一個不管你盈虧,你只要有交易就要繳稅的,那一個是有所得,就要課稅,有虧損要准予減除。兩個稅目,在經濟活動的不同端點,然後稅制適用的法理基礎又不一樣的情況底下,如何能夠說他是以稅代稅,那大概是我們的立法者或者是我們國家的法制環境很特殊,認為證交稅可以代替證所稅。

簡單來講,交易稅並沒有辦法適當地替代所得稅,因為交易完全顯現不出稅捐負擔能力,所得可以顯現出稅捐負擔能力。交易,只能說顯現出他增加行政上的成本。理論上,交易在一個未受管制的情況底下,透過供需雙方看不見的手,有人需要有人想要,有人不要,那這個時候他們透過契約的締結,他們就可以自然形成交易行為。交易稅的課徵其實是在因為他產生行政成本。為了反映行政成本,製造行政成本的人要去負擔該項行政成本,因此交易稅的正當理由,在於費用的回饋跟負擔。你增加行政成本,原則上就透過你的交易稅的課稅來回復你增加出來的行政上的成本。所以交易稅本質上可以被規費給替代,也就是如果增加的行政成本是在行政機關裡面,其實可以收取行政規費來替代交易稅的課徵。那麼交易稅也因此在現代法秩序,特別是在量能課稅底下,並不存在著正當性。交易稅,毋寧是因為他的歷史久遠,到目前為止,我們仍然存在著許多的交易稅,作爲量能課稅原則的前導,但是不是真正反映出量能。因為交易之後就是消費,那麼原則上他只是去推定交易越多、交易越大量的,你消費能力更高,從而你應該要更有能力去負擔稅捐。那么,在這樣一個前提底下,交易稅基本上不足以適當地去反映出一個人的稅捐負擔能力,換言之,買賣越多的人不一定有更多的稅捐負擔能力。相對於此,有所得的人通常就有稅捐負擔能力。因此,在分離課稅的前提底下,證券交易所得,仍然應該要被核實課稅。

臺灣的證券交易市場基本上是集中交易市場,沒有道理不能核實課稅,所以不願意核實課稅,純粹就是一個政治上的選擇,就是要不要做的問題,要做可能會部長下臺兩次,就這個差別而已喔。不做,那就代表著我們國家的立法者告訴你,努力工作賺錢,你要被課40\%的所得稅,但炒股票是不用繳稅的。就是在告訴你這個基本價值,歡迎大家來到一個炒地皮跟炒作股票不用繳稅的世界裡面。我們歡迎大家來到這個世界,就這樣而已。所以,不需要找好工作,你應該是想辦法裡面去股市裡面去賺到大筆大筆的錢,這個是立法者告訴你的立法上的價值。我同意資本利得的這種交易所得應該要低課稅,各國之所以對證券交易所得稅不會像薪資所得課這麼高的稅負,是因為資本很容易在全球化底下瞬間移動。哪個地方課高額的所得稅,很快資本就會流失到不課高額資本利得稅的地方,這一點完全可以清楚,可以完全理解。所以當周邊國家對資本利得不課高額稅賦的時候,在臺灣單獨要推出一個高額的資本利得稅,確實在現實上不容易可行,特別是當左右鄰近是香港新加坡。因為香港跟新加坡本來就把自己定位為是一個資金停泊港,就是要讓國際資金可以快速來回。香港確實真的不課資本利得稅,所以在當地炒的地、炒股票,原則上只課印花稅,相當於我們的契稅,相當於我們的交易稅,確實是不可所得稅。我們可以很充分地明白,確實,在客觀環境上面,如果周邊國家不課資本利得的所得稅,那我們要單獨去課的時候,我們確實不會讓資本停泊在本地。

但回過頭來,這個確實是一個價值上的取捨,在本地,如果當你課所得稅最高到40\%的時候,也沒有道理,對資本的在本地的取得完全一毛所得稅都不課,因為他等於是在鼓勵人們用錢去賺錢,繳的稅負比用勞動力賺錢繳的稅負更低,鼓勵你去用資本的方式去賺取到報酬而不用課徵任何稅負。這個無論如何,在現代的立法者的價值底下,即使是在國際競逐資本的前提底下,是不具有正當性。

從而,資本利得仍然應該要以核實課稅為前提,也許稅率不高,但仍然必須要繳納一個最低該要繳納的稅負。以臺灣所謂的證交稅替代證所稅,他的稅率是3‰,簡單而言不到1\%。雖然他是用交易總額的3‰,不能夠完全跟所得額的1\%相比,因為交易總額的3‰不等於是所得額。我們可以統計臺灣的證交稅跟在市場上,因此如果有獲得所得的話,那麼他到底實質實際有效稅負是多少,如果不到1\%,他就傳遞出,我們努力工作的人繳的稅,甚至是比在股市裡面炒地皮的某一些知名人士都還要繳的稅負更多。這是在講比例,不是在講數額,因為有一些人詭辯說,他就算繳3‰,他也是繳比你多。你還真的有夠悲慘,因為你很努力地賺了錢,你繳了5\%以後,你的是我比他的3‰還少,因為他是賺幾千萬幾億的3‰,所以他的數額比你高。但在這種前提底下,我們還是會質疑這裡面的一個立法者的價值。資本利得稅,確實在全球化競逐資本前提底下,不太容易高課,但一點所得稅都不課,則是沒有正當性的。臺灣的資本市場裡面不課所得稅,也會扭曲我們的稅捐負擔分配。舉例而言,當你交易所得不課稅,股利所得卻要被課28\%。你到底是在鼓勵長期持有,還是鼓勵短期的套利投機行為?這個是你的一個法秩序上的一個價值上的乖離,也就是本來你應該要鼓勵張忠謀先生長期持有臺積電股票,而不是鼓勵他短期套利。短期套利只有3‰交易稅,長期持有卻要交到28\%所得稅,那我當然是選擇短期套利。當然,張忠謀先生不會用這個稅捐負擔來評估他自己要不要持有臺積電股票,這是另外一回事,我們剛剛講的是小股東的持有股票的方式。

那我們今天先跟各位談到這裡。

第二個階段,其實是在講應稅跟免稅所得,跟第一個階段的差別就在要有法律明文規定為前提,才可以享受稅捐優惠。第一個階段是沒有法律規定,從解釋上也應該做理所當然的推定跟推理,這個是他們的區別實義。

我們今天先跟各位談到這裡。

\hypertarget{section-11}{%
\chapter{20231016\_01}\label{section-11}}

\begin{longtable}[]{@{}l@{}}
\toprule()
\endhead
課程:1121所得稅法一 \\
日期:2023/10/16 \\
周次:07 \\
節次:1 \\
\bottomrule()
\end{longtable}

上周講了應稅所得跟免稅所得的辨別,也就是在稅捐客體的辨認的第二個階段。

我們接下來今天進入第三個階段,也是跟稅捐客體有關。

\hypertarget{ux6240ux5f97ux985eux578b}{%
\section{【所得類型】}\label{ux6240ux5f97ux985eux578b}}

我們請各位看所得稅法第14條,也就是各類所得的類型的這個部分的說明。

首先從綜合所得稅的意義跟角度來看,照道理來講,作為綜合所得稅,名稱就叫綜合,分類是沒有意義的。因為,既然稱之為叫綜合所得稅,簡單來講就是薪資所得跟營業所得、薪資所得跟財產交易所得,1分錢就是1分錢,1塊錢就是1塊錢,那既然如此,為什麼還是需要分類?所以首先第一個我們必須要談一下,為什麼有14條這個規定的必要性,跟區分的實益。

\hypertarget{ux4e09ux9edeux5340ux5225ux5be6ux76ca}{%
\subsection{【三點區別實益】}\label{ux4e09ux9edeux5340ux5225ux5be6ux76ca}}

雖然我們的稅法稅目稱之為叫綜合所得,但實際上所謂的綜合所得,他的概念是將各類型所得放在同一個稅基底下,不分彼此你我,也就是如同剛剛所講的薪資所得,跟營業所得,乃至于跟財產交易所得1塊錢就是塊錢,沒有什么差別。但為什麼各國還是會有這一類關於類型的規定,主要是來自於,現實上,我們確實對不同類型的所得做不同的計算的方式,這個地方會導出他的所得的數額,也就是淨額,這個計算最終的標準是不一樣的。

我們舉例而言,薪資所得跟營業所得就有不同的計算所得額的標準方式。

薪資所得在釋字745號之前,原則上他的規定是收入等於所得額,並沒有成本費用可供扣除。相反的,同一個業務的同一個工作的內容執行,假如你是一個營業的狀態,或者是你是一個執行業務者的狀態,你則至少在一定的要件前提底下有一定的成本費用是可以做減除的。從而雖然叫綜合所得稅,但實際上在成本費用的認列上面,也就是最終的淨額所得的計算上面,其實還是有所差異。這個是特別在實體法的部分,因為說是綜合所得,但所得的額度的計算上面來講,會導致他真正在類型上是有所差異的。

正因為所得額的計算有所差別,理論上,綜合所得稅是不分類型,這個分別是沒有意義的,但實際上這個分別在各國的所得稅法裡面仍然會產生,因為他立法上面會產生一些差別。這是第一點我們所提到的。

第一個,理論上,綜合所得稅1塊錢的薪資所得跟財產交易的譬如說虧損,是得併計的。也就是你在不同的類型所得裡面,你在某一類裡面有盈餘有所得,你在另一類有虧損,所謂綜合所得的概念,應該是你可以甲類跟乙類盈虧互抵計算,這才叫綜合所得。但現實上我們并不是如此。我們各類型所得的實務操作,原則上所有的數額都只計算到零為止,沒有虧損的跨類型計算。雖然名稱叫綜合所得稅,但實務的操作上卻是跨類型的盈虧互抵不存在。一個薪資所得者,譬如說他有薪資所得,他有財產交易虧損。我舉例而言,一個公務員也是有薪資所得所得,他買賣黃金虧損,那可不可以把薪資所得跟買賣黃金的財產交易虧損來去做綜合所得稅概念底下的盈虧互抵呢?因為照綜合這個概念,應該是不分類型才對。答案是不可以。我們只容許同一種類型,在法律有明文規定的情況底下去做同一個類型的跨年度盈虧互抵。財產交易是唯一一個可以做同一類型內跨年度盈虧互抵,是唯一被容許的。其他的類型我們不容許跨類型,也不容許跨年度。

我們的實務操作並沒有很明白的依據,他是透過一種類型的解釋體系的解釋法,也就是法律有明定的只有給財產交易虧損做跨年度盈虧互抵,不給非財產交易所得的其他類型有這樣一個跨年度盈虧互抵。因此,他透過這種反面解釋的方式,法律給你這個類型的跨年度盈虧互抵。那至於為什麼跨類型不可以?待會我們來看一下所得稅法的14條文字,從那個地方裡面,實務上認為不給跨類型的盈虧互抵的這種可能性。

因此,這是第二個,為什麼在我們的稅法的規範上面,或是我們在法律適用上面,我們仍然還是必須要去討論關於所得類型這樣子的意義跟區別的實益所在。

這兩個都是實體法,也就是稅捐數額的計算上面是有直接密切的關聯性。剛剛第一個我們講到不同的類型,有不同的成本費用認列的方式、準許的範圍。在釋字745號以前,薪資所得原則上沒有成本費用可以做認列扣除,他是一種用概括形態的成本費用,去做薪資所得的計算,這個是我們到後面會配合主觀淨值原則的時候,會再跟各位去說。到所得稅法第17條的規定的時候,再一併去做更進一步的說明,先請各位記得有這樣一個實體的,也就是稅額計算上的區別實益。第二個是,所得稅裡面,儘管稱之為叫綜合所得稅,但實際上我們實務操作上是不容許跨類型的盈虧互抵的計算,這個也是一個區別實益。當然,更進一步去講的話,就是不同類型的所得會產生不同類型所得該有的成本費用,也就是說,為什麼去區別類型,原則上你的成本費用只能夠是來自於該類型直接產生出來的成本或者費用才準予減除,也就是在客觀淨值原則底下,所謂的所得額來自於收入減成本減費用減虧損後的净額,這個彼此都是聯繫在你所做的那個該項行為,他在事務之本質上必然會產生出來的或依照法律規定必然產生出來的增加或者減少。收入是增加,成本費用虧損是減少。從而在這裡面,特別在學會計學裡面,他們有稱之為叫成本費用去對應收入的對應原則,成本費用去配合收入的這個對應原則,在我們稅法上面把他稱之為叫關聯性,就跟你的活動有關,或者是因為那個地方翻譯不太一樣啊。他的德文叫Veranlassungsprinzip,因為那個活動的起因,因為那個活動跟那個活動有關的可以認列,跟活動無關的,不能夠認列該相關的成本費用。會計上是直接把那個數額裡面的成本費用去對應收入,這個從法律的觀點來講不太對。因為我們不是收成本費用去對收入,我們是對活動。會計學上面,在算他的會計淨值的時候,是成本費用去對應收入原則。法律上的概念是不是對應收入,萬一你的活動沒有收入,那怎么對?萬一沒有收入,我做這件事情完全沒賺到錢,但我有投入成本費用。只要是應稅的營業執業活動,對應的是活動,而不是對應收入。這個是我們跟財會之間的差別,我們講的對應是在對應那個活動,只要他是應稅活動,即使他沒有正的收入,產生成本費用,客觀淨值原則的計算底下他會有虧損,他仍然還是可以被認定為應稅的所得計算裡面的項目,因為,我因為該項營業活動或是執業活動,我沒有產生任何收入,但我有成本費用支出,從而成本費用減掉收入,我有一個虧損,那這個虧損等於是我營業或執業活動產生出來虧損,儘管這個概念似乎有這樣一個差別,但大多數非學會計或非學稅法的人不太能夠去理解這中間的差異。

稅法上,我們不是對應那個收入,是對應活動。翻譯上面來講,我目前為止,我自己也沒有找到一個比較比較合適的方案,因為他的德文Veranlassung,是一種跟某一個東西有關聯性,起因於這個活動而產生出來的收入或成本費用。那會計上因為是直接成本費用去對應收入,他就忽略了跳過去對應那個活動的那個概念。德文裡面是直接對應那個活動。那也因此,我最早自己是把他翻成關聯性,是因為我們的所得稅法24條,就是因為跟營業活動有關之一切成本費用,所以把他用這樣子一個方式去做說明。總而言之一句話,在這裡面,你的相關的收入跟成本費用,因為在各類型所得裡面他有對應的活動,而因此產生出實體法上的差異,也就是稅額計算稅負上的差異。

最後一個區別實益在稽徵程序上面,也就是說,各種不同類型所得,他會因為類型的不同,而可能會有不同的所謂的稅捐徵收的程序。一般而言,薪資所得者會有就源扣繳程序。但你如果是自營活動的營利事業或執行業務者,由於所得稅的收取,本身是不一定有就源扣繳。因為我如果做一個律師,我去執業,客戶並沒有必要要就源扣繳,所以是我透過結算申報程序。我如果是一個獨立開店的業者,我也沒有從顧客那個地方有就源扣繳程序,特別是對方,如果他也是一個境內的居住者,那這個時候他沒有就源扣繳義務啊,除非他是一個國外的納稅義務人,他是一個非境內居住者。不然的話對方也是境內居住者,我賣東西給對方,對方並不需要去做就源扣繳程序。我有所得,是透過所得稅的結算申報程序。

從而我們從實體法到程序法,因為有這些區別實益的存在,因此我們仍然還是必須要去談所得稅法第14條的各類型所得。簡單而言,就是綜合所得稅並非如其所名的所謂的綜合在同一個稅基底下去適用同一組的稅率。我們的綜合所得稅並不是如此的完整,其實還是有各類所得的區別實益,不管是成本費用的計算,不管是成本費用的歸屬認列的方式,也就是剛剛所講的關聯性原則,或者是在成本費用我們特別是在跨類型的盈虧互抵的計算上面來,至於稅捐的稽徵程序上面來講,我們的所得稅的類型分類是必要的。

基於這樣子的一個說明,我們展開今天所得稅法第14條的規定。所得稅法的第14條從歷史的演進而言,他是逐漸慢慢變化而來的。各位可以翻開來我們第14條的規定。我們的14條的規定,本身是一個在歷史演進的下的產物,也就是說,他本身體系不太清楚。

\hypertarget{ux7b2cux4e00ux985e-ux71dfux5229ux6240ux5f97}{%
\section{【第一類 營利所得】}\label{ux7b2cux4e00ux985e-ux71dfux5229ux6240ux5f97}}

所得稅法的第14條的第一類裡面,理論上,總共包含三種不同類型的所得。我們來看一下所得稅法第14條的第一類被稱之為叫營利所得的類型,實際上他包含了三種不同類型的所得。

第一種類型的所得,我們可以把他稱之為叫做獨資商號跟合夥事業經營成果的營業所得。14條,第一項第一類裡面,他稱之為叫營利所得。我先把其中一個類型叫獨資跟合夥經營成果的盈餘,這個部分我把他單獨拉出來,稱之為叫。
因為這一類型的所得,除了資本投入以外,一般而言還需要勞動力的投入。這個稱之為獨資合夥經營事業的營業所得。

第二種類型是純資本投入,也就是透過股本投入公司合作社或者是其他法人類型,比如說有限合夥啊,你透過股本的投入,因此而產生出來的不管你是公司合作社或者是其他法人類型,我們把他稱之為叫股利盈餘。他是一種資本投入所產生出來的孳息。

第三種,我們把他稱之為叫一時貿易盈餘,著重在他的一時性。上面的這兩個類型,在某程度上面,有繼續經營的概念,也就是他有一段時間的持有,因此才能產生包括孳息在內的股利盈餘,包括你的經營成果的營業所得。一時性,在這裡面,對應繼續性,沒有說出來的文字是,繼續性。上面這兩類,有一個繼續性這當中。這個不是我隨便亂說,因為待會我會跟各位重組排列我們的類型所得的概念。

這就是我們的稅法。根據立法者排列出來的體系,我們把他稱之為叫外在體系,是立法者自己的排列順序。但實際上雞跟兔放在同一個營利所得這個類型底下,是雞兔同籠,把不同類型放在同一個類型底下去做稱呼。我們透過類型,是一種價值判斷,也就是一個有意義的分類標準,去做分類,這個被稱之為叫內在體系的分類的標準。內在體系的分類標準可以讓我們有辦法去,對於同類做平等原則的適用。特別是在立法政策上,特別是在法律解釋適用上面,所謂的等者等之,你必須要同類才能等,雞跟兔沒有辦法比,雞跟兔是不能比的。同類這一件事情來講,必須要在本質上相同,像我們剛剛這個地方,我們的立法者的外在體系分類就是一類型分三類,但實際上在這裡面,他裡面有繼續性性質的勞動力加資本投入,其實他是勞動力為主,資本為輔的一種經營成果。也就是獨資跟合夥,不會只有資本投入,他主要是來自於他的勞動力投入,只是他是一個獨立性的繼續性的付出,而因此產生經濟上的成果。所以德國人的法律定義他就很清楚的告訴你,什麼叫營業所得?獨立繼續性地投入市場經濟活動。人家的法律規定就很清楚的告訴你說在這個類型裡面,該有的價值標準。法學後進跟前進的差別就在掌握分寸標準上的精準性。這就是我們跟他們的差別。不是我們不好,是因為我們這個部分,確實立法者反映出他的所處的社會的研究水準。簡單來講,就是學者分不清,立法者當然也分不清。反過頭來,立法者分不清楚的情況底下,法律適用的操作上面也就很容易類型混肴,放在一起。

比較法的作用就是我們看別人怎麼分類,哦,原來他這樣分類,他這個分類有沒有意義跟道理?我們學到這個分類標準,然後運用在我們自己本身的分類的適用上面,這就是社會科學。社會科學是透過一套可檢驗的標準來適用在自己本土的社會,而不是全然照抄。社會科學,是觀察自己所處社會的問題,運用一套可檢驗可被重複檢驗的方法,適用在你自己所處的社會裡面,這個才叫社會科學。

這也因此透過這3個,我們首先先跟各位去做分類,因為待會我要做重組排列。我們先把這個不同類型做區分。第一類的所得叫營利所得,其實包含了繼續性跟一時性的兩種經營的成果的所得,都放在這裡面。繼續性的這一類型裡面,區分純資本投入的跟所謂的資本加勞動力,特別著重在勞動力投入的這一種營業所得。這裡面總共三種不同類型的所得,這個區別是有實益的。你如果今天有勞動力投入的話,一般而言會有成本的費用的問題。可是如果你沒有勞動力投入,你就只有投入的資本裡面會產生成本,而一般而言不太會有費用。當然,交易本身會產生費用。

一時貿易盈餘,是著重在一時性,因此在實務上面來講,不以之為常業的這一類的,經營成果才會成為叫一時貿易盈餘的態樣。舉例而言,像老師也有一個一時貿易盈餘。雖然我的主業是教授,但我副業可能做別的工作的時候,這個時候我就可能會有一個一時貿易盈餘,待會跟各位作說明。

\hypertarget{ux7b2cux4e8cux985e-ux85aaux8cc7ux6240ux5f97}{%
\section{【第二類 薪資所得】}\label{ux7b2cux4e8cux985e-ux85aaux8cc7ux6240ux5f97}}

接下來第二類叫薪資所得。薪資所得的類型特徵在他是一個純粹勞動力付出,而且是一個非獨立性的勞動力。相對的,營業所得,也是一種勞動力付出,但是他是一種獨立性的勞動力付出。簡單來講,薪資所得的勞動力的付出,是有一個雇主的,因此薪資所得的類型特徵,在於透過勞務的付出,而該項勞務本身非具有獨立性------他是非獨立性的勞動力付出而取得的經濟上的成果。

這裡面的公教軍警公私事業職工,這個只是在跟你講,不管是公或私領域,只要透過勞動力付出而產生出來的勞務,提供產生出來的對價,因此就產生薪資所得,在這裡面不分你是在公共行政領域裡面或是私領域裡面的。包括公務員的薪水,雖然公務員是參政的一種形式。公務員的參與國家的政務,本身是一個參政權,不一定被理解為是一個工作權。當然這個跟各位在這個學憲法的時候,到底,成為公務員,是工作權保障的範圍,或是參政權的保障範圍,也許會有一些不同的歧義的看法,但這裡面不管是公或私,原則上是你透過勞動力付出,因此產生出來的取得的對價,就屬於薪資所得。

老師是在公立大學裡面授課,那我主要是透過勞務的付出,因此取得報酬對價。為什麼這是一個非具有獨立性的勞動力?因為我的勞務付出是國立臺灣大學底下的,只是我在禮拜一的下午1點20分的時候,來到1302的教室上課。所以,我的勞務付出是非獨立性勞務。如果我自己開班授課,各位來上課,這個叫獨立性勞務。這個時候我就是一個獨資商號,或者是合夥事業。

所以,陳清秀老師去東吳大學上課的收入,爲什麽不是一個獨立執業的律師的執行業務所得。為什麼?因為他是受東吳大學的指示監督,只是而在固定的時間地點付出勞務去取得對價。主治大夫也是如此,你的所謂的勞務付出是在醫院的指示底下而做出來的勞務付出。雖然在你付出的內容上面具有高度專業取向。例如說我上課的稅法的內容,國立臺灣大學,不會具體地指示我要上什麼,所以我其實可以唸法條就好,我唸完法條,大家混完兩個小時時間就可以回去了。你頂多就是評價我,老師教得很爛。但我已經是教授了啊,不然你是要怎樣?就算評分很爛,我也是這樣,我照領薪水。

這叫非獨立性的勞務的付出形態。勞務的付出,本身是受僱主的指示,雖然在勞務的付出內容上啊,具有高度的專業性,這個時候並不改變他在勞務付出上,原則上是透過雇主的指示而提供勞務,也因此他的報酬本身不直接來自勞務提供的相對人之間。例如各位就是顧客,國立臺灣大學就是雇主,提供勞務服務,只是柯格鐘到這個時間地點來對各位來付出這個非獨立性勞務。

假如我今天是開補習班,各位因為慕名而來,我收錢,那這個時候我就變成是一個前面講到的營業所得者。

我們跟各位去談到不同類型的所得,這個區辨是非常重要,就如同剛剛回到我們今天講課的主題,為什麼分類重要啊?因為這個區辨決定了你的所得成本費用的計算,乃至於稽徵程序,也會有所差別。

第三類薪資所得,這裡面,公教軍警公私事業職工的薪資跟提供勞務的所得,換言之,他這裡面一個重要的類型特徵有寫出來,叫做勞務,但沒有把另外一個重要的類型特徵叫非獨立性寫出來。公私事業職工,他意味著不管是公或者是私------在高權行政領域或者是非高權領域裡面的私法領域,只要有勞務的提供,但非基於獨立性的提供勞務的話,這裡面就會被認為是薪資所得的類型。這個是在這裡面我們立法者的外在體系,沒有表現出他背後該有的基本價值,因為你沒有去表現出那個獨立性,你就不容易去區辨營業所得,跟後面我們會講到執行業務所得,乃至于所謂的薪資所得,這三種不同的勞務所得的類型。他們有一個共同特徵,都是勞務付出,但獨立跟不獨立,結論差很多。

接下來我們跟各位提到薪資所謂的勞務付出,有沒有說是限於合法?答案是,沒說。沒說,代表在這裡面啊,你只要是勞務的提供而取得的對價,原則上就是薪資所得。在德國,是利用職務之機會而取得的這類型所得範圍。在臺灣的實務上,所謂的提供勞務,只限於來自於雇主提供的所得。所以,如果你是利用職務之機會而取得的業務回扣,或者是貪污收賄,不是來自於雇主,而是來自於第三人,不是來自於合法,而是來自於非法利用職務之機會而取得的所得,在我們的實務上不列入薪資所得,而是放到第十類,其他所得。

我們再講一次。我們的法律上的規定其實並沒有講合法非法,也沒有說是從雇主。理論上來講,不論合法非法,不論是不是從雇主或雇主以外之第三人取得。理論上如此,因為你只要是非基於獨立性勞務之付出,或利用其機會而取得,理論上應該都是在計算上面來講,列入薪資所得。因為他的條文規定只有講:凡公、教、軍、警、公私事業職工薪資及提供勞務者之所得。

所以,非法打工的人,他非法打工,他受僱主指示,在後面洗碗,就被查到,他是一個逃逸的外籍移工。那這一種情況底下,他一樣,有勞務付出,他的所得理論上還是來自於薪資非努力勞務的所得,但在臺灣把他因此變成是其他類型所得,也就是第十類,而不是第三類。我們的所得稅的認定對薪資所得是僅限合法,僅限雇主給付,第三人給付不算。所以例如服務費收入,像餐飲業的服務生。在外國很多餐飲業的服務生都會獲得額外的小費,小費是誰給?客人給的。特別在美國消費,往往小費是這一些服務生的主要收入來源,因為他們的起薪是比較低的,所以他們這個國家的文化,是透過他提供很棒的服務來獲得顧客的小費,不是固定的。所以如果你服務很好,你可以給到25\%,這個是美國的標準。當然到後來慢慢變質,好像不給的話,就好像會覺得有一點不太習慣。比如說,明明這上面定價是100塊,付出去的時候要付125塊,那為什麼你支付我110塊?我服務很差嗎,為什麼你不給我25\%?可能服務生會這樣問。德國的習慣是,給到整數。比如說27塊,那你就可以給到28,29,30塊,當然你的整數如果是40塊,服務生也當然很樂意。小費計算的方式跟標準,每個地方的文化不太一樣。要注意在臺灣並沒有這個文化。臺灣的服務費收入都是由雇主收走啊,所以他本質上並不直接成為受僱的這些員工的所謂的其他所得,還是會從雇主那邊進去,所以原則上還是屬於雇主給付的薪資所得。

我們在實務上面,第三人給付的所得,通常是代替原先的這個雇主給付。我舉例而言,在德國勞工做團結抗爭的時候,也就是工會出來去做罷工的時候,工會會發給代金。也就是你這一段時間裡面,假如雇主沒有依照約定繼續給付薪水的話,那麼他們的團結權裡面就包括了由工會給付這段時間裡面的代金。也就是,做抗議,假設你沒有拿到你該有的薪水的時候,你先前繳的這些工會的保險費就會支付你的這個代金。抗議用的這個代金,本質上是所謂的薪資的替代,從而他是薪資所得的類型。在德國,抗議代金理論上就一樣要計入薪資所得的類型,美國文化可能會有不太一樣的這個方式。但理論上基本的概念是,不論合法非法,不論來自於雇主或第三人,只要是因勞務付出而取得有關聯的所得,這個就是薪資所得的類型。

\hypertarget{ux7b2cux4e8cux985e-ux57f7ux884cux696dux52d9ux6240ux5f97}{%
\section{【第二類 執行業務所得】}\label{ux7b2cux4e8cux985e-ux57f7ux884cux696dux52d9ux6240ux5f97}}

我們回過頭來再來講第二類。第二類執行業務所得:凡執行業務者執行業務者的所得,這個就是執行業務所得。至於何謂執行業務者,則請各位去看所得稅法第11條第一項的規定。我們的所得稅法第14條的第一項第二類執行業務所得關於執行業務者的定義規定,在所得稅法第11條的第一項。

所得稅法的第11條第一項,本法稱執行業務者,係指律師、會計師、建築師、技師、醫師、藥師、助產士、著作人、經紀人、代書人、工匠、表演人及其他以技藝自力營生者。

我們的外在體系,也就是立法上的技術,是透過例示的方式列舉幾個執行業務者的類別,再加一個概括性描述的條文規定,也就是以技藝自力營生。從這個「自力」,你就可以看出來,這邊的勞務付出是一種獨立營業的態樣。自力營生者的所得,我們稱之為叫執行業務所得,我把他就直接寫執業所得。簡化一點,所謂的執行業務者,簡稱執業。

原則上以這所謂的四大師------律師、會計師、醫師、建築師啊,跟「師」有關的,被認為是一個執行業務者。

我們透過這些例示的規定,大致上理解立法者想認為心中所想的執行業務者有哪些,但他為了怕將來還是會產生疑義,怕未能窮盡類型,從而立法者在後面補上一句描述執行業務者的類型特徵。因此,執行業務者這個概念,他是一個類型概念,就像各位在學民法的時候的買賣、租賃、承攬、僱傭契約\ldots\ldots 他是在描繪客觀的社會現象,但有時候這種描繪在立法技術上無法窮盡其類型,從而立法者把幾個他認為比較典型類型先告訴你,然後為了怕這幾種類型以外,可能還會有其他類型的時候,那這個時候他透過這個類型的描述特徵,他說他們是用以「技藝」「自力」「營生」,去參與經濟活動,透過這3個類型特徵的描述,去進一步描繪所謂的執行業務者。

這個是我們在方法論上被稱之為叫概念底下的一種主要的類型叫類型概念。他的相對分類是這樣,就是法律概念不確定。法律概念是,例如常規交易,例如一般租金水準,這個本身就不特定內涵。

但屬於比較特定內涵裡面確定的法律概念裡面,我們有一個類型,叫典型的分類概念。典型分類概念就是法律,透過他的定義,以數字和公式的方式截然劃分區別的標準。譬如成年跟未成年、自然人跟非自然人。這個就叫典型概念,或者稱之為叫純正概念、分類概念,因為他可以透過數字和公式去做典型截然劃分。

譬如說,18歲成年。就算你聰明才智跟愛因斯坦一樣,你只要16歲,你仍然是未成年。在法律上面你18歲滿的那一天開始,你就成年了,你可以獨立的為自己去做契約上的締結行為。在此之前,你就是未成年人。當然還有什麼未成年人或者是限制行為能力人,他這個都是用透過一個數字的方式,可以去做一個截然劃分。這個稱之為叫典型概念。

大多數的法律概念不是這種典型概念,也不是不確定法律概念,而是類型概念。也就是透過類型的特徵描述去做一個法律涵攝內容的全面包攝,但這種包攝由於他本質上需要透過解釋,所以他的邊界類型是浮動的,是不清楚的。各位可以明白這3個之間的差別嗎?

不確定法律概念與確定法律概念。確定法律概念裡面,又再分成典型概念跟類型概念。

類型概念裡面,就是用類型特徵去描述類型特徵裡面的核心概念。

以執行業務者而言,就是以技藝自力營生。參與市場經濟活動,這個就是營生的概念。這個是他的核心概念的描述啊。透過例示類型的標示,讓法律適用者在適用上可以去特定他適用的範圍,所以他是一個確定法律概念的適用範圍,但即使是確定法律概念,他仍然也有一定程度上的不確定性。他沒有辦法像典型分類概念一樣,是可以階段劃分的。各位要瞭解這個差別,因為關於法律概念的分類,這個就是方法論的入門。

在這個情況底下,執行業務者是一種類型概念,透過四大師的描述,大致上讓你清楚知道何謂執行業務者。但這個描述往往雖然描述出來,仍然無法窮盡其類型,因為何謂「以技藝」「自力」「營生」,這幾個特徵仍然在實務上面難以絕對區分。技藝,什麼是技藝?比如說,醫師、藥師、助產士\ldots\ldots 復健師呢?復健師算不算執行業務?復健師如果是執行業務者,那幫忙喬骨頭的喬骨師呢?你可以取出各種很接近的類型,在這個地方我們仍然要去判斷他算不算以技藝自力營生?

這正是類型概念的不確定性,雖然說是確定法律概念,但其實包含不確定性。這正是所有的法律都需要解釋的原因。因為法律不在於背法條,而在透過這些類型概念,一方面去評價外在體系的判斷標準,有沒有把該有的特徵把他標志出來,另外一方面就是在這一些該有的特徵底下,我們去對個案個別裡面的涉及到的業務,我們去做判斷是不是所謂的執行業務者。

執行業務者的類型特徵是這樣,也就是這四大師的類型,其實他們有一個共同特徵是經由大學體系的教育底下所養成,再加上國家考試。這是「師」的內容。不管是對會計師、建築師\ldots\ldots 一般而言,在傳統的西方社會裡面都是透過學院式的教育,然後經過一定程度上的國家考試。換言之,一個執行業務者,他的執業內容是經過一定的養成以及國家考試而產生出來的業務內容。

非透過這個學院養成加國家考試,如果不需要這個內容,就可以自己出來付出勞務的,原則上就變成是營業所得。

我再講一次,執行業務所得是業務裡面的一種類型,這種類型是經由學院之養成加國家考試,這一種類型特徵把他獨立出來,就被稱之為叫執業所得。你先客觀辨識這事實,接下來再來談適當不適當的問題,因為我說到這裡,你很快就會發現,所謂的學院養成跟國家考試都是一個在遠古時代在前現代的時候,就延續而來的某種執業養成的路徑。

麵包師傅,就算沒有學業養成,他好歹也要經過國家考試,雖然看起來好像沒有上個禮拜剛舉行那樣子的全國大家結合在一起的律師考試。美容師傅、麵包師傅,大家也都是「師」,國家考試及格。你是國家考試,我也是國家考試,我只是沒有上你們的那個臺大法律系去上課的養成而已。沒有不同,沒有職業歧視。換言之,執業所得,某種程度上面來自於傳統跟文化裡面所承認的一種,要經過特殊的養成跟國家考試的及格的程序而產生出來的一種業務類別。執業所得只是一種特殊的業務型態,這種業務型態當然在某些國家裡面,特意給他一個比較特殊的定位。譬如說傳統上德國律師跟臺灣一樣不能廣告,無限責任------也就是執業上的疏失,是無限責任。他不能跟賣商品一樣誇贊他的功效。像我們律師,誇讚功效就是,委任我,一定贏。訴訟沒有穩贏的。醫療,沒有穩贏的,沒有一定會治好。醫生説,相信我,我幫你開刀一定沒問題,人就算死了也可以活回來。不存在這種可能性。而且律師負擔的是無限責任。這個被認為是一種特殊的信賴關係,也就是對他所提供的勞務的內容具有特殊勞務的信賴關係。

由於律師跟受委託人之間的這樣一個忠實保護的義務,要知悉的業務秘密,他必須要去忠實地去以他的利益來去做最佳的考量。

那麼同樣的,會計師跟建築師都同樣的有這樣一個類型特徵,被稱之為執行業務者。但論其實際,他是一種特種勞務型態。只是被國家考試跟特殊的教育養成被培養出來。

但實務的操作上,其實就算你沒有拿到國家考試及格,你出去執業,密醫算不算醫生?答案是,密醫也是醫生,只是他沒拿到國家考試及格的資格。所以如同我剛剛前面在講薪資所得一樣,非法執業,他只要是一個獨立性勞務,在概念上仍然是一個執業所得,跟他有沒有取得國家考試及格的證書是沒有必然的關連性。密醫理論上還是醫生執行業務所得,只是他沒拿到國家考試及格證書。那就更不用講,沒有經過學院養成,連我們律師考試都不一定要經過學院養成,你可以用同等學歷去報考,你只要拿到必要學分。

所以其實執行業務所得是一種特殊執業類別,我們把他從某些執業類別獨立出來而已。這是第一種執業所得。

第二種執業所得,透過人的智慧發揮而產生出來的這一些創作性活動,像作曲家、音樂家、表演家、舞蹈家,這些都是透過他的人類的智慧財產的創作活動所產生出來,這一類也稱之為叫執行業務者,這是第二類。我們沒有很清楚地去表示出來這一個類型特徵,這裡面只有一個很概括的叫表演人。表演人,包括作曲家、音樂家、作家、專欄記者------非受僱於報社,他不是直接受僱,他可能是一個很有名的專欄作家,像德國很多專欄作家,不一定受僱於圖報,或者是鏡報,結果他寫出來,大家搶著要。這一類的所謂的獨立專欄作家,這個就是所謂的以他的智慧財產而產生出來,獨立付出勞務因此而取得的所得。他是一種特殊勞務的類型。

第三種就是特殊技藝,這裡面就包含了工匠跟經紀人。這種獨立勞而這種勞務的態樣是跟,營業所得是不太容易去劃分的。像是,修鞋的師傅,算工匠,還是算獨資商號?商港裡面有一個引水人在幫忙,在外國輪船,要進港出港的時候,將船舶引出引進。引水人是算哪一種?因此,獨立的技術到底是哪一種?在這裡面,類型概念特徵存在著不確定性。

但歸類為營業所得,跟歸類為執行業務所得,原則上所謂的執行業務所得,就是以「四大師」為首的一個特定的養成國家考試為他的要件,但又被放寬認定到,以智慧財產的表演創作為主的,像舞蹈家、作曲家、音樂家、歌唱家,增加了這個剛剛所提到的這一些第二類的執行業務者,第三類的是以技藝自力營生,以經紀人跟工匠為代表,但這個工匠本身就是一個跟其他營業的類型難以區辨的。

我們待會會去講營利事業。你去看營利事業,就是提供工商農林漁牧礦冶,這裡面特別是商的裡面,商業包括貨物跟勞務的提供,那這裡面的勞務提供究竟跟執行業務者有什麼差別?歡迎各位去做個案個別的判斷,因為這裡面仍然存在非常高的不確定性。我們只能簡單來講,牙醫師幫你植牙,這個是他的執行業務,但同時間賣牙膏牙刷,別人都可以做,那這個時候的所得就會是營利所得。這樣各位該聽得懂嗎?同一個執業,我作為律師,我可以上法庭,這個叫執行業務所得,但我如果做諮詢顧問,做一個package的打包或銷售服務的話,那這個就可能是營業。這樣各位清楚嗎?分類很重要,因為涉及所得成本費用乃至於稽徵程序。這個是環環相扣的問題。

我們先休息一下。

\hypertarget{section-12}{%
\chapter{20231016\_02}\label{section-12}}

\begin{longtable}[]{@{}l@{}}
\toprule()
\endhead
課程:1121所得稅法一 \\
日期:2023/10/16 \\
周次:07 \\
節次:2 \\
\bottomrule()
\end{longtable}

\hypertarget{ux7e7cux7e8cux8ac7ux7368ux7acbux52deux52d9}{%
\section{【繼續談獨立勞務】}\label{ux7e7cux7e8cux8ac7ux7368ux7acbux52deux52d9}}

\hypertarget{ux7e7cux7e8cux8ac7ux57f7ux884cux696dux52d9}{%
\subsection{【繼續談執行業務】}\label{ux7e7cux7e8cux8ac7ux57f7ux884cux696dux52d9}}

執行業務所得,其實他是一種特種業務的態樣,也就是說他是一種特種職業的態樣。只是說透過11條第一項的規定所提出來的一個以技藝自力營生者,這樣一個類型特徵,能不能足夠的去前面去做涵蓋而已。由於差別對待,各位看到14條第一項,第一類營利所得,我們根據各式各樣不同的營利所得的類型,有成本費用的這個認列的問題。特別是獨資跟合夥,因為在獨資跟合夥這一類型裡面,在我們傳統上面,過去的所得稅法都把他當作是營利事業的一種形態。因此各位看到,我們所得稅法第11條的第二項,他也把獨資跟合夥列為一種職業的類別。

第11條的第二項:
本法稱營利事業,係指公營、私營或公私合營,以營利為目的,具備營業牌號或場所之獨資、合夥、公司及其他組織方式之工、商、農、林、漁、牧、礦冶等營利事業。

組織型態,包括了獨資合夥公司跟其他組織方式都在內。營業的內容包括了工商農林漁牧礦冶等。因此11條第二項跟11條第一項執行業務者,他們的共同特徵都在透過自力提供一定的勞務內容,也就是獨立性地,提供服務的內容或是勞務的內容,或是商品的內容,那差別只是差在於執行業務者他的勞務的特殊性。

勞務其實是11條第二項裡面的經營事業的其中一種類型而已,因為經營事業包括了「工、商、農、林、漁、牧、礦冶」。其中「農林漁牧礦冶」的部分,作爲所得稅法裡面的一種類型,也是我們待會會跟各位進一步去探討的。

所得稅法第14條第一項第六類:自力耕作、漁、牧、林、礦。

這個地方,很有趣的,不是有趣,很無聊的,把「農」變成是「自力耕作」,然後「礦冶」後面那個「冶」就不見啊。就這樣啊,這個我們的立法者的外在體系就不太容易對應的起來。你在前面11條第二項,你講到所謂的「工、商、農、林、漁、牧、礦冶」。好,然後再第14條的第一項第六類,裡面所謂的「農」,變成「自力耕作」。當然這個解釋上應該是同一個啊,就是說自力耕作,這個是指農業上的所得。

「漁牧林礦」之所得,待會我們再來進一步去講「冶」,因為也這個部分,特別是跟工匠有關,只要你是以勞務為內容的話,那他基本上就會是屬於執行業務所得。因爲工匠的技藝在11條第一項裡面,是列為執行業務。

所以,我們先從現行法律規範裡面,跟各位去解讀,所得稅法第14條第一項第二類的執行業務所得,跟第一類裡面的營業所得,去做一個區別。待會去跟第六類自力耕作漁牧林礦之所得做一個分別。但這三類所得,他們的共同特徵,都是獨立的去營生,獨立透過他的勞務服務或者是商品的提供,而去賺取經濟成果,也就是取得所得。取得經濟上的成果,這個就是他的所得,只是我們對於他的勞務內容做分類。

第一種抽出來的執行業務所得,根據11條第一項的規定,以四大師為首的,透過特定的養成教育,加上國家考試。

第二種,以人的精神創作所產生出來的經濟成果,作家、作曲家、音樂家、演唱家,這一些都是。專欄的作家、自由的記者,這個都是。運動員也是,高收入的那些運動員,雖然他看起來,如果照我們的看法好像看起來是受僱於球團,但這個在德國都是屬於執行業務者。因為他怎麼去踢球,基本上他有很高的自由度。這個地方確實會存在著不同國家對這個部分的定義上的差別。

然後第三類型就是獨立的自力營生。像工匠跟經紀人。所以你看,保險經紀人,就是屬於執行業務。不動產經紀人,也是屬於執行業務者,而不認為是一個營利事業。

那我們在這裡面,特別是第三類獨立技藝自力營生,這個部分,其實已經跟11條第二項裡面的「工、商」,尤其是那個「商」相關,因為「商」,本質上就是各種各式各樣的勞務跟服務之提供。

從而從法律解釋上來講,執行業務是特種的商業服務,就這樣而已,他是被特別拉出來而已。那這個特別拉出來,當然他有區別對待的實義,因為我們現行法裡面把他列入不同類型的所得:一個是第二類的執行業務所得,一個是回到第一類的營利所得,就是應該被稱之為叫營業所得,因為他是一個獨資或合夥經營事業的這種形態。

所以,原則上,法律適用上是,如果有執行業務的類別的話,優先適用。他是特別的一種業務類別,現行法律上是一個特別的業務類別,立法論上則可以廢止執行業務者,回到一般業別就可以。他只是一種業別上面透過成本費用去做個別的計算而已。

當然執行業務者確實有特殊的業務上要求,比如說不得正義做誇張不實的廣告,以及要負無限的責任這件事情。當然,其實隨著律師會計師醫師,本身在執行業務上面,我們現在已經開放所謂的社團法人。律師會計師,尤其是醫師,可以用社團法人的方式去執行業務,不像早期只有獨資,自己經營職業。各位,我們在律師業裡面,你看到最多的就是自己打天下闖江湖的這一種自己執業的。數個人一起合夥執業,像現在我們的大所,大部分都是合夥型態的,但法律上他其實也可以用社團法人的形態。現在的法律是這樣,我查過。醫生最多。醫生為了要阻擋那個無限責任。因為如果你是今天是一個合夥一起執業,死一個個人,大概你們那個那間醫院就會倒了。

這樣你知道嗎?所以像安隆案,會計師,你今天做一個那個簽證,在美國就倒了。類似的案子,其實在臺灣也應該要倒才對。因為那種都是是無限的責任啦,你把他想像成是一個無限責任的公司的話,哇,那只要一出事,大家就全倒。所以其實理論上來講,如果他成立社團法人,他就可以透過有限責任的方式去把責任擋住。當你可以這樣做的時候,你其實跟一般工商服務有什麼差別?有什麼區別對待的道理?

所以,也許我們的法制的變遷,就是所謂的四大師執業的法律環境變遷,其實也會影響到稅法上面。稅法上我個人的看法是認為沒有區別對待的理由,沒有區別對待的正當性。但現行法制上是有區別對待。因為一個是職業所得,一個是營利所得,也就是營利所得裡面的獨資跟合夥的營業類型。

\hypertarget{ux8fb2ux6f01ux7267ux6797ux7926}{%
\subsection{【農漁牧林礦】}\label{ux8fb2ux6f01ux7267ux6797ux7926}}

好,接下來我們先跳過去,講第六類。「自力耕作、漁、牧、林、礦」,簡單來講叫「農漁牧林礦」所得,我自己就直接就簡稱叫「農林所得」。當然了,在臺灣,林業不會比漁業更多啊,當然你要講農漁所得也可以,只是順不順而已。什麼叫自力耕作漁牧林礦(簡稱農林所得)呢?就是從大自然界的第一次產出,尤其是從土地第一次產出的,透過你的勞動力投入在大自然界,特別是從土地。那裡面漁業養殖漁業算不算土地啊?你可以跟我吵。尤其是如果說去捕魚的,去海洋,這個算不算土地?好啦,我就跟你講,就是大自然嘛,好不好,土地跟海洋裡面產生出來的。你從那裡面第一次產出的,這個就叫農林所得。你去播種,土地長出來稻穀的,去種果樹的,這個都叫農林所得。

農林漁牧,一樣喔,你養羊,就是「牧」。從大自然界第一次產出的,這個就叫做農林所得。如果不是從大自然做第一次產出,而是做一定程度上加工處理的,這個就不再是農林所得,這個就是營業所得。
你只要是獨立的話,你就會是營業所得。所以營業所得,在概念上,是執業、農林所得的補充性類型。農林跟執業所得,則是營業所得的特別類型,是一種特殊的勞務付出類型,跟特殊的商品服務類型。

農林所得,是對於大自然界第一次產出,剛剛所講的從土地海洋,從這個大自然界裡面,你去養殖耕作,因此產生出來的第一次產出,這個部分叫農林所得,之後的加工,這個就已經不再是大自然界第一次產出哦。所以你今天去捕魚,做成魚丸打包賣出去,這個不是農林所得,那個叫工商農林漁牧礦裡面的商業的一種類型,營利所得。

那當然,你今天的組織型態也會影響到你是屬於公司合作社或是獨資合夥商號而做這個區別。

因此,透過你的勞務的付出,特種勞務的被稱之為叫農林,或者是執行業務者,其他一律就是所謂的營業所得。

我待會,我們把他重組以後,各位就可以很清楚的知道這一個類型的類型特徵,主要的經濟成果是來自於一種獨立性的參與市場經濟活動,透過他的勞務再加上資本,而勞務在這裡面是以獨立性,是不是用技藝為主(決定他是執行業務或者是一般營業的標準),是不是從大自然界而產生出來的商品、貨物(決定他是農民或者是一般營業),而做這個差別對待。

這個是我們在談到的所得稅法第14條第一項,第六類跟第二類跟第一類裡面的營業所得之間的差別。

\hypertarget{ux975eux7368ux7acbux52deux52d9}{%
\section{【非獨立勞務】}\label{ux975eux7368ux7acbux52deux52d9}}

接下來我們來談,非獨立勞務,第一個類型就是薪資所得。

那我們接下來跳過去,講第九類退休所得。

第九類:退職所得:凡個人領取之退休金、資遣費、退職金、離職金、終身俸、非屬保險給付之養老金及依勞工退休金條例規定辦理年金保險之保險給付等所得。

退休所得跟薪資所得,類型特徵都是來自於一段期間的非獨立性的勞務的付出。在非獨立性勞務的前提下,從現在的僱傭關係,從現在的非獨立勞務的付出裡面,直接從僱傭關係的雇主所取得的,依照我國的現行法規定,這個是薪資所得的範圍。等到你的僱傭關係結束之後,這個時候所產生出來的不管是來自於雇主,或者是保險單位------我們保險單位包括了保險的機關(構)------依照法律規定所做的,或者依照契約所做的保險金給付。這一個部分就會變成是第九類的退休所得。特別是照顧退休後的這一段期間裡面的,因為退休關係而產生出來的保險金給付或退休金的給予,這個是屬於退休所得。

當然,我們法條的規定是夯不隆冬把很多東西稱之為叫退職所得。退職所得跟薪資所得的差別在,對於僱傭關係的現在的給付,或者是對將來的給付,也就是你領到的這一個所得,如果是來自於先前的僱傭關係的延續,產生出來的不管是雇主給付還是保險單位所做的給付,依法依契約所做的給付,理論上就會進入退休退職所得的範圍。

所以,我們的法條規定是退職所得,在德國,是薪資所得的一環。因為德國的薪資所得,包括了現在的非獨立勞務所提供的對價給付,以及為了將來,以及為了過去的僱傭關係。

將來,譬如說,假設各位在日本,大四的學生通常就已經被大企業預訂,喔,你畢業就直接進我們Sony工作。你現在領到的Sony提供給你的,就叫做為了未來的僱傭契約所做的給付,所以你領到薪資所得。薪資所得包括為了將來,為了現在以及為了過去的僱傭關係,因此所取得的給付,不管你是來自於雇主,第三人,包括保險單位所做的給付,都是薪資所得的範圍。所以在德國,退休金給付是薪資所得的範圍,包括你的遺孀遺屬,繼續從雇主或者是保險單位所取得的報酬,仍然是薪資所得的範圍。因為德國認為這是同類型所得,都是來自於一個僱傭關係產生出來的,對現在過去和將來的給付,不管取得所得者是不是勞動付出之本人,都是薪資所得,從而是同一種類型。

但在臺灣,我們把他分兩種不同類型,而且這兩種不同類型的分類,對退職所得是比較優惠的分類,各位可以去看第14條的第九類的退職所得。第14條的第九類的退職所得有各式各樣不同的計算方式。因為我們的退職領取的方式有不同的領取方式。這裡面一次領取者,可以去後除掉一定的數額啊,一次領取的話可以扣一個15萬啊,在15萬以下的話,所得額為零,如果超過15萬,那么則乘以他退職的年資,未達30萬,則以半數為所得額,超過30萬部分則以他的乘以退職服務年資之金額部分,全數為所得額。

也就是在我們退職所得的計算方式裡面,根據你一次領還是是分期領,而做不同的區別對待。立法者並沒有很清楚的交代,但我們可以去這樣理解,當你是一次領取,會有累積多年所得一次大量實現,在綜合所得稅是累進稅率的前提底下,會被大量一次急劇增加稅負的問題,從而在我們的第九類第一款,一次領取的計算方式上面給予特殊對待。

我個人不認為他是一種稅捐優惠。因為這是為了避免一次領取退職所得產生的所得急劇集中在一年實現,產生大量稅捐負擔的問題。雖然確實你可以去講,那都是同樣是累積多年所得一次大量實現,實現立法者為什麼在14條第二項的變動所得,跟14條第一項第九類的退職所得裡面做不同的區別對待?

沒錯,很好,這就是各位學過平等原則,上過課以後,你就自然會產生出來的思考,立法者做了一個可能說不出來正當理由的區別對待。同樣是累積多年所得,但我們的立法者做了一個不同方式的區別對待。有沒有違反平等原則,確實存在著疑慮。但回過頭來,如果光以這個條款來看,他並不應該直接就被認為這是一種稅捐優惠規範,理由在於累積多年所得一次大量實現會有稅負級距而增加的這種稅負提高的不利,這是為了避免這種不利益,只是他採取的手段,對同樣的累積多年所得一次大量實現,做了不同對待。在這個地方一碼歸一碼。

我還是要跟各位強調,我不認為那個是稅收優惠,而是避免稅負級距增加,產生出來的不利益負擔,從而用這一種看起來不太連貫的方式。有一個算數的計算方法,這種計算方法半數所得半數課稅,然後有一些是所得免稅,那是認為說哦,如果是這種情況底下,我們就不計入所得的課稅的計算的範圍,那當然超過一定數額的話,我們就用半數啊,如果超過一定數額,則用全數。

這裡面確實存在著立法者的形成空間,立法者也在這裡面對同樣性質的累積多年所得,做了一個不同對待。但這個地方一碼歸一碼,個別去處理。

我現在要講的是分期給付。退職所得,當他用分期給付,用按月給付的時候。今天用按月給付,全年領取總額可以減除65萬後的餘額來做為所得。所以假設薪資所得者退休後所得替代率是100\%,我又選擇按月給付,我根據所得稅法14條第一項第九類的第二款規定,我可以減除65萬的餘額來做為所得額。這就是很明顯而赤裸裸的對現在的薪資所得者不利差別對待。講清楚明白一點,就是退休的比在職工作的,稅負上更好,因為他可以扣60萬,你不行。

這個如果沒有違反平等原則,那還有什麼違反平等原則?我如果是一次領,當然你會說,為了避免稅負級距產生不利的對待啊,這種情況啊,我當然我可以選擇,用一個比較有利的計算方式。可是當我今天是分期給付,說實在話,為了過去的僱傭關係跟為了現在的僱傭關係的區分,假設你今天的所得替代率100\%。當然因此我們要檢驗,目前的職業年金保險的制度有沒有所得的100\%替代? 假如啦,我只是假設一個狀況,假如所得替代率,退休後,像在希臘可以110\%,又有我們這個65萬的減項,那請問你,你今天50歲可以退休,你會不會退休? 傻瓜才不退。太愛國了,想不清楚才會不退。

因為我現在如果在職的時候領100塊,退休可以領110\%,然後我又可以有65萬的免稅額,為什麼不退休?

這個就是稅制做出區別對待。像分期給付這件事情,尤其讓類型特徵就更接近他跟薪資所得者的同樣的待遇的情況,為了現在跟過去的僱傭關係而受領的給付,做不同的區別對待,在我國法制裡面特別看得出來他作對現職的勞務付出者,他有不利的稅捐待遇的問題。

當然我們這個地方是以退職所得,跟其他的這個薪資所得去做一個對比。本質上面退職所得,其實包括了獨資跟合夥事業,為了獨資資本主跟合夥事業的事業組,也就是執行業務合夥人跟合夥人本身他們所提列的退休金給付。因此,退職所得,在我們的法律規定裡面,也包括了營利事業,以獨資跟合夥作為前提,未來,他為自己所提供的退休後的給付。雖然這個地方在我們的實務上,這一類型操作,獨資跟合夥為自己提列退休金的情形並不是那麼普遍。但理論上來講,退職所得也可以在前面,你是一個勞務付出,你只要是一個營利事業的形態的獨資跟合夥,其實還是有可能,還是會存在著這樣的一個退職所得的形態。

我們把勞務付出的類型就說到這裡。接下來進入資本的投入。

\hypertarget{ux8cc7ux672cux6295ux5165}{%
\section{【資本投入】}\label{ux8cc7ux672cux6295ux5165}}

資本投入所取得的第一種類型的所得,最大宗的叫做孳息。因為所有的資本投入,只要持續一段期間,一般而言就會產生孳息的所得,這類孳息的所得有以下的不同的名稱,包括了第一類,透過投入股本,進入公司合作社或有限合夥成為股東,因此而產生出來的第一類,我們稱之為叫股利盈餘的營利所得。這是第一種孳息的類型。

第二種孳息的類型是第四類,利息。也就是資本的型態,如果是金錢或債券,包括公債、公司債、金融債券跟各種短期票券在內,所有的這一類都叫做利息所得。

資本的本體如果是股本就會產生股利盈餘,是第一類所得,法律名稱叫營利所得。如果你的資本類型是金錢或債券,包括公債、公司債、金融債券跟短期票券,各種存款跟其他貸出款項之利息。在這裡面,公債不分是中央國債或是地方政府發行公債。公司債,我們也不分。原則上公司在跟金融債券是發行機構的差別而已,也就是債務人主體不同而已,短期票券這個分類,基本上他也是一個短期的資金融通手段,只是6個月以內而已。所以這裡面,原則上6個月以上是債券,6個月以內的短期資金融通工具,這個就稱之為叫票券,資本本體其實都是來自於金錢,跟金錢轉換成有價證券的標的,只是差在6個月以上稱債券,債務人主體是國家或公法人稱之為公債,或者是私法人裡面的公司或金融機構,差別差在這裡面而已。但基本上都是資本的本體是通貨,或者是以通貨為標的物的所謂的融通資金的工具,也就是借錢,就是資金的借貸。

通貨是一個國家發行的法定貨幣,你可以把他理解成他是持有者對國家的一個債權的主張,這樣各位可以理解嗎?美金等於是我對美國政府的一個債權主張,美國政府要擔保我在市場上對市場上實體物的購買力。所謂的通貨膨脹,就是這個國家發行出來的通貨,本來是作為支付工具,但無法有效地購買他先前所承諾的那些物品,這個時候就叫通貨膨脹,相反的就叫緊縮。新臺幣亦如是,新臺幣是對中華民國的一個請求,確保他在市場上實體物購買的實質的購買力的一個憑證。所以基本上債權是通貨的衍生性商品。今天如果我借給你錢,我把他拿去做有價證券化,變成是債之衍生性金融產品,這個就是衍生性金融商品的類型態樣。我可以再繼續,我可以做swap,做交換,我對你的債跟你對他的債,我們做交換,然後我再把他去做有價證券化,就再發行出去,這個就是衍生性金融商品的概念。

回到我們這裡面,股利盈餘這個地方,如果你的資本類型是通貨獲得,是通貨衍生性出去的債權的話,原則上你取得的都叫利息。

資本的本體如果是第五類,接下來進入到有體物。不管是不動產或者是動產,就會產生租賃所得。第五類裡面還有一個資本的本體是無體財產。無體財產的權利金所得,也就是他是一個使用權的讓與,而不是所有權讓與。資本本體是無形資產,無體財產。無形資產是會計用語,那在我們法律上面就是無體財產。無體財產使用權的讓與,這個是權利金。有體財產裡面,我們再來去看,不動產跟動產則是租賃所得;以金錢為對象的,我們再分成是股本的股利盈餘營利所得或者是利息。

這幾個都是產自於一段期間裡面的孳息所得,因此只是差在資本的本體類型有差別而已。他們都是一種孳息所得的類型,包括:
第一類的股利盈餘,排除掉獨資跟合夥這種營業所得,也就是你透過你的股本,你去投資公司合作社或有限合夥組織,因此所產生出來的這個孳息,這個叫股利盈餘,也就是第一類的營利所得。
第二種所得,資本的形態是通貨,或者是以通貨為標的債券,6個月以內的叫票券,這一類的所得孳息就稱之為第四類的利息所得。
如果你的資本類型是有體物,不管是動產或不動產,因此產生出來的孳息,這個叫租賃所得。
如果你的資本是無體財產,包括專利、商標或者是著作權,你授權給他人行使而取得的,這個就叫權利金所得。

這些是因為財產本體和資本的差異,而產生不同的所得的類型,在這裡面,包括在孳息所得這個概念底下。

最後面我們再來談一個財產本體可以做交易,因此財產類型的所得會產生第二種類型所得,叫做財產交易所得,又稱為轉售價差,又稱為資本利得。財產本身可以產生一種轉售上的價差。我們的法條規定稱之為叫財產交易所得,在第七類。外國立法例上有可能稱之為叫capital gain,叫資本利得稅,那一般通用的說法叫轉售價差。

\hypertarget{ux5c0fux7d50-1}{%
\section{【小結】}\label{ux5c0fux7d50-1}}

財產會產生兩種所得,一種叫孳息,一種叫資本利得。所以臺灣俗話講「錢四脚」,就是錢又可以賺孳息,又可以賺利得。為什麼「人兩脚」,因為你的勞務投入只能夠一天24小時。所以臺灣俗話講「人兩脚,錢四脚」,就是這樣來的。

我透過這個圖形,我就告訴各位一件簡單的事情,找好工作不一定是好事。有本事跟銀行撈錢出來,你賺價差的這個叫資本利得,賺孳息差的就是你的利息所得。所以人永遠跟不上錢的速度,所以資本家們是用錢去賺錢。我們傳統的教育告訴你應該要找好工作,好好努力學習法律。說穿了,也就是這樣,法律上,同樣是所得,但做不同對待,所以要賺錢絕對不是找好工作。要賺錢,就是想辦法用錢賺錢。這是這一堂課給各位最大的禮物。

那為什麼會有這樣的一個差別?因為稅制上面就是根據這個類型做不同的區別。勞動力投入,獨立跟非獨立這兩種類型。非獨立裡面,現在的勞務投入,取得薪資所得。因過去的非獨立勞動力投入取得的,這個叫退職所得。

獨立勞務,分三種類型:普遍性的,作為補充性的叫營業所得;特殊勞務的執業所得;特種貨品,大自然界第一次產出的,農林所得。

財產的所得,這個部分。第一,孳息。資本的形態是股本的,叫股利盈餘。資本的形態是通過通貨、債券或者票券的,這個叫利息。資本的形態是有體物的,是租賃所得。資本的形態是無體財產,叫權利金所得。

財產,還有另外一個,是資本利得。

各位,腦袋裡面有沒有這個圖像?你有這個圖像,不用看所得稅法第14條,大致上就可以全部掌握。這樣可以嗎?這樣你才會知道什麼叫等者等之,什麼叫稅捐中立性?同樣一個類型的東西,理論上要同等對待。你如果沒有這個類型,對你而言,所得稅法就是一堆枝枝節節的法律條文的拼湊,你才會一天到晚那邊字字句句斟酌。這真的是沒辦法的事情,因為我們的立法的外在體系極其糟糕,沒有辦法看出來那個同類。真正好的立法體系,就是可以透過立法的結構就讓你分出來。德國原則上大概就會有這樣一個立法結構。所以學德國的所得稅法,確實有幫助於人們從法條規範,就很快可以掌握他們該有的類型。但臺灣不能,還真的是,沒辦法,不太容易。此前有同學問我說,老師,有沒有課本?我就跟你們講沒有課本。我們現在目前的稅法教科書有這樣去分類所得嗎?有這樣的類型,就可以看得出來這個不同類型的手段不同對待,真的是有區別的。

那我們今天先談到這裡。所得的類型,除了繼續性,我還有一個一時性的,還沒講。所以我們下個禮拜再從所得的類型再發展下去,包括一時貿易,包括第八類的機會中獎,包括第十類的其他所得。下個禮拜我們都會去談。各位,這個禮拜務必要把這幾種類型在腦海裡面把他串起來,把他連起來,那這樣你才會比較有一個整體性的概念。

我們今天先說到這裡。

\hypertarget{section-13}{%
\chapter{20231023\_01}\label{section-13}}

\begin{longtable}[]{@{}l@{}}
\toprule()
\endhead
課程:1121所得稅法一 \\
日期:2023/10/23 \\
周次:08 \\
節次:1 \\
\bottomrule()
\end{longtable}

\hypertarget{ux7e7cux7e8cux8ac7ux6240ux5f97ux985eux578b}{%
\section{【繼續談所得類型】}\label{ux7e7cux7e8cux8ac7ux6240ux5f97ux985eux578b}}

在上個禮拜,我們講到所得的類型裡面,我們區分勞務的投入或是資本的投入,二者原則上應該都具備一個特徵:繼續一段期間內才能獲得財產上的增益。

(圖示,勞務所得、資本孳息、財產交易之類型)

特殊技藝獨立執業的這一個類型,在工商的部分,其實毋寧比較是偏向於獨資跟合夥。因為如果他是屬於公司組織的公司、合作社、依有限合夥法成立的組織型態,這一個區塊,原則上就會進入營利事業所得稅的稅捐主體的範圍,課徵營利事業所得稅。當然在這裡面有交界的領域啦,因為這裡面是一個獨資跟合夥。我們在民國86年以前,全部都是營利事業所得稅,因為獨資合夥跟公司、合作社,那時候還沒有出現有限合夥,基本上是營利事業的不同組織的形態。86年以前的所得稅對獨資跟合夥是最不利的。因為他們原則上被課了兩次稅捐,當時候是25\%課一次,75塊,再來乘以40\%,所以等於是25+30,是55塊錢的稅負,等於稅後净所得只有剩下45塊。86年以前是這樣。87年開始到民國107年之間,我們兩稅合一的制度,這個時候他是消弭了,特別是獨資跟合夥的重複課稅,因為只剩下對個人的營利所得的課稅。但造成了非獨資合夥,特別是一般組織型態利用公司合作社,在課徵稅負上的極大的依從跟稽徵成本上的負擔,也就是那個兩稅合一的設算所得稅額扣抵制度。設算扣抵稅額的公司帳戶,對特別是公開發行股份的公司來講,因為他們的股東本來就是一直不斷在變化,在所謂的基準日之前,基本上還是一直可能在變化的過程當中裡面。從而造成了上市櫃公司一個極大的,我們稱之為叫依從成本上的負擔,因為公司為了要設兩稅合一的可扣抵稅額帳戶,一方面很大很大的成本,為了要避免超額分配盈餘,又做了一個可扣抵稅額帳戶的可扣抵稅額比例的分配,也就是當年度你原則上可以被扣抵的稅額,只有在那個限度,也就是要分配給自然人股東的那個盈餘限度範圍內,你們去做課稅,但你如果是做過度的盈餘分配的話,這個地方他有限制的規定。

也因為這個限制的規定,造成了一方面是國家稅收短收,因為本來86年以前,我們是兩稅分立的。對公司營利事業所得稅來講,等於是前階段的稅收,國家收不到,後面到個人營利事業的時候啊,登記個人股東的時候,才會課到所得稅。然後為了防堵大家不分配盈餘,在這種情況之下,我們又加徵一個在其他國家稅制裡面沒有的,變成只課徵未分配盈餘所得稅,但實際上我們既損失大量稅收,又增加高度的依從成本,以及,關於可扣抵稅額的比例而造成徵納雙方極大的爭議。

從而我們終於才在107年的時候把他廢掉,廢掉以後,現在留下一個尾巴,就是這個地方改變成,公司課徵一次的營利事業所得稅,現在是20\%,公司如果發放盈餘,給同樣是公司的,我們再也不課稅,因為第42條的規定,我們就不課稅。只有在公司繼續分配盈餘給自然人的時候,我們才對自然人再去課徵那一次的個人的所得稅,但原則上透過用分離課稅28\%,或者是做稅額扣抵,8萬元額度內。複雜的稅制,這就是我們現在的兩稅合一的這個制度的本身的演進的過程。那留下來獨資跟合夥呢,則是用一套,在德國稱之為叫做兩重計算的方式,在臺灣沒有一個正式的稱呼。所謂的兩重計算,就是營利事業這個層次的時候,先算一次營利事業所得額,但不課稅。然後第二重的計算則是按照他的股份就直接分配盈餘給獨資資本主跟合夥事業合夥人。所以第一重這個營利事業所得稅這個層次裡面,只有做帳務上的計算營利事業所得,然後按個別的合夥股東的股份,在第二重計算裡面,就直接認為他已經把盈餘分配給合夥人,然後計入他個人的營利所得。所得的類型是屬於我們的講的營利所得的類型,就是股利盈餘的分配。

但留下一個尾巴,在我們所得稅法第71條第二項裡面說,在這裡面獨資商號跟合夥事業還是要做營利事業所得稅的申報,只是不用繳而已。繳稅是到了他們的股利盈餘分配給自然人的股東時,因為他就做了第二道的計算,股利盈餘的計算,從而把那個營利所得就直接算入獨資資本主或合夥人的個人的營利所得,來繳納綜合所得稅就好。

他是兩道計算,但是申報這個階段仍然不可免,就是營利事業所得稅還是要做申報。71條第二項有做了一個很偉大的怪異的規定是,但書,如果你是小規模營利事業,則不用申報,我幫你算。一個德國制度裡面簡單的穿透稅制,在臺灣是分兩個階段,然後做申報稅的不同處理,也就是說原則上你還是要報,但小規模營業的營利事業不用報,我幫你算。

所以這個地方,在稽徵程序的協力義務上面來講,有申報跟有不申報的,但都不需要繳。什麼時候繳?直接就算入個人的營利所得,以個人綜合所得稅的方式去繳納。這就是關於我們的這一種穿透性的稅制裡面的這個獨資跟合夥商號,這就是我們偉大而難以説明的規定,因為到目前為止,目前為止也沒有一個正式的名稱的稱呼去說明這件事情。

\hypertarget{ux4e00ux6642ux6027}{%
\section{【一時性】}\label{ux4e00ux6642ux6027}}

我們今天跟各位從這個地方再開啟另外一個可能性,叫一時性。一時性,意味著他是一種偶然,機會。遺產稅也被認為是因為你爸爸死掉,被繼承人死亡所留下來給你,我們不能預測誰死亡,但是他死掉的時候突然留給我的。

我在這個地方,我先跟各位去描述一時性。在我們的所得稅裡面有三種類型的一時性的所得。

第一個是一時貿易盈餘。
第二個是第八類的機會中獎,競技競賽及機會中獎。
第三個,是第十類的其他所得。

這三類都是偶然機會發生的。一時貿易盈餘,本身就告訴你,一時性。老師的職業本業是教授,但我也有一個一時貿易盈餘,因為我在我家的上面裝一個太陽能光電板。你如果去德國,可以看到現在目前很多家庭,房屋上面本來以前不管是鋪什麼東西,現在很多都是改鋪成太陽能電板。

其實我們臺大如果有興趣的話,搞不好可以在我們頂樓上面鋪一個。很多公寓大廈,現在都是用這個租給人家,然後就可以收錢,是一個創造收益的一個收入來源。所以要透過規模經濟的方式的話,飛機這樣過去的時候,看到整個臺大一片全部都是那個太陽能光電板。

我在我家裡面上面裝了一個太陽能光電板,我賣電給臺電,這個叫我的一時貿易盈餘。我不是以之為業,但我兼差賣電。但我們這樣講,我如果經常性賣電,我賣的規模不是只有我家頂樓,我去你家,去學校,去農場,我去到處去房屋上面蓋這個太陽能光電板,我就會從「一時性」變「繼續性」。然後如果我自己去做這個生意,這個是獨立勞務裡面的哪一個?

營業所得。

類型化的功能,就是幫助各位整體掌握區分彼此你我。

這個圖請各位帶在腦海裡面,你就永遠不會忘記。所得的類型其實有包含「繼續性」、「一時性」,這也是我現在跟各位講,為什麼所得,我們不是用市場經濟活動的那個所謂的源泉理論,因為源泉基本上是掌握孳息跟勞務,資本利得就沒有掌握,所以我們的所得稅法絕對不是用源泉理論。相反的,我們是包括繼續性一時性,我們的勞務裡面不管你是在是在私的事業,或者是在公領域裡面服務,只要你付出勞務的,基本上都是所謂我們薪資所得所掌握的公私事業職工裡面的薪資所得的範圍,所以我們也不是全然採取所謂的市場所得理論。我們是把公私領域裡面的,包括你的各式各樣的勞務提供或者是非勞務提供,有可能是來自於其他所得。因為我們的薪資所得的類型,只限於現在合法雇主提供的,如果你是一時之機會利用職務工作之機會看到顧客的錢包,突然就想起來,哦,原來我可以把他偷走。竊盜。不法的哦。不法的行為,利用職務之機會收取的回扣,公務員利用職務之機會所收取的賄賂,就會是其他所得。要不要課稅?答案是要,只是不是原來的薪資所得,就這樣而已。

做不同類型的歸類,會幫助我們去比較同跟不同。也許這個地方,獨立性勞務不需要分那麼細。農林、執業跟營業,其實是營業的特殊變化特殊類型而已。是大自然界的第一次產出或是延伸的加工性商品,乃至於特定的職業類型的一種特定的形態,基本上差別在成本費用,差別在勞務的提供規制管理上又會有不同規定。但是對稅來講,我們看的是錢,我們不是在看工作管制規定。從而對稅來講,錢才重要。這裡面因此重要的區別標準叫「獨立性」。何謂「獨立性」。

這個差別,我在上個禮拜跟各位講過。陳清秀老師,平常在執業做律師,他那時候在東吳兼差的時候,還是一個執業律師。今天上法庭賺到的報酬,這個叫執行業務所得,可是今天到了東吳大學去授課,是受東吳大學的指示,告訴你幾點幾分到哪上什麼課。上課的內容,儘管是你的專業,可以去自由發揮決定你要告訴學生的內容,但你仍然是受東吳大學的指示。換言之,收不到學費,不會減少你的報酬,也就是不管盈虧都不會影響其報酬給付之本質。從而,陳清秀老師,從東吳大學取得的叫非獨立性勞務的薪資所得,但同時間可以執行律師業務。所以一個律師到學校演講,他會有兩種所得,一個是他執業的執行業務所得,一個是薪資所得。

我們任何兩種類型都可以比。我在考試的時候就可以考這個,我讓各位去比較不同類型,因為這就是區分差異你我。

一時性裡面包括一時貿易盈餘。老師有一位學生寫的碩士論文就是一時貿易盈餘跟營業所得跟財交所得這三個的區別。

我舉例而言,我賣襪子。你不會今天我賣襪子三雙,你就認為我賣襪子,我有取得所得,有一時貿易盈餘了。不會吧,因為賣襪子以這樣薄利的本質特性,一般來講,連稱之為一時貿易,都很難認為是。所以賣三雙襪子叫財產交易所得。
但是,我自己去跟人家進貨,一直賣一直賣,在某程度上就會開始進入一時貿易。然後之後越賣,生意越大,就會變成是營業所得。各位清楚嗎? 賣東西看客體。

在德國賣房子,你一年賣一棟,算不算營利事業? 要看客體,不是只看頻繁度。這裡面講的繼續性跟一時性,基本上是看客體、價格跟交易的性質上的頻繁性的可能。賣三雙襪子不會讓你變成營利事業,但一年賣三棟就會讓你變營利事業。因為房子高單價。

德國有一個「五年三個客體」界線,是跨年度去觀察一個人從事該項行為的頻繁跟密集性,判斷是否繼續性的勞務。你一年賣一棟房子不立刻就構成所謂的這個營利事業賣房子的營利事業,五年內賣三棟以內,還是財產交易所得。任何一個5年的期間,只要超過三棟以上,第四棟開始,就會量變變質變,讓你變成是一個營利事業的所得,從而變成是德國法制裡面被稱之為叫人合性的營利事業。一樣還是透過兩重計算,但就會變成是一個營利所得的類型,而不再是單純的財產交易所得。

這個叫類型特徵,每一個概念都叫類型概念。因為他是有邊際界線的差別。實務的有趣就在這裡,實務的複雜就在這裡。沒有一個全然絕對的標準,而是綜合整體觀察。
任何兩類三類,我都可以把他湊起來去比對。那就看各位期中考試嘍。

對於一時貿易盈餘,老師到目前為止沒辦法給各位明確界限,因為他不能以之為主業,但又不是單一的一次性的財產交易轉售價差。到什麼程度叫一時貿易盈餘?我跟各位舉個例子是我賣電,你可以想像我賣電不是以之為主業。

那我現在舉一個例子,我是一個醫生,然後我非常喜歡日本,去日本旅遊的經驗非常多,所以我就在我的部落格裡面開始去寫日本旅遊經驗的這一些分享。分享分享到最後面,有些人說啊,柯醫生,可不可以會帶我們去日本做一個日本團旅遊啊。不錯哦。好,那我們就來做日本團吧。但這日本團是這樣,就是買機票十張有送一張啦。就簡單來講就是,大家一起出團啊,大家都買一張機票啊,航空公司會有一個折價,這一張折價就送我了。這樣可以嗎?沒問題吧?所以在這種情況底下,那個折價,你獲得一張免費的機票,算不算你有所得?有所得的話,哪種類型?有趣吧!這是實務案例哦。

因為不同所得類型,有成本費用認列的問題。哪一種類型,會有對應的成本費用認列的限制、證明方法、影響所得額的計算。
所以即使我們是綜合所得,我們仍然很大程度上會需要分類。

機會中獎、還有其他所得裡面最典型的叫賭博的收入,他們都是一時性的。機會中獎,不會「以之為業」。但如果基本上還真的有人以參加各地舉行的競技競賽為他的目標,藉此賺取獎金跟報酬的,還真的有可能轉成「以之為業」。

在德國,一次玩撲克牌,這個叫一時性的所得,在德國是完全排除在課稅所的範圍內,因為他們認為十賭九輸,原則上你應該輸,所以我乾脆就不算入所得的範圍。所以用這種方式把他切出去課稅所的範圍,所以你去拉斯維加斯去拉霸,在德國,你賺到錢都不算所得,虧損也不給你認列。德國人的做法就是把他從一時性繼續性切開來,只有繼續性的活動才是課稅所得。這種好聽一點叫一時性,難聽一點,叫射悻性很高的這種中獎的,不管是合法或非法,原則上在德國被稱之為嗜好,不列入課稅所得範圍,嚴格地切割開來。因為德國人認為有所得就要課稅,但有虧損要準予減除,不可以做半套。
美國法,做半套的話也可以,只有賺到錢的才算所得,虧損不準你盈虧互抵。
我們的稅法就受美國影響,就是說,OK,如果你是有賭博行為,有所得,併計綜合所得課稅,但有虧損,不準予減除。我們的做法是這樣。因此一時性所得也是所得稅的所得,從而我國不採源泉理論。我們是採純資產增加說。不管是繼續性或一時性,或者是採跨越市場所得理論的營利所得,我們原則上只要賺到錢,都會被認為是所得。不管你的行為來自繼續性、一時性、合法性或非法性都是一樣,都是所得,只是歸類為哪一種類型的差別而已。

儘管是這樣,我還是要把一些排除在現行的課稅所得範圍的部分跟各位作說明。第一個部分,財交所得。首先,我們排除掉土地的財產交易所得,105年以後排除掉房屋跟土地的財產交易所得,改課房地合一稅。財產排斥土地,因為土地改課土增稅,105年以後排除掉房屋加土地的房地合一稅。第二個是,財交所得排掉證交期交的財產交易所得,4-1條4-2條,這個地方我們不課證交期交的交易所得稅,所以透過這個立法的圖形,立法者告訴你,要炒地皮要炒股,賺的錢都不用課所得稅,好棒棒。這個就是我們立法者鼓勵大家做的事情。

你這個地方,不管你投入多少的勞務薪資,所得者永遠是最上面這一個,收入跟成本費用被認列的最嚴苛的,也是最沒有機會去做核實課稅的所得類型。越往下的包括利得裡面的證交期交跟房地產的部分,臺灣向來就沒有做實質核實的課稅。所以,不要以為立法者多公平。他在鼓勵各位炒地皮跟炒股,就這樣而已。因為他在稅制上是被有利對待,大家理所當然推想他應該是受國家鼓勵的行為。你可能第一次聽到,但我跟各位講,現實就是如此,也沒有。

好第二個部分,我們把一時性的所得裡面的因為爸爸死掉偶然留下來的遺產跟贈與拉出去分離課稅,所以遺產跟贈與,爸爸生前給的是贈與,爸爸死後給的叫遺產。爸爸不管生前給贈與死後給,都是一時性機會性的所得,沒有違法的問題,他是合法性的,那這種所得為什麼要分開來?簡單來講,就是怕一時性加進來,累進稅率級距升高,就這樣而已。

我們對繼續性所得,也把其中某些類型以變動所得的名義把他拉出去做部分免稅部分課稅,也就是變動所得的一半課稅,一半免稅,所以也是從繼續性裡面只是因為他是累積多年所的一次大量實現,從而,像農林所得,像漁業的一次出海打漁,累積多年所得會透過變動所得的態樣,把他做部分的課稅部分的免稅。這個是我們現行的法律規定所架構出來的分類所得的類型,我們以這個方式來跟各位做一個基本的說明。

這個所得類型的結構我們待會留著,因為我們接下來要跟各位去講所得的實現時間跟所得的實現地點,這個部分有具有要按照所得的類型去做實現的時間跟地點的一對一的判斷的需要。我們待會兒再進一步做說明。

先休息一下。

\hypertarget{section-14}{%
\chapter{20231023\_02}\label{section-14}}

\begin{longtable}[]{@{}l@{}}
\toprule()
\endhead
課程:1121所得稅法一 \\
日期:2023/10/23 \\
周次:08 \\
節次:2 \\
\bottomrule()
\end{longtable}

我們到這個階段,主要就是經過了所得跟非所得的分類,然後所得裡面又分成應稅所得跟免稅所得的這個分類,那麼也進入到所得的類型。

\hypertarget{ux6240ux5f97ux4e4bux5be6ux73feux6642}{%
\section{【所得之實現時】}\label{ux6240ux5f97ux4e4bux5be6ux73feux6642}}

我們接下來進入客體的最後一個階段。原則上所得是不管是勞動力或者是資本投入,不管是繼續性或一時性,你只要因此產生所謂的經濟上的成果,被稱之為叫所得的這個成果出現的時候,在實務上面就會開始要去對你去做所得額的計算跟課稅。

但最後一個階段,我們仍然還是必須要去講,這個經濟成果的表現的方式,因此我們要判斷,所得的實現(實現了沒有)、實現的時間,以及所得的實現地點在哪裡?實現是一個經濟成果的表現方式。要討論實現時的問題,是因爲由於稅捐是以金錢為內容的給付義務,所以所得的實現就是經濟成果的實現,原則上是以金錢為內容的實現,這個時候我才會產生稅捐負擔能力。當我沒有實現以金錢(通貨)表現的方式的話,這個時候雖然我有可支配性,但我沒有辦法拿來做繳稅的内容。

我跟人家做交易,我拿到一張票據,原則上我不能拿這一張票去繳稅啊,因為國稅局不收票,尤其是不收客票。不是你自己本人開啊,那是你本人開可以嗎?也不可以,對不起。國稅局原則上只看到通貨,或者是約當通貨的現金的這些表現的,也就是我們在所得稅法第14條第一項第四類裡面,短票。

「三、短期票券指期限在一年期以內之國庫券、可轉讓銀行定期存單、公司與公營事業機構發行之本票或匯票及其他經目的事業主管機關核准之短期債務憑證。」

也就是,所謂的短期票券,被稱之為幾乎約等於現金,因此又被稱之為約當(約略該當)現金的短票。當然,約當現金,你也可以說他不是現金,因為只有中央銀行發行的法定貨幣才會具有完全的作為通貨的本質跟功能。通貨是拿來作為一種支付工具,除了通貨膨脹或緊縮產生實質購買力的差異以外,原則上一等於一,我們不在這上面去做價值上的變換。通貨膨脹或緊縮,在稅裡面,是被忽略掉的。當然例外的情況,通貨膨脹或緊縮,當一個國家的幣值變動很大的時候,國家必須要有一個反應機制去反映出人民的這個幣值變動極大情況底下所產生的實質購買力的差距,簡單來講,就是依據通貨膨脹去調整當代的那個幣值,透過直接用免稅額後扣除的方式,直接去把通貨膨脹或緊縮所產生出來的差異給消減掉。但原則上通貨本身在稅制裡面來講,就是一個全然的支付工具,而不計算其跟實質購買力之間的差距。

所得之實現,以現金跟約當現金為標的,因為我們繳稅,原則上是以金錢為內容,所謂的金錢是指國家發行的法定貨幣,而在我國的稅捐實務上面來講,跟現金接近的約當現金的這些短期票券,原則上也被認為他已經實現了稅捐上的負擔能力。

所得稅法第14條第二項規定:「前項各類所得,如為實物、有價證券或外國貨幣,應以取得時政府規定之價格或認可之兌換率折算之;未經政府規定者,以當地時價計算。」

這個地方那個取得時啊,取得時就是我們關於所得的實現時的基準點。原則上是以你可排斥他人去干預而取得該項標的或標的物的支配權力之時作為取得時間點的一個判斷基準。所得稅法第14條第二項規定的另外是針對你取得的,如果不是現金或約當現金之實體物,包括有價證券或外國通貨,這個時候則以表定價格,也就是表定的兌換率,換算成以新臺幣,也就是我們的現金為基準的計算標準。

這個就是14條第二項關於實現時間的規定。不是很清楚,因為他其實並沒有直接表現出來現金跟約當現金作為一比一的計算標準,他的表現方式反而是用所謂的,如果你取得的是實物有價證券或外國貨幣,由於他不是本地的通貨,因此他透過一個用政府表訂價格的方式也就是兌換率去折算成本地的通貨,這個時候才實現了稅捐負擔能力裡面的稅捐負擔能力的計算基準。

簡單加以說明,14條第二項是我們的實現時間點。因此,任何只要具有市場流動性之物,只要可支配,也就是可排除他人之干預而歸屬於所得的取得者,這個時候就是一個取得的時間點。儘管沒有換算成現金,在計算上面來講,是用政府表訂價格去計算現金支付的能力。這個14條第二項因此是我們現行法裡面唯一一個表述實現時間的法律的明文規定。

從稅捐本質上是金錢出發,但你取得的標的物或標的本身不是現金的話,則以政府表訂價格,或者是政府規定的兌換率去計算你的現金支付能力,成為稅捐負擔能力,這個是14條第二項關於實現時間的計算標準。也就是在這裡面,政府表訂價格雖然並不代表實質的購買力,因為這一些表定價格往往跟現金會有一段差距,而且因為不是現金通貨,從而他會有所謂的變現的風險。

我拿到一個有價證券,但這個有價證券也許有市場上的價值,有財產上的價值,但仍然不是現金啊。因為我不能直接以這個有價證券去繳稅。因此在這裡面會存在著一定程度上的現金支付能力計算上的差距。就這一點而言,只要變化不大,是屬於立法者價值決定,我個人認為應該基本上尊重這個決定,因為他的變化不大。他的變化是在假設現金,因為本身沒有價值的變化。當然你可以跟我跟我講說我現金怎麼會有價值的變化?柯老師,你到底是活在哪個世界裡面。對了,我知道,但是稅捐裡面,他必須要有一個基準點,就好像我們任何象限,我們怎麼樣,都是要畫出一個零點嘛。這個零點就是假設現金本身實質購買力不變,在這種情況底下,你拿到的東西,好特別是比如說你在外國工作,那你取得的是直接人家發給你美金或港幣啊,那你這個時候你不能繳美金跟港幣啊,那這時候我要怎麼算我們的稅捐負擔能力,那就是依政府表訂價格,把你的美金跟港幣換成新臺幣。

雖然政府表訂價格不是時時刻刻即時對應真正的實質的購買力,比如說現在新臺幣在貶值的情況底下,照道理來講,我應該算比較多一點啦,對不對哦,這個著理論上來講是這樣,那我現在幣升值的前一陣子,新臺幣兌日圓升值的時候啊,那我到來講,我應該沒有那麼高的稅捐負擔能力啦。那這件事情是14條第二項跟你講,立法者就這樣說了算了。在總是會有一個你沒有辦法完全接近的這個情況底下,原則上只要沒有差距太遠的話,那這個通常就會被認為這個是符合稅捐負擔能力的量的課稅原則,特別是在以實現的時間點。

比較大的可能問題,會是出現在依權責發生制。權責發生制是你沒拿到錢,但權利義務已經發生的話,那這個時候我就算你有所得,權責發生制是權利義務發生時就入帳,就算有所得,你比較容易產生那個時間差。

好,回過頭來,那為什麼營利事業就用權責發生制?這個是又是另外一個層次,因為在營利事業,原則上,他就假設你會按商業會計法編列財務報表。所以當你有財報,我稅務我就遵財務直接用一套帳。從財務到稅務裡面就一次去算你的所得,不然我每一次都要去做這種調整很累啊,營利事業要做一個財報給債權人跟股東看,又要做一份稅報給國稅局看,搞死人的。所以稅務遵財務。所以我們才一律採權責發生。避免營利事業的納稅義務人的依從成本過高。在這個情況底下也是一種實用性原則。

因此,量能課稅原則裡面的實現原則有比較高的彈性。原則上是現金收付。但營利事業權採用責發生制,這是另外一個層次的實用性原則,也就是因為稅務遵財務,營利事業不用做兩次帳。兩次帳的意思,不是指內帳外帳,是財報一份帳,稅務又做一份帳。這樣很麻煩。所以現在稅務遵財務,就只要做一份帳,再做稅的帳外調整,這樣就可以。這樣聽得懂嗎?

兩份帳,內帳跟外帳,這絕對是逃漏稅行為,不要在那邊胡搞瞎搞。兩份帳的意思是指說,如果營利事業賺錢,通常一般的情形是他要做一份財報給債權人跟股東證明我棒棒嘛,我好厲害,我賺到錢,對不對?相反的,他面對國稅局的時候,因為我的好棒棒,這是我的經濟成果,國稅局會以之作為課稅基礎,所以我會再做一份的稅務的帳冊來跟國稅局申報我賺多少錢嘛。那這個時候,基於實用性原則底下的我們稱之為叫稽徵經濟原則,那簡單來講就是我讓你只做一套帳就好。那你就做出財報,然後稅務在做個別帳外調整,那這個時候降低了納稅義務人為了要提供財報跟稅報的兩本帳的差異產生出來的依從成本,從而讓營利事業可以在計算所得的時候,直接用一套帳來記帳,來做課稅所得額的計算。只是指依照個別法律規定去做稅的帳外調整,基於稅務目的去做帳外調整。

最典型的帳外調整,權責發生制是這樣,比如說我今天因為排放廢水,被環保局裁罰。在裁罰的時候,這個就要入帳,在財務上權責發生了啊。你說我不一定要繳罰款,因為我會打行政救濟啊,沒關係,打行政救濟,你打不管你打贏打輸,這個是另外講。反正你被環保局裁罰的時候,你基本上就要先行入帳。入帳以後,作用就是避免營利事業膨風,說,我好棒我好棒,多發放盈餘。各位了解那個意思嗎?就是說,哎,我明明賺100塊,為什麼不能發100塊錢的盈餘?答案是因為你被環保局裁罰了50塊啊。雖然你現在實際上還沒有繳納罰鍰,你還在打行政救濟,搞不好你會贏,那沒關係,等你贏了再說,我現在要先入帳。先入帳,你的這個盈餘就只有50塊錢,因此你要發放盈餘是以50塊作為基礎,而不是以100塊作為基礎。

財務有一個叫謹慎或稱之為叫保守原則。這就是財務會計裡面的謹慎保守原則。在謹慎保守原則底下,權責發生的時候,原則上就入帳。當然還有更前端的,就是「期望」可不可以入帳?當然不可以。「期望」只是一個你主觀的想像,沒有發生法律上權利義務的那種期望,不能直接拿來做財務報表編製的基準。各位可以聽得懂嗎?我預期我公司10年後會上市,上市以後會跟馬斯克的那個特斯拉一樣,你信不信?反正我講你就信。可不可以把這種主觀期望放進去?不可以啊。謹慎保守原則說不可以把營利事業自己的主觀期望入帳,你必須要到法律上面的權利義務,也就是權責發生,這個時候,原則上就要入帳。對財務會計的人而言,他認為反而不要現金入帳,因為現金可以被操縱。各位可以知道,財務會計上的基本原則概念了。不能用主觀期望,也不是用後面的現金入帳,因為現金可以因為你現金的支付而去操縱所得實現的時間點,或是以盈餘發生的時間點。所以對財務會計學的人來講,他認為就權責發生是最典型的,這個就是他們講的實現。財務會計學者講的實現原則是指權責發生。所以當你今天賣出去,你有法律上的權利就要入帳,沒收到錢,只是應收款項,或者你應取得的債權,如果是現金直接就入帳,就這樣而已。相反的債務也是如此,應付款項,你沒付,你也是一樣要入帳,付出現金一樣會在你的帳戶裡面表現出來。

從稅務的觀點來看的話。所得的實現原則上是金錢的實現或者約當現金的實現,一旦用債權作為計算基準,反而會產生稅捐負擔能力的扭曲。只是這一種稅捐負擔能力的扭曲,在營利事業所得稅計算裡面,權責發生制是基於稅務遵財務的實用性原則,我們就用一套帳。但如果個人的話,因為個人原則上沒有商業會計法的適用餘地,個人就是用現金收付。現金收付才真正表現出稅捐是以金錢為內容的給付義務的本質。

好,我現在跟各位談一個釋字722號沒有很清楚呈現的一個概念,就是,當我是一個獨資合夥商號,而我的營業規模接近一個組織型態的營利事業,我只要依照商業會計法去記帳,你為什麼不給我用權責發生制?各位了解那個意思嗎?這個地方是不是跟營利事業所得稅接軌,對吧?只是組織型態的差別而已啊。

我現在假設哦,我是一個獨資合夥商號,我做的事業規模很大,跟你們營利事業沒差多少,而且重要的是,我也用商業會計法的方式去記帳。你為什麼不給我用權責發生制?因為你們營利事業這邊用權責發生制啊,我這邊用現金收付制啊。但當我的標準一模一樣的時候,我做的規模沒有比你差,我的記帳的標準也是一樣,我不是單純計日記帳而已,我還記到了商業會計法裡面的帳,那請問你為什麼在平等原則底下不給我適用權責發生制?

這樣各位清楚知道哪些地方是等跟哪些地方不等。你不記帳,就是現金收付,你記帳,依標準規格去記帳的話,讓你有機會轉成權責發生制。。

所以今天本質上相同的東西才同樣對待。等者等之,是指你們兩個做到一樣的方式,當然任何事物都會有等或不等的問題。其實獨資跟合夥就算營業規模大到跟營利事業一般的公司合夥組織一模一樣,你也可以講他們組織方式不一樣啊。這個不是這麼重要,真正的重點是在,在這裡面,如果他用商業會計法去表現出他的盈餘跟虧損,那他就有權利透過他的選擇權的行使,讓他自己轉換成一個用商業會計法用權責發生制來認列所得的,這種營利事業的態樣。

因此,當我一個事業組織體,不管我是農林、執業或者是營業,當我是用商業會計法的方式去做記帳,我的營業規模到達一定程度的話,我的資本的這個勞務的經營有獨立性,要繼續性,我的勞務做到這種程度的時候,我原則上可以轉成用營利事業的方式去課徵我的營利事業所得稅。

這正是我們在釋字722號裡面大概所提到的,就是說一個執行業務者,他不能只是因為工會的代收轉付這樣子一個形態,因為公會代收轉付,根本沒有改變他的性質。而是看執行業務者本身的執行業務的內容,他本身是不是以商業會計法上商業會計的方式去表現出來他的盈餘跟虧損。

這個地方,特別是對醫生,醫生從他的診所業務開始出發到社團法人型的聯合診所,聯合的這個醫療服務的提供,甚至是到大型醫院的提供,基本上只要用商業會計法的方式去記帳,容許醫生們可以選擇用權責發生制,因為醫生最容易有諸多的機器設備,他需要按逐年提供逐年認列的方式去認列。牙醫師要有機器設備,或者是一個醫院,只要一定規模以上,就會有需要。所以你會看到實務上大部分都是醫生,希望自己是執行業務者,可以比照營利事業。他們當然還是不認為自己是營利事業,但他們希望比照像營利事業的權責發生制,可以把他們的機器設備的成本給攤提在這一段經營期間當中。那我們律師跟會計師,剛好我們比較是人類精神的創作啊,我們的機器設備基本上不太多啦,我們不像醫生,機器設備,他們相對比較多,所以律師跟會計師事務所裡面,如果嚴格來講,不是不可以,但比較少。因為做比較複雜的會計帳冊用商業會計法的話,就代表著你的相關的東西,沒有學過會計的大概不會怎麼去記那個帳。會計師自己做自己的帳當然是沒問題,律師大概就會有這個困難。

所以一般而言,你想要改用權責發生制,那個過橋的做法,基本上就是按商業會計法去做,那理論上來講,在這裡面獨立的勞務就有可能讓你應該要有這個機會,可以選擇用權責發生制來認列你的課稅所得。

回到稅捐本質上是金錢給付,就是現金收付,特別是對不記帳的所有的這些人,這一些人基本上一定都是用現金收付制。14條第二項並不是很明顯的,理由在因為沒有講你的所謂的取得時間,也是一個排斥他人的干預,而可以支配,原則上是取得現金,而是他直接跳過去,就規定說,如果你是拿到現金以外,有體物或外國通貨,則以政府核可的兌換率或者是比例去做計算。為什麼要做這樣一個比例計算?答案是因為我們要用我們的通貨來計算你的稅捐負擔能力。這個是14條第二項的規定。

\hypertarget{ux6240ux5f97ux5be6ux73feux5730ux9ede}{%
\section{【所得實現地點】}\label{ux6240ux5f97ux5be6ux73feux5730ux9ede}}

OK。最後一個問題。實現的地點,原則上必須在中華民國境內。因為我們的所得稅法第2條第一項跟第二項都必須,所得實現在中華民國,才會有中華民國來源所得的問題。因為我們綜所稅的所得,是沒有就境外來源所得課稅的,我們只就中華民國來源所得。不管你是我們的境內居住者或非境內居住者。我們的所得稅法又更強化了那個屬地主義的判斷。相對於此,如果是德國跟日本的所得稅法,你是我們的稅籍居民。稅籍居民,不管你是來自於德國和德國以外的所得,原則上都要課德國的所得稅。日本亦如是。你在日本取得跟日本以外取得的所得,你只要是日本的稅籍居民,他就要課全球來源所得。這都是採屬人主義的國家。因此對他們來講,分辨實現地也許沒有你想像的那麼大的區別實益。

好,回過頭來。我們還是要去講為什麼實現地這一件事情在我們的法律裡面很重要,是因為從民國96年開始,我們已經將境外來源所得,改課所得基本稅額條例的最低稅負制。所以不要說再用以往的觀念,所得稅法只就中華民國來源所得課所得稅所得,所以我們就不需要去辨別實現地。

不是這樣,因為臺灣並不是真的全然採用屬地原則,我們其實是變相地從民國96年開始,2007年開始,我們改課所得基本稅額條例的最低稅負制。所以我們的所得稅是境內來源課所得稅的所得,非境內來源所得改課所得基本稅額條例的所得。就這樣而已。這個立法者的價值決定,我認為違反平等原則,但他仍然是立法者的價值決定,沒宣告違憲之前,他就是長這個樣子。

所以為什麼我們最後一章還是要去辨識,是不是中華民國來源所得,是有區別實益。因為中華民國來源所得課所得稅所得。因為非中華民國來源所得不課所得稅所得,改課所得基本稅額條例的個人所得基本稅額,個人的最低稅負制。

立法者做了一個不同的決定。原則上我們就稱之為叫立法者的價值決定。德文講價值決定臺灣到目前為止沒有統一的翻譯,那我直接就用德文的【Wertbestimmung】,所謂的價值決定。

好,這個價值決定有沒有合憲,我個人認為是不合憲。我已經跟各位講過,綜所稅所得免稅額一點點,累計稅率最高到40\%?所得基本稅額條例呢?他是把境內境外合在一起給你一個670萬的免稅額,然後稅率比例稅率20\%。所以光這個稅制並排出來,你就知道,臺灣,你取得境內來源所得,你可以被課到最高40\%。可是如果你同樣這一筆所得,取得全部都是境外來源所得670萬免稅,就算要課,也是20\%。這不是標準的差別對待嗎,對你取得境外來源是提供優惠鼓勵,請問還有什麼情況底下可以正當化這種立法者價值處境?他就是歧視你,怎樣,你跑不掉,關起門來自己打,就這樣。這是一個明顯的對自己本國的取得本地來源所得的稅負上的不利益對待。當然你取得比較高的境外來源所得的話意思是你很有本事,很好,我們可以670萬的免稅額度。因為我們的所得基本稅額條例是境內境外合在一起算一個670萬的一個免稅額度,所以我現在必須要假設,如果境內零,你全部都是在美加取得所得,那這種情況你真的是蠻被鼓勵的,因為你只有超過670萬以上才會被課稅,而且就算被課稅是20\%。所以假設你賺1000萬,你炒美股,炒日本股票你賺了1000萬,講白一點,就是在我們的稅制裡面,就是減掉670萬,然後再乘以20\%。可是你如果在境內,你當然炒臺灣的股票更好,因為臺灣根本連證所稅也都沒有喔,那你在臺灣如果有1000萬的所得的話,你一定是最高邊際,因為我們的免稅額就這麼一點,那原則上就適用最高邊際稅率的40\%,你可能就會有比較多的稅負。

這個是我們的這個,實現的地點,是有重大的區別實益,因為我們的立法者在這個地方做了所得基本稅額跟所得稅的差別對待。在進入真正要去談中華民國來源所得的那個認定標準之前,我們還要先跟各位說一下,在兩岸人民關係條例裡面,我們是用臺灣地區居民跟大陸地區居民去描述我們的稅籍居民。因此,根據兩岸人民關係條例第24條的規定,只要你是臺灣戶籍,在臺灣,他把臺灣地區來源所得跟大陸地區來源所得全部算做中華民國來源所得課所得稅所得。如果你是來自於港澳地區,根據港澳關係條例28條的規定,你如果取得的是港澳地區來源所得,則不認為是中華民國來源,所得是境外的所得,就是不是課徵綜所稅所得,而是改課所得基本稅額條例的所得。如果你是美加,那就真的本來就是境外,所以他就一定是課徵所得基本稅額條例的所得。

你只要是設戶籍在中華民國境內。你取得了臺灣地區跟大陸地區來源所得,所以臺幹,我不管你派到那邊多久,你只要在那邊取得大陸地區來源所得,你也是綜所稅所得。

如果你是被派到香港,好,派到香港去的話,那就香港地區來源所得,根據港澳條例的規定,不課綜所稅所,改課所得基本稅額條例的所得。

那你如果更有本事,你跑去美加賺到錢啊,那就本來就是境外啊,這個本來就是所得基本稅額條例的所得。

所以我們這樣子哦,錢不一定全部都課綜所稅。當你是我們的稅籍居民,戶籍設在中華民國境內,那你就會是臺灣跟大陸算綜所稅所得,香港跟美加都算所得基本稅額條例。

我們接下來講第二種情況,你戶籍不設在臺灣啊,你戶籍改設在大陸,也就是你現在已經變成是大陸中華人民共和國的戶籍的居民的話,你僅就取得臺灣地區來源所得課所得稅所得。

他透過戶籍的方式去區別區辨課稅所得的來源的課稅的稅負,所以你是設籍在大陸。哦,當然了,這個其實到底設籍在大陸還可不可以繼續保有中華民國國籍,這個我不太確定,這個我不敢跟各位100\%地說,我只能說確實我們用戶籍去做區別啊。在臺灣跟大陸之間不用國籍區別,他是用戶籍做區別,因為我們兩岸人民關係條例適用戶籍去做區分,臺灣的地區的居民跟大陸地區的居民。好,這個的區別,他的區別實益是臺灣地區的居民,不管你是來自於臺灣跟大陸,全部都課所得稅所得。如果你是大陸地區的居民,那只就臺灣地區來源所得。

那我們想,大陸地區居民取得大陸地區來源所得,你可不可以課稅?你敢課稅試試看,習近平在北京賺到了錢,你可以跟他課稅嗎?你課不到的,在想什麼。

大陸地區的居民,他只有取得臺灣地區的來源所得,這個時候他才課我們的所得稅的所得。

大陸地區的居民如果取得港澳地區來源所得跟美加地區來的所得,講白一句話,干你屁事?干什麼事?你能課稅嗎?你當然課不到稅了。你能怎樣?香港政府給你申報資料嗎?或者是你期待他們進來的時候,希望他們過境給你提交稅捐申報書嘛?那不可能啊。所以,如果真正是第三地的人民。港澳地區在我們的法制現況裡面,稅制的現況裡面,基本上都是當作境外地區,所以香港的居民澳門的居民,他到這個到大陸地區取得所得,因為香港澳門地區有很多人跑去他們北上進内陸去發展,這個當然不會課我們中華民國所得稅,也不會跟我們有什麼所得基本稅額條例的連結。

港澳地區的居民如果真的來臺灣取得臺灣地區來源所得,而這個時候,我們一樣是課所得稅的所得,因為他是我們的屬地來源所得,他不是我們的稅籍居民,但他會取得屬地來源的綜合所得稅所得。那港澳地區的居民,如果他取得美加來源所得,那當然就不干你事啊,你跟他沒有稅制上的連結。

所以我們大致上就跟各位以這個我們的兩岸三地跟真正的境外地區基本上做一個區別。立法者的價值決定在兩岸人民關係條例,跟港澳條例,以及所得稅法裡面去做稅基上的不同切割跟對待。這個就是我們的偉大的稅捐立法者所做出來價值決定,這種價值決定,基本上體現在實現地的不同,會有稅負計算上的差異。稅負計算的真的差異很大。講白一點,到大陸發展的臺幹,基本上綜所稅是很多的。就賭你不敢換戶籍,你敢勇敢地換戶籍,還真的只課臺灣地區來源所得,就改課大陸的所得稅。

\hypertarget{ux7d50ux8a9e}{%
\section{【結語】}\label{ux7d50ux8a9e}}

我們今天先暫且到這裡。因為我們下個禮拜,請各位務必準備一個「中華民國來源所得認定原則」,因為那個是一個非常長的財政部的解釋令,請各位先把「中華民國來源所得認定原則」先下載。我們下個禮拜要講所得稅法第8條的規定,但第8條的規定有寫跟沒寫差不多,因為他幾乎是14條的重聲,再加一個中華民國來源。我們真正實務的判斷上,則是要回到剛剛跟各位講的中華民國來源所得認定原則。這一個中華民國來源所得認定原則,根本不長期為法規命令,就是一個行政規則。一個行政規則,實際決定的到底要課綜所稅所得,還是是課所謂的所得基本稅額條例的所得。這個基本的價值決定,這個是重大的價值決定,如果用依法課稅原則來講,我國稅法根本就沒有遵守依法課稅原則,因為所得稅法第8條根本就沒有規定這麼細的事項,這些全部都是用剛剛所提到的認定原則,去做一個行政規則上的調整。看起來是一個事實認定規則實際上則是切割稅基的規則規定,所以在依法課稅原則底下,我個人認為應該要用制定法或者是法規命令的這個長期的規定。我們的所得稅法第8條要重新調整才對。但如果所得稅法第8條把整個所得來源地認定原則塞下去,你就知道我們所得稅法會膨脹多大。

我們下個禮拜會再跟各位去講這裡面的來源認定原則其實是有秩序的標準的認定。下個禮拜,除了那個所得來源認定原則以外,我們就是要按這個(所得類型)圖表來跟各位去講他的來源地鎮定大致上是按你的所得類型去做區別的。法律完全看不出來,但是所得的分類真的是決定我們在判斷你是不是中華民國來源所得的一個非常重要的區別標準。大致上我們做一個對照就可以比較清楚的知道背後的基本原則。

今天先到這裡。

\hypertarget{section-15}{%
\chapter{20231030\_01}\label{section-15}}

\begin{longtable}[]{@{}l@{}}
\toprule()
\endhead
課程:1121所得稅法一 \\
日期:2023/10/30 \\
周次:09 \\
節次:1 \\
\bottomrule()
\end{longtable}

\hypertarget{ux6240ux5f97ux5be6ux73feux5730}{%
\section{【所得實現地】}\label{ux6240ux5f97ux5be6ux73feux5730}}

所得的客體,最後一個段落,就是關於所得的實現地。

首先跟各位談一下,探討「所得實現地」的這個問題的實義所在。

首先,所得稅法第2條的規定,只有實現在中華民國來源的所得,才會跟我們的所得稅法產生屬地上的連結。
也就是,只有中華民國來源的所得,我們才能對該所得產生課稅上的權限。

在國際租稅法的領域裡面,對於一個國家或地區作為高權的主體,對於個別的稅捐,或者是我們稱之為叫稅捐財,對個別的經濟活動或者是經濟成果產生課稅權限的正當連結,在國際租稅法的領域裡面有一個所謂的真實連結的標準或原則。

這個真實連結的原則是指說一個國家的課稅主權,可以對系爭的經濟活動,乃至於他的經濟成果去做課稅,要不是是屬於屬人上面有所連結,例如說他是你的稅籍居民的身份,那麼這個時候,國家或者課稅主權區,原則上會對稅籍居民所產生出來的經濟活動跟他的經濟成果享有課稅上的主權。

另外,第二個就是所謂的屬地的連結,也就是這個經濟成果,經濟活動是在你的領土領域範圍內裡面去做去進行的,或者是因此去取得的經濟成果。因此真實的連結原則是一個國際租稅法裡面的一個不成文的原則,但也正因為他是一個不成文的原則,這個是一個以規則為秩序的國際之間相互約束的一個課稅上的原則。這不是一個明文規定,因為國際租稅法裡面從來沒有明文說所謂的真實的連結。但所謂的真實連結,其實也在強調各國課稅主權的行使,不能任意地無限制地行使在,對這個課稅主權國或地區而言,他是沒有屬人或沒有屬地連結的情況。

我們隨便舉一個例子來講。比如說美國總統拜登,他是在美國擔任總統的,或者是任何一個美國人,他跑到日本去工作,那麼不管是屬人或者是屬地,我們國家都跟該項經濟活動或者是該項經濟成果是沒有關聯的。你當然可以說,哦,我可以對美國總統拜登或者是任何一個美國人在美國工作,我們也享有對他的課稅主權。但這一種課稅主權的行使,並不讓你自己是一個主權國家或地區來正當化。任何國家的課稅主權的行使,他必須要有一個屬人或者屬地上的連結,這是基於這樣的一個真實連結原則。

那麼我們的所得稅法,他本身在所得稅法第2條裡面,原則上,不管你是不是我們的境內居住者,我們的第2條的第一項跟第二項都採取屬地連結。簡單而言就是,只有中華民國來源之所得,稅籍居民的人,原則上是按結算申報程序去進行綜合所得稅的報繳程序。非稅籍居民也只有在有中華民國來源所得的時候,他才會課徵我們的綜合所得稅,也許是用就源扣繳程序,而不是用結算申報程序。因為在那個地方,我們在稅捐稽徵程序裡面,對非稅籍居民,在理論上以及法規範上,無法要求非稅籍居民去做每一年結算的申報程序,從而他會用一個比較特殊的稅捐計算程序,通常就是88條,跟第89條所講的就源扣繳程序。

因此我們的所得稅法,不管是不是稅籍居民,我們都採屬地連結。

當然我們在前面的課程裡面有跟各位去講過,其實我們對於稅籍居民的連結是用一個所謂的,如果是中華民國來源所得,是課綜所稅所得,如果不是中華民國來源所得,則改課所得基本稅額條例裡面的,個人所得最低稅負制個人所得基本額。

換言之,這個地方其實我們也有某種程度上的稅籍居民的概念,只是我們並沒有把他放在同一個所得稅裡面的稅基底下適用同一組稅率,我們是把他做區別對待,一個是課綜所稅所得,一個改課所得基本稅額條例裡面的個人所得基本稅額,這樣一個制度而已。

\hypertarget{ux6240ux5f97ux4f86ux6e90ux5340ux5206ux5be6ux76ca}{%
\section{【所得來源區分實益】}\label{ux6240ux5f97ux4f86ux6e90ux5340ux5206ux5be6ux76ca}}

我們首先第一個去談,為什麼要有關於中華民國來源所得的的判斷,或者這個條文規定的區分的實益爲何。

個人產生的中華民國來源所得,只要你是我們的稅籍居民,原則上你適用的是綜所稅的結算申報程序。

雖然你是我們的稅籍居民,但非取得中華民國來源所得,那則改課所得基本稅額條例的最低稅負制。

非稅籍居民則更是只有在中華民國來源所得的時候,他才會課綜所稅的所得。

我們不會對非稅籍居民去課所得基本稅額條例,因為他本來就根本地沒有任何上的稅籍居民身份上的連結。

那么當然這一個所得來源地的判斷,在營利事業所得稅裡面就扮演更重要關鍵的角色。因為在營利事業所得稅,採取跟個人綜合所得稅不一樣的立法的方式。我們這個學期沒談到營利事業所得稅,不過各位可以去看一下所得稅法第3條的第二項的第一句的規定:「營利事業之總機構在中華民國境內者,應就其中華民國境內外全部營利事業所得,合併課徵營利事業所得稅。」

「營利事業之總機構在中華民國境內者」,我們可以簡單稱之為是「稅籍企業」,或者是「稅籍居民企業」。

當然我這個地方講「稅籍企業」或「稅籍居民企業」,這個只是一個學理上稱呼,因為我們的所得稅法本身並沒有對他做一個固定名詞的稱呼,而是用一個比較繁瑣的「總機構在中華民國境內之營利事業」。每一次你只要去講到這個概念的時候,我們沒有法條長的定義,只是一個學理上一般把他稱之為叫「稅籍企業」或「稅籍居民企業」。

德國則把他稱之為叫「無限制納稅義務人」,無限制納稅人有自然人的無限制納稅義務人,有營利事業的無限制納稅義務人。德國用一個比較統一性的稱呼去稱呼這兩者的概念。而我們的所得稅法本身沒有,我不知道為什麼他不要,我只能說法律條規定就長這個樣子,用一個比較繁瑣的「營利事業之總機構在中華民國境內者」。

所得稅法第3條的第二項的第一句的規定:「營利事業之總機構在中華民國境內者,應就其中華民國境內外全部營利事業所得,合併課徵營利事業所得稅。」

也就是總機構在境內之稅籍居民企業或稅籍企業,就其境內外全部之營利事業所得,也就是當你在營利事業所得稅的課稅裡面,你是我們的稅籍企業的時候,原則上不分境內外來源都是課徵所得稅的所得,也就是營所稅的所得。

那么對營利事業是否就沒有區別境內外的實義呢?因為第3條第二項第一句,是採全球來源所得都要課稅。那麼是否在營利事業屬於稅籍企業的時候就沒有區別「是否屬於中華民國來源所得」的區別實益呢?答案是否定的。

因為不要只看第一句,因為還有第二句。第3條第二項第二句裡面規定:「但其來自中華民國境外之所得,已依所得來源國稅法規定繳納之所得稅,得由納稅義務人提出所得來源國稅務機關發給之同一年度納稅憑證,並取得所在地中華民國使領館或其他經中華民國政府認許機構之簽證後,自其全部營利事業所得結算應納稅額中扣抵。」

很長的一句話,這個也是我們所得稅法立法上的這一種,我個人認為很差勁的一種立法上的技術。因為這一段話很長。當然了,德國所得稅法條例也許也都是很長。不過臺灣的條文通常都是把程序法跟實體法放在一起規定。這個條款是一個很典型的實體跟程序放在一起的規定。

實體是指哪些呢?你從境外取得的所得,可以依所得稅法的規定來做可扣抵稅額的扣抵的計算。這個就是實體法規定。

至於後半段:「得由納稅人提出來源地國的稅務機關所發給之同一年度的納稅憑證」,原則上這個就是程序上的規定。也就是該筆所得在所得來源地國被該國的稅捐稽徵機關做過做課稅權限的行使,那么納稅人則取得繳納稅款的納稅憑證,經由我國在當地的使領館或其他經中華民國政府認許之機構的簽證後\ldots\ldots{}

當然了,這個名詞也是很不精確的。「簽證」是什麼東西?簽證是在講什麼呀?是在講國家的旅行的管理機關所發出來的一個準許旅行的一個證明文件嗎?簽證當然不是指那個意思。而是由我國駐外的使領館或者經我國政府授權許可的機構,這一些我們透過用認證的方式,認證這個確實是一個該國稅捐稽徵機關,課徵稅捐的一個繳稅的證明。這個地方其實是外國公文書的確認的程序。所以我們才說這個是一個程序法規範。

因為外國可扣抵稅額,必須要這一筆所得,確實在外國經各該國家的稅捐稽徵機關課稅,然後經過我國駐當地的使館或者是受公權力授予權限的機關去認證該納稅證明,作為一種證據方法來做可扣抵稅額的扣抵適用。

因此第3條第二項第二句的規定,是一個區辨,只有境外來源所得,在當地確實被課所得稅,而你拿到一定的程序上的認證的憑證以後,你可以來做外國可扣抵稅額的計算。

從而,辨明是否中華民國來源所得,正是第3條第二項第二句裡面,外國可扣抵稅額的一個依據。簡單來講就是只有境外來源之所得,被境外的有權機關課徵稅捐而取得一定程序上的認證的該納稅憑證,可以做外國稅額扣抵。

因此,即使是屬人主義的稅籍企業,雖然是就全球來源所得,但還是要去辨明,該所得究竟是境內或境外來源。

屬於境內,也就是認定為是屬中華民國來源之所得,沒有適用3條二項二句的餘地,因為,本質依我國的稅法規定,認定為這就是我國來源所得,并不是所謂的外國來源所得。只有外國來源所得,在各該國家經由該國的稅捐稽徵機關課稅的納稅憑證,經由我國駐在當地國家的使領館,或者是經政府認許的機構,給予這樣一個認證以後,才會有可扣抵稅額的這樣一個實體法上的效力。

辨明是否中華民國來源所得,不管是稅籍企業或者是非稅籍企業,都是有實益的。

對非稅籍企業,我們只有在真實連結原則底下,才對該非稅籍企業或者是非稅籍的居民產生對他的課稅權限。以所得稅的課稅權限而言,就是綜所或者是營利事業所得稅。

如果是稅籍居民企業,雖然是採取全球所得來源課稅的制度,但只有非我國來源所得,在各該國家被該當地的稅捐稽徵機關課徵了所得稅,然後經由一定程序去取得的這個憑證,才能夠作為可扣抵稅額的依據。

以上,是首先跟各位去談,為什麼要去談「是否中華民國來源所得」,是有區別的實益。

不管是決定我國有沒有課稅權限的行使,在真實連結原則底下,原則上只有屬地的所得,我國才可以去行使課稅權限。即使是屬於全球來源所得的,也是決定是否適用外國可扣抵稅額的依據。

簡單來講,就是劃分各國課稅權限的重要規定,所以理論上應該要由稅法本身去做規定。這個是根據重要性理論。在德國法制上面,一般而言,作為課稅劃分稅基的規定,是稅捐構成要件裡面的重要的要素,應該要由立法者自己來去做決定。那么,在間接民主底下,是由人民所選出的代表,自己的代表選出來,才能決定自己要負哪些稅捐,也就是「無代表即無稅捐」的這樣一個概念的實踐。那麼也因此,在德國認為這個是重要的,要人民自己來做決定。理論上來講,應該要由制定法來規範,連授權給法規命令都不可以,因為這個是要人民的代表自己來做決定的。

不過在臺灣我們是採一般法律保留原則,那麼只要授權明確的話,原則上還是可以把這些事項授權給法規命令,由行政機關依授權明確性原則,在目的內容範圍許可的範圍內去做,關於稅基範圍劃定的這樣子的一個法規命令的制定。這個是我們的實務現況,跟德國大致上的差距,也順帶跟各位去提到稅基範圍劃定是重要的構成要件,從而理論上,在依法課稅原則底下,應該要由立法者,適用嚴格的法律保留,國會保留原則才是比較妥當的。

\hypertarget{ux7b2c8ux689dux4e2dux83efux6c11ux570bux4f86ux6e90}{%
\section{【第8條,中華民國來源】}\label{ux7b2c8ux689dux4e2dux83efux6c11ux570bux4f86ux6e90}}

回過頭來,也因此,我們在第8條就設了所謂的中華民國來源的判斷,這個條文規定。

第8條的規定,「本法稱中華民國來源所得者係指左列各項所得」。總共十一款的規定,看似完全符合剛剛我們所講的法律保留啊,甚至你可以講哦,這個就是國會制定的,是立法者自己制定所謂的中華民國來源所得這個概念。

因為我們是綜所跟營所合併立法的體系。換言之,這個條款規定同時間可能適用在綜所,同時間也可能適用在營所。從而我們對於這個條款規定,要個別去判斷,這個是適用綜所的規定,這個是適用營所的規定,還是兩者皆有適用。那麼跟我們同樣採分稅立法方式的,譬如說德國或者是日本,綜所跟營所,是分開立法。從而如果有類似的條文規定,你可以很清楚而明確地知道,是只是用在綜所,或者是只是用在營所。如果要有規定同樣適用的話,一般而言,立法技術上會採用準用的方式,也就是德國法人稅會去準用德國個人稅,也就是所得稅裡面的條文規定。

臺灣則不是。自始,我們從1905年開始,因為這一部所得稅法最早是大陸時期的「所得稅條例」,一開始就是把對個人跟對營利事業放在同一部稅法裡面規定。這一部法律,我們自始就是把個人稅(個人所得稅)跟營利事業的稅負(營利事業所得稅)放在一起。所以我們雖然採用分稅立法,但我們在綜所跟營所的時候則是合併立法。

從而,第8條的規定,理論上可能適用營所也可能適用綜所,當然也有可能同時適用。這是作為一個,我們是分稅立法,但綜所跟營所卻又合併立法的立法模式,特殊產生出來的一些問題。

我們接下來看一下第8條的規定,看起來是符合立法者自己決定何謂中華民國來源所得,但實際上,你如果把第8條的規定去跟14條的規定做對照,你赫然發現第8條幾乎是第14條規定之重申,再加一句「中華民國來源」。

第8條的第一款跟第二款,還不這麼明顯,但各位可以繼續來看第8條的第三款規定。第三款的規定其實對應14條裡面的第二類跟第三類。因為第二類是執行業務所得,第三類薪資所得,都是屬於勞務付出,而取得經濟成果,差別只差在執行業務,主要是獨立勞務,薪資所得也是勞務,只是是非獨立勞務,原則上是受僱主指示而付出勞務。不管是獨立或非獨立勞務,在第三款規定第8條第三款規定裡面他就加上一句,叫''在中華民國境內提供勞務之報酬''。這一句話基本上只是把勞務這個類型的型態,透過第14條第二項第三項,把他抽象為「勞務」,加上一句「中華民國境內提供之勞務」。

簡單來講就是這個條文本身本來應該是要去判斷關於勞務類型的所得,如何決定是否是中華民國的勞務所得,也就是中華民國所取得的境內提供勞務的所得。因為勞務可以是幾種,我們可以根據他的類型態樣,可能是提供者,可能是受領者,可能是實施的地點。

我再講一次,勞務是一個無形的人力的付出,從概念上,判斷的基準,可以是提供者的所在地,勞務的提供者,假設他可以跨境提供勞務,那麼勞務的「提供者」會是一個判准。

「受領者」又是一個判決。

也有可能是以「勞務獨立實施的地點」作爲判斷。我們舉例而言,要蓋一棟特殊的建築,假設臺北的101,開放國際標國際團隊來競標。那今天會有美國日本,各國的團隊都來競標。建築勞務是一個很典型的獨立勞務的類型。如果得標的是美國團隊,提供者是美國人,受領者則是臺北市政府,受領的地點則會是在臺北市,也就是中華民國來源。實施的地點有可能因為是101,所以是在臺北市。

但假設如果我們今天把實施的地點搬到第三地。換言之,這個受領者在第三地有一個建築勞務,這個時候在概念上,他就有可能是提供受領跟實施的三個不同地點,來決定勞務究竟是否屬於境內來源所得。是哪一個?提供?受領?還是具體的實施?你看第8條第三款的規定,「中華民國境內提供勞務」,在講什麼?講提供者?還是在講受領者?還是在講勞務實施地點?哪一個?答案是,不清楚,因為他沒講。只是加上一個「境內提供勞務」,而這句話有講跟沒講沒兩樣。至少在我的觀點來看,我看不出來他講什麽。

第四款規定:「四、自中華民國各級政府、中華民國境內之法人及中華民國境內居住之個人所取得之利息。」

這個還算好,各款規定裡面有些確實是有意義的。

然後第五款。租賃,是一個有體物的租賃,所以條文是寫中華民國境內之財產,因租賃而取得之租金。這個說他有講清楚,也可以說有,說沒有,也可以沒有,因為他就是再加上一個「境內」,財產是在中華民國境內,你因為該項租賃而取得的租金,這個是中華民國來源所得。

那么第六款跟第七款,後面都是加一個「中華民國境內」,這個立法上的方式是對應著第14條總共有十類的所得,然後加上「中華民國境內」。

我們現在稍微整理一下,第一款跟第二款基本上對應第一類所得,也就是營利所得,拆分成兩款規定,一個是公司,第二款規定則是合作社跟合夥組織。公司的股利盈餘,跟合作社或合夥組織所發放的盈餘,這個是8條一款跟二款,對應第14條的第一類營利所得,因為我們14條第一類裡面,基本上就是公司合作社獨資合夥商號,有股利盈餘,有這個經濟上的成果,加上中華民國境內。這個地方條文規定的解析,我們先暫時說明完之後,我會回過頭來,再來跟各位去講他背後真正的意義在哪裡。

第三款規定對應第14條的第二類跟第三類。

第四款的規定對應利息,也就是14條的第四類。

接下來第8條的第五款規定,租賃而取得的租金,對應的是14條的第五類中的租賃所得。

第8條的第六款規定對應的是第14條的第五類裡面的權利金所得。

第8條第七款的規定對應的就是回到五類裡面的租賃所得。

第八款的規定,「中華民國政府派駐國外人員及一般雇用人員在國外提供勞務之報酬」,這個是可以認為是第三款規定的特別規定,專指中華民國公權力高權團體,派駐國外之工作人員及一般雇用人員提供勞務所獲得的報酬,這個待會我會給各位做一個整理。第八款基本上是第三款勞務的特別規定。

接下來的第九款規定,對應的是所得稅法第14條的第六類,自力耕作漁牧林礦所得,也就是農林漁牧礦冶。有一個第九款規定,是把「工商」又算進去,因為工商是營利事業的一種類型,工商只要不是執行業務者的獨立勞務的付出,原則上就是營利事業的類型。所以第8條的第九款規定,同時對應所得稅法第14條的第六類的農林漁牧礦冶的綜所稅的所得來源地的判斷標準,但同時間他也是營利事業所得稅裡面的所謂的中華民國來源所得的判斷標準。因為在這裡面,工商事業正是營利事業最主要的類型。

第十款規定,競技、競賽、機會中獎,對應的是所得稅法的第八類,競技、競賽及機會中獎的所得。

第十一款的規定,中華民國境內取得之其他收益,則對應的是14條的第十類其他所得。

從這個條文的對應關係,因為我們是綜所營所合併立法的體系,從而在一個條文裡面各款對應不同的稅目會有不同的適用範圍的劃定。在這裡面,第8條的規定,立法技術上是,把第14條的十類所得,加上一個「中華民國來源」。這一種立法導致這個條文規定的規範意旨是不明確,不清楚的。

正因為不明確不清楚,從而在實務上面才另行公布一個「所得稅法第八條規定中華民國來源所得認定原則」。

因為你的母法本身就基本上在立法技術上,只是將各類所得加上「中華民國來源」而已。從而實務上根本無從,從這一些所得稅法的母法規定裡面直接去認定究竟何謂中華民國來源所得,以決定到底外國可扣抵稅額有多少,到底要課綜所稅的所得有多少。

在這種情況底下,實務上,從而從98年9月3號財政部公布「所得稅法第八條規定中華民國來源所得認定原則」,最近修正日期是民國112年10月13日,台財稅字第11204568350號令。我們現在是10月30號,所以是17天以前。

這一個「所得稅法第八條規定中華民國來源所得認定原則」,替代了所得稅法的規定第8條的規定,成為實務上操作認定,是否中華民國來源所得的主要的判准所在。

如果我們不顧慮依法課稅原則,實務上從來基本上是不看第8條規定的,而是看這一個「所得稅法第八條規定中華民國來源所得認定原則」,簡單講,「認定原則」。就跟我們的稅籍居民的概念一樣,所得稅法上沒有清楚的劃分的標準,什麼叫你的經濟跟生活重心所在。

從而這一些屬於本來在法制上被認為是重要的構成要件,在臺灣所得稅法看不到,因為法律條文本身就模糊。至於為什麼法律條文本身模糊,立法者為什么不定清楚?我個人的看法,這只是我個人的觀察,也許我有偏見。我認為財政部送進去就是這種條文。看你立法者會不會改,你立法者如果蓋個橡皮圖章就通過的話,那條文就長這個樣子。因為立法者是立法委員,沒有比財政部更高度的專業。而財政部為什麼要這樣做?很簡單,我送一個抽象低密度的立法,我之後我只要改來源認定原則,我就可以實質變動我們的課稅權限的行使範圍。就這麼簡單。這也許我是小人之心,財政部也許認為說,沒有啦冤枉,柯老師不要常常講我們是這樣子啦,我們立法者都是這樣子,我們去他們搞不好也改了很多啊。對,沒有錯,但你送進去的草案可能就長這個樣子,你財政部希望的就是這種低密度抽象模糊規範的立法。這個認定原則,完全沒有授權依據,以一個事實認定行政規則的方式,實際上替代了所得稅法第8條的規定。

這正是在我國實務上面,當一個低密度立法的母法,連授權的法規命令都沒有講授權,然後實務操作就是用一大堆的行政規則。我們試著想想看,如果沒有這些行政規則,你怎麼去判斷所得來源地。當然待會,我會帶領各位同學我們去看一下第8條的規定,其實這裡面還是有一些蛛絲馬跡,可以去理解,究竟對應的那一類所得,我們怎麼去判斷是否為中華民國來源所得,但不是每一次都能夠對應出來。像我剛剛跟各位談的勞務所得的來源地的判斷,除非是剛好是中華民國派出去的這一些政府的去國外去行使國家公權力的外交的行為的,或者是軍事上的這些行為的人,否則原則上,你根本就看不出來第三款規定裡面他講的到底是勞務提供、受領,或者是實施地。你根本看不出來。待會我們去看來,來源所得認定原則,可能你就會知道啊,什麼叫做勞務提供地。

我會跟各位講,如果你考第8條的規定,往往只能考抽象的法條,那純粹背法條沒有意義,因為實務操作根本就是放在來源所得認定原則。也就是法條跟實務是有差距的。那麼在這個地方我們還沒有去講來源所得認定原則之前,我們還是回到本來,其實在立法上應該要很清楚地去具體指明,何謂中華民國來源所得。

因此我們先來看一下我們現在目前的所得稅法第8條的規定,是不是可以從這些文字裡面可以去瞭解立法者究竟背後所認為的中華民國來源所得,到底是在講什麼?

我們以第8條第一款規定為例。第8條第一款規定是這樣:「一、依中華民國公司法規定設立登記成立之公司,或經中華民國政府認許在中華民國境內營業之外國公司所分配之股利。」

我們先看第一款第一段,依中華民國公司法規定,設立登記成立之公司,所分配之股利。請各位同學,你跟第二款規定對照。第二款規定:「二、中華民國境內之合作社或合夥組織營利事業所分配之盈餘。」

兩相對照一下,因為他們本來都是對應的14條第一類的營利所得。看第8條第一款規定跟第8條第二款規定,差在哪裡?第8條第一款,如何判斷,股利盈餘是否來自中華民國?依中華民國公司法設立,對吧?只要你是依中華民國公司法設立的,那你發放出來的股利盈餘就是中華民國來源所得,就是第一類裡面的股利盈餘所得。所以,我們在判斷第一類的營利所得,以公司發放的股利盈餘為例,不是看經營地點,而是看你是依哪一國法設立。依我國法設立的,由你發放出來的股利盈餘,這個就叫中華民國來源所得。因此相反的,非依中華民國設立的公司,除非該當第一款裡面的後半段,那個「經中華民國政府認許在中華民國境內經營」,理論上你只要非依中華民國公司法,發放出來的股利盈餘就不會叫中華民國來源所得。所以當我今天掛一個,我是美商、新加坡商,就算在境內有所經營,他的經營成果理論上是在中華民國境內經營的經濟成果,沒關係,只要是由公司經營的成果,再把他的經營成果以股利盈餘發放的方式給你的話,他是外商公司的,就不叫中華民國來源所得。

這個就是第8條第一款,實際上這個就叫做準據法。這個也是我們在下個學期,我們講營利事業所得稅第3條第二項第一句。看起來叫總機構在中華民國境內,實際上總機構在中華民國境內判斷標準就是依中華民國公司法設立的,就叫總機構在中華民國境内。也就是你依我國公司法設立的,就是屬於我們的稅籍企業。

我們的立法跟實務上的操作,沒有很清楚地透過法條文字,去做一個連結。用第3條第二項第一句,叫「總機構在中華民國境內之營利事業」。

那麼到了第8條的第一款規定,他很清楚的告訴你,依我國公司法設立的,叫我國的來源所。但第二款就沒有很清楚,因為第二款他寫什麼,他寫「在中華民國境內之合作社或合夥組織」。理論上,如果要用同樣標準,那就是依我國的合作社法而設立之合作社,他所發放的股利盈餘則就是中華民國來源的股利所得。同樣的道理是合夥組織跟獨資商號,其實在我們的經濟法制裡面,獨資跟合夥是依商業登記法規定去做工商事業登記。所以,如果立法上體系上要一致,是依我國法設立之公司、合作社、獨資及合夥事業,所發放之股利盈餘屬於我國來源所得。

這就是透過一跟二款的規定,我們把他綜合起來,去解析第8條何謂中華民國來源所得。你只要是依我國法設立的第一類的所得,原則上就是我國來源所。條文本身並不非常清楚,但實務的操作確實是依照這個來做他的標準。獨資跟合夥由於在我們的現實實務上,很多生意,獨資跟合夥根本不辦商業登記。從而他才會用事實上你的經營地點在中華民國境內,來做為第二款的標準。為什麼他是用經營這個概念?因為在臺灣的獨資跟合夥商號,很多是不按商業登記法去辦理登記的。開店營業,本來是要先辦理登記,但他沒有去做工商登記。

稅法上的登記叫稅籍登記,他其實跟經濟法裡面的工商登記是分開來的。工商登記是指依商業登記法、合作社法或者是公司法所設立登記的組織型態的這種營利事業組織。所謂的稅籍登記,是指為了課稅之必要性。營利事業一般來講,做工商登記之後,就要做稅籍上的登記。所以才一邊歸經濟部,另外一邊歸財政部主管。所以歸經濟部管的,這個就是財務會計,國稅局財政部管的這邊就叫稅務會計。稅務跟財務是兩件事啊,不是一件事,是有兩個不同主管機關。

財稅這個是稅務的問題。工商登記,各位在學公司法的時候,你常會講到工商主管機關,經濟部才是他的主管機構。那為什麼經濟部講的財務會計,怎麼一下子會變成是稅務的計算基準啊?答案是,我們在這裡面財務會計跟稅務,為了要讓營利事業透過,記一本帳,就可以兩邊都應付,所以我們是用「稅務遵財務原則」,做帳外調整就好。財務的這一邊,原則上是做工商事業管理使用。你的財務上的報表要讓投資者,主要是股東跟債權人,能夠理解你到底賺多少,能夠理解我要借你多少錢,能夠理解我要不要投資你。誠實地讓人們知道你到底賺多少錢。這個是財務會計的主要任務,這就是商業會計法裡面相關規範的目的所在。由於這一個商業上的語言,擬募集資金對象的主體不會是只有國內的投資者,你往往都是要對外,對國外的投資人去募集資金,因為你可能會有外國的股東,可能會有外國的債權人,從而在商業領域裡面,他是國際語言,這些國際之間的商業會計語言,以一種習慣法的姿態而呈現在財務報表裡面。這個就是財務會計,為什麼會用IFRSs用所謂的國際財務報導準則。因為你只要是面對全世界的債權人跟資本主的話,也就是股東,你講的話當然是要他們聽得懂的。所以你用百分制的這種分數,到國外去看不懂。為什麼要ABCDEF這種這種評分標準,這個就叫國際語言。我們以前都是用100分的這種方式,這個叫我們本地語言。如果你募集資金,只是面對本地,你用本地語言沒問題。你拿著你在學成績99分來跟我講,那我知道哦,這個是好學生。可是99分,對美國人對歐洲人沒有意義,因為他看不懂99分是什麼意思。但你如果跟他講叫A+,這樣大家看的懂。這就是財務會計的語言,具有高度國際化。因為我要讓別人知道你的財務健康狀態,你是健康寶寶,還是是一個打腫臉充胖子的狀況?

相反的,稅務不是這樣。稅務是本地的,因為他面對的主要是國稅局,國際租稅法沒出現之前,稅務主要面對的是國內的稅捐稽的機關。從而為了不讓營利事業做兩套帳。一個是財務帳,一個是稅務帳,所以我們就是稅遵財,再做帳外調整就可以。在這樣的一個前提底下,稅法規範是内國法規範,是內國法制。沒有任何一個國家會把自己的課稅主權交給外國人去決定你有沒有課稅權限,去行使。

正是因為如此,稅法是法律,稅法是本國法。學財務會計進入稅法的領域,就一定要學法律的語言。所以各位,稅法是什麼?不是純粹的會計。財務會計不代表稅法,也不要混在一起。正因為如此,在稅法的領域,我們希望法律本身規範明確,符合依法課稅原則。當然這個還有很長的路走。

所得稅法第8條的規定,立法技術,我個人認為很糟糕,因為他什麼東西都沒講。當然你仔細分析,好像也有一些有講。但實務的操作,也正是因為他有些沒有講得很清楚,我們在實務操作往往都是看來源所得認定原則。其實來源所得認定原則,他在第2點跟第3點裡面,他就真的就寫,剛剛我跟各位講的,依我國法律設立登記的,這個就叫中華民國來源所得。這其實就是這樣一個相互的來回對照。我們今天會跟各位去講,何謂中華民國來源所得,就這樣子的一個法制不是很完整的一個狀態所產生出來的問題。

我們先休息一下。

\hypertarget{section-16}{%
\chapter{20231030\_02}\label{section-16}}

\begin{longtable}[]{@{}l@{}}
\toprule()
\endhead
課程:1121所得稅法一 \\
日期:2023/10/30 \\
周次:09 \\
節次:2 \\
\bottomrule()
\end{longtable}

\hypertarget{ux7e7cux7e8cux7b2c8ux689dux4f86ux6e90ux6240ux5f97ux8a8dux5b9aux539fux5247}{%
\section{【繼續,第8條,來源所得認定原則】}\label{ux7e7cux7e8cux7b2c8ux689dux4f86ux6e90ux6240ux5f97ux8a8dux5b9aux539fux5247}}

來源所得認定原則,常常更新,所以實務上根本不可能考這個。因為你考可能立刻考題出來就過時。所以各位不要傻傻去背這個,沒意義。我只是透過這個跟各位講我們的法條規定,財政部可以自己就透過這個解釋令公布實際上變更是不是中華民國來源所得的判斷標準,實際上,決定課稅權限的大或小。你覺得這樣符合依法課稅嗎?你如果認爲符合,我跟你是不太一樣的感覺,就是這樣。

剛剛有幾位同學來問。我跟各位說一下,我對這些原則的看法,基本上就是,立法低密度,標準不明。這一些其政規則,因為都沒有授權,都會用事實認定標準的這種法律形式出現。他在第一點裡面就沒有講我是從哪裡來的。如果有授權依據,一般而言,法規命令會自己說,我是哪一個條文授權規定。我們的稅籍居民的認定,也是一樣,他都不會講自己的授權依據,所以在實務上呈現的態樣,都是用事實認定的標準,以行政規則的姿態出現。可是他實際上就決定了我們課稅權限到底有多少的所得的範圍。

來源所得認定原則,為什麼會到五頁?你如果講要依法課稅,那就把這5頁的規定全部放進去所得稅,所得稅法的條款就會爆掉。其實我們這樣講,常常是他該有的劃分標準沒出來,但不該有的劃分則是過度劃分,就叫做過猶不及。各位能明白那個意思嗎?法制落後國家的最通常現象往往不是他沒有法律,而是那個標準,那個區分界線在哪裡,不清楚,或者被清楚出來的規範,你認為他過猶不及。這是法治落後的很標準的現象,就是那個分寸抓不清楚。

因此,法律人的任務就是,如果可能,我們當然透過法律解釋。如果連法律解釋都做不到,我們就只能要改變立法。這個就是我跟各位去講過,其實法律人在稅法的領域有一些比較,我個人認為是蠻特殊的任務,因為我們是追求法治國。你任何萬千政策好意都要透過法規範來實現,不是你一個人說了怎麼算。因此所有我們今天跟各位講的判準,老師當然會希望說,欸最好,我們所得稅法第8條就這樣定,可是,不是老師一個人說了算,這個標準一定要入法,否則這個標準,就會一直產生一種浮動不明或過猶不及,這種現象會不斷的發生。

我們看,這個所得來源地的認定標準。因為他不是「條」,是用「點」的方式。這也是另外一個,我常常實在是很受不了財政部地方,就是不太區分實體法和程序法,對於稅捐的主體、客體、稅基、稅率,這些構成要件,沒有那個清楚地分明。其實以我自己的了解,就是他們都是個別承辦的人,參考一個外國法,然後就直接抄過來,像作文一樣。所以你看到我們現在的條文,就是看起來像作文,這個就是實務上很常見的情況。

\hypertarget{ux7b2c2ux9edeux80a1ux5229}{%
\subsection{【第2點,股利】}\label{ux7b2c2ux9edeux80a1ux5229}}

我們看第2點喔。第2點很長。第2點第一句:

二、本法第八條第一款所稱「依中華民國公司法規定設立登記成立之公司,或經中華民國政府認許在中華民國境內營業之外國公司所分配之股利」,指依公司法規定在中華民國境內設立登記之公司所分配之股利。

依我國法設立的公司,這個公司發放的股利,這個叫中華民國來源所得。這個就是我剛剛跟各位講的,這個叫準據法。因為你去辦公司登記的時候,他上面就會寫本公司是依公司法第幾條,設立的在哪裡,比如說在臺北市大安區幾號幾號,負責人是誰。這個就是在實務上,總機構在中華民國境內。所以總機構在中華民國境內,我看一些教科書,有些會寫說,哦,我們有分三類,一種叫做實質管理處所,一種叫做準據法地,而我們就不是實質管理處所,我們也不是準據法,我們叫總機構在中華民國境內法。我都看不太懂。

就好像我在看那個憲法111條,什麼叫均權原則?不是聯邦制,也不是單一制,我們兄弟個人獨獲之創見叫均權制。好吧,不知道你在講什麽。當然啦,看起來都符合一個事務法則,就是說均權原則是依事務之本質分類,凡事有歸中央者,要統一者則歸中央,凡事有個地方個別做不同規定者就歸地方。這不是廢話嗎,不是本來就應該如此嗎?

「遇有爭議時,由立法院解決之。」就是,如果不能決定的話,則由立法院決定之。我告訴你,就是這句話。這句話才真正表現,我們就是個單一國家。你不要只看前面那個,因為當中央跟地方有爭議的時候,立法院是誰的機構?是地方的嗎?是中央的立法機構。所以我們就是一個徹頭徹尾的單一制國家。有爭議,就我說了算,我中央說了算。這就是一個很典型的單一制,你不要只看那個名目的那個名稱。

同樣的道理就是在這個認定原則第2點裡面。其實他要講的就是,從這個依據我國法設立的公司所發放的股利盈餘,這個就是我們的中華民國來源所得好。第2點裡面有一個但書的規定,就是第二句的規定:「但不包括外國公司在中華民國境內設立之分公司之盈餘匯回。」

如果是依外國法設立在臺灣的分公司,也就是他是屬於外國公司,但在境內,他只是以分公司的型態,當分公司把這個盈餘匯回去給他的總公司的時候,這個盈餘匯回,不視爲中華民國境內來源所得。他的條文規定是指說,臺灣分公司匯回去給外國總公司的這個股利盈餘,不是中華民國來源所得。這個條文規定,在我個人來看,比較不太對。因為他的本意應該不是在講中華民國來源,而是指說分公司匯回給總公司的這一筆,不認為是股利盈餘。這個是客體的否認,而不是所得來源地的否認。各位能夠知道老師在這裡講的意思嗎?他的概念應是說分公司把股利盈餘匯回去給境外的總公司,這一筆股利盈餘的匯回,不視為是一種股利的分配。他是在否認他是股利分配的概念,而不是所得來源地的否認。其實他還是中華民國匯出去的,只是他不是股利盈餘。

我們的條規定是看起來有一點點說,這個不是中華民國來源所得。這個是第2點。

第2點,第三句的規定:「依外國法律規定設立登記之外國公司,其經中華民國證券主管機關核准來臺募集與發行股票或臺灣存託憑證,並在中華民國證券交易市場掛牌買賣者,該外國公司所分配之股利,非屬中華民國來源所得。」

第三句的規定,是在講TDR,臺灣存託憑證。美國或新加坡上市公司,是在各該當的公開發行股份的公司,那么透過在本地在臺灣發行所謂的TDR,這個叫臺灣存託憑證。一股的臺灣存託憑證對應一個外國的股份,對應一個新加坡或美國的股份。基本上,TDR,所謂的存托憑證,是一種衍生性金融商品。因為在各該美國跟新加坡都是在該地發行的有價證券,而你是有價證券的再次延伸。你在新加坡上市的,在美國納斯達克上市的,發行一個在臺灣以一定的比例去對應的外國公司股票,所以臺灣存託憑證是外國發行有價證券的衍生性金融商品。反之亦然。臺積電是在臺灣發行的公開發行股份的上市櫃公司,那麼他在美國的這個就叫存託憑證,就叫做ADR,如果是全球範圍發行的,這個就叫global的GDR。如果是在歐洲發行的,就是EDR。他基本上對應的就是在本地公開發行的一個有價證券之衍生性金融商品。一般而言,這個價格會用一個比例的方式去做計算,然後在這個地方,一個ADR或TDR會對應一個在外國發行的一個所謂的有價證券。

好,我們的第三句規定的意旨是說,當發行在臺灣的存托憑證,那么你透過外國公司所發給的股利盈餘,那麼透過TDR的方式因此而產生之股利盈餘的分配,則認為不屬中華民國來源所得,也就是他是屬於各該發行的國家的該地的來源所得。美國發行的就是美國來源所得,新加坡的就是新加坡。臺灣的TDR這個部分,原則上這個就是一個有價證券,就是一個股權的衍生性金融商品或是債權的衍生性金融商品,這種衍生性商品,則回到最初的他的這個資本市場裡面由該地去課徵他們的所得稅。這個是第2點的第三句,他所想要表達的。也就是如果是存託憑證,不論是否是依照我國法律發行的。只要你是這一類的,債或股票的衍生性商品的話,原則上回到他原先所聯繫的那個基礎證券來判斷他的所得來源地。

在我們所得稅法第8條第一款規定,有這個條文嗎?應該沒有吧?從哪裡得到的這個規範的依據呢?我也看不出來他到底是從哪裡來的。有沒有道理呢?這個是另外其次的問題。在我自己個人來看,如果臺灣的TDR,那臺灣的這個部分就是臺灣的那個所得。我們這個地方做股利盈餘的分配,我們雖然聯繫的是美國或是新加坡的一個上市櫃發行的股份,沒有錯,他確實是這樣再回來臺灣。衍生性金融商品的特色,就是他可以不斷延伸,你在美國發行了,他可以不斷地透過這種衍生的方式再衍生,所以你知道這就是2008年以前衍生性金融商品所造成的過度信用擴張的問題。其實是同一筆債權或是同一筆資本,但不斷地透過這個方式,他就可以創造很多的利潤出來。衍生性金融商品就是一個無中生有的東西。所以,非常恐怖,也非常的厲害。我個人認為是,如果是懂的人,就覺得,哇,那個真的是fast money啊,就很快的錢,就是他可以不斷地衍生。衍生性商品本質上是可以沒有限制的,特別是在葛林斯潘當美國的聯準會主席那個時候,因為他基本上非常放任金融機構去發展各式各樣的衍生性金融商品。在他的在任那幾年的時候,基本上就是美國華爾街市場,股票翻好幾倍的都有,每個人大家都是賺得這個分紅賺到不行啊。不過,很快的喔,就是基礎證券,如果崩塌,那就會連帶到後面的衍生性金融商品。那這個是我們跟各位簡單談第2點。

\hypertarget{ux7b2c3ux9edeux76c8ux9918}{%
\subsection{【第3點,盈餘】}\label{ux7b2c3ux9edeux76c8ux9918}}

很快的,第3點。

三、本法第八條第二款所稱「中華民國境內之合作社或合夥組織營利事業所分配之盈餘」,指依合作社法規定在中華民國境內設立登記之合作社所分配之盈餘,或在中華民國境內設立登記之獨資、合夥組織營利事業所分配或應分配之盈餘。

所以,就是依我國的合作社法設立登記的合作社,所以他其實跟第2點的一我國公司法設立登記公司所分配的盈餘,他都是準據法。只要是營利事業的股利盈餘,原則上是依準據法。可是因為獨資跟合夥,理論上你只要是依商業我國商業登記法所設立登記的獨資跟合夥,那他發放的股利盈餘,也是屬於中華民國來源所得。可是沒有設立登記的,在這裡面,只要你經營地點在中華民國境內,才有可能是中華民國來源所得,這個時候就要跳過去,去適用第8條的第九款規定,就是經營地。因為你沒有準據法的第一款跟第二款,那你這個時候就要跳過去,去適用第九款的規定,你的經營地點在中華民國境內,這樣才是屬於中華民國來源的所得。而這個第8條的第九款規定,對應回來,在來源所得認定原則裡面,是第10點,看第一句:本法第八條第九款所稱「在中華民國境內經營工商、農林、漁牧、礦冶等業之盈餘」,指營利事業在中華民國境內從事屬本業營業項目之營業行為(包含銷售貨物及提供勞務)所獲取之營業利潤。

所以,你沒有依據準據法,那只要你有實質經營本業的營業項目的營業利潤,你的營業行為的經營地點在中華民國境內,這個時候就是中華民國來源的股利盈餘。

這個是我們透過對相關條文的解讀跟配合,去做解讀。

我假設各位手頭上有這樣的一個要點。考試是不會考啦,我也不會這麼不上道地考這種很無聊的考試啦。但是我們學習還需要知道背後他到底是什麼原理原則啊,就這樣而已。所以我雖然很清楚的告訴各位,不會考這種無聊的考試,但是我們總是要有一些準據法的觀念。各位在將來也許你不看這個來源認定原則,但是你在腦海裡面大概知道說,哦,他可能會有哪些標準。

將認定原則的規定整理一下。

第一類的股利所得,原則上,依準據法。這一套標準,我跟各位說,他其實是美國的標準,美國就是用準據法。這個我在前面,不知道是哪一次上課的時候跟各位講過,美國基本上是按你按哪個州法設立的,你就是那個州的公司營利事業。這個是美國式的標準,像德拉瓦州,我有跟各位去講過說,在美國,不管你事業做多棒,在東岸西洋做得棒,很多公司都跑去德拉瓦去,依該州的州法去設立營利事業,所以他們那個州法裡面有怎麼樣的組織形態,原則上,當他發生爭議的時候,就回到德拉瓦州去訴訟。這個就是各位在公司法裡面常聽到的德拉瓦州的法律,那你只要是以美國的法律設立的,在臺灣的標準裡面,這個就是美商,這就是外國公司。依我國公司法設立的,原則上,這個就是我國的境內的營利事業,這個的標準,是形式標準。

相反的,屬於歐陸法系的國家,他們比較會用不是形式標準,而是用實質的經營地點在哪裡、實質的營運地點在哪裡,去判斷。法國、德國,歐陸法系國家比較偏重用實質的標準,像愛爾蘭,其實也是用實質營運的為標準。

因為我們用形式準據法,這個時候很容易讓人們去產生規避之可能。所以我現在只要依外國公司法設立,我在臺灣做境內分公司臺灣分公司,根據我剛剛跟各位念到第2點的第二句的規定,我在臺灣賺的錢,匯回去給國外的總公司,不算中華民國來源所得哦。所以,他只需要用外國法設立,然後在臺灣作為分公司的組織形態,原則上他只就境內,也就是中華民國來源所得,課一次我們的所得稅,然後他把盈餘,課完以後的盈餘,要匯回的時候,這個不會再被第2次課稅。那匯回到他自己所屬的國家地區的時候,他是用英屬維京群島法設立的,英屬維京群島不可所得稅。所以這個時候他就能夠實際上有效地降低稅負。

相反的,臺灣的臺積電,他是依中華民國公司法設立,所以他在臺灣賺到的錢,他在美國賺到的錢在歐洲賺到了錢,原則上,全部都要算我們的營利事業所得稅。只有在你在個別國家被人家課所得稅,根據我們剛剛念到那第3條第二句的規定,從外國來源地所得被外國的稅捐稽徵機關課稅的,那麼你就拿著那個納稅憑證,經我國駐外的使領館或認許的機構去做認證以後,哎,你就可以拿來做外國可扣抵稅額的稅額扣抵。這個就是總機構在中華民國境內,全球範圍內的所得,和外國可扣抵稅額,但原則上你必須是外國來源,經過該外國的稅捐稽徵機關,做過課稅。沒有課稅的話,原則上你的外國可扣抵稅額,還是一樣是零。你會全部都歸入中華民國所得稅法,去課中華民國裡面的營利事業所得稅。

我們大致上的課稅的結構是這樣。

我們的稅法裡面規定,就是你依照我們的公司法設立的,那你就是總機構在中華民國境內,全球來源所得都要課稅,可是如果你依照外國法設立的,特別是設立在臺灣的是分公司的話,是美商的臺灣分公司,或者是英屬維京群島臺灣分公司,那麼原則上就是臺灣分公司只在臺灣課一次營利事業所得稅,然後盈餘匯回的時候就不用再課稅。臺灣分公司只就臺灣來源所得課稅而已,不就美國或歐洲的賺的錢課稅。因為他在我們的稅法裡面他叫,非稅籍企業。非我國的稅籍企業,只就臺灣地區來源所得課我們的所得稅。他的稅負差別就差在這裡。所以各位可以去看一下實務上有多少公司是這樣設立的。你可以去看一下有多少公司用這招方式順利的就可以把盈餘。

在這種情況底下,他就可以把美國或歐洲的所得,想辦法,不用課到所得稅。臺灣的所得,他跑不掉,他基本上會被課一次。臺灣境內來源的所得課一次所得稅,那這一個課完以後,再把那個盈餘匯到他自己的境外公司。那一家境外公司位在租稅天堂地區,BVI,開曼,或者是巴拿馬,這一些被認為是租稅天堂的地區或國家,那麼這個時候因為這些地方基本上不課所得稅或者所得稅負極低。這種情形底下他實際有效的稅負可以很低。他可以在全球賺了好幾百塊,臺灣只有100塊,在美國賺500塊,在歐洲賺600塊。全球加起來,他本來是1200塊的所得,可是他只有就臺灣的100塊課了20塊錢的所得稅。所以他的實際有效稅負就是20÷1200,1.67\%。你看,一個正常企業在臺灣要繳20\%。我光設一個準據法,我就可以只剩下不到2\%。繳稅的是笨蛋,繳稅的是傻瓜,那麼愛國,不知道什麼這個叫實際意義。這就是為什麼這一套制度本來就是荒謬到極點,就是對比較有錢的人比較有利。因為你薪資所得沒有這種機會,沒有辦法這樣晃來晃去,可是我要做生意的話,我用外國法,我就可以有效地把盈餘利潤留在海外,你為什麼不做?做生意本來就是這樣子。實務的操作的這個現象,你越是清楚,你就會越知道,哇,我們的稅負分配之不公。雖然有理想,可是在政治上要動,真的不是太簡單。就跟我們到處都是夜市裡面的攤販。我不知道各位是怎麼看的,我沒有特別的意見,但德國這一段這一段時間開始要耶誕夜市,因為他們只有耶誕節前這一段時間,有開這種耶誕夜市。那為什麼他們不把耶誕夜市常態化?就像我們臺灣一樣,每天都有夜市,只是分不同時間。他就是一個城市的管理各項機能的失效,才會產生到處都是夜市。第一個,人家正常開店的要有一個店舖經營,夜市不用,只要一個地方圈起來就可以了。第二個,環境管制規範、排水規範、僱傭條件規範、勞動管制的規範,然後課稅規範,全部都放在一邊的,這個就是夜市。

我只是跟各位講理論,如果剛好各位的身邊周邊的親朋好友是在夜市裡面工作,我沒有貶低的意思,我完全沒有貶低的意思,我都會覺得你賺錢是好事。作為稅法老師,只有一件事情,你賺錢了沒有?你賺錢了,請你繳稅。就這樣,這應該不算過分啦。我自己認為不是很過分。不曉得是不是有人認爲,要繳稅就是很過分,那如果是這樣,我也沒有什麼好講的。

我只是跟各位講,在臺灣做到稅捐負擔分配正義,確實是不太容易,我知道政治上有很大的困難。就像我之前跟各位講的,居住正義不是那一次游行那幾個人上去講話,他們根本搞不清楚,居住正義是什麼回事,弄不清楚為什麼居住正義長這個樣子。我說實在話,那個真的就很典型的political的動員,那個根本就是搞不清楚狀況。臺灣的居住正義,單一這個環節,他沒有辦法透過,用一個什麼政府把售價降低就好了。你知道政府售價降低,受傷的會是現在一群中產階級,擁有不動產的人,真的會整個崩塌,是整個的秩序上的崩塌。任何一個這樣的民主國家都不可能這樣,都不可能像你那些上去去講的人那樣子去做。

我只能跟各位說,稅制裡面有很多不合理的現象,但,怎麼去做起?確實,作為稅法老師,我也不敢跟各位講,依照我的方法一定能做得到,不是。因為我就不是獨裁者啊,不是我說了算。民主國家,他必須要透過很多的説服,這個說服首先是必須要有很多人認知到這個問題的存在,並且有意願去改變。民主就是這樣,沒有捷徑,也沒有比較遠的路。就是這樣,因為他就是必須要透過說理説服。説理說服的前提是先認識到問題的存在。我們現在的問題是連認識到這個是問題的存在,都不一定。正因為如此,很大程度上我們很多人自己也並不知道,我們的稅捐負擔是長這個樣子。

這個是這個課程,我自己個人認為,很需要,特別是法律背景,去多了解這個領域裡面的狀況。我當然知道改變不是一天就能做成的,可是完全沒有認識,這個才是比較大的問題。也許改變可能是老師已經死了好幾十年之後的事情,但只要有認識,才會有真正改變的開始。

我自己當學生的時候,到時候的人,炒地皮算正常啊,我沒有拿來用,就算剛好而已。這個你可以講,老師,你開公司這麼好用,你為什麼自己不去開?對啦,就是有些事情,我雖然知道,我不會去做啊。這個還是有不同的觀念。你也可以說,老師就是因為可能沒錢啊,所以你就不會去幹這種事情。也不完全是這樣。但我知道確實是這個一個方式,如果我把我變成是一個臺灣分公司的話,哇,我的稅負就變成只有1\%而已,當然重點是要你賺錢。開公司不是目的,開公司的目的就是要賺錢。這是真的,開公司,就是要賺錢啊。為了要賺錢,他才會有繳稅的問題。如果你真的是虧損,在老師的立場,虧損就要求認列減除,就這樣。

所以,有所得就要課稅,這是半句話,有虧損要準予減除,你不能只講前面,後面就不講。所以,講了前面,我是一個右派,講後面的我看起來變左派。不是這樣,因為你話不能講半句。

我以這樣一個方式去說這件事情的時候,也希望各位能夠在此同時了解我們從稅捐法治國家的角度瞭解我們的弱點,或者是我們的問題可能會出在哪。

好,現在回到要點裡面。

\hypertarget{ux7b2c4ux9edeux52deux52d9ux5831ux916c}{%
\subsection{【第4點,勞務報酬】}\label{ux7b2c4ux9edeux52deux52d9ux5831ux916c}}

第4點,第一項規定,

本法第八條第三款所稱「在中華民國境內提供勞務之報酬」,於個人指在中華民國境內提供勞務取得之薪資、執行業務所得或其他所得;於營利事業指依下列情形之一提供勞務所取得之報酬:\ldots\ldots{}

看第一分句,這個地方,於個人是同時適用三種類型的所得就是薪資、執行業務所得,或其他所得,也就是一時性的勞務所取得的所得。然後,第二分句,於營利事業,則指以下之情形。以下下列之情形,提供勞務所取得的報酬好第4點的第一項第一款的規定:「提供勞務之行為,全部在中華民國境內進行且完成者。」

這個地方,才開始終於出現幾個我認為比較關鍵的字眼,叫做勞務的進行跟完成地點在中華民國境內,這個就叫做營利事業所取得的勞務的報酬是中華民國來源所。我個人的理解是,這個就是勞務的實施地點在中華民國境內。我以剛剛那個例子來跟各位說一下,假設臺北市政府,在香港有一塊地,想要去蓋一個一個特殊的建築,然後開放給全世界的建築師團隊來競標進行該項工程。這個也不難舉例,因為我們早期臺灣有很多農耕隊在國外啊,比如說派到沙烏地阿拉伯去,在當地需要做建築的時候,我們有一個勞務的提供、受領,跟實施地點,這三個都有可能會主張自己是屬於所謂的勞務的取得地點,也就是取得勞務的經濟成果的課稅來源地所得地。「提供地」就是指,比如說是美國建築師團隊,他取得這個案子,那美國國稅局會主張,因為我的團隊去,可能是在阿拉伯所做建築工程嘛,那這個建築工程是因為美國的建築師團隊去標到取得的案子,所以我們是勞務的提供者,這個地方因此享有他的課稅主權。另外,第二個是,欸,我們中華民國是勞務的計畫的受領者,因為是我們去開啟這個國際標的,那也因此中華民國在這個地方就取得一個課稅的權限跟主張。那勞務實施地點的沙烏地阿拉伯也可能會主張,欸,你的勞務就是在我境內,雖然你的業主是中華民國的政府,雖然你提供勞務的是美國的團隊,但你勞務實施地點是在沙烏地阿拉伯境內,那這個時候就有可能,沙烏地阿拉伯會主張是在他境內進行及完成的。

依照我們第4點第一項的第一款規定,就是只要你全部在中華民國境內進行跟完成,這個就叫中華民國來源所得,所以他的標準是以勞務實施地點作為判斷標準。

第二款,「提供勞務之行為,需在中華民國境內及境外進行始可完成者。」,勞務是在境內跟境外都提供的話,那這個也叫中華民國來源所得。

第三款的規定,「提供勞務之行為,在中華民國境外進行,惟須經由中華民國境內居住之個人或營利事業之參與及協助始可完成者。」好這個地方終於出現了一個實務界人人嘴巴常講的概念。勞務,要有我國境內的個人或營利事業的參與跟協助,只要有我們的人參與跟協助,這個就讓中華民國來源所得。實務界判斷勞務的所得來源地,幾乎就是指有沒有我們境內的個人或營利事業的參與跟協助。有參與跟協助,就是中華民國來源所。

正是這個條文,我告訴各位,只要有我國人的參與跟協助,即使我們不是一個勞務的提供的主要那一方,也不是受領方,甚至勞務實施地點都不在我國境內,我們也自己認為我們有課稅主權,而且根據我們的第4點第一項的第三款規定,這個地方是中華民國來源所得,還不認為是外國可扣抵稅額,可以扣抵的外國來源所得喔,是中華民國來源所得。這個條文規定,基本上大幅度擴張我國課稅權限所及,而且基本上否定了外國可扣抵稅額的適用可能。講白一點,就是因為以我國的角度來講,我們認為這是中華民國來源所得。勞務,只要裡面有任何一個中華民國的個人或營利事業去參與或協助,這個勞務報酬,就算是中華民國來源所得。這個課稅範圍畫得可真大,真的非常大。

也因此,在實務上面,這一種大幅度擴充自己本國課稅主權,從自己國家的角度去考慮這種租稅的情況,我們這個是一個蠻典型的情形,而且是違背了真實連結的原則。

屬地原則,確實讓所得來源地國享有課稅權限。可是不能夠,本來是屬地,因為當地的課稅權限是由於,國家提供給你當地的基礎設施,提供給你當地的這個各式各樣的行政上的便利,讓你因此在當地取得經濟成果,從而當地才取得一個課稅權限的屬地上的連結。結果我們的法律規定是,只要有我們的參與跟協助,這個就叫中華民國來源所得。這樣的我個人認為是廣泛而無邊際的課稅,其實是牴觸國際租稅法上的真實連結原則。可是這只是我個人的說法,因為實務上每天都這樣。只要司法機關一天不廢掉這個所得認定原則,繼續用這個標準,那這個就是我們的實務操作。

話說回過頭來,法官如果不按這個標準,參與跟協助,來認定,那請問法官又要用什麼標準?這是法官的困難。法官之所以我今天我曾經想過我要是法官,我為什麼不大膽地宣告,我不用這個「中華民國來源所得認定原則」?因為我不用這個,又沒上過柯老師的稅法,我真的不知道要怎麼用。因為我看所得稅法第8條第三款規定,只有講,所得來源地,境內提供勞務\ldots\ldots 什麼叫境內提供勞務?哦,實物說,只要有中華民國個人跟營利事業參與跟協助,這個就叫境內提供勞務,看起來好像也不能說沒有道理啊,因為我們有貢獻嘛?所以這樣就叫做中華民國境內提供勞務。可是這個其實跟屬地的連結是不穩固的。屬地連結,人家國際租稅法裡面的常設機構,就是至少要超過6個月以上的連結。你説,只要參與一下這樣就有了,這也未免太過大幅度地擴張我們的課稅範圍。沒有錯,國際租稅法大家都會想辦法去做自己的課稅主權的連結。沒有錯,但是,至少要有一個最低限度的真實連結。要麼屬人,要麼屬地。屬地要有一段期間得繼續性的存在,這就是常設機構的概念。

德國是這樣,常設機構,原則上就是要半年以上,工程的常設機構要更長期間。因為工程本質上就是一段時間裡面的繼續性存在。他們可以透過雙重租稅的協定,甚至是放棄對那些境外常設機構的課稅權限。可是屬地連結要建立在一定期間裡面的繼續的存在。單純只是偶一的偶然性的參與跟協助不應該構成屬地的連結,這是我們的租稅法跟別的國家不太一樣的地方。也是因為這樣,我個人認為,我們的租稅法非常的不國際化,沒有跟國際規則準則走在一起,我們是一套兄弟個人獨獲創見之法律規範。我們自己關起門來。因為是所得來源地,所以要求你付錢給對方,你就要就源扣繳,所以倒楣的是誰?因為這個是中華民國來源所得,對不對?假設對方是美國的建築師團隊。臺灣要付錢給美國建築師團隊,他要求臺灣的機構要就源扣繳,你不就源扣繳,就補你稅,罰你的錢。所以最後面倒楣,還是我國的營利事業?話說回來,我們在這麼不利的稅法環境底下,我們還有這麼強的國際競爭力,還有真的蠻讓人喜出望外的。真的不太知道,哇,我們怎麼這麼厲害!好棒喔,竟然在這麼不按照國際規則走的這種現況底下,還有這麼強的競爭力。這一點,我到目前為止也是不太明白。我們臺灣廠商這麼厲害,可以有這麼強的全球趴趴走的競爭力。可能有另外的方法節稅,這樣也許可以。就大致上是這樣的。

稅制規則的合理化,我相信是稅捐法律人,重要的公法上的任務。這個是在這裡面最後面跟各位提到。

今天時間上來不及,那下個禮拜我會在借用前半個小時的時間,把所得稅法第8條的規定,我會把幾個重要的基準原則說出來。比如說,不動產,只要跟不動產有關,一定是不動產的所在地,因為涉及領土主權的部分。跟不動產有關的,所以像財產的租賃、財產交易所得,還有農林漁牧礦冶。各位還記得我講農林漁牧礦冶,跟大自然界第一次產出,所以「地」在哪裡,那個地方,原則上有屬地連結。

我們下個禮拜跟各位去講,國際租稅法裡面的,幾個重要的這種連結的原則。屬地連結,其實有一個標準,基本上這5個標準下來,我們的規則會比較清楚,也比較知道到底應該怎麼做連結。不是一個參與協助我們就這樣就連結了。這個是跟各位先做一點簡單的補充説明。我們下個禮拜就這個部分再花半個小時時間跟各位去做一下講解。

稅捐課客體之後,我們就進入稅基。稅基才真的是考試上的重點啦,因為稅基比較容易考試啊。

我們先到這裡。

\hypertarget{section-17}{%
\chapter{20231106\_01}\label{section-17}}

\begin{longtable}[]{@{}l@{}}
\toprule()
\endhead
課程:1121所得稅法一 \\
日期:2023/11/06 \\
周次:10 \\
節次:1 \\
\bottomrule()
\end{longtable}

今天我們先繼續,所得稅的稅捐客體的最後一個部分,關於所得的來源地的判斷,接著在這個之後就會談到今天我們非常重要的關於稅基的部分。

所得稅的稅捐客體的最後一個部分,關於所得的來源地的判斷。在上個禮拜我們跟各位提到,所得稅法第8條本身的規定,並不是很清楚,去指明何謂所得來源地的判斷。也因此,在實務上,我們透過財政部的「所得稅法第八條規定中華民國來源所得認定原則」,作為一個判斷上的依據,這個來源所得認定原則,看起來是一個,一般性抽象性的行政規則,因為他本身並沒有法律的明文規定的授權依據,等於是以所謂的事實認定原則的方式出現。這樣子的一個條款規定,儘管實務上自認為是一個事實認定的一個規則,但,以老師個人的判斷來看,我認為他還是一個關於我國有沒有課稅權限的劃分,是屬於稅基的規定,那應該是要由立法者,透過制定法,或是授權明確的法規命令來做規範,才符合依法課稅原則。但現實上他就是沒有這樣一個授權。那司法實務上的困難是,我如果不知道這個來源所得認定原則,也不太容易有一個客觀上的一個判斷標準。

那從我們今天要跟各位去談的,就是一個應然的狀態。也就是說,實務上其實並沒有這樣一套的判斷標準,就是我們在司法的實務累積上面,他並沒有形成一個比較穩固的一個見解。

老師接下來跟各位談的,關於所得來源地的判斷的問題,涉及到國際租稅法的問題,另外一方面也是參考那個德國的立法上的規定,來做一個綜合的說明。

\hypertarget{ux6240ux8b02ux7684ux4e2dux83efux6c11ux570bux9818ux57df}{%
\section{【所謂的中華民國領域】}\label{ux6240ux8b02ux7684ux4e2dux83efux6c11ux570bux9818ux57df}}

那我們首先跟各位先談到中華民國所得來源地認定,我們現在目前的所得稅法第8條規定,本身並不是非常恰當。因為所謂的中華民國,在這個地方,不應該透過意識形態而去將所謂的臺灣地區跟大陸地區,都認定為叫中華民國所得來源地,這種判斷標準是一個意識形態,而不是法律原則的一個判斷標準。因為很簡單,中華民國,臺灣根本沒有對大陸的有效管轄權限。一個美國人在大陸地區獲得所得,你總不會認為他應該要繳我們的所得稅,只因為他是中華民國來源所得,這一個純粹基於意識形態的這樣一個標準。你完全不會認為任何人到中國大陸地區去獲得所得,因為他是中華民國來源,所以我們跟他產生屬地的連結,因此他要繳我們的所得稅。這個道理是說不過去的。

從而我們的所得稅法第8條第一項,其實應該要明白的定義,所謂的中華民國來源所得,我們只及於我們現在在國家主權有效行使範圍內的臺澎金馬地區,這一些才是我們的所謂的中華民國來源所得。要去連結所得稅法第2條第一項所講的中華民國來源所得,所謂的屬地連結是你的課稅管轄權限,能夠有效的連結的領土、領空跟領海。這個地方跟你憲法愛怎麼講,統一前的所謂中華民國主權能及的範圍,不同。你根本就沒有辦法有效去管轄大陸地區,就不要談,所謂那個地方也算中華民國來源所得。

所以第8條第一項,其實理論上應該是,本法所稱的中華民國來源所得是指中華民國現在有效管轄的臺澎金馬地區的領土、領空跟領海。

領土,待會跟各位去談到,為什麼他是所得來源地的一個判斷標準。因為領土涉及到一個國家的不管是課稅管轄區或是司法,比如說刑事法上的有效管轄,這個都是一個地域管轄的一個非常重要的一個範圍。

另外大概就是領海。依照國際法,一般而言是我們的海岸線外推再3三海浬以內,基本上會是屬於我們的領海範圍。簡單而言,就是在領海範圍內的話,那原則上還是屬於我們的課稅權限所及的中華民國的屬地的來源。200海浬的經濟海域,那個不是領海的範圍哦。200海浬的經濟海域,那個只是在,國際級的經濟行政法裡面,認為國家在那個地方有專屬的漁業上面的採摘的權利,但那並不是領海的範圍。這個要還是要請各位同學要認弄清楚,我們所謂的領海,一般而言是指你的海岸線外推。國際法之間有一些是根據大陸棚的規定,還可以繼續往外推,那個是要看看國際法的規定。

那當然還有領空,當然在這個地方的領空並不至於讓你導出,就是只要是上面飛過去的所有的可能涉及到課稅管轄的事情,比如說外國的航空器飛過去,因此就要課我們的這個所得稅。並不是以這樣一個方式。那也因此,領空這個概念必須是在國際法裡面所承認,認為在這個部分是屬於中華民國在課稅權限所管轄的範圍。

另外,第二個要解決的問題是這樣,在這裡面有許多我們並不是,跟那個領土的範圍剛好重疊一致的,有一些是物理上在境內,但被認為是境外;相反的,也有一些是物理上境外,但應該被認為是境內。這一些主要是我國駐外的使領館。

我國駐外的使領館在這個地方,物理上是境外,但在課稅權限裡面,一般是被認為是我國的境內區域。也就是在中華民國駐外的使領館裡面所發生的,屬於應受課稅適用對象的,比如說在那個地方的勞務的付出。我們駐外的使領館一定會有派駐在外的這些公務人員跟非公務人員的使用,包括大使或是領事,甚至是派駐在外的這些負有公法上任務完成的這些武官跟外交人員。那么當然也有不執行公務的,純粹是技術性事務的,比如說是清潔事務,或者是屬於這個一般事務處理的,也就是並不是在執行公務的這一類型。當然這一類,一般而言,也可能會派本地的人員去前往,但原則上並不具有執行公物公法上的這樣子的一個勞務性質。從而,在這一塊,物理上境外但在國際法上被認為是境內的這個領域裡面,所發生出來的跟課稅管轄權限,特別是在這上面行使國家公權力的公務員跟非行使國家公權力的受僱員,因此所產生出來的勞務的付出所獲得的所得,依然是屬於中華民國來源所得。

相反的,外國使領館在中華民國境內,這個在物理上是屬於我國境內,但是在法理上,是外國的課稅管轄權限所及的範圍。

一般而言,在維也納公約底下,屬於駐外使領館範圍內的這一種在物理上雖然不是中華民國的境內,或者相反的,在物理上是中華民國境內,但在課稅管轄權限行駛上面的時候,處於駐外使領館的這個部分,當然必須要在我們的法律裡面有一個明文的規定。

除了這一種,透過維也納公約所認定的一些所謂的這個物理上境內或境外,而視爲物理上境外,跟境内的這一種情形,還有是屬於我國籍的船舶及航空器。這個也同樣都必須在我們的所得稅法第8條裡面去做明文的規定。

凡掛我國旗的我國旗的這個船舶或航空器,比如說長榮海運陽明海運,或者是長榮航空公司跟中華航空公司,原則上面他在這一個航空器或是船舶內所發生的,物理上可能是在公海,或者是可能在其他國家。

從而所謂的中華民國領土、領空,跟領海第一項的規定,跟第二項,物理上境外視爲境內,或相反的是,物理上境內視為境外,以及我國籍的船舶及航空器,原則上面都是屬於我國的中華民國領域範圍內的課稅管轄權限所及的,中華民國領域來源所得,中華民國來源所得。

你看這幾個條款規定,在我們的所得稅法沒有規定,在我們的來源所得認定原則也沒有規定,那請問你怎麼認定這個算中華民國來源所得?你只有加一個第14條裡面的所得類型,加一個中華民國來源所得,等於是你有規定跟沒規定是沒兩樣的意思。這個也是我們的法制上面,我個人認為,稅法規定不僅是規範密度過低,而且是常常是該有規定沒規定。

沒有規定,這個時候會讓司法機關,往往沒有受過學校裡面養成教育,或是在司法實務上,沒有形成一個比較一致性的見解的話,他會不知道要怎麼去做判准。

這個是我們跟各位提到第一項跟第二項所談到的這些規定,也就是,所謂的中華民國領域,這個到底是指什麼地方跟範圍。

\hypertarget{ux6240ux5f97ux4f86ux6e90ux5730ux4e4bux56dbux500bux5224ux65b7ux6a19ux6e96ux6982ux8aac}{%
\section{【所得來源地之四個判斷標準,概説】}\label{ux6240ux5f97ux4f86ux6e90ux5730ux4e4bux56dbux500bux5224ux65b7ux6a19ux6e96ux6982ux8aac}}

那接下來我跟各位談第三項的規定,也就是說,何謂中華民國來源的所得,要按所得類型去做分類。也因此各位還記得我之前在講所得客體類型的時候,有跟各位提到一個非常重要的分類標準。就是按你是繼續性跟一時性分別。是勞務或者資本的投入,做兩種不同的類型。按照是勞務裡面的非獨立勞務或者是獨立勞務,分兩種不同的類型。資本類型裡面,按照是孳息或是財產的交易所得分類。

各位都還記的這個圖形嗎?還記得哦,希望各位一定要記得,我沒有暗示什麽。但你不記得的話,我一定會用某種方式讓你記得。

非獨立性的勞務跟獨立性的勞務,三種不同的類型,依照你的業務內容。農林,大自然界的第一次產出。特殊的勞務,執行業務者。以及其他的工商農林漁牧礦冶的營利事業。工商農林漁牧礦冶的營利事業,記得跟農林所得怎麼區別嗎?還記得嗎?不記得,我也一樣提醒各位,我會用某種方式讓各位記得。執行業務跟營利事業又怎麼區別呢?還記得嗎,都是獨立型勞務。

營業所得,我講的就是獨資或合夥去經營工商農林漁牧礦冶,那裡面雖然有農林漁牧礦冶,但他跟農林的所得最大的區別是在,是否是大自然界的第一次產出,或者是加工的產品的差別。執行業務所得,原則上在我們的法制環境裡面,主要是指三種不同類型的特殊勞務的所得,除了大學養成的國家考試,除了人類智慧財產權的相關的,比如說表演作曲藝術啊這一類的,以及特殊勞務的特殊技藝的這一類所得以外,原則上其他都是屬於營業。

資本裡面,我們依照資本的類型分成的股利、利息、權利金還有,還記得嗎?這四種不同的資本的形態產生出來的法定孳息。

那麼所有的資本都還會有第二種所得,叫財產交易所得,這個就叫資本利得,capital gain的概念。

我們把這個客體的類型列出來。我們接下來一個一個跟各位去說,所得來源地怎麼去判斷。

\hypertarget{ux6240ux5728ux5730ux8ab2ux7a05ux539fux5247}{%
\subsection{【所在地課稅原則】}\label{ux6240ux5728ux5730ux8ab2ux7a05ux539fux5247}}

第一個標準。涉及到領土、領海的領域的這個部分,會有一個所在地的判斷標準。你只要是中華民國境內的農林漁牧礦冶,以及財產跟孳息裡面有一個類型是租金。因為租金可以是不動產的孳息。還有財產交易所得,由於你的財產可以是不動產。不動產的話,只要你是所在地在中華民國,我剛剛講的第一項,中華民國境內的領土、領海的範圍,原則上,這個就是所在地課稅原則。在哪裡,那個地方就被認為是所得的來源地。

國稅,基本上我們不分是在哪個縣市自治團體範圍內。這個,主要是區隔我們跟外國之間的課稅管轄權限。這個是所在地的課稅原則。

只要是這裡面跟不動產有關的,原則上都是所在地。跟土地、海域有關的都是所在地課稅原則。就是,這個東西物理上在哪裡。除非你有第二項,我剛剛所講的剛好是屬於物理上的境外,但在國際法上被認為是我們境內。

這個叫所在地課稅,就是這個東西在哪裡,就有那個地方被認為是所得來源地。

\hypertarget{ux52deux52d9ux7d93ux71dfux5730ux5c65ux884cux5730}{%
\subsection{【勞務:經營地、履行地】}\label{ux52deux52d9ux7d93ux71dfux5730ux5c65ux884cux5730}}

第二個,我們來看一下勞務的類型,裡面分獨立跟非獨立勞務兩種類型。

屬於獨立勞務的,有以下這幾個類型:執行業務、營利事業。這一些獨立的業務的類型基本上具有一個特色,他是一種獨立經常性的經營活動所產生出來的一個所得的類型,所以他是一個active income,因此這個會適用「經營地」來作為判斷標準。經營地會聯繫在國際租稅法裡面的「常設機構課稅原則」。因為activity,你的活動,要有一個活動的主體存在。國際租稅法裡面因此稱之為叫「常設機構課稅原則」,在我們的營利事業所得稅裡面,我們到下一次談營利事業所得稅的時候,會講固定營業場所或是營業代理人。那個概念,基本上也都是在講你的經營地點,常設機構經營地點。那當然如果你是一個營利事業,你自己本身營利事業的經營地點,也就是你實質的營運地點。歐陸法系國家大部分都是以你實際經營的地點來作為判斷的標準,而不是相反的,以美國為代表的,適用準據法,依照準據法地。

很抱歉,我突然忘記,準據法的據怎麽寫。因為現在寫東西不太像以前早期我們都用手寫,大家都現在用打字,所以我自己也常常會忘記哪些字怎麼寫,如果各位同學寫錯,我基本上我也比較寬認啊,因為我自己有同樣的問題,我不能苛責各位同學。我只有對某一些字眼,我想我會比較敏感一點,比如說稅捐救濟程序裡面,原文寫這個「復查」,你如果寫成「覆查」或「複查」,這個我會比較敏感。我們稅捐救濟程序是「復查」,這個如果寫錯,我看到我會稍微跳起來一下子,要稍微注意一下。

第三個我們講非獨立勞務則是以履行的地點,也就是勞務的實行地點。因為獨立跟非獨立的差別在,獨立,是一個自己去承擔財產上經營的風險而去經營的一個行為產生出來的經濟成果,我們稱之為叫執行業務所得,或者是營業所得。非獨立勞務的薪資所得,他的勞務的付出,原則上是受僱指示,從而他產生出來的所得,我們把他稱之為叫非獨立的薪資所得。對未來做給付的,原則上是退休所得。這一類的所得原則上是以當初他勞務在哪裡付出,勞務在哪裡履行來判斷。你勞務在哪裡付出,原則上就會是那個地方的課稅管轄權限所及。

\hypertarget{ux50b5ux52d9ux4ebaux7a05ux7c4d}{%
\subsection{【債務人稅籍】}\label{ux50b5ux52d9ux4ebaux7a05ux7c4d}}

那么最後還有一個,是這一類,很大程度上排除掉不動產以外的,其他的包括股利包括利息,包括權利金,原則上他都是由債務人,給錢的人的稅籍來作為判斷標準。因為他本身的資本型態,跟「所在地」最沒有關聯的大概就是現金通貨,就是你借錢這件事情,因為現金本身不具有個別性獨特性,只是全然一個支付工具而已。在虛擬通貨興起的時代裡面,你更難以認定那個通貨到底是哪個主權國家所管轄,或者是所給予的。因此在判斷標準上面來講,以利息為代表的這一類所得,不看是用什麼通貨去給付,是誰給付才重要。也就是採用屬人上的連結,給錢的那個人來做為所謂的所得來源地,而通常一般而言,會適用債務人的稅籍國來作為判斷標準。

這裡面我們總共四個標準,第一個,不動產的所在地,第二個,獨立勞務的經營地,第三個非獨立勞務的履行地、實行地,以及第四個,所謂的資本的債務人的稅籍地。

這四個標準,理論上來講,應該要在我們的所得稅款第8條的第三項裡面,做一個明白的指示,這個就是立法者做成一個價值決定的一個判斷標準。

當然,接下來我們會進一步去延伸,因為透過這幾個基本的標準,所在地的話,會有幾個所得的類型,由於他並不一定能夠完全被涵蓋,但大致上來講,這四大標準也是國際租稅法裡面去判斷所得來源地的來源地判斷標準。

非獨立勞務類型,我們有兩種類型所得,一個是薪資所得,一個叫退休所得。退休所得是過去勞務的未來繼續性支付,對吧?因為我現在退休之後沒有付出勞務,那也因此我在概念上沒有勞務的繼續性付出。這個時候退休勞務有可能從勞務所在地轉成,債務人支付,也就是以債務人的稅籍地來作為他的判斷標準,這個是有可能的。比如說你退休之後移民到國外,移民到泰國去,那麼當投保單位繼續給你錢,你當然可以以他是一個過去勞務本來是在臺灣境內所付出的勞務,那從而臺灣對這一筆退休所得享有課稅上的管轄權限。那同時另外一個方面,就是他是臺灣的投保單位,也就是保險單位所提供的給付,從債務人的觀點來看,他也會是以債務人的稅籍國,也就是是屬於我國投保單位所支付出去的保險金給付,會是我們的來源所得,我們擁有課稅上的管轄權限。會有兩種標準的可能。

接下來我們看到農林,農林漁牧礦冶,我們剛剛跟各位講過,基本上是大自然界的第一次產出,從而跟屬地管轄,領地跟領海的管轄權有密切關聯性,這一類所得在國際租稅法裡面,基本上都是領土優先,就是誰的領土發生的,那就是他們的所得,你不能把自己的手伸到別人家的領土的範圍內裡面,去要求課稅管轄權限。就如同我剛剛一開始跟各位舉的例子,美國人到北京工作,你憑什么課他的稅呢?就算他偶爾經過臺灣的桃園的機場,你也不能說欸,你要繳完稅你才能過境啊。我們沒有那個對他的一個所謂的那個真實性的連結。他不是我們的稅籍居民,他根本沒有在中華民國境內取得所得來源,你憑什麼對他去課稅?我會特別講這個,是因為我們只對設籍在臺灣的人,他不管取得臺灣地區來源所得跟大陸地區來源所得,他全部都被認為叫中華民國來源所得,課徵綜合所得稅的所得。這是我們的兩岸人民關係條例第26條第一項的規定。一個全然是基於意識形態所做的不利本國人的一個規定,我個人認為是一個違反平等對待原則的一個規定,是很沒有道理的一個規定。

回過頭來,屬地的這個部分。跟屬地有關的不是只有農林,特別是不動產裡面的租金,還有不動產的財產交易所得,這三種所得都是適用屬地原則,所在地原則,也就是東西、土地在哪裡,該當地因此就取得了課稅上的管轄權限。

從所在地原則會衍生出來我們的第八類,我們有一個競技競賽及機會中獎。在所得稅法第14條的第八類。比賽舉行地點在中華民國境內,所發出去的競技競賽及機會中獎的獎金所得,這個就是中華民國來源所得。所以,假設在本地,舉辦一個國際菁英邀請賽,或者是舉行一個大樂透遊戲。這個都是屬於,只要是競技競賽及機會中獎的舉辦地點在中華民國境內,那這個獎金原則上就會是中華民國來源所得。這是以舉行地點來作為判斷標準,你可以理解,舉行的基本上就是所在地的延伸,他跟領土管轄主權也有比較密切的關聯性。

那么屬於獨立性勞務的,排除掉農林所得,剩下的,包括執行業務者的勞務,跟營利事業的商品跟勞務,原則上因為他具有繼續性的經營活動的性質,只是勞務內容有所區別而已,所以一般而言是以經營地聯繫著常設機構課稅原則,不管你有沒有成立成立一個法人組織的型態。常設機構,本身不一定要是法人組織,非法人組織也可以。因此獨資跟合夥,如果是一個法人組織,那當然就是你的組織的這個經營地點在哪裡,那么這就是你的所得來源地的判斷標準。獨立勞務跟繼續性的貨物跟勞務銷售行為的地點基本上是經營地。

\hypertarget{ux5be6ux8ceaux5224ux65b7ux6a19ux6e96}{%
\section{【實質判斷標準】}\label{ux5be6ux8ceaux5224ux65b7ux6a19ux6e96}}

全球化底下,經營地點,可能是遍佈在全球範圍內。數位化經濟時代來臨之後,用更容易讓那個經營地點本身是不太容易去判斷的。也因此在這裡面,所謂的經營地點,一般被理解,叫做決策做成的地點,他的經營管理階層在哪裡做成決策。那因此在實務上面以歐陸法系為代表的幾個採取實質判斷標準的這一些國家,通常會以所謂的「實質營運管理地點」來作為判斷標準。這就是我們所得稅法第43-4條的PEM的由來。

正因為我們在實務上面並不是採實質營運地的標準,我們是採準據法地。我們準據法地這個判斷標準,不是只有課稅而已,基本上營利事業所得稅跟公司法都是採用準據法地,也就是依中華民國公司法設立的,就稱之為叫中華民國的公司。那如果你是依非中華民國,外國公司法設立的,就被稱之為叫非中華民國的公司。我們是以比較形式的標準去判斷是否為中華民國的公司。同樣以比較接近這種形式的標準來判斷,這個叫總機構在中華民國境內或非在中華民國境內。那也因為這樣子的一個形式準據作為判斷標準,從而我們在理論上面為了防杜,透過設立準據法來規避我國的所得稅法的課稅,營利事業所得稅裡面基本上需要PEM的制度,叫做「實質營運地」的判斷標準。

這個立法是出來了,但是我們現在目前為止43-4條的規定,目前為止還沒有預定在甚麼時候正式施行。不像43-3,我們已經從今年開始正式實行。但43-4什麼時候正式實行,目前沒有一個比較明確的一個時間。雖然有法條的規定,但並沒有正式施行的日期。

我們在稅捐規避行為裡面,大概你只要用外國法設立的話,原則上就不叫總機構在中華民國境內,所以你去看實務上一堆用外國法設立的。本來一般而言呢,不是以我們的法律設立的,他不能夠在中華民國的資本市場去募集資金,可是在我們的相關的法律做變更之後,這一些公司用外國法設立,但他也可以在中華民國境內的資本市場裡面去募集資金,這一些就是所謂的被稱之為叫KY公司,是一些境外的公司。他理論上來講,本來應該是從境外的資本過來,但因為我們現在目前為了要擴大國內的資本市場,也容許他在境內去做募集資金的動作,但相關的財務上的考核制度,我們卻在這個部分的建構上面,沒有做得比較完整。因為這一些相關的,他只是單純依據外國法去設立,所以就認為他是一個境外公司的這樣子的一個法律上的認定,我個人認為是一個非常草率,而且是一個非常容易被規避的一個規定。以前還可以透過資本市場,你如果是境外,原則上必須要把資本匯入,你才能夠在本地進行相關的投資或營運管理的行為。可是現在目前是,你用境外公司的法律設立的,那你還是可以在本國境內去募集資本。而你的所得,凡是非從臺灣所取得的,一切境外的所得,基本上就被認為是屬於由外國的課稅管轄權限所及。但這一些往往都不是在高稅負的這些國家裡面去設立,他有可能都是屬於在租稅天堂的地區國家所設立的境外公司。

我們在實務上面,就這一塊領域來講,我們並沒有很完整的一個防杜稅捐規避行爲的相關條文的規定。

原則上,經營地是一個實質判斷標準,以判斷所得的來源地國,那麼經營地點只要在該地,一般國際租稅法裡面稱之為叫常設機構,就是,你在該地經營原則上要有一個固定營業場所在該地經營,那這個時候那個地方就被稱之為叫所得來源地。稅籍居民國,雖然也可以課稅,但稅籍居民國,對所得來源地國行使的課稅權限要予以尊重,因此他要透過稅額扣抵的方式,來避免重複課稅。

重複課稅的避免義務,是稅籍居民國家的立法上的義務,而不是所得來源地國的立法上的義務。所得來源地國主要是避免,做了過度不當的屬地的連結,像我之前跟各位講到的那個例子,那個是一個不當連結的例子。

\hypertarget{ux975eux7368ux7acbux52deux52d9ux8207ux71dfux5229ux4e8bux696d}{%
\section{【非獨立勞務與營利事業】}\label{ux975eux7368ux7acbux52deux52d9ux8207ux71dfux5229ux4e8bux696d}}

那非獨立性勞務,這個地方也是一個蠻有趣的一個類型。營利事業經營的內容也包括了勞務的付出。勞務的付出,在營利事業而言是經營的問題,不是勞務的履行的問題。因為營利事業的勞務的付出,本來就是透過自然人付出,但自然人的付出,這個是歸屬於營利事業本身的經營內容。換言之,營利事業透過他所僱傭的受僱人付出勞務,這是在營利事業這個層次,要判斷他的經營地點是否在中華民國境內。那個付出勞務那個人,那個是自然人,那個人,他的勞務付出叫非獨立勞務,這個時候我們才會有非獨立勞務的履行跟實現的的判斷問題。各位可以聽得懂這兩個不同層次之間的關係嗎?

我一個經營者,我一個公司,我可以派一個受僱人,柯格鐘,到美國去執行我的勞務,那這個時候是我這一家公司有沒有在美國經營的行為,這是一個層次的問題。另外,第二個,我派柯格鐘到美國紐約去執行我們這個公司的營業的內容的時候,這個人在當地,他自己本身有一個所謂的非獨立勞務,依據其履行地點是否在美國紐約,因此,美國取得課稅管轄權限的問題。這是兩個不同層次的問題。各位可以明白嗎?

法人的經營跟受法人雇主指示,而在當地付出勞務,這是兩個不同層次的問題。第一個層次是營利事業的經營地問題。第二個層次則是受僱人的勞務履行地的問題。所以因此,當你第一個層次,營利事業的經營地點,如果我只是偶一為之,在當地派一個人,比如說,我在當地並沒有一個固定營業場所,我只是派一個人去那邊去接洽客戶,談成了生意以後,就把這個生意轉回來,讓我們的營利事業可以把貨物跟勞務銷售在該當地區。在當地沒有常設機構,從而他不會構成營利事業的所得,要課徵美國當地所得稅的問題。各位清楚嗎?

相反的。如果是受僱主指示而到當地去付出勞務的話,他是一個非獨立勞務,因為他的勞務本質上是要歸屬給他所屬的組織,作為經營的內容去判斷。但這個受僱的人,他本身是一個自然人,他的勞務履行地點是在美國紐約,這個時候一般我們以自然人的所得判斷標準,就是以他在當地停留是否夠久作為判斷標準。原則上就是183天。也就是當你受僱主的指示派到當地去,你在當地停留超過一個歷年的一半以上,你就會因為停留夠久,這個自然人本身構成了取得當地來源所得,因為你的勞務地點是在美國紐約當地。這個時候有沒有構成營利事業的經營所得的經營地點?這個是兩個不同層次的判斷。

這個時候,原則上你付出去的所得,在當地,不管是當地雇主給付,因為我們派出去,可能也會有當地的我跟那一邊的營利事業做了接洽,所以我派出去的人可能會取得當地的雇主,當地的人所給付的所得;也可能是本地的雇主付給他外派,讓他外派到國外去的所得。這個地方屬地的連結之外,在加屬人的連結來綜合評價,就該所得是否屬於源自於該當地的所得來源。

也就是勞務的履行地點,一般而言,超過歷年的一半以上就會被認為是屬地的來源所得。但如果沒有超過6個月,只要非由當地雇主給付的,我們這邊付出去給他的,那原則上對方那個地點並不是所得來源地,而是我們付出去的錢,我們派出去的員工,那我國還是擁有課稅上的管轄權限,而不是當地的勞務履行地的課稅管轄權限。因為第一個,他停留不夠久。第二個付錢的人債務人,並不是當地的雇主。這個是依據他的停留時間,以及債務人是哪一國的稅籍居民去做綜合評價判斷。

一般而言,超過歷年的一半以上,所得來源地就會是在當地,不管你是國內或境外的雇主所做的支付。短於183天的話,則在國際法上是依照兩國之間各自所得稅法的規範,如果有可能,透過雙邊租稅協定,去維持,或者是退縮各國的課稅管轄權限,這是透過雙邊租稅協定去洽商規範。

\hypertarget{ux975eux7a05ux7c4dux5c45ux6c11ux5728ux6211ux570bux5883ux5167}{%
\section{【非稅籍居民在我國境內】}\label{ux975eux7a05ux7c4dux5c45ux6c11ux5728ux6211ux570bux5883ux5167}}

請各位因此來看一下我們的第8條第三款規定的但書:「但非中華民國境內居住之個人,於一課稅年度內在中華民國境內居留合計不超過九十天者,其自中華民國境外僱主所取得之勞務報酬不在此限。」

我們透過第8條第三款但書規定,假設一個美國公司派一個美國的業務代表,到臺灣來來指導我們臺灣的公司,關於他的勞務要怎么去完成。因為像技術服務常常有這種情況。那個美國公司派過來臺灣來做技術服務指導的這個美國人呢,他是非中華民國的境內居住者。他在我們境內,他停留不超過90天。而且他的所得也不是來自於我國的雇主所給予的報酬,也就是根據契約上的約定,並不是中華民國公司付給他的,而是由他的美國公司繼續付給他的。從而透過第8條的第三款但書的規定,原則上我國課稅管轄權裡面,他直接把這一塊切掉,我不課。

所以外國的稅籍居民在我國境內停留不超過90天,而且他的所得是由外國雇主付給的,那原則上非屬中華民國來源所得,我們不課。即使他的勞務在這90天以內,是在中華民國境內付出,但這還不是中華民國來源所得。理由,第一個,他在中華民國境內停留不夠久,第二個付錢的不是中華民國的雇主。不是透過契約由中華民國境內的營利事業,或者是個人付錢給他。

除了這個比較明顯的,我們把他切割出去,不在我們的課稅管轄權限範圍內,基本上幾種組合類型就比較沒有明確的標準。

如果90天內停留,但是由我們的雇主,依照契約的規定約定付給他的。請問這個要不要課中華民國來源所得? 要課,就要開始用我們後面會講的就源扣繳程序規定喔。因為他只要是中華民國來源所得,付給他的人,要依法律的規定要去做就源扣繳,因為他是一個非境內居住者,他是一個非我國的稅籍居民。

超過90天以上的,我們沒有答案。也就是你90天以內的由我國的公司所給付給他的,因為我們請外國技術人員來,我們可能會給他一些比如說職務上的津貼,讓他住房住宿舍,或者是有所謂的加給。除了外國公司給,依照合約的規定,臺灣的公司也要給他一筆錢,那這筆錢算不算中華民國來源所得?如果是的話,那這就會要課徵所得稅的所得,要用就源扣繳,而這個問題的答案是,我之前跟各位講過,有沒有中華民國境內的個人跟營利事業的參與和協助。這樣就是要有參與的協助。

因此,在我們的所得稅法裡面,停留90天以內,外國雇主支付,我們明確切割出去,不屬中華民國來源所得,不課所得稅的所得。

除此之外的勞務付出,只要有境內之個人或營利事業的參與或協助,就很有可能依來源所得認定原則來做一定的評價,那當然來源所得認定原則裡面所講的協助或者是參與在第4點的規定裡面,有一些要件要求,尤其是第4點的第三項規定,協助或者是參與,是指說要提供設備整人力專門知識或技術等資源。只要有我們境內的個人或營利事業,提供設備人力專門知識或技術等其他資源,這個時候就可能被認為是所謂的參與跟協助。但不包含勞務買受人應配合提供勞務所需之基本背景相關資訊跟應行通知或確認之聯繫事項。

我個人當然知道,這大概想說要講的就是說我只是填表的話,這樣不算是參與跟協助。我只是填表,因為我們的營利事業跟對方營利事業簽約,原則上面我們有簽約相關的資訊去做提供,我只是告訴你我們這邊的這一些的背景知識跟應行通知或確認的聯繫事項,單純的一個資訊告知,不被認為是一個參與跟協助。可是任何只要有積極性地去提供設備人力專門知識或技術,不管是有形或無形之資產的技術上知識上的提供,只要有一個積極參與的行為,不管比例多少,原則上就會被認為是有我們的參與跟協助,因此屬於中華民國來源所得。

這個條文規定極大程度範圍地去擴張我們課稅管轄權限所及,背離了一般,如果在判斷經營地點的時候,必須要是常設機構,這樣一個國際課稅裡面的基本原則。你看起來會覺得好像我們是對外國的這個營利事業去課稅,因為他是外國到臺灣去取得中華民國來源所得的這種情形,但正因為他是一個非稅籍居民,取得中華民國來源所得,所以我們會用就源扣繳程序,基本上倒楣的會是跟外國做生意的,這一些我國的稅籍居民,包括營利事業在內。

我大致上把境內來源的所得,裡面的這獨立勞務的付出,跟非獨立勞務的付出,以及財產的所在地介紹完。休息完之後,我們再跟各位很快的講一下債務人的稅籍居民國,特別是關於他的權利金的給付的部分。因為權利金的給付,往往跟登記註冊地點相關。因為權利,專利、商標這種權利,原則上是一個無形資產,是一個無體的財產,所以會跟他登記在哪裡會有密切的關聯性。因此,根據債務人的稅籍國,未必會跟登記國是同一個國家,也因此在實務上面有可能是登記國家,也可能會行使課稅權限,認為只要是我國的專利商標,我們是所得的來源地國。但國際稅法裡面,原則上是以債務人的稅籍國來作為判斷標準,也就是是以人而不是以那個權利的這個登記地點來作為判斷標準。

這個地方我們待會休息完之後,我們很快地再跟各位講一下資本的孳息的這幾種類型,判斷的標準。

我們先休息。

\hypertarget{section-18}{%
\chapter{20231106\_02}\label{section-18}}

\begin{longtable}[]{@{}l@{}}
\toprule()
\endhead
課程:1121所得稅法一 \\
日期:2023/11/06 \\
周次:10 \\
節次:2 \\
\bottomrule()
\end{longtable}

\hypertarget{ux5b73ux606fux5340ux5206ux8cc7ux672cux985eux578b}{%
\section{【孳息,區分資本類型】}\label{ux5b73ux606fux5340ux5206ux8cc7ux672cux985eux578b}}

這節課,我們談一下關於孳息的這一類型。因為不同的資本類型,在標準上面可能會有不太完全一樣的地方。

\hypertarget{ux80a1ux5229}{%
\subsection{【股利】}\label{ux80a1ux5229}}

我們來看一下股利所得,股利盈餘。我在8條的第一款規定跟第二款規定裡面有跟各位提到過,你是依中華民國公司法設立的公司所發放的股利盈餘,原則上這個就中華民國來源所得,第二款規定亦如是。沒有寫,依合作社法,或者是依商業登記法所設立登記的合作社,或者是合夥組織或獨資商號。雖然沒有這樣寫,但從我們的認定原則裡面,基本上也是看,你是依據哪一國法來設立而發放的股利盈餘。我們在這個之前跟各位提到過,這個其實就是一個準據法,但轉過頭來,他其實就是在講債務人。債務人是誰,那他就是那個地方的來源所得。換言之,付錢給股東的那個人,付出去的那個人,他是哪一國的稅籍事業,那個地方就會是所得來源地。第一款跟第二款,其實就是債務人法,也就是以債務人的稅籍國來作為判斷標準。

這個稅籍居民,我們是用準據法判斷。但有沒有取得中華民國來源所得,則是用債務人法,也就是給錢的那個人,他是哪一國的稅籍居民,來作為判斷標準。

股本的股利所得,原則上是以債務人法,那債務人的稅籍的判斷標準,不是以實質的經營管理地點,而是以他的準據法來作為判斷標準。

勞務的旅行,第8條的第三款,境內提供勞務,其實在我們這裡沒有直接講他是獨立勞務或是非獨立勞務,按照我們的來源所得認定原則,第4點,也就是所得稅法第8條第三款規定,把營利事業他把執行勞務的部分,包括獨立和非獨立,全部都放在第8條的第三款規定裡面。實在這裡面應該要分兩個部分,應該是提供勞務,分成獨立跟非獨立勞務這兩種類型,那麼非獨立勞務,原則上第8條第三款的規定,應該是以勞務的履行,實行的地點來作為判斷標準。可是只要你是獨立勞務,則應該以第九款規定,也就是在中華民國境內經營工商農林漁牧礦冶等業之盈餘。

執行業務也是一種勞務的經營的行為,因為他是一個獨立勞務的付出,只是我們的法律上規定,我們並沒有第九款規定是適用在執行業務者。第九款規定適用在營利事業的經營上面。執行業務者,我們適用第8條的第三款規定來做判斷,因此在第8條第三款規定裡面,其實綜合包含了兩種,獨立勞務跟非獨立勞務的類型。從理論上來講,應該是區分經營地點跟勞務履行地點啊,當然這個地方也是在判斷你的中華民國境內提供勞務這個標準上面。

\hypertarget{ux5229ux606f}{%
\subsection{【利息】}\label{ux5229ux606f}}

第四款規定好,就涉及到了自中華民國各級政府、中華民國境內之法人及中華民國境內居住之個人所取得之利息。這個是由我國付出去的利息。我國付出去的利息,也因此利息的所得是按債務人法,也就是誰付錢的,原則上那個就是中華民國來源所得,只要是中華民國的政府、法人或個人付出去的,就是中華民國來源所得。這個就是債務人法。

這樣各位前兩種的資本的孳息的類型就出來,一個是股利盈餘,一個是利息的類型。

\hypertarget{ux8ca1ux7522ux4e4bux5b73ux606fux79dfux91d1}{%
\subsection{【財產之孳息,租金】}\label{ux8ca1ux7522ux4e4bux5b73ux606fux79dfux91d1}}

第8條第五款規定,在中華民國境內之財產因租賃而取得的。什麼叫境內之財產,就是你的財產為在中華民國境內,所在地點在中華民國境內。特別是因不動產租賃而產生出來的租金。當然第8條的第五款規定,不一定是不動產租賃,動產,只要在中華民國境內,其租金所得原則上就是中華民國來源所得。當然我們的困擾會是在,如果這個動產會移動的話,例如船舶或航空器,則應該是依照他的船籍國或航空器的登記國,作為他的所在地國。這個是第五款規定,我們有法律上沒有規定清楚的地方,因為是像我國籍的航空器跟船舶,他雖然是動產啊,但是在判斷標準上面來講,原則上是以船籍國或者是他的登記國來作為他的判斷標準。

還有其他的動產,也可能會移動。其他的動產的話,我們看一下在營業稅法的關於動產的部分。營業稅法第4條,貨物會有這一種不斷在移動的可能性,在營業稅關於貨物銷售地點的判斷,是以貨物的交付,如果需要移運的話,起運地在中華民國境內。另外第二個就是銷售貨物之交付地,如果無須移運的話,則以所在地點在中華民國境內。營業稅法的第4條,對於貨物這一類的動產的判斷標準,第一個,貨物在哪裡,那這個就是在哪裡的貨物銷售行為,這個是以貨物所在地作為判斷標準。這個標準原則上跟我們的動產,物在哪裡,一樣的判斷標準。可是當動產會動的時候會移運的時候,營業稅法的第4條是以起運地,而不在於到達地點,甚至在概念上,也可能會有經過第三地轉運至地點的可能,那么營業稅法裡面,在判斷貨物銷售行為的銷售地的時候,是以貨物在哪裡起運,作為判斷標準。

這一類的所得,在所得稅裡面要不要採取同樣標準,也就是當租賃之財產持續在移動,屬於船舶跟航空器,原則上是以登記國為基準,這個是我們剛剛所提到的。

其他的財產有移動的問題的話,那么原則上就有可能以移運之地,也就是起運之地點來做為跟所得來源地一樣的判斷標準。這個是當是一個動產會持續運動狀態底下所產生出來的所得,跟營業的這種銷售行為的判斷標準。

\hypertarget{ux6b0aux5229ux91d1}{%
\subsection{【權利金】}\label{ux6b0aux5229ux91d1}}

最後一個叫權利金,回到我們所得稅法第8條的規定,所得稅法的第8條裡面的第六款規定:「專利權、商標權、著作權、秘密方法及各種特許權利,因在中華民國境內供他人使用所取得之權利金。」

看起來是以權利使用的地點為標準,沒有直接去聯繫那個「他人」,是否是我國的稅籍居民,如果那個「他人」是我國稅籍居民,當然你可以理解,一般而言會在我們境內使用,那個「他人」可能是我們的稅籍居民。以我們的第8條的第六款規定,是以權利的使用地點作為判斷標準,這個地方權利的註冊地、使用地,跟債務人國,這個都可以是這一類的無形資產的課稅管轄權的主權國家或地區。因為他本身的特性是一個無形資產,他必須要透過註冊跟使用,以及給付對價,才能夠產生經濟上的成果,所以這三方都有可能依據各自所得稅法之規定,而採取他們的課稅管轄權的規範。以我們的第8條第六款規定,看起來是使用地,但國際租稅法裡面往往這一類passive income,是一個消極所得,原則上坐著躺著就會自動進來。那這一類的所得,往往都是以債務人的稅籍國來作為判斷標準。因為我從那個地方去取得所得,這個人不管他在哪裡使用,付錢那個人,付錢的那個國家,就會被認為是所得來源國。第六款規定,我們採取的是「使用地」,沒有對那個「他人」做特定的描述。

\hypertarget{ux8ca1ux7522ux4ea4ux6613}{%
\section{【財產交易】}\label{ux8ca1ux7522ux4ea4ux6613}}

第8條第七款,你只要是在中華民國境內的不動產跟動產,特別是不動產,這個是採所在地課稅原則。如果是動產的話,我們剛剛跟各位講過屬於動產的話,原則上船舶跟航空器,屬於我國籍的船舶航空器的轉讓,他基本上還是中華民國來源所得。如果你是一個不斷移動當中的財產,那理論上這個地方的境內財產也包含了權利的交易。權利,包括專利跟商標這些無體財產的交易。

第六款跟第七款規定,差別差異在第六款是使用權的授權給予,是權利金所得。如果你是所有權的讓與,則各國原則上會以註冊的地點來作為判斷標準。因為他是哪裡來的,那他就會是哪裡的專利跟商標權。

對那個境內財產交易這件事情的話,我們的法條,第8條的第七款規定本身並不是很清楚。如果不動產,那不動產所在地,沒問題。那如果你是動產的話,原則上船舶航空器當然各自會依照那個船籍國國或是航空器的登記國,來作為所得來源地的判斷標準,但如果他是一個持續移動的動產或者是不動產的話,會依照他的登記國,或者是動產的起運的地點,來作為判斷上的標準。

\hypertarget{ux653fux5e9cux6d3eux99d0}{%
\section{【政府派駐】}\label{ux653fux5e9cux6d3eux99d0}}

那麼,第八款的規定:「中華民國政府派駐國外工作人員,及一般雇用人員在國外提供勞務之報酬。」

這個要注意,這邊必須是視為境內的地方。不然我們的課稅管轄權限,無權去管轄,物理上境外的地區,我們卻對他們去做國外勞務提供的報酬的課稅。所以第八款的規定是指中華民國政府派駐國外之工作人員及一般雇用人員,在國外的相當於我們的使領館範圍內所提供的勞務,因此所產生出來的報酬。他如果不是在使領館範圍內提供的報酬,我們對他根本也沒有管轄權限。我們政府派出去的國外工作的員工,不履行公法上任務的這些人員、一般雇用人員在國外提供勞務,他已經是境外,根本不在我們的課稅管轄權限範圍內。

這個條文規定,不管是在主體跟適用的客體範圍上面都不盡清楚,都不是一個很完整的法條規定,一般僱傭人員要特別注意到,我國政府派出去國外的人,有一些是實行公法上勞務的,也就是他在境外是行使我們的外交管轄,外交的權限的這些人員,他才會是前半段那個部分的適用人員。因為還有一部分的事務性勞務,有可能是派遣我國人員去那個地方,比如說派人去那邊做清潔打掃,可是因為各國為什麼會有這種派駐這種國外的情況,最主要是避免萬一所在地國你跟所在地國基本上是處在一個比較緊張的關係底下的話,你可能在當地雖然找一個清潔工或是找一個來做打掃的人,可是可能是對方的間諜。所以我們有可能派出去的,有一部分是屬於履行公法上勞務的,這一些外派的公務員,不管你是文官和武官,另外一方面,我們也有可能是派遣出去國外的事務性工作的這些人員包括負責影印機,包括打掃清潔,甚至包括住宿服務,我們可能都會有派駐國外非履行公法勞務的這些,一般的派駐人員。這個部分,只有前半段,是維也納外交公約裡面所明白認為,只有這部分的公務人員的所得,所在地國不能對他課稅,至於後半段這個部分,這個是要依雙方國家的交涉跟談判去做認定的,並沒有必然要免稅。即使他工作的地點可能是在境外使領館,可是由於他非執行公法上的勞務,這個部分是容許由雙方就是派駐國跟駐在國,兩方面去做協調,來達成相互一致的協議。如果你給我免,我也免你的,那這個雙方是互惠的,但沒有必然免的道理。只有執行公務的人人員,他才會有必然免除的適用,這個是在維也納外交公約裡面所規定的。

目前我們的外交處境的特殊,往往沒有國對國之間的外交關係,因此透過授權給公權力私人行使的這一些被授權的人員,只要是實行公法上勞務的這一部分的人員\ldots\ldots 我們當然不是直接是維也納外交公約裡面簽署適用的對象,可是在這裡面,由於我們在外交上面採取的是受委託公權力授權給私人行使的方式,因此我們派出去的跟相對方派過來的人員,基本上都是透過我們跟對方之間雙方的互談以後,才因此會有適用我們第8條的第八款的規定。

比如說美國啊,美國派駐到臺灣的,跟我們派駐到美國的臺灣駐美國的代表處、辦事處裡面的相關人員,基本上也是適用這樣一個條文的規定。雖然我們不直接適用維也納公約裡面的相關的條文規定。維也納公約,是要我們跟A國跟B國,有一個正式的外交關係,彼此之間是執行公法上任務的這些外交人員,原則上一方是不能夠去對他方課稅。這個人員都是在履行各該國家的公法上的權限。未經允許而進入他們的外國使領館的範圍內,這個是被認為是戰爭行為。這個1978年的伊朗德黑蘭的那個學生,爲了抗議美國,衝進去那個美國的外交使領館去綁架相關的人員,這個在國際法上,就是視為戰爭行為。這個叫沒有宣告,就發生發動戰爭的行為。所以美國他的反擊就會是依照國際法裡面的,你既然對我宣戰,那我也對你去做這樣子的一個宣戰的行為。

所以,外國使領館並不是一個國家的課稅主權所及,即使在物理上是屬於該外國境內範圍,但並未經過允許,他基本上就是不能去進入,在這裡面去行使,包括課稅管轄權限在內的各國的主權。這個是維也納外交公約裡面。所對各國之間的相互往來所適用的基準的行為上的規範。非履行公法任務的這一部分,並不在當然適用範圍內,而是從許由雙方各國來互相來去談判而去做個別的決定。你可能互不免除,也可能互相免除,或者是單方免除,這個都有可能,但原則上要不是互不免除,要不就互相免除。

\hypertarget{ux5883ux5167ux7d93ux71df}{%
\section{【境內經營】}\label{ux5883ux5167ux7d93ux71df}}

第8條第九款的規定是指境內經營,也就是經營地,原則上應該是屬於獨立勞務,不管你是貨物跟勞務的銷售,第8條第九款規定,應該是以經營地點來作為判斷標準。這個就是獨立勞務,或者是營利事業經營的判斷標準的原則。那當然我們這個地方關於境內經營這一類的事件,由於在我們的法條規定裡面本身並不是很清楚地去呈現出來。這個地方因此實務上透過來源所得認定原則。

原則上如果一個passive income,變成是一個activity的一個經營的內容,變成是積極的所得來源,這個時候會被認爲,你的事業經營就在該當地該當一個常設機構,從而會適用第九款的規定,也就是以經營地來做判斷。

這也是雙邊租稅協定裡面會涉及到的,也就是說,營利事業,當你的消極性的行為反覆為之,會讓他變成是積極性的行為,而轉變則是一個營利事業所得的態樣,適用經營地的判斷標準,也就是以常設機構所在地作為一個課稅的所得來源地。

\hypertarget{ux7af6ux6280ux7af6ux8cfdux6a5fux6703ux4e2dux734e}{%
\section{【競技、競賽、機會中獎】}\label{ux7af6ux6280ux7af6ux8cfdux6a5fux6703ux4e2dux734e}}

最後一個是關於第十款的規定,也就是競技競賽,沒有很清楚地說明,境內參加是什么樣的意思。其實在這裡面,應該是以舉行地點作爲依據。因為競技競賽及機會中獎,一般而言,會有一個競技競賽舉辦的地點,而以這個地點成為所得來源地的來源地的判斷標準。

\hypertarget{ux5176ux4ed6ux6536ux76ca}{%
\section{【其他收益】}\label{ux5176ux4ed6ux6536ux76ca}}

最後一個是十一款的規定,在中華民國境內取得之其他收益。我們在這裡面屬於一時性的,我們跟各位列出幾種類型,一時貿易盈餘,競技競賽及機會中獎,還有一個就是具有射悻性的非法行為的判斷標準。

我們第十一款規定,在中華民國境內取得。什麼叫中華民國境內取得?這是在實務上不容易判斷的地方。你的貿易,如果是一時貿易,這種經營行為本身是在中華民國境內,理論上來講,用第九款規定就可以,而不需要去適用第十一款規定。那麼第十一款規定往往是對於非法經濟活動,你在非法的經濟活動的活動地點在這個境內所取得的,那因此就會是屬於該地的來源的所得。

我們實務上有來源所得認定原則,但長達5頁的認定原則,往往甚難給予司法機關一個客觀上比較明確的判斷標準,我們透過解釋上的說明,原則上四個原則:經營地、勞務實現地、所在地,還有就是債務人地。就是債務人,給付的這個人的稅籍。這四大標準是所得來源地的判斷標準,所連結的是我們的稅捐客體,所得來源。我想我們就到這裡跟各位做一個客體方面的結束。

\hypertarget{ux7a05ux6350ux5ba2ux9ad4ux5c0fux7d50}{%
\section{【稅捐客體小結】}\label{ux7a05ux6350ux5ba2ux9ad4ux5c0fux7d50}}

稅捐客體,涉及到總共四個層次不同問題:是所得、非所得的問題,是應稅所得跟非稅所得的問題,是屬於何種所得類型的問題,跟所得的實現時間跟所得實現地點的問題。每一個區分都有區別的實益。這個是我們跟各位談到稅捐客體的部分。

\hypertarget{ux7a05ux57fa-1}{%
\section{【稅基】}\label{ux7a05ux57fa-1}}

14條結束了,之後就進入到第17條的規定。稅基的概念是客體的數量化,也就是每一個都要把前面講的客體帶上一個額度,所得額。比如說你是第14條第一類的營利所得,稅基就是營利所得額。

稅捐客體的數量化被稱之為叫稅基。為什麼被稱之為稅基?他是稅捐計算基礎的簡稱。稅額就是你的稅捐負擔,稅捐負擔的計算基礎是透過稅基乘稅率,得到稅額。

因此在稅捐法總論裡面,我們提到所謂的總額主義,在講稅額。形成稅額的基礎有兩個,一個是稅基,一個是稅率。稅基怎麼來?是看稅捐客體,也就是稅捐主體獲得的稅捐客體,透過計算方式把他換成稅基,乘上稅率,得到稅額,這個就是你的稅負,就是課稅處分裡面所載的那個稅額。

納稅義務人不服的,往往是,我知道我要繳稅,我有營業行為,我有所得,但我所得額沒那麼多。所以一般而言,納稅義務人不服,會展現在他對稅基的計算基礎上的不服。正因為稅基是所有的課稅構成要件裡面,我把他稱之為他是最重要的構成要件,因為他承接了前面的稅捐客體,之後就是稅額,所以他是關鍵要素。稅基,因此要由立法者自己決定。基於重要性理論,立法者要自己決定應稅所得額,到底怎麼算出來,應稅銷售額到底怎麼出來的。這個要由立法者自己做決定,這個是重要的,重中之重的構成要件要素。

那么另一方面,所有的稅制改革裡面的最重要的基礎,是稅基,而不是稅率。

非法律學者往往把他們的稅改重點放在稅率。稅率,就是去調整那個比例,但任何明白稅額是透過稅基乘稅率的人,再怎麼樣的一個簡單的道理,這個只是稅捐負擔比例的高低的調整而已,稅基才是決定稅額的最重要的基礎,所以稅改重點是稅基,不是稅率。但很可惜的是,以往操縱臺灣,包括產官學界裡面,所謂認為的稅法的專家,幾乎都是以財稅為背景的,他們把重點都一直放在稅率。所以我們2009年的遺產贈與稅法的改革,一直都是在講稅率。我們的房地合一稅,所謂的對短期持有,要加強他的稅捐負擔,我們也是把重點都放在稅率。

但法律背景的我們認為,應該重點是在正確的稅基,才有正確的稅額,才有正確的量能課稅。如果沒有正確的稅基,稅率,尤其是累計稅率,只會扭曲稅捐負擔分配的結果。例如,沒有被適當扣掉的稅基,就會加重納稅人的稅負,會扭曲稅捐負擔分配的效果。從而,未來如果哪一天,各位同學有人有機會成為立法院裡面的立法委員。你可以講,整個稅制改革的重點是在所謂的寬稅基、低稅率,廣稅基、低稅率。稅改的重點,一般稱之為叫廣稅基、輕稅率。稅率輕不輕,要看前面稅基,稅基廣,那當然稅率就輕,稅基窄,國家還是要錢啊,那當然稅率就會高,可是越高的稅率會說引起人們不斷地逃漏稅基,因為這是相對應的。越廣的稅基,會讓人們會覺得,他的稅捐負擔能夠在公平的計算方式底下算出來,這個才是符合稅捐負擔分配正義的稅法。

也因此我們進入所得稅的稅率之前,稅基是最重要的構成要件。理論上應該要由當事人在稅捐申報的時候提供正確的數額,來讓稅捐稽徵機關一方面依職權調查,你提供的這些數額對不對,另外一方面,他當然也要依職權調查,查核有沒有未申報的所得。這個正是協力義務。協力義務,透過納稅人告知正確的所得額。我們的所得額,進入營利事業所得稅裡面通常會有收入額,成本費用額,甚至有虧損額度,所以在這個地方,我個人建議,任何稅基的計算都應該要加入「額」,而不要單純只講「收入」,收入是客體概念,算成數額應該要加一個額度的「額」。在這裡面,收入額、成本額或費用額、盈餘額跟虧損額,都一樣,我個人是認為都要加「額」這個字眼,才能彰顯出客體數量化的稅基名詞。

不過很可惜的是,我們現在目前實務上基本上都喜歡用收入、成本、費用、虧損,都是用這些比較有可以指稅捐客體,又可以指稅基的名詞來做描述。

這個是首先我們跟各位去談到稅基跟稅捐客體之間的關係。稅基作為稅捐計算稅額負擔計算基礎,是一個非常重要的概念。

\hypertarget{ux5ba2ux89c0ux8207ux4e3bux89c0ux51c8ux6240ux5f97}{%
\section{【客觀與主觀凈所得】}\label{ux5ba2ux89c0ux8207ux4e3bux89c0ux51c8ux6240ux5f97}}

所得稅法14條,主要是客體的類型,14條裡面也是要去計算出他的客觀凈額的範圍,客觀凈所得的範圍。簡單來講就是個類型的所得,是透過14條的規定,以收入減成本減費用,甚至必要的時候減除掉法律所允許的虧損。14條的規定,因此是屬於我們在稅基計算上的前階段,也就是第一個階段叫,各類所得的淨額。

那么14條算完之後,會有一個各類所得的綜合總額。我們請各位同學來看到第17條的規定:「按第十四條及前二條規定計得之個人綜合所得總額,減除下列免稅額及扣除額後之餘額,為個人之綜合所得淨額」

這邊的「個人綜合所得總額」,我個人認為啊,「總額」會讓人家以為是沒有扣掉成本費用的。

那個地方其實是兩階段的計算方法。14條15條跟16條,這幾個條文算起來是為了要算客觀凈額。從第17條開始,是要進入主觀淨額、主觀净值的範圍。也就是市場經濟活動算出來的净額,之後就進入到我們稅法裡面,維繫個人生存,維繫個人跟個人一起生活的配偶及受扶養親屬的生存需求。

因此,前面14、15、16條在立法的結構上,是客觀凈所得的計算範圍,接下來進入17條的是主觀凈所得。稅基,是分兩階段計算。第一階段,客觀淨所得。第二階段,主觀凈所得。

第一階段,法律學、經濟學都承認,客觀淨值才是稅捐負擔能力的計算基準。

第二階段,只有法律學者認為很重要,甚至重要性高過客觀凈所得。因為我們認為賺錢不是為了繳稅給國家。賺錢是為了要養活我自己,跟我自己親愛的配偶跟家人。從而主觀凈所得的減除,負有憲法上維繫生存權保障的這樣一個任務,比客觀淨所得,在規範的位階上,雖然排列在後面,但憲法上的意義則更為強烈。

前階段這一部分的客觀凈所得,如果不準許解除,侵害的毋寧只是工作財產權。後半段的這個部分沒有核實的反應,則侵害到憲法上的生存權保障。

也因此,我們在稅法總論裡面區分了主觀淨值跟客觀淨值原則,最主要的表現就在這裡面。這一個表現在個人綜所稅裡面,透過14、15、16條,客觀淨所得的計算,第17條開始,這個是主觀凈所得。

第17條的主觀凈所得,在法人沒有這個問題。因為法人沒有維繫生存的概念,法人是一個人的組織體,是一個人的集合體,就算採用法人實在說,法人仍然沒有像個人自然人一樣的維繫生存的必要或需要。法人沒有吃飽穿暖的問題,沒有維繫人性尊嚴的問題。儘管你可以講,法人在基本權的保障,依其性質,有同樣適用的可能性,但維繫生存的需求這一件事情並不存在。

從而在這裡面,我們進入第17條之前的時候,有幾項基本的觀念要讓各位清楚。

第一個,經濟學者普遍並不認同主觀凈所得。他們認為,稅捐只有客觀淨值,沒有主觀淨值,主觀淨值已經影響到正確的稅捐負擔能力的計算。以他們來講,賺錢凈所得就是指客觀淨值,之後就是稅捐負擔能力的起點。對法律學而言,我們賺錢不是為了國家繳稅給國家,我們賺錢要先養活自己。從稅法的角度,稅捐負擔能力是主觀淨值。這個主觀淨值也表現出來稅法跟社會法的一體兩面。因為,我賺錢是養活我自己,然後我還要養活我這一些受我扶養的人,這些人包括配偶跟受扶養親屬。在社會法裡面有一個補充性原則,國家是補充性的介入,當一個人自己沒辦法養活自己,也要先由其他負法律的扶養義務人先養他,如果連這個負法律扶養義務人也養不起他,我國家才來介入,用社會救助法把你撈起來,建構社會安全網,把你扶助起來。

從而在法秩序的一體性地下,社會法跟稅法,共同建構起的共同的法律價值,就是我養我自己的,我當然可以扣除,我養我負有法律扶養義務的這些人,因為國家也只有在我養不起他的時候,事後性地、補充性地,才去承接這些人,從而這一部分的生存權保障,在我可以照顧他的時候,都是屬於不可自由支配的所得,應該要準許減除。

這反映的是憲法位階的生存權保障,從而在我們的釋憲實務上,一般涉及生存權保障的財產權問題,會提高其違憲審查的位階到中度的審查基準,背後的理論基礎就是在這樣一個前提基礎上。第17條的規定,因此是生存權保障的稅法上的實踐。

在進入所得稅法的規定之前,我們不可避免,還要再談一個概念,在納稅者權利保護法裡面,有一個基本生活費用不課稅原則,在納保法的第4條。根據地納保法第4條第一項,「納稅者為維持自己及受扶養親屬享有符合人性尊嚴之基本生活所需之費用,不得加以課稅。」

第4條的基本生活費用不課稅原則,這個條文規定現在在目前的實務上,幾乎取代了所得稅法裡面的關於免稅額跟扣除額的規定,是因為如果基本生活費用不課稅原則,他直接用一個額度就涵蓋了所有的這一些維持生存的費用所需的額度的話,那麼納稅人必然不需要去適用所得稅法裡面的免稅額跟扣除額規定。

簡單來講,就是,除了免稅額以外的所有的生活費用,全部都會被納保法第4條基本生活費用規定所涵蓋。

基本生活費用不課稅原則,理論上應該是立法者據以調整所得稅法的免稅額跟扣除額的一個誡命義務,也就是他應該要時時調整所得稅法裡的免稅額跟扣除額,俾便反應納保法第4條的基本當代基本生活費用。簡單而言,就是的納保法第4條的規定應該是作為所得稅法的免稅額及扣除額,立法作為義務的一種誡命。立法者應該要與時俱進檢討,我們現在的免稅額跟扣除額規定,俾便能夠實質反映當代生活的最低生活費用,而不是另外取代他,把他取代掉。那這樣會造成了我們所得稅法裡面的這些相關規定在現實上的適用就不太具有意義。納保法增訂這個條文理論上來講,應該是促使立法者,負一個作為的誡命義務,調整所得稅法裡面的免稅額扣除額的規定。

那這個是我們對納保法還有對所得稅法兩者規範之間的體系性的瞭解,彼此之間的關係,所產生出來的這樣子的一個法律適用上的關係。

進入到第17條的規定,他的法條規範結構上總共有三項的規定。所得稅法的第17條的規定非常的長。第一項裡面總共就有3個或4個不同的名詞。

第17條的第一項第一款有「免稅額」規定。
17條的第一項第二款的第一目,有「標準扣除額」。
17條的第一項第二款的第二目,有「列舉扣除額」。
17條的第一項第二款的第三目,有「特別扣除額」。

各位很抱歉,立法者不是我,我也沒辦法,就長這樣啊。所以17條第一項,我們總共有4個專有名詞,這4個專有名詞,免稅額之外,標準跟列舉二擇一,之後再有一個特別扣除額。

所以規範結構是這樣,免稅額先,然後接下來標扣跟列舉扣除二擇一,然後再加特別扣除額。

請各位務必把這個法條適用結構在這麼複雜的規範邏輯當中,外在體系很不好,但至少這一個適用的順序,原則上各位一定弄清楚。

免稅額,標扣跟列扣是二擇一相互替代,然後再加特別扣除額。

那我現在跟各位說一下,我們在禮拜六的時候補課的時候,我們再跟各位進一步去說明。免稅額是維繫物理生存需求,基本上就反映出當代生活的維持溫飽,就是吃飽傳染好,還有第二個這個部分叫標扣跟列舉扣除額,他的概念上是維繫物理生存需求之外,因為人不只是一個動物的人,除了吃飽穿暖以外,人也是一個群居的動物,需要跟別人有所往來連結,所以他有一個維持社會生存的需求。這個維持社會生存需求的這個額度,則由中間這個標口跟列舉扣除額,用二擇一的方式去反映。第三個叫特別扣除額,因為個人的特殊因素給予特別考量而加進來的扣除額。

各位一個一個去對照,免稅額是物理生存,基本上反映出維繫物理生存,你可以講馬斯洛需求的層級,人有一個至少要吃飽穿暖的需求。然後接下來是要有一個大家可以跟人往來的需求,至少我是個人,我不是只有吃飽穿暖,我不是一個動物的人,我還是一個我需要有住的地方,讓我知道我朋友來找我的時候,他可以找得到我,而且我要活得健康。所以在這裡面,我們把這一切把他稱之為叫維繫社會生存的需求。這兩個合起來就是基本生活費用。基本生活費用就是剛剛我們提到了納保法第4條所講的基本生活費用。接下來還有一個叫特別扣除,這個特別扣除額講什麼?因為你個人的年紀身心狀態而產生出來的特殊的個別的需求而給予扣除。這裡面有包括量能課稅跟稅捐優惠,理論上來講就是你個人做了一些好事,那我就給你特別的額度。

那基本上這個概念在這裡面,我們反映出來的就是,一個納稅義務人,除了獲得市場經濟活動的客觀凈所得以外,繳稅的稅捐負擔能力也要扣掉維繫個人及家庭成員的維繫生存需求,及個人特殊因素所產生出來的這些需求。最典型的就是身心障礙者。因為我們前面的算法基本上是以一般人的平均來做為比較的標準,那你如果有身心障礙的特殊因素的話,再給你加上去。所以身心障礙特別扣除額是一個很典型的因為個人的特殊因素產生出來的扣除額。

那我們在禮拜六的時候會跟各位把這整套的稅基主觀凈所得的計算方式來跟各位做一個說明。對各位而言,這個部分是屬於考試上比較重要的地方。說實在話,客體的部分有時候是去做區辨,當然也很重要,我也不否認他是很重要,但考試比較不太好考試。稅基比較好考,特別是净所得的這個部分。

我們今天先講到這裡。

\hypertarget{section-19}{%
\chapter{20231111\_01}\label{section-19}}

\begin{longtable}[]{@{}l@{}}
\toprule()
\endhead
課程:1121所得稅法一 \\
日期:2023/11/11 \\
周次:10\_1 \\
節次:1 \\
\bottomrule()
\end{longtable}

下個禮拜一,後天的考試,各位同學都很關注。老師在題目裡面是分兩個大題,下面再拆一些小題。比照國考,大家帶紙本的法條。如果各位像我一樣,我也會習慣在上面做一些劃線註記,這種都沒有關係,包括你這上面有大法官解釋,我都沒意見,因為現在社會本來這些資料就常常都是會連在一起。像我們在德國考試,稅法是兩大法典,就是紅紅的很大,就這樣,大家都帶著。法律系學生很好辨識啊,跟會計系學生都一樣,很好辨識。會計師就拿個一個計算機,法律系就拿個六法。我們就是拿這些,小六法或者是大六法都可以,隨便你,帶你自己的法典。

我至少會希望各位同學對法典稍微要熟一下,其實這個也是有助於各位如果要參加國考。其實國考不難,都是法典的運用而已,我就不太知道為什麼大家這麼畏懼財稅法。財稅法每一年考試都人數最少的。不知道什麽原因。那其實都是法條的這個運用而已。所以,我這兩個題目基本上也都是法條運用而已。就請各位就帶著法典。

當然在學習上面,我會希望各位有一個總體的概念,這也是老師在教學上面一直在強調的重點。我們其實不是注意到樹木而已,我們要有一個「林」的概念。那「林」的概念就是這樣,因為我們稅法本身,我記得我在第一節課跟各位講過,有那個內在、外在體系。就是說,稅法,其實他本來應該是依構成要件逐次展開來做相關規範,可是因為我們的稅捐法律條文的規定,他並沒有很好的外在體系做法,不像民法。民法,因為是歐陸法系最早發展的,所以體系化做最好。而且因為我們抄別人的,抄得還不錯,體系化就做得很好。凡是我們自己立法的,我們體系化都做得很爛。你不能怪他們,因為在他那個背景裡面,真的沒有這個體系。那沒關係,我們自己要來做自己的體系。

所得稅法其實就是構成要件一步一步,稅捐主體、客體、稅基、稅率,就這樣把法條的規定一個一個排列。就好像我們稅捐主體的構成要件,就是境內居住者跟非境內居住者,包括稱之為叫稅籍居民跟非稅籍居民。

稅籍、國籍、戶籍這本來就不一樣的概念,我第一次上課就跟各位講過了啦,有時候會找國籍、戶籍來做聯繫因素,但不是必然的,因為稅法本身就有自己的規定嘛。我們的兩岸人民關係條例確實是稅籍連戶籍,但戶籍不等於國籍。那國籍可不可以當作稅籍的聯繫因素要件?可以啊,我們的遺產贈與稅法就有把國籍聯繫進來,但是我們所得稅法是沒有連結國籍這個要件。

所以,不同的法律有自己的稅捐主體,這個叫權利能力的概念,就像民法有權利能力,民訴有當事人能力是一樣,這個本來就依照各法律去規定他自己的權利義務的歸屬主體。所以這個地方,稅捐主體、稅籍居民跟非稅籍居民,聯繫的構成要件,法條裡面就有規定。境內居住者的構成要件在所得稅法第7條第二項,那第7條第三項就是非稅籍居民的構成要件。

好,客體,我在這個地方花很長一段時間跟各位去講課題,包括了所得的概念所得跟非所得,然後第二個層次叫應稅所得、免稅所得,然後接下來是類型。所得本來是綜合所得,但我們就有分類所得的概念,這個不僅是因為我們的立法本身,一開始是從分類所得逐次慢慢演變,變成現在的綜合所得稅,另外一方面是我們分類所得的概念,有分類的實義,在徵收跟稅額的計算裡面,真的會有區別。那麼這個分類所得的概念之後,我們談到所得實現的時間跟地點,然後,實現的時間跟地點裡面有會有一個客體跟主體之間的這個叫歸屬的判斷,原則上所得的取得所得者,這個就歸屬給他,但那個在稅總裡面,特別在講那個經濟實質叫實質課稅原則,就是在那個歸屬主體的判斷上面,可能會有產生類型的變化,還有那個歸屬主體的變化。就是我透過一個人去取得的話,到底是他的所得還是我的所得?這個叫歸屬的判斷。在我們所得稅裡面,其實這個歸屬判斷,客體跟主體之間的聯繫關係也是非常重要的,這也是我們在稅捐實務上最常見的爭訟實務的類型。像借名行為就是最典型的。借名行為算誰的錢?出名人的還是借名的人。這個是實務上很大的爭議點所在啊。我如果狠一點,我就多考一點借名,我這次不會考,但實務上這類非常多。規避,也是屬於這裡面再考那個主體客體歸屬上面。

\hypertarget{ux7a05ux57fa-2}{%
\section{【稅基】}\label{ux7a05ux57fa-2}}

客體之後進入稅基,上禮拜一的時候,我們開始講稅基了,但其實稅基在某程度上還是會跟客體有連結。

在還沒有進入稅基之前,客體算出來,這個在我們的法典上面稱之為叫應稅所得總額,這個名詞出現在第14條,第一項本文規定應稅所得總額是以客體的十類所得的總額計算。

在這裡面,稅基的計算,我們的法律的規定是在第17條裡面,有幾個專有名詞是17條第一項第一款的免稅額,以及所謂的標標準的扣除額跟列舉扣除額,第2層次的17條第一項第一款跟第二款的規定,以及最後面這個叫特別扣除額。稅基的計算是經過這3道,17條的第一項第一款,17條的第一項第二款第一目標準扣除額,第二款第二目的列舉扣除額,17條第一項第二款第三目特別扣除額,這3道過濾以後,才會算出一個應稅的所得净額。14條的應稅所得總額,經過免稅額扣除額以後,這個叫應稅所得净額。

法條是這樣規定,但我們在釋義學的理解是這樣,就是說,我們在法律規定的層次裡面前面的第14、15、16,都是客觀凈所得、客觀淨值原則的計算,第17條則是主觀淨值主觀淨所得的計算。然後17條之後,先不用管他,比如說17條之二,那個是退給稅款,例如重購退稅的規定,這個條款規定本身是稅捐優惠,是構成要件的附加,所以在舉證責任分配上面是由主張退給稅款的納稅義務人來負責舉證責任,證明該項課稅構成要件。

在這個過程當中,法條規範結構是長這個樣子,看起來很完美,但是你看哦,我們現在的立法者,林林總總加了14-1,後來有14-2、14-3,然後後面還有14-4到14-8。你看我們的法條規範結構,這真的是修法,都不覺得他很醜。亂插進來這些法條,整體的規範結構就變調了。

本來是前面14、15、16是客觀凈所得,17條是主觀凈所得,但現在我們開始插進來一些法條。

我們現在來看一下14-1條的規定,14-1在講什麼?講公司在債券利息所得的扣繳。14-1這個條文規定,法律體系的位置,其實應該放在14條的第四類。因為那個地方,他本來就是利息所得放的位置。

「個人持有公債、公司債及金融債券之利息所得,應依第八十八條規定扣繳稅款,不併計綜合所得總額。」

14-1條,首先就破壞綜合所得課稅原則,自己做分離課稅,但法條位置也不是放在14-1,是應該放在第14條的第四類,利息所得那個地方。也就是你應該要放的話,也是放在第四類第四款去增訂一個條文規定。立法上的體系位置,要對應在對的位置,那個法條規範結構才不會被扭曲掉。然後14-2因為被刪掉了,我們就暫時不講。

14-3,你看一下法條規定,誰要適用?同學有上老師的稅捐規避課程的話,都知道這個條文本來是對應66-8,就是個人未分配盈餘的課稅,防堵利用公司來規避股利盈餘分配的規定。所以我們在66-8,隨著兩稅合一制度廢止之後,我們再增訂了14-3,目前這個條款只有14-3的第二項看得出來他是所謂的隱藏性盈餘分配的概念。14-3的第一項,你看他是個人營利事業都適用,不只適用在綜所稅也適用在營所稅。第14-3,因為這個立法條文,他本身他的位置是在綜所稅。如果你是適用綜所和營所,那你應該放到總則。可是因為他這種抽象性立法的方式,導致了他其實是稅捐規避條款,依照我們的立法體系,14-3,應該跟納保法稅稽法放在同一個位置去分辨。我們14-3的規定,是防杜個人或營利事業利用中間公司(德文是Zwischengesellschaft),來做股利盈餘分配。這個中間公司如果在境外,叫做受控制外國公司,如果在境內,叫做阻擋股利盈餘分配的公司。這個就是14-3,本來要適用的對象。14-3本身的規定本來就是對中間公司做規定。如果是個人利用中間公司,這個中間公司在境外,那個叫做個人境外受控制公司,那個是所得基本稅額條例12-1條要規範的對象。所以14-3跟所得基本稅額條例12-1是重複規定。實務上沒有人在用14-3,也是因為那個是境外公司,那就直接用個人所得基本稅額條例的規定。然後。

14-3的這個法條,因為他的適用主體包括了個人跟營利事業,所以體系上的地位來看,是一個很怪異的存在,他不應該放在所得稅法。就算放在所得,稅法應該放在通則性的規定裡面。14-3應該是放在稅捐稽徵法。稅稽法被修掉12-1條以後,他就應該放在納保法裡面做規定。

接下來是14-4到14-8,這一大段,這個叫房地合一稅。房地合一稅怎么去算稅基。房地合一稅,我們的法條規定是一個獨立稅目,但沒有稅目名稱。我們的稅法體系真的是體系亂到不可以,因為你條文明明就說所得稅是課綜所跟營所,沒有房地合一稅這個稅目,結果你法條規範結構其實還是有。亂到不行,如果真的要制定這個房地合一稅,可以啊,你應該是在第二章綜所、第三章營所,那之後就第三章之一,房地合一稅嘛,可以吧?因為你是對個人跟營利事業課徵,另外單獨拉出來去課一個稅,那這個時候他的法條規範結構才完整。

我們現在,我真的,說實在,真的是\ldots\ldots 我雖然在財政部訴願會,有時候對個別訴願,可以發揮一點點的影響,但是他整個的立法政策,說實在,有時候實在是使不上力。立法,不太會問法律學者的意見,當然問法律學者通常都反對哦。我不是反對,我有跟你講要怎麼做啊,像這種東西本來就是不應該這樣子去弄,而他還是這樣子去弄。那個條文哦,聽說他就是去直接去抄那個某一個外國的一個制定法,然後就這樣啊。人家制定法長這個樣子,他說,他們法律就這樣規定。那規定真的很粗糙哦。這是體系這個部分,14-4到14-8其實應該是另外拉出去,在所得稅裡面用一個專章規定。簡單來講,他是稅基的分離,個別課稅,就簡單這樣子去做說明。

\hypertarget{ux76c8ux8667ux4e92ux62b5}{%
\subsection{【盈虧互抵】}\label{ux76c8ux8667ux4e92ux62b5}}

我們就來看14、15、16。第14條本來是各類所得的淨值計算,14條裡面本來應該也要有收入減成本,減費用,甚至必要的時候,虧損的減除,也是客觀淨所得、客觀净值的範圍。虧損減除不是主觀凈所得喔,虧損減除是客觀淨值原則的一部分。營利事業或者是個人的執業活動,這個活動產生出來的收入,這個部分是正的收入,聯繫這個營業活動或職業活動,你的投入,分成成本跟費用。所以所謂的净所得,是收入減成本費用。你這個營業活動可能成本費用大於收入,這個時候會產生虧損嘛,對不對?你有第一個營業活動,你有第二個營業活動,你第一個營業活動有收入大於成本費用,所以你有盈餘,但你第二個營業活動是收入低於成本費用,所以你有虧損。那你要不要讓人家盈虧互抵?應該不用我特別講什麼概念,這個本來就應該要讓人家盈虧互抵啊。除非你第一個是應稅,第二個是免稅。如果是免稅,就切開來。但如果都應稅,要不要讓人家盈虧互抵?要啊。你所得稅不能講一半,有所得,就應該課稅。對啦,我承認。有虧損呢?要給人家減除啊,對不對?正確計算净所得的概念,不能只看正的,不看負的,這樣沒有道理啦。

所以客觀淨所得或客觀净值原則,除了釋字745號所講的成本費用概念以外,其實虧損不是稅捐優惠。我再講一次哦,虧損減除不是稅捐優惠,他是客觀淨所得的部分,你不要一天到晚用立法理由立法理由。立法者偏心你不知道嗎?立法者心是偏的。因為我的心也是偏的。

他的立法理由說,欸,因為這個所謂的凈所得這件事情,我們是按年度稅。年度稅沒那麼高尚。年度稅是一種技術性原則,因為我們所得稅不可能到你人死了,再來算所得。營利事業不能到你解散的時候,我再來算營所,因為那個時候做收不到稅。對吧?所以年度稅是一種技術性原則,作為技術性原則,不能高過於目的性原則。像我們講量能課稅,量能課稅導出來的是終生所得概念,終生雖然不可行,這個理想太高遠。終生是不可行的,但容許一定年度的跨期間盈虧互抵總可以做吧?這樣這樣各位可以聽得懂這個概念嗎?我們現在實務上,包括最高行代表的實務,一直認為跨年度盈虧互抵,就是優惠。不是啦。跨年度是客觀淨所得客觀淨值原則裡面的一環。收入要減成本費用,這個是14條裡面該有的。因為收入減成本,可能收入大於成本費用,你會有盈餘,如果收入低於成本費用,你會有虧損,所以這裡面還有第二個內容,叫做盈虧互抵。

一個營利事業只要做兩個活動。比如說我舉例而言,我是臺大的教師,然後我同時間去買賣黃金。不可以嗎?誰說不可以?我就這樣啊,不然你是要怎樣,我就真的買賣黃金那樣啊。我買賣黃金,我賺錢要繳稅對不對啊,我虧損呢?我買賣重金屬,我可能真的會虧損。信不信由你,真的會,我真的會虧損。

好,那這樣我有薪資正所得,對吧?但我有財產交易的虧損啊,你為什麼不準我盈虧互抵?我是同一個主體,你要計算我的稅捐負擔能力,我在我的薪資所得,我有100塊,但我的財產交易我有虧損,我有50塊虧損,所以我真正净所得其實是50塊。

當然你可以講說,臺大教師怎麼可以又再去操作衍生性金融商品,或是有這種這種股票交易買賣行為?好,我這樣講啦,也許我違反公務員法,我違反教師法。我應該要倫理道德高尚,我怎麼可以去操作股票或是操作重金屬。不應該去做這些事情。這個是規制,但另外我們回過頭來,除了規制之外,基本上這些不是非法活動,賺出來的錢,要進入所得課稅範圍,沒有意見,但虧損要準予減除。

但我們現在所得稅法裡面,一個隱含著所謂的凈所得原則的概念,是各類净所得。我們14條的各類所得,是各類凈所得,不準跨類的跨類盈虧互抵。像我剛剛講的第三類薪資所得跟第六類財產交易所得(弧虧),做盈虧互抵,不可以。

那你說為什麼不可以?答案是因為我們現行法,就把這種跨類、跨年度盈虧互抵認為是優惠規定,必須法有明文才給優惠。這個理所當然,如果他真是優惠,那當然就必須法有明文。

為什麼這一類的跨類跟跨年度盈虧互抵是不被允許的?各位來看,所得稅法第17條的第一項第二款第三目特別扣除額之一的財產交易損失。

你看哦,只有財產交易損失可以在法有明文規定情況底下同類,三年間,跨年度盈虧互抵。只有財產交易損失在法有明文規定的情況底下,可以同類,但不能跨類哦,因為只有財產交易所得可以扣除財產交易損失。可以跨期間,但是只有在三年期間內。跨期間盈虧互抵。

透過這個條文規定的反面解釋,所有其他跨類,跟所有其他所得類型的跨年度盈虧互抵,一律不準。這個就是目前實務的釋義學解釋的操作。明明法條叫綜合所得,但說實在的,根本就不準跨類,就沒有綜合這個概念。這是分類所得的時候才會出現的概念,因為分類本來就是一條路走到底。你如果盈餘虧損,原則上就是減資,就這樣而已。但我們有綜合所得之名,我們並不存在著跨類盈虧互抵的可能。

然後跨期間盈虧互抵呢?可以,只有財產交易虧損是被允許的。所以這個概念也同樣貫徹到個人所得基本稅額條例,只有法律明定的跨期間才可以,不然就都一律不可以。

這個是所得稅法第17條第一項第二款第三目之一,特別扣除額的規定,他的體系位置跟主觀淨值無關,應該移動到第14條的財產交易所得第七類,就是財產交易損失的跨期間盈虧互抵規定。法條規範位置應該是14條,第七類,你要增訂第四款規定,到那個地方才對,因為他們概念上叫客觀凈所得。

所以我現在把我們的客觀淨所得的條文規定,重組一遍喔。各位要跟上來哦。

第15條的規定,第15條是合併申報的數個主體的規定,對不對?因為他是家戶所得課稅制。15條第二項的規定,有關於薪資所得,有提到,關於減除的免稅額。我們先來看一下第四項跟第五項的規定。15條的第四項跟第五項的規定是關於可扣抵稅額計算的規定,也就是8.5\%計算可抵減稅額的規定。這個第15條的第四項規定,要放到我們整個算出應稅所得額,這個地方稅率的規定是第5條第二項的規定,要放到整個稅基稅率之後算稅額的地方,去做,可抵減、可抵稅額的計算。這個稅額扣掉可抵減稅額以後,才是最後面實際上的結算稅額。

第15條的第四款規定,在規範的位置上,要在客觀凈所得、主觀凈所得、稅率,算出稅額,這個叫應納稅額,然後扣掉可抵減稅額以後才算出來結算稅額。他是應該放在稅基乘稅率得稅額之後的可抵減稅額這個位置,這個位置才會是正確的。

第五項的規定,因為他是28條的營利所得的分開計稅,第15條第五項的規定,因為他是稅基分離,所以是客體的第一類營利所得稅基分離,然後連結稅率的規定,做分離稅基固定比例計算。

分離稅基,加上28\%的固定比例。15條第五項的規定,適用全戶可扣抵稅額再加分離稅基與固定比例二擇一的方式去課稅。這個叫二擇一的營利所得課稅。條文規定的位置,是在可抵減稅額,是放在稅額之後。如果是28\%的第五項規定,則是第一項的稅基分離,加上稅率的個別規定,而採取28\%。因為我們本來的比例是在稅率,是在第5條第二項裡面的規定,那這個時候是二擇一的方式去做課稅。

這是我們在2018年之後的兩稅分立的課稅的制度,當然實務上把他稱之為叫可抵減稅額跟固定比例的二擇一課稅的制度。

接下來16條,是承繼第15條第一項的營利事業,如果是由配偶跟納稅人一起經營或分開經營的話,第16條的規定則是盈虧互計,因為我們15條第一項配偶經營的已經算入納稅人經營的範圍啊,所以16條的規定本來就應該直接接在第15條第一項規定之後的位置。

\hypertarget{ux751fux5b58ux6b0aux4fddux969c}{%
\section{【生存權保障】}\label{ux751fux5b58ux6b0aux4fddux969c}}

我們接下來看17條的第一項的規定。第一項底下的第一款免稅額,我們在上個禮拜跟各位提到,免稅額,這個部分是關於物理生存需求。

我們的17條第一項第二款,第二個層次的標扣跟列扣,是二擇一。本文規定,納稅義務人就標準扣除額或列舉扣除額,得減除其中一個項目。這個部分反映出來的,稱之為叫社會生存需求。因為納稅義務人,不是一個動物的人,有需要跟社會上其他的人,彼此之間往來交友,這個稱之為叫物理生存需求之外的社會生存需求。

這兩個概念加起來叫基本生活費用。因為一個人,賺錢,第一個,要維持自己物理生存需求,能夠吃飽穿了,第二個,我是一個群居的人,我要跟人有所連結,所以我要活得健康,我要有居住的地方,這個時候我才是一個活得有人性尊嚴的人,而不是只是吃飽穿暖的人而已。在這一個概念底下,這個叫做社會生存需求。

在這個之外的特別扣除,則反映個人的特殊狀態,尤其是身心年齡狀態產生出來的特別需求。這一種特別需求,透過特別扣除,整個都是反映出一個當代社會維繫個人生存需求,所謂的生存權保障的不可分割的概念範圍。生存權保障在法律上面做了一個,維繫物理生存、社會生存以及按照個別需要而產生出來的考量,這個叫特別扣除額,以此反映出社會法裡面特別強調的生存權保障。這個生存權保障,在我們憲法的第15條裡面。

第17條的這一塊叫生存權保障。前面14、15、16,主要是來自於對客觀淨所得客觀淨值原則的反映。客觀淨所得的反映在經濟學界裡面,基本上也強調,收入減成本費用,盈餘減虧損,這個是客觀凈所得。後面17條這個部分是主觀凈所得。算完以後,應稅所得的淨額,才乘稅率,得到應納稅額,再扣掉可抵減稅額,得到結算稅額。這就是我們整體的稅法的規範結構。

理論上,應該是一個法條一個法條去對應。

稅捐主體中華民國的境內居住者之個人,取得,理論上應該是全球來源的所得,依照客觀凈所得計算之客觀凈額,減除掉本法所規定之維繫物理生存需求的主觀淨額之後,乘上我國所規定的稅率,得到應納稅額,減除掉事先繳納或者依照法律規定的可扣抵稅額以後得到的結算數額,這個叫結算稅額,進入稅捐申報跟繳納程序。這一段就是稅捐債務法,這樣各位可以理解嗎?

法條規範結構本來應該是一個一個去做對應,而且因為我們的外在體系看不出來,所以請各位回去自己做。因為你做起來,就會比較容易知道這個法條之間的對應關係。

\hypertarget{ux5217ux8209ux6263ux9664ux984d}{%
\section{【列舉扣除額】}\label{ux5217ux8209ux6263ux9664ux984d}}

根據這個對應關係,我接下來要跟各位講的,這裡有6個列舉扣除額的項目,我現在把法條本來該有的位置,他應該要放在哪裡,我們重新做一個排列組合。

各位來看一下列舉扣除額。

所得稅法第17條第一項第二款第二目之一,捐贈。捐贈跟主觀凈所得一點關聯性都沒有。因為捐贈的扣除額是來自於對慈善公益之行為的鼓勵,所以本質是稅捐優惠。其實跟主觀淨值的個人特別需求也沒有關聯性。如果要放正確位置,應該是拉出去,就像稅捐優惠的重購退稅一樣,理論上他應該是拉出去。但如果放在17條,應該放在特別扣除額,也就是我們在特別扣除額裡面,把那個概念放寬一點,就是包含個人應受獎勵的社會上的這些各項行為,被鼓勵的各種行為,也放到特別扣除額裡面,當作是一種特別的考量,特別的扣除因素。所以,捐贈,要放到特別扣除額。

第二目之保險。保險是反映醫藥及生育費的另類支出。簡單來講就是,醫藥生育是保護納稅義務人及其受扶養親屬的身體健康跟生育這一個需求所需的所謂的生存需求的概念的範圍。一個人除了養活自己以外,我也要活得健康,因此會生病的時候,我會有醫藥費的支出。涵蓋醫藥及生育的費用,是對家庭,如果要生養生育這個功能的話,會是不可或缺的一個部分。在這裡面都是屬於維繫生存需求的範圍。而保險費是cover這樣的一個醫藥及生育費。因此,這兩個,在概念上面來講,只要保險費涵蓋的醫藥及生育費的範圍,原則上不能重複減除。超過醫藥及生育費的保險費的支出,只要是具有投資性質的保險費支出,這個時候他會跟後面的儲蓄投資特別扣除額重疊,就有可能會拉出去,另外加以對待。

保險費要區分其性質,做個別的對待。

醫藥及生育費剛剛就講過,主要是維繫身心健康以及生育需求的這樣子的一個維繫生存的費用。

災害損失則是個人遭受的天災地變的事由,所以應該是特別扣除額。

購屋借款跟房屋租金支出,這兩類支出基本上是維繫納稅人住的需求,從而他是列舉扣除額的項目範圍,所以放在這裡是對的。只是購屋借款是30萬利息扣除,房屋租金支出則是一戶12萬,對應出來的就是一個月租金1萬塊可以被扣除,但房屋購屋借款利息是30萬,可以扣除。原則上就是如果是全額貸款的話,以利率1\%來算的話,30萬會對應出來,你可以購買的房子,跟你可以租的房子來比較看看,哪一個是屬於在稅制上相對會比較有利。這個都被認為是滿足維繫生存需求的範圍。

接下來來看,特別扣除額的第三目之一,財產交易損失,這個要放到第14條的第七類,財產交易的凈所得的計算範圍。

接下來的薪資所得特別扣除,這個也應該放到第14條的第三類薪資所得的計算裡面。這個是概括扣除額的規定。第17條的這個薪資特別扣除,要放過去跟14條的第三類薪資所得一起,以二擇一的方式去做規定,也就是要不適用特別扣除額一個概括性扣除額度,不然就是用14條的第一項第三類的列舉必要費用,包括職業專用服裝費用,包括進修訓練費用,跟職業上工具支出的,二擇一的方式去做規定。所以,薪資特別扣除額的規定要放到14條第一項第三類裡面的列舉必要費用,去做並列適用的規定,因為這是客觀淨所得的部分。

那麼,接下來是儲蓄投資特別扣除額。儲蓄投資特別扣除額,原則上被認為是一種獎勵儲蓄投資行為的扣除額,就跟捐贈一樣。捐贈的扣除額,某種程度上是獎勵公益慈善,那獎勵你照顧好自己老後未來的生活,這一種儲蓄投資特別扣除額,因此放在特別扣除這個項目內。

身心障礙特別扣除額,是一個最典型的,因為個人身心障礙特殊狀態所導出來的一個特別扣除的因素,所以身心障礙特別扣除額,放在特別扣除額裡面,或者移動到列舉扣除額的項目裡面也很適當。

接下來我們來看教育學費特別扣除額。教育學費的教育,是指大專以上院校,不是在講義務教育的這個學費。義務教育理論上沒有學費,但我們知道實務上不一定真是如此啊。因為我們有一些學校特別是私立學校,還是會有收學費。但這個地方條文規定就是大專以上,大專以上不在義務教育的範圍,所以他是一個鼓勵你進修的,獎勵進修的教育學費特別扣除,所以放在鼓勵這個部分。

然後幼兒學前,幼兒學前是因為你的扶養對象是幼兒,從而幼兒在5歲以下,會有要照顧他的個別的需求,所以也是屬於因為年齡產生出來的個別需求。

長期照顧,則也是屬於他的身心狀態的特殊狀態的,產生個別出來的個別的照顧需求。這一個長期照顧特別扣除額,因此跟醫藥及生育費不完全相同。醫藥生育費可能會有包括的長期照顧,但長期照顧不一定會完全被醫藥生育給涵蓋。因為我們現在的長期照顧,主要是這個照顧人力的這個支出的部分,而不全然是屬於的醫藥生育費用。

基於這樣子的一個整體的觀察,我們重組一下法條規範的結構。

免稅額,有一個一定年齡以上的加計50\%,就是納稅義務人跟配偶年滿70歲,就老年人加50\%的免稅額。他跟幼兒的照顧特別扣除額一樣,都是因為年齡而產生出來的特別需求。因為身心障礙或特殊狀態產生出來的需求是身心障礙特別扣除跟長期照顧特別扣除,這個都是他的身心狀態產生出來的特別需求,在概念上都可以放在特別扣除額。

照顧一般平均水準下的物理生存需求,以及維持身心健康跟住的需求是社會生存需求,是前面這兩個作為反映物理生存跟社會生存需求的兩個主要費用。接下來的特別扣除額,一個脈絡,是因為身心跟年紀的特殊狀態所產生出來的特別需求,另外一個脈絡,則是受鼓勵獎勵優惠的特別扣除額。這個是我們現在所得稅法在目前稅基計算,該要分出來的規範結構。

再重申一次,前面這個部分,一直到維繫個人生存需求這個部分,我個人認為都是憲法15條生存權保障的範圍,因為他跟個人的人性尊嚴有密切的關聯性,所以在相關規範的位階審查上面,就是憲法位階,憲法的審查上面,是屬於比較嚴格一點的,也就是中度的實質關聯性的審查的基準。

但如果是屬於客觀凈所得,甚至是後面的稅捐優惠的事項。稅捐優惠,讓立法者享有比較高的立法形成空間,因為這本來就不是反應量能課稅原則的必要,而是立法者根據立法政策上所做出來的決定。因此這個獎勵的部分,即使放到特別扣除,他的法律性質,也不應該跟其他的維繫生存的需求等同對待,要做區隔,要做不同的區別。

這是我們整體理解我們的稅法規範結構。透過內在體系,我們稱之為叫主觀凈所得、客觀凈所得的區隔,稅基乘稅率得到稅額,這個是依法課稅原則的稅法的基本原則。在這樣的整體結構底下,重新解析跟建構我們的所得稅法。所得稅法的條文規定,因此需要做一定程度上的重新的排列組合。

客觀淨所得,原則上要放在14條的各類所得裡面去做規定。

接下來第15條是兩個主體的合併計算。15跟16,因為我們的所得稅法是把兩個主體合在一起去做綜合所得課稅的稅捐主體的適用範圍。也就是第15條適用的,原則上就會在16條裡面去做合併計算所得,合併計算盈餘跟虧損。因此這也是客觀淨所得的反應。第16條的規定,也是如此。

進入17條,則是主觀凈所得、主觀淨值原則的反映。當然,在法律規範結構上,稅捐優惠,應該另外拉出來,有一個17-2的相關規定,去做個別的規範。稅捐優惠的規定,其實回饋到最前面的第4條的規定。第4條裡面,我們有諸多屬於應稅跟免稅所得的規定,那個地方都是稅捐優惠的規定,要拉出去。拉出去,去做一個跟稅基有關或稅率有關的個別規定。這個才是所得稅法應然該有的一個圖像。

所得稅法,因此是依照課稅構成要件各項要素逐次展開的法條規定。稅基乘稅率得到稅額,這個是我們在這個稅捐實體法裡面最終所呈現出來的法律效果,才接著接下來的稅捐稽徵程序。

所以我們接下來談個別的問題,跟稅基計算里面有關的,做個別的調整適用,但規範結構裡面的重新調整,希望各位可以透過這堂課程,大致上有一個想像中的圖像,就是一個所得稅法規範的結構的圖像。從稅捐主體的境內居住者的這個概念,到客體的所得,到應稅所得額的計算,各類所得的净額的計算,進入稅率乘稅基的稅額的計算,結算稅額,這個之後才是接著稅捐稽徵法,也就是所得稅法第71條的結算申報制度,跟就源扣繳制度的規定。

我們在這裡面要有一個稅額計算的規定,但我們法條規定看不太出來有稅額計算的規定,因為他是放在各個不同的稅的客體的規定裡面去做分散式的規定。

這是我們目前的規範結構。

我們先休息。

\hypertarget{section-20}{%
\chapter{20231111\_02}\label{section-20}}

\begin{longtable}[]{@{}l@{}}
\toprule()
\endhead
課程:1121所得稅法一 \\
日期:2023/11/11 \\
周次:10\_1 \\
節次:2 \\
\bottomrule()
\end{longtable}

\hypertarget{ux5ba2ux89c0ux51c0ux503c}{%
\section{【客觀净值】}\label{ux5ba2ux89c0ux51c0ux503c}}

所得稅法第14條第一項,第一類所得,營利所得,分三種類型的所得。

一種叫公司合作社,或者是其他法人類型,像有限合夥組織。股利盈餘,是因為資本投入而產生出來的所得,或稱營利所得。

第二種,獨資跟合夥,也就是人合性組織,投入勞動力加上資本而產生出來的盈餘分配,那這個我們法條也叫做盈餘分配,但我比較認為這個是營業活動,所以我把他稱之為叫營業所得,是一個獨資合夥的營利事業的營業所得。

還有第三種,叫一時貿易盈餘,我們的所得稅法裡面又再把這種,偶一為之,或兼營這個事業的型態的貿易盈餘,放在這裡。

以上總共有三類,統稱叫做營利所得,分成了股利盈餘分配、營業所得,跟一時貿易盈餘,這三類。

那請各位注意,在這裡面一時貿易盈餘,跟股利盈餘分配,跟這個營業所得的分別。特別是營業所得跟一時貿易盈餘,基本上還是會有成本費用投入的概念。

股利盈餘分配的話,基本上是,你的投入資本是你的成本,因此產生出來的股利盈餘,原則上是收入減成本。但營業所得,跟一時貿易盈餘,都會有投入,有包括成本跟費用的問題,所以在這裡面,第一類所得,透過條文規範結構,不太看得出來有客觀凈額的原則,或客觀淨所得原則的概念,但仍然他應該還是會有。

因為營業所得跟一時貿易盈餘,至少投入的成本費用,都應該在法律裡面要反映出來,這樣一個客觀凈所得的概念。

我們到第二類執行業務所得,各位就可以看著這個條文,很明顯看得出來有成本跟費用的概念。第二類:「執行業務所得:凡執行業務者之業務或演技收入,減除業務所房租或折舊、業務上使用器材設備之折舊及修理費,或收取代價提供顧客使用之藥品、材料等之成本、業務上雇用人員之薪資、執行業務之旅費及其他直接必要費用後之餘額為所得額。」

各位,收入減除成本費用,畫起來。你因為你的職業活動獲得的正所得,這個叫收入,那你的投入,你會根據你的投入的內容去做切割,一個你的投入內容裡面就是提供給事務所所需要的經營的場所,這個房租跟折舊,然後器材設備,他的折舊跟修理費,還有你的藥品跟材料,這個叫成本。費用的部分就是你業務上雇用之人員的薪資、旅費及其他直接必要費用。各位把那個「直接必要費用」畫起來哦。

收入減成本減必要費用後的餘額,這個就是客觀凈所得、客觀淨值原則的適用。這個是第二類第一大段。

接下來第二段的規定,叫做協力義務,就是你至少要備製的帳簿是什麼。這個條文規定,執行業務者至少要設日記帳。你就回去看前面的第一項,營利所得,營業所得營利事業,其實本來應該要備製的帳簿。

獨資跟合夥,是根據第21條的規定,要根據稅捐稽徵機關管理帳簿憑證辦法,設置21條所講的帳簿跟憑證。

營利事業如果你是獨資跟合夥,你是營業所得的營利事業,那原則上是適用21條的帳簿憑證管理辦法的,這些帳戶跟憑證。

相反的,執行業務者則是用第二類第二段,根據第一句的規定,就是至少要備製日記帳。執行業務者不需要備製多麼繁雜的記事本,但至少最低限度就是要日記帳。

所以我們現在從這個條款規定哦,如果你能夠按照營利事業獨資跟合夥適用的這個商業會計法,如果你用商業會計法去記帳,那麼你就應該有權利去要求要適用權責發生制。也就是說,你在第二類執行業務所得裡面,法律規定只是日記帳而已,但如果你今天自己去用比較嚴格的標準,用商業會計法的方式去記帳,那這個時候就可以有權利要求,我可以用權責發生制。

德國的所得稅法就是這樣規定。因為執行業務者跟營利事業差別,只是在業務執行內容的差異而已。他們都是獨立勞務的類型。只是記帳方式的的差別。複式簿的這一種記帳方式,有資產負債表的,這個是商業會計法裡面所做的要求。那麼如果你是執行業務者,你是比較簡便的,就是你簡單就流水帳就好。那流水帳,你記出來的話,原則上用收付實現制。但如果你今天做了一個T字帳,你做了一個資產負債表的這樣一個列帳的方式的話,那麼在德國法律上是,執行業務者可以選擇改用權責發生制。

這就是我們的釋字722號,在講的是說,哎,你的營業事業的比較複雜,會計比較複雜,其實就是根據你的記帳方式,你呈現出來的是流水帳,我用收付實現制。但如果你用商業會計法,用資產負債表,用損益表這一種方式去做你的帳戶憑證的記帳。你用會計師記帳的話,那原則上你就適用商業會計法的方式記帳。這個時候你可以選擇改用權責發生制。權責發生制,可以讓你的收入跟成本費用的對應上比較平衡。因為收付實現,基本上是看現金收入,所以有時候多有時候少。少的時候只剩下盈餘虧損至零而已啊。所以這個時候對一個執行業務者,如果他的收入也是會有常常這種起起伏伏的話,那麼原則上用權責發生制,對他來講會是比較收支比較一個平衡的狀態,繳稅的狀態也會比較平衡。這個是德國的所得稅法規制結構上,容許納稅人可以做這種稅捐上的選擇權,但是你要以記帳作為前提。

權責發生制,不是你隨便能用就能用。你只要沒有好好記帳,其實沒有人有辦法知道你這個成本費用是對應哪一些的收入。你的成本費要去對應這一個收入的話,你就必須要有一個記帳作為一個依據。

因此第二類裡面的第二段,這個是協力上的義務。那第一段則是在講收入減成本費用的淨值,客觀淨值,客觀凈所得。

所得稅法第14條第一項,第三類薪資所得。原則上是薪資所得獲得的收入,減掉,兩種可能性,一個是列舉必要費用,這個就是14條第三類第一款,第一目第二目第三目。這個是列舉必要扣除額。這個列舉必要扣除額,跟所得稅法第17條的第一項第二款第三目之二的薪資所得特別扣除,這個地方,是要二擇一的。

原則上,你可以列舉的話,你就適用列舉,你不能列舉,就適用那個薪資特別扣除額,因為那是一個概括的成本費用的概算。因為他的法律性只是概括成本費用的概算,從而,你沒辦法列舉的,那概括成本費用概算,就是用薪資特別扣除額。

那我跟各位講,這立法者很賊,你看到他每一個額度都要求要有一個上限扣除額度,而且扣除比例很低喔。他的扣除比例,14條的這個各類就是全年減除,不能超過薪資總額3\%,而且每一個項目你不能併計,就是各個項目加起來3\%+3\%+3\%。所以答案是這樣,因為我們現在目前薪資特別扣除額,假設是用20萬來算,那你20萬這樣用3\%來換算下去,幾乎你的薪資所得總額要超過222萬以上,而且你各個項目都剛好就是各個項目都剛好平均分配。

就是你的薪資所得總額222萬,然後你的進修執業工作上的支出跟職業服裝的費用,這3個平均分配,這個時候你才有可能用列舉必要費用方式扣除,而達到跟特別扣除額相同的效果。或者你有列舉扣除其中一項,你要薪資所得總額666萬,扣除足額3\%,才可以扣到20萬。這個幾乎是高級名模才用得到。講白一點,就因為名模,才有可能可以扣到那麽多的職業服裝費。職業專用哦,這個關鍵在那個專用哦,只要職業不是職業專用的,就是一般場合也可以用,那這樣就不算了。所以那個職業服裝就是要讓人家很印象深刻,就看得出來你只有在那個場合能穿得。比如說,比如說玉山銀行不是有那個綠色的制服。我想,平常人不會穿綠色啦,除非你是藝人。那個就真的是職業專用服裝。如果各位將來執行律師業務啊,那你穿的那個法袍也算職業專用服裝,請問那個會到20萬嗎?應該不會啦。

自己不是財政部的人員喔,其實我自己推測,實務上能真的用到進修訓練3\%加職業工具3\%,然後再加職業專用服裝3\%,能夠平均分配到這麼高的這個不太容易。

我就跟各位講,有一個費用,沒有算在這裡面,交通。不是所有人剛好住在公司旁邊,交通費用基本上不給扣,所以你住遠是你家的事情,你最好是住在學校公司上班附近啊,不要通勤費用太高的。所以你在臺北市上班啊,你去住桃園,那是你家的事情。你每天這樣來回,看看有沒有超過20萬?那個是你家的事情。

所以我自己個人推測,大部分的人應該還是用薪資特別扣除額。這個條款規定是釋字745號,由這個名模林若亞那個案件,還有另外一位就是我們的陳清秀老師,好不容易爭取過來的,說,欸,違反客觀凈所得原則,所以我們立法者應該修正。立法者就修了一個讓你實際上用不到的條款,所以還是回到薪資特別扣除額。我們的條文規定是長這個樣子。

是14條第三類薪資所得,第四款規定。薪資收入包括一時性跟跟繼續性的。一時性的包括了那種津貼紅利獎金各種補助,這個都還是算喔,但是必須都是雇主給的,所以你利用職務受賄之行為而取得的,原則上不是薪資所得,是其他所得。

如果是退休後給付的,不是來自於雇主的話,就是退休所得,是第九類,不是第三類。

領錢的人如果變了,比如說遺孀遺屬,領取這個被繼承人生前的那個部分的所得,這個也是屬於退休金的所得,而不是薪資所得的範圍。

第五款規定,請各位注意,我們只有在勞保條例許可的範圍內,這個部分才給予免除薪資收入課稅。這個地方要請各位特別注意。其實退休金給付並沒有全額免稅,是要看是不是在勞保給付的收入範圍內。這個地方只要在每月工資6\%的範圍內的話,則不計入薪資收入,給予免稅的稅捐上的優惠的規定。

第四類是利息所得。來自於金錢或者金錢的衍生性的債券,不管是國家發行公債或公司發行的公司債,或是金融機構發行的金融債券,或短票。短票是約當現金,就幾乎接近現金,所以跟現金跟存款一樣,這個都是利息的所得。利息的所得原則上在我們14-1條的規定底下,是做分離課稅。

所以我剛剛有特別跟各位講14-1條的規定,要放到第14條的第四類這邊來一併考察。這邊條文的規定,基本上都是在14-1裡面做了一個特殊的規定.

原則上,利息所得10\%,分離課稅,所以各位有錢應該要賺利息所得,因為只有10\%,比股利所得更低啊。當然他的缺點會是在利息所得,基本上報酬不高而已,大概是這個差別,但他的稅率基本上相對是一個比較低的稅率。所以雖然利息所得是綜合所得的範圍,但現實上來講,是分離課稅,而且是比較低比例的課稅。

所得稅法第14條第一項,第五類,租賃所得。我們先講第一款。這個地方很明顯看出來,收入減成本減費用的客觀凈額。財產租賃所得跟權利金所得以全年租賃收入或權利金收入減除必要損耗及費用。必要損耗就是你的那個租賃物的損耗折舊,這個之後減掉費用後的餘額,這個是所得額。

那接下來的第二款到第五款,是關於收入計算方式的一個防杜規避的條款。

那我們來看一下。第一個是永佃跟地上權。只要你是設定地上定期的永佃跟地上權的話,被認為都是所得的範圍。你取得類似押金的話,也是要按一年期定存利率計算租賃收入。

然後如果是租金比較低的話,這個叫一般租金標準。所以借朋友住的話,也有可能就會按一般租金標準來設算你的租賃收入喔。

這個還真的會發生實務上的爭議。就是你借朋友住,你不收錢,國稅局還是會設定,你會有一般租金所得。

當然,法律條文這樣規定,你可以去打釋憲,不過目前為止好像也沒有因此犯案的這個問題。這個條款目前還在,以一般租金所得的標準設定你對他有一個相當於是租金的收入。當然,法律上面來講,不一定是用這個方式去做規定,立法者做了價值決定,是設定認為你有一個一般租金的收入,其實你也可以當作贈與,不過贈與不一定會在免稅的範圍內。

這個是,我們現在目前的法律上的條款。

第六類,叫農林漁牧礦冶,這一種農林業所得,那原則上有收入減成本費用,但這個地方僅限於是大自然界的第一次產出的農漁牧林礦冶,這是大自然界的第一次產出,有收入減成本費用。

第七類的財產交易所得,請各位要搭配第9條的規定,關於這個財產交易所得,跟財產交易損失的概念。以及第17條的第一項第二款第三目之一,財產交易損失特別扣除額。財產交易所得僅限於扣除財產交易產生出來的虧損,跨年度,三年內的盈虧互抵。

所以第14條的第七類,一樣有收入減成本減費用。你因為出價取得以交易時的成交價額減除原始取得成本,然後減除必要費用後之餘額為所得額。

在這個裡面,客觀淨所得裡面,沒有特別去談到所謂的這個虧損的部分,則要搭配17條的規定,這樣你才會比較清楚地去適用。

第八類,競技競賽,這個條款規定的話,因為第三款規定,政府舉辦之獎金的話,原則上就不併計綜合所得總額範圍內,像財政部的統一發票的兌獎,基本上就是直接依照所得稅法第88條的規定,扣繳稅率原則上是20\%,所以是用按你取得的部分直接扣20\%。所以你雖然開到1000萬的獎金,你不用開心拿到1000萬,其實只有800萬而已啊,因為財政部就把20萬扣掉了,不併計所得總額。

第九類,退職所得。退職所得也是一樣喔,退職所得一定是法律明文規定的退休金給付範圍。不是法律明文規定的,就不在退休金範圍。像實務上最常看見的就是你可能跟雇主之間有紛爭。勞工跟雇主之間有紛爭,那雇主可能一次給付給你的,好了好了,趕快離開,趕快離開啦。就是給他一個和解金,或者是一個損害賠償。這一個可不可以當作是退休金給付退職所得,被認為是這個計算上面有一個比較優惠的方式?答案是,只在法律許可的退休金給付範圍內,也就是個人依照保險給付,像勞保,或者是這個軍公教人員的這種年金保險給付範圍內,只在法律所規定的年金保險給付範圍內提繳的部分,按照年資,抵繳的年資,按薪資收入狀況計算的。

所以,如果你拿的是從雇主那個地方拿到一筆的退休金,我在實務上看過非常多的類型,像從外商公司退休下來,有時候外商公司為了要給你一個不競業的獎金,意思就是,不要你又出去給我投到那個對手競爭公司,或自己去創業啊。那很多人就會主張什麼?主張說,欸我這個退休了,因為我是離開的時候一次拿的。對不起哦,你退休一次拿的是所得沒有錯,但不算退職所得,因為不是算在勞保條例裡面算的這個範圍,那個範圍的金額不會很高。那其他的不是退休所得,就變成是其他所得。

所以我們的其他所得,第十類是包山包海的,只要前面沒有了,全部都流入其他所得。只要沒辦法歸類的,全部都歸到其他所得,但這裡面其實理論上來講,應該是前面可以盡量歸類的,就要歸到他所屬的類別裡面去,比如說非法打工也算是打工,這個也是薪資所得的範圍啊。

第14條的第二項規定,這個是實現原則的規定,也是那個我們實務上面,按照那個計價課稅原則裡面的政府表訂價格的計價方式,也就是實價課稅原則。實現跟實價都是在第14條第二項規定裡面去做表現。

14條第三項的規定是得其多年所得一次大量實現的變動所得,只限於法律所規定的四種類型,其他的就不承認是變動所得。因為被認為是半數所得課稅半數也說是一種稅就優惠。作為一種稅捐優惠要例外從嚴解釋。那這裡面其實我個人也會認為累積多年所得一次大量實現,其實並不全然是一種稅捐優惠,是因為累積多年所得一次大量實現,在累進稅率底下會有稅負急劇(集聚)的效應,產生稅負增加的效果,從而理論上,變動所得這一種方式課稅並不全然是稅就優惠,而是為了避免急劇(集聚)的效應產生出來的這種稅負上可能不利的待遇。我個人比較不傾向,認為這全然是一種稅捐優惠。

當然啦,這種變動所得的這一種切割方法,是一個蠻粗暴地簡單的方式的切割方法,那確實可能在這個情況底下,帶有稅捐優惠的性質,我不排除這個可能性。

接下來我們14-4到14-8,我們就留待到所得稅法四,再來跟各位談。

15條是配偶跟納稅人的合併申報,以及納稅人跟受扶養親屬的合併申報。我們到考完試之後,我們再來跟各位講家戶所得課稅制.

15條跟16條的規定基本上是對應的,因為你既然家戶合在一起,夫妻各自經營營利事業,有營利所得,當然要併計啊。這句話不是廢話嗎?你不能夠前面那個地方有所得要併計,但是虧損就不能夠比併計減除。所以你看第16條的規定就是夫妻兩個人各自經營營利事業,那你只要併計的話,核定了盈餘虧損就可以併計。

第16條的規定,當然你可以理解為是跨主體的盈虧互抵,可是是因為我們是採家戶核定課稅制,也就是納稅人跟配偶可以合併申報,這是同一個主體的關係。

\hypertarget{ux4e3bux89c0ux51c0ux503c}{%
\section{【主觀净值】}\label{ux4e3bux89c0ux51c0ux503c}}

我們接下來進入主觀净值。

\hypertarget{ux514dux7a05ux984d}{%
\subsection{【免稅額】}\label{ux514dux7a05ux984d}}

主觀净值的免稅額規定,他是一個物理生存需求的體現,他的概念是反映出當代維繫個人物理生存最低需求的額度,也就是至少吃飽穿暖所需要的額度。所以立法者有誡命,而不是立法形成空間,立法者有誡命應該要反映當代社會維繫物理生存需求的最低額度。這個額度不應該低於維繫同樣是物理生存需求的,在社會救助法裡面提供給的最低生活費。

簡單來講,有賺錢繳稅的生活水準,維繫生存需求不應該低於社會法上的標準。當然不是說沒賺錢繳稅的就怎麼樣,而是你在維繫物理生存需求這件事情上面稅法跟社會法不能背離。至少,最低限度,他們的生活水平,應該是一樣的喔。當然,如果真要不一樣,也不應該是稅法低於社會法。就是社會法裡面的最低生活費,跟納保法裡面的基本生活費,我們的概念,用不同的名詞,意義在於稅法裡面的基本生活費用不應該低於最低生活費用,不應該低於社會救助法裡面的最低生活費用,因為他反映出來的就是在當代裡面,維繫生存的最低需求。

當然,社會法也不應該低到讓人民無法生存的地步,因此誰的額度低,不足以反映出來當代的維繫生存需求,那是個別法律領域要去做調整,來反映出當代維繫生存需求的這個額度啊,這個是立法者的誡命,不在立法者的形成空間範圍內。簡單來講,立法者的形成空間受到這個誡命的限縮,立法者應該要反映出當代維繫生存的需求。

接下來的17條的第一項,各款規定,除了納稅人本人以外,也包括了配偶跟受扶養親屬。這個時候就帶進來婚姻跟家庭的制度保障。為什麼我可以減除配偶、受扶養親屬的免稅額?因為我養他。因為我們是婚姻成員,因為我們是家庭的成員,所以除了維繫個人本身的生存權以外,我養婚姻配偶,養受扶養基礎,這同時也在保障社會裡面的婚姻跟家庭制度。所以這裡面不是生存權保障而已,也包括婚姻跟家庭受憲法保障的這個概念。當然同性婚要一併進來,因為在這裡面,同性婚在法律上的待遇要等同於是異性婚姻所產生出來的待遇。簡單來講就是17條的第一項第一款體現生存權保障,在此同時也是一個婚姻跟家庭受稅法保障、憲法保障的一個體現。

好接下來跟各位簡單來講,未成年。未成年的子女受納稅人扶養沒有問題,但是成年子女跟納稅義務人的直系血親尊親屬,其實在法律上面來講,他們未必實際受納稅人扶養。納稅義務人對於直系血親尊親屬,可能有法律上扶養義務,但是他未必真的有在履行扶養義務。所以其實法律上的這個條文規定,就容許變成是納稅人可以選擇,我可以去做,把我的納稅義務人的受扶養親屬計入。其實他沒有真正受扶養,只是在法律上有扶養義務,納稅義務人把他合併申報,所以這個稅制,如果你把他合併申報,要損益同歸,也就是萬一納稅義務人的受扶養親屬,假設有隱匿所得的話,因為他隱匿所得,這時候要加計進來,將來就會提高你的所得稅的稅負。這樣各位清楚嗎?

在這裡跟強制合併申報的這一群親屬團體是不太一樣的喔。強制合併的這一些配偶跟受扶養親屬,特別是納稅人的未成年子女,因為他沒有選擇可能。從而他這裡面的損益同歸這一件事情,我們下一次上完課之後,我們再來跟各位去談。這裡面還有另外一群人是成年的親屬,其實他未必受扶養,雖然有法律上的扶養義務,但未必受扶養。也因此在這個部分,理論上來講,你如果把他申報進來,你行使的一個合併申報的選擇權,那這個時候損益要同歸,意思就是他的所得要算進來,他的虧損理論上當然也算進來。但如果他有漏報所得,這個時候補稅也會補到你納稅義務人身上。

這個地方是17條的第一項,第一款的免稅額規定,反映出維繫物理生存需求。

剛剛我有提到老年人增加50\%,老年人其實這個增加50\%,跟幼兒的學前扣除額每人扣除12萬,以及長照跟身心障礙者,老師個人認為他是因為身心跟年齡而產生了特別的需求。當然,老年人的需求跟幼兒的需求不太完全一致啦。老年人主要其實是更增加醫藥以外的其他的生活上的費用的支出。其實他吃的穿的反而比較少,吃穿不一定比較多啦,但一般而言,可能增加其他的營養補給品,或者房子要加裝無障礙設施啦,這些相關的費用。那就跟幼兒的情形大致上類似,是因為年齡,因為照顧幼兒,往往也是增加許多的生活費用上的支出。

所以在這個地方,就是老年人的免稅有50\%,跟嬰幼兒的幼兒學前扣除額,以及身心障礙特別扣除額,以及長期照顧,老師個人認為他應該要在平等原則下做統一的平等原則的規範。原則上是放在一起去做統一的規定會比較適當。這個是建議而已喔。當然,因為目前法律是做分別不同的考量啊。

接下來我們談到關於未成年子女就學的這個部分。未成年子女,因為一般來講就是受納稅人扶養,可是如果是屬於已經成年的而有身心障礙或無謀生能力的話,那這個部分原則上必須要由納稅人提供證據資料來讓主管機關依職權調查。主管機關當然可以依職權調查,可是因為這些相關的事由,必須要由納稅人在一定程度上,對於無謀生能力這一件事情提供資料。特別是在這裡面無謀生能力的判斷的話,要有機關根據納稅人提供的證據資料來去做調查,才有辦法去做一個比較完整的判斷。

接下來我們來看那個標扣跟特別扣除,因為是二擇一。

看一下,列舉扣除額,基本上他的主體適用範圍就是那時候有配偶跟受扶養親屬,你回過頭來去看標扣。標扣,只有納稅人跟配偶,少了一個納稅人的受扶養親屬,所以適用標扣一定不利。因為法制上只允許單身跟有配偶如果你是合併申報,你有受扶養親屬適用標扣一定不利。標扣,我就不信有多少人會用。如果你是一個家裡成員,大部分的情況都是因為適用列舉扣除額。

列舉扣除額,是納稅人配偶跟受扶養親屬,三個主體都可以適用,所以實務上應該標扣可以適用的情形不多,會喪失了規範的意義,就是本來標扣是為了要避免列舉的複雜跟困難,所以就用一個概括扣除額,但實務上基本上沒有用,因為只有單身跟有配偶的才可以用。如果你有受扶養親屬的話,基本上沒有受扶養親屬的標準扣除額,所以這個額度在實務上適用的可能性比較少。

\hypertarget{ux5217ux8209ux6263ux9664ux984d-1}{%
\subsection{【列舉扣除額】}\label{ux5217ux8209ux6263ux9664ux984d-1}}

捐贈。捐贈就要請各位注意,因為這是一個稅捐優惠,所以捐贈扣除額的規定,我們後來做了一個限制17-4的規定,非現金捐贈。這個條款規定,本來就稅收優惠嘛,結果實務上就一堆賣不出去的東西就來捐贈,賣不出去的電腦就捐贈給學校,賣不出去的圖書,就捐贈給受贈的單位,反正那些受贈單位很happy啊,為什麼因為他不用花錢就可以拿到一個財產。特別是在實務上,有一段時期就是偏鄉的那些地方自治團體,你捐土地給他,他好開心啊。為什麼?某一個澎湖縣的地方團體,擁有全臺北市最多的公共設施保留地。因為大家做節稅規劃的,大家都想到這個事情。而其實,你去看了這臺面上有一些人啊,那個做節稅規劃的時候,就是在做這種事情啊。他把臺北市公共設施保留地哦,就捐給澎湖某一個鄉公所,因為他們效率很高,一天就可以開給你捐贈證明哦。所以到年底的時候要報稅之前的時候趕快捐,當天他就開給你。

開給你什麼?公共設施保留地的價格是這樣,公告現值是100塊,市價可能是只有10塊到30塊。假設是10塊,我是一個地政士,我就去買下來哦,市價10塊我去跟那個地主用20塊錢買下來,然後你要做節稅規劃的,我賣給你30塊。所以你是30塊取得對不對?然後你再把這一塊地捐給澎湖的那個地方自治團體。

以前新北市也有一個很有名,專門接受這個捐贈,就是一天內,行政效率都超高,就當天就開出來捐贈證明。因為他只要查那個價格就可以,公告現值。所以我用30塊買一塊公共設施保留地,我就拿到一個上面寫著100塊錢土地的價格,然後我就在報稅的時候,因為我通常都是我的所得稅額到邊際稅率的40\%,所以我上面就寫著,我捐了一個100塊的土地給那個澎湖或是新北市那個鄉公所,所以我拿到多少錢的可扣抵稅額?40塊。

這樣各位知道這一個過程當中,大家都很happy。因為那個公共設施保留地,地主一般來講是政經弱勢者,他們家之所以被劃到公共設施保留地,就因為你是政經弱勢。因為人家政經强勢,剛好都是劃在緊鄰公設保留地的建地。這就是地方的邏輯。地方邏輯就這樣,你們家比較弱,你們就是公共設施保留地,就是公園預定地。剛好我們家跟你緊鄰,我就是建地。就是道路劃過去,劃到你們家,沒劃到我們家。是這樣啊,劃完以後,這塊地還是公告現值100塊,但是市價立刻跌落變成10塊。那就會有土地掮客就出來,就這樣,你那塊地哦,沒用了,真的是沒用,因為你又不能做什麼啊。那現在有人願意出20塊,我跟你講,那個農民很開心。10塊錢的市價的有人出20塊,真是開心死了,所以你很快開心地賣給他,20塊錢。

那這個20塊錢的地,你要做節稅規劃,你來找我,30塊賣給你。我就賺10塊。這個比土地仲介還好賺欸,土地仲介只能賺總價的6\%,100塊錢的土地賣出去,我才只能賺6塊。我這樣一來一回之間我就可以賺10塊,而且因為他持有成本只有20塊錢,他可以去貸款嘛,這基本上都比較低的貸款,然後我就可以拿20塊換到30塊。

需要節稅規劃的那個人,他拿來這塊地,他30塊買的,他就捐給地方自治團體,拿到40塊錢的抵稅權,所以他也拿回了10塊錢,大家都happy,只有一個不happy,國家。

其實還有一個團體啊,為什麼?因為那個地方通常是他們那一個上級的地方自治團體都沒拿到錢,都是下級的地方自治團體拿到錢,所以他心裡也很恨,你們怎麼可以有這麼多錢?下級地方自治團體:啊,我們效率高啊,大家都喜歡我們啊,都捐贈給我們。

國稅局後來就去查,哎,你們所有買這些,捐這些地的,來,拿出當初購買的證明。你不拿出來,你怎麼會拿出來,你拿出來就只有拿到12塊錢的抵稅權。你捐給地方自治團體要你去買的,對不對?你買一塊地看起來是100塊,但是最後捐給地方資產,你其實也只是捐那30塊而已啊,如果真要算,你抵稅權只能30×40\%,只有拿到12塊錢,那這樣你一定不會拿出來。

所以最後面,這個國稅局就是說,你不拿出來,我就用16\%給你算,就是當初你這一塊地看起來是價值100塊,我就用16塊給你算,所以立刻所有人全部都,哇天啊,如果是這樣,那我不就虧了嗎?我拿30塊,我本來至少可以拿12塊錢的抵稅權,因為用40\%去算啊,如果給我用16塊,我再拿一個40\%,我就只有5、6塊錢的抵稅權了。所以那時候就發生很多爭議,然後就一律都去做訴訟,一律全部都輸,結果到最後面,去打一個釋憲,釋字705號,說這個違反依法課稅原則,所以不準,不準國稅局用16\%。所以後來實務上就說,不準的話,那我們就訂一個17-4。就這樣。這個條文就這樣來的啦。

非現金捐贈,一律就是拿出你當初的成本。所以訴訟打了一圈以後,沒用。實務上就是這樣,現金貨真價實,大家沒有什麼好說,但是實物捐贈,原則上用成本算。你拿不出成本的話,未提出非現金的成本的話,我就用市價行情給你估算。這就是17-4條的規定。

講回過頭來,為什麼捐贈可以用這個區別現金跟非現金捐贈?很簡單,因為他就是鼓勵慈善,立法者,享有比較高度的立法形成空間。我也贊成,如果不是現金的話,嚴格去區別,這個是必要的,因為說實在話,那個東西可能是沒有流動性的的東西,像被亂葬的、亂丟垃圾的土地,捐給你市政府,我看你收不收。那個收到的市政府跟那個縣市團體大概都投大,一個頭兩個大。為什麼?要清理,光被自己的環保局一直開罰單就夠了。為什麼?因為都沒清理,亂葬崗、亂倒垃圾土地,那個高度被嚴重了。

臺南的路,我們那一帶,整個都被當時的台鹼,整個都被污染,地下水汞污染,光排除那個要幾百億的錢。臺南市這種當然說沒錢,我怎麼會有錢?

你看這個條款看不出來端倪,實際上就是這樣。

回來,保險費。保險費對應醫藥及生育費,這個都是屬於維繫物理生存需求、社會生存需求的最低費用。

購物借款跟房屋租金支出,這個也是一樣。

災害損失,拉到特別扣除額。災害損失應該包括他的個人受不可抗力的災害損失。如果是人為的災害損失,比如說被切到被強盜,這個就不算。

其實我是覺得這個不太有道理啦,不一定要天災地變。這個當然了,因為如果是被竊盜或者是被強盜,這個也可能要有一定的證據方法,不能說我被強盜,所以我就沒錢。有沒有報案?沒報案。損失了多少?黃金幾百兩啊。你黃金以前購買的單據沒有保留?你隨便說一個就黃金幾百兩不見了。怎麼有辦法這樣子去做一個災害損失的認定呢?這個是一個困難點。

購屋借款跟房屋租金支出,這兩個要對應啊。因為有錢買房子的30萬利息支出,沒錢買房子只能租的,只有12萬,一個月就是1萬塊錢。這個1萬塊錢的房子租金在臺北市大概租不了什麼房子,雅房應該可以,沒有廁所的雅房哦,搞不好是地下室或是頂樓陽臺,大概是可以啊。這個其實在臺北市根本就對那個這個租屋的人不太夠。

\hypertarget{ux7279ux5225ux6263ux9664ux984d}{%
\subsection{【特別扣除額】}\label{ux7279ux5225ux6263ux9664ux984d}}

特別扣除額就是交易損失跟薪資所得特別扣,都要放到第14條去。

儲蓄投資特別扣除,要放在特別扣除的項目裡面。

身心障礙跟教育學費。身心障礙,這個是身心狀態。教育學費是特殊的鼓勵的教育的費用。

最後一個就是,幼兒學前跟長期照顧。在我們的17條的第三項規定有排富條款,這個是不對的條款。

照顧幼兒是理所當然會增加支出的費用,不管是有錢人沒錢人,照顧幼兒就是要這麼多錢啊。有錢人照顧幼兒也是要錢啊,難道他就不用錢?你可以說那個錢,他不那麼care沒關係。可是這個是反映出幼兒維繫生存需求的額度的範圍。那長期照護也是一樣,有錢人照顧自己要長期照顧的受扶養親屬,也是必然會產生這一些費用。我個人認為這個排富條款根本就混淆了稅法跟社會法的界限。正確的排富是放在社會法,是放在稅捐優惠。排富不應該放在反應量能課稅原則的客觀跟主觀淨值裡面。如果你放在這裡面,那就是混淆了,量能跟稅捐優惠的差異。

\hypertarget{ux7a05ux6350ux69cbux6210ux8981ux4ef6ux5c0fux7d50}{%
\section{【稅捐構成要件小結】}\label{ux7a05ux6350ux69cbux6210ux8981ux4ef6ux5c0fux7d50}}

除了合併申報制度,跟後面的重購退稅的規定以外,我們就將課稅構成要件的規定介紹到這。

稅率,就是看第5條第二項的規定。

這也是我們大致上在期中考之前的内容,希望各位,對依法課稅原則的構成要件的部分,要重新組合。

我們的考試原則上會在構成要件裡面去做題目的設定。可能主體,可能是客體、稅基。就這樣,大概就這幾個部分,課稅的基本原則,因為稅法、所得稅法要適用的這些基本原則還是不變的。這個地方給各位做參考,預祝各位考試順利。

今天到這裡。

\hypertarget{section-21}{%
\chapter{20231120\_01}\label{section-21}}

\begin{longtable}[]{@{}l@{}}
\toprule()
\endhead
課程:1121所得稅法一 \\
日期:2023/11/20 \\
周次:12 \\
節次:1 \\
\bottomrule()
\end{longtable}

\hypertarget{ux671fux4e2dux8003}{%
\section{【期中考】}\label{ux671fux4e2dux8003}}

期中考試的,第一題,是關於,所得稅法2條一項規定適用的稅捐主體。適用的稅捐主體,是我國的所謂的境內居住者的個人,相關的條文規定在,第2條第一項以外,那就是在第7條。適用的稅捐主體是境內居住者,或者我們也可以稱之為叫我國的稅籍居民。

關於只就境內來源所得來課稅的立法理由的說明,這個是不是符合量能平的那個的課稅原則?從這個全部所得全部課稅原則來看,並不符合,量能課稅的基本原則底下的一個子原則,也就是全部所得課稅原則。這個也是,我們在所得稅法的這個條款規定上面來講,比較不符合量能課稅原則之處。

當然,我們在民國95年,在2006年的時候,我們訂了一個所得基本稅額條例的規定,算是在一定程度上將全部所得的部分啊,就是境外來源所得,我們透過用所得基本稅額條例的方式去補充,去做課稅。因此,在這個地方,以所得稅法跟所得基本稅額條例整體來看的話,確實在這個限度範圍內,裡面也就境外來源所得去做課稅。但即使是如此,我們並沒有符合這個量能課稅原則裡面,特別是境內境外來源所得應該要同等對待的,基本精神,也就是平等課稅原則的精神,去平等地計算稅負。以所得稅法跟所得基本稅額條例這樣的一個這個課稅的模式來看,我們還是沒有符合量能課稅原則。

實務上的考量,認為有幾個理由,關於境外來源所得不易控制,關於工業化國家的稅率比我國高,所以課稅也難以增加稅收。有沒有增加稅收並不是在這裡面,立法者自己去考慮的,也就是說,簡單來講,不是稅捐增收入增加,所以就可以,也不是稅捐收入收不到就不可以。毋寧應該考慮的是稅捐負擔的公平性的問題。也就是說,作為我國的稅籍居民,理論上應該要去就境內境外來源所得,都依照法律的規定,來去做稅捐上的申報。

所以,這個制度,在全部所得課稅原則底下,法律上就應該要有相同的規定,讓納稅人去做稅捐上的申報。因此,對所得來源不易控制,以及難以增加我國稅收,這個並不是一個正確的一個考量。這個全然基於我國的稅捐稽徵程序上面是否能控制,以及稅收是否能增加作為考慮,並不是,在稅捐立法上面的完全正當的考量。在稅捐立法上面來講,仍然必須要以這個為前提,來做稅捐立法上的課稅。

順帶透過這個方式跟各位說到,對我國來源所得不易控制,這一件事情,其實在目前的實務上面,是透過稅捐資訊交換的方式去獲得,境外來源所得的相關資訊。然後對所謂的工業化國家的往來,所以我們並沒有課稅的實益,這一句話也不成立。作為國際上的往來,並不是只有工業化國家而已,那它其實也有相對稅率,相對我國來講是比較低的稅率的國家,那當然在這種情況底下,不是以所謂的稅收增加考量來作為他,這個評斷上的依據,這個是對第一題,提到全部所得課稅原則。即使是有考慮到所得基本稅額條例,我們並沒有符合稅捐負擔的平等原則。

第二題的話則是稅捐稽徵機關依照檢舉的資料查得,某甲在同一個年度內買了10筆的預售房屋,預售房屋的權利的證明,那么這位某甲在這一段時間內把這10筆的紅單轉讓獲利。

第一小題的問題,稅捐機關查得這個轉帳紅單的獲利之後,除命補繳稅款以外,又給予應繳稅款而未繳稅款的裁罰,這部分是不是構成了一行為二次處罰?

本題的關鍵在,補稅並不是裁罰。補稅是一個不利益。裁罰是一個制裁性的行政處分。所以補稅同時並裁罰,並不構成所謂法治國原則底下的一行為二次處罰,這個對非稅捐法律人是蠻普遍被誤解的一個道理,認為補稅就是一種裁罰。因此第二題的第一小題,稅捐稽徵機關對納稅人未申報營利所得所做的補稅跟裁罰的處分,並不構成法治國原則違反的一行為二次處罰。補稅是根據稅捐負擔能力所做出來的計算,換言之,有所得,理論上就應該要繳納申報,來負擔應負的稅捐。因此補稅是量能課稅原則的維護,是一種確實是對納稅人構成一個不利益的金錢給付的負擔義務。裁罰則是屬於對過去行為的非難,也就是你應申報未申報,應做誠實的申報而未做申報。因此第二題的第一小題不構成一行為的二次處罰。

第二題提到甲主張紅單買賣,那麼這裡面特別涉及到土地的部分,是不課徵土地所得稅。這個是非常常見的一種抗辯的方式,也就是納稅人主張,根據我國的所得稅法的規定,所得稅法第4條第一項第十六款的規定,對於土地,免納所得稅的所得,從而他是不用申報的,因此我沒有申報,本來就很正常。

本件個案,其實是預售屋的買賣。換言之,在這裡面至少就預售屋這個部分,他本身是一個權利的買賣行為,並不是一個所得稅法第4條第一項第十六款所講的土地,即使是土地,他是一個土地移轉登記請求權。所以第二小題的題目,他其實在實務上,是一種權利讓與。就算是以不動產的土地為標的,仍然是權利讓與。這個是第二小題裡面,對於實務上,當事人往往主張土地出售本身不課徵所得稅。沒有錯,所得稅法4條一項十六款,土地的出售不課徵所得稅,但性質並不是免納所得稅的稅捐優惠。其實是所得稅,跟土地增值稅,避免重複課稅,因此不是一個稅捐優惠的規定。話說回頭,這個地方並沒有土地的移轉的問題。預售屋是一個還沒有建築完成的房屋,所以它是一個權利的讓與。實務上對以往這種紅單的讓與,基本上就是以權利讓與為由來做稅捐上的核課。從而構成了,一個權利讓與的財產交易的所得。這個是對第二題的第二小題,就是關於對此沒有所得稅的申報義務,他主張的是土地的不動產的這個轉售。這裡面只有土地,才有可能有這樣一個抗辯主張成立的適用。但,不管是預售屋的部分或者是土地的部分,其實都不構成土地移轉不課徵所得稅的適用。至於土地移轉不課徵所得稅,並不是稅捐優惠,其實是分離課稅。

本題這個地方,無論如何,是一個這一個權利得讓與的概念。

因此第二題的第三小題,這個是在,房地合一稅沒出現之前,在我們實務上,對這一類的房屋,如果假設跟這個建設公司買了既成屋的買賣的話,那這個部分會不會影響到關於當事人的所得類型的判斷?第二題的第三小題裡面。特別是關於房屋的買賣這個部分,他到底是用財產交易所得,還是是營利所得?

在房屋財產交易買賣裡面,一次性的交易行為,會構成的房屋財產交易所得。但是在一段時間內的密集連續的交易行為,會讓這個行為變成是一個營利事業的營業上的行為。簡單來講,就是透過整體觀察綜合評價的方式,一個人在單一的一個年度內買賣超過10筆以上的,就算是既成房屋,不會再是財產交易所得,而是一個經常性繼續性的參與市場交易行為,從而是一個營利所得的態樣,也就是第一類的營利所得。

我們以往在紅單交易裡面基本上也是同樣的看法。買賣紅單本來就是一個權利讓與,本身是一個財產上的交易行為,但你如果在一段期間內不斷地去做買賣,那這個就會構成了所謂的營利所得,也就是從第七類的那個財產交易所得會變成的是第一類的營利所得的概念。也因此,這一類的行為以往在實務上面來講,是構成第一類,營利所得,以營利事業的方式去做課稅。

因此第二題的第三小題,一方面涉及到關於稅捐客體的類型判斷,當然也會因此連動影響到對稅捐主體的判斷,也就是這個人,會變成從個人的方式,變成是一個營利事業,這個是在實務上面,在我們房地合一稅沒有出現之前,我們在所得稅的課稅的方式。反覆不斷買賣交易行為,構成了繼續性,正因為它持續性地繼續性地參與市場交易的行為,讓它變成是一個營利所得的態樣。這是在第二題的第三小題。

在德國實務上有一個看法啊,叫做五年三個客體界線,但不寫到這個沒有什麼關係,因為老師只是借這個案例跟各位去說,對於買賣的標的物本質上是做整體觀察綜合評價,要看那個標的物本身,是不是原則上你要以之經營作為一個營利事業的營利目的。如果賣三雙襪子,不會構成一個所謂的繼續不斷的營利行為,因為本身襪子的單價並不高。但是房子的單價是比較高的,也就是說,在這個地方,德國的實務上發展出對一個人,有沒有構成營利行為的判斷,是跨年度去觀察一個行為人在任何一段5年內,只要賣三棟房子以上,第四棟開始,就會被認為他是一個營利事業。這個叫五年三個客體界線。但這個就是,整體觀察綜合評價的意思。即使在一個年度內賣三雙襪子,不會讓你變營利事業,因為襪子單價低。但賣三棟房子就會讓你變成營利事業。也就是你一年內賣一次,兩年內第二次再賣,三年內第三次賣,第四年開始,如果假設你又再賣了第四棟房子,在德國實務上就會構成了營利事業。換言之,本來是房屋財產交易所得,會變成是營利事業的營利所得,改以營利事業所得來計算課稅。因為正常來講,不會有人一天到晚都在用房屋來作為他的一個買賣交易的行為。這樣也是一個對一個社會上的一個實際經驗觀察,確實不是形諸於法律條文裡面的一個判斷標準,是由司法實務所發展出來的。

那臺灣在這個地方早期實務上一直都沒有很明確的標準,甚至,實務上流行一個叫做一年六個客體,也就是當你一年內只要不賣超過六戶就被認為不是營利事業,不課徵營業稅,因為房屋是要課營業稅的喔,所以就不課徵營業稅,你就不會課徵營利事業所得稅。那這個一年六戶的界線,即使是以本件這個地方來講,他也是一年賣超過十戶以上。所以啊,這個在我們的實務上面來講,也會認為是一個營利事業,那會同時該當了營業稅的申報繳納義務,也會有營利事業所得稅的申報繳納義務。簡單來講就是一個行為人會因為他的市場參與行為,他對一個客體的反覆繼續性的參與,讓他從個人的性質,而轉變成是一個以繼續性、經常性地參與市場的經濟活動的態樣,而變成是一個營利事業的性質。這也是我們在這裡面營利事業跟個人主體之間在稅捐主體上的差別。

從而第二題的第三小題涉及到客體的判斷,那也涉及到主體的判斷。以德國跟臺灣的實務上,基本上都認為他是一個營利所得。

不過當然很多個個案事件裡面,也有當事人會主張認為說你繼續性的判斷標準不明確。對,沒有錯。

正是因為整體觀察綜合評價,所以這個確實會存在著不確定性。就算是以五年三個客體為界線,那是不是買賣三棟房子以內就不構成營利事業,第四棟就突然變成營利事業?對沒有錯,確實在實務上會有產生這樣一個,量變變質變的道理。你不斷地這樣子去做,從客觀上他才有辦法去評價,你是一個營利事業,持續性繼續性地去參與市場上的活動。

即使構成了,個人,也會從財產交易所得第七類的所得,變成是第一類的所得的類型。這個是實務案例。老師出的這個題目,其實是從實務案例把它變換出來的,那這個是在房地合一稅沒有出來之前。房地合一稅1.0版出來,民國105年之後,預售屋的買賣就已經被列入房地合一稅課稅的範圍,所以目前的法制,房地合一稅裡面,原則上會把預售屋的買賣以他的轉售價差減掉土地增值稅的稅基,然後其餘剩餘的所得額,就改課房地合一稅。

所以本件個案發生在100年度,那時候並沒有房地合一稅。在那個年代,只要是紅單的權利讓與的行為,那以權利讓你的這個法律規範來看的話,本身是一個財產交易所得。一次讓與是財產交易所得,兩次讓與,是財產交易所得,到達一定的程度上面,一年內的十次讓與,比較清楚而明確地,會被認為是一個營利所得的態樣。

我個人也認為這個看法是正確的,雖然確實存在著不確定性,就是對於繼續性跟一次性的買賣交易行為的區分。

其實,買賣交易的行為跟一時貿易盈餘跟經常性的營利所得,本來就具有很高的不確定性,就是那個界線是不太清楚的。這也正是在訴訟實務上或稽徵實務上面,各位作為法律者常常會面臨到我體個案裡面,必須要去做綜合考量,依據客體、市場上的單價以及交易的實務的現況的一般情形來做判斷。

我剛剛舉的例子,三雙襪子不會構成一個營利事業,因為襪子單價不高,只是一年賣三雙襪子,不太可能以之為業。所以你賣三雙襪子會是一時性的財產交易所得,比較是屬於偏向財產交易的盈餘。但一年內賣三棟房子,第四棟出來的時候,就有可能就會變成了,營利所得的態樣。

以上是我們在上個禮拜舉行的期中考的部分。先檢討一下這個問題的方向跟答案,大致上給各位做參考。也歡迎各位同學,如果你自己寫題目的時候,你有不太一樣的想法,也可以。

老師改考卷,我總是要有一個特定的方向來作為一個評分評價的標準。如果各位同學在個別的答題裡面啊,你覺得你自己是很有理由跟說服力的,也可以來個別表示你的意見。沒有一定非得要照剛剛老師所說的這樣的一個做法。不過你要有具有充分的說服力。有沒有說服力,確實是以老師主觀的心證為基準。你可以來表示,你有什麼樣的看法不同的意見,老師會聽聽看你的說法,是不是可以成立。答案並沒有一定要非得照老師的。因為訴訟就是這樣。我也許我在這次的看法裡面比較偏向國稅局或是稽徵實務司法實務的看法,但你也可能採取不太一樣的看法。其實改考卷只是用一個答案去做衡量標準,那個答案未必是真的是標準答案。因為某種程度上,確實是會有開放性。因為這個開放性,所以老師並沒有一定非得要以我個人的答案作為標準。如果你不能說服我也沒有關係,那只是以老師的標準,你沒有達到這個標準而已。但歡迎各位同學未來在稅捐的訴訟實務上可以提出你的觀點。你只要能說服法官,你個案還是可能會贏。法律確實存在著有價值判斷,因此會有一些結果的開放性。

歡迎各位同學,也希望各位同學能夠,不是因為老師講的,所以你就覺得只有這種思考的方式,也希望你有不同的思考的結果和方式出來。這個是法律在價值判斷底下的結果上的開放性。

到這裡有問題嗎?沒有上課上提出問題也沒有關係,你下課再來也可以,因為我們今天有進度。假設各位沒有問題的話,我們接下來就繼續上課。

\hypertarget{ux6240ux5f97ux7a05ux6cd5ux898fux7bc4ux7d50ux69cbux56deux9867}{%
\section{【所得稅法規範結構回顧】}\label{ux6240ux5f97ux7a05ux6cd5ux898fux7bc4ux7d50ux69cbux56deux9867}}

我們在考試之前跟各位提到的,我們稍微簡單的回顧一下,我們整個法條的規範結構。

第2條的規定,區分了稅籍居民跟非稅籍居民。第2條的第一項是稅籍居民,也就是境內居住者,原則上只就中華民國來源所得課稅。在中華民國來源所得課稅底下,根據兩岸人民關係條例,戶籍設在臺灣的稅籍居民,原則上是就臺灣跟大陸地區來源,所得全部都課中華民國來源所得稅,都是所得稅的所得。第2條第二項的非稅籍居民,也就是非境內居住者,這個非境內居住者,原則上也是就中華民國來源所得課稅,但他的課稅申報的方式,或者是繳納稅款的方式,我們今天會跟各位去講。因為我們的繳納稅款方式有兩種,一種叫結算申報。2條第一項的稅籍居民原則上用結算申報程序。2條第二項的非稅籍居民,則是用「就源扣繳程序」。

因此,我們回顧一下我們前面講的稅捐主體,稅籍居民跟非稅籍居民。我們的法律規定,原則上全部都是就中華民國來源所得課稅,當然這個中華民國來源所得的中華民國的概念,根據兩岸人民關係條例啊,是指臺灣跟大陸地區,只要你是戶籍設在臺灣地區,那原則上就是臺灣大陸全部都叫中華民國來源所得。那至於港澳跟境外(真正的境外:美、日)這個是不課所得稅法上或所得,是改課所得基本稅額條例的個人所得基本稅額。

我們的所得稅法,原則上只就中華民國來源所得課稅,所以他的區分實益主要是在稽徵程序上面。因為所得稅法裡面只就中華民國來源所得課稅,稅籍居民跟非稅籍居民都是只就中華民國來源所得課稅。所以實體法上,沒有什麼差別。差別主要是差在稽徵程序上面。因為我們的所得稅法的第2條第一項啊,一樣是就中華民國來源所得課稅。但是他是按照所得稅法規定的結算申報程序,課徵綜合所得稅,這個是第2條第一項的規定。

第2條第二項,非稅籍居民除本法另有規定外,其應納稅額分別就來源扣繳。所以我們會跟各位去談關於結算申報跟就源扣繳程序。

第2條第一項的稅捐主體的區別實益,如果以德國跟日本為例,或者是以美國為例,只要是稅籍居民,原則上是全球來源所得課稅,也就是都是課同樣一種稅法上的所得。那我們的所得稅法比較特殊,我們同樣都是只就中華民國來源所得課稅,那差別是差在稽徵程序上面。第2條第一項的稅籍居民,「依本法規定」,沒有寫什麼叫依本法規定。解釋上是依本法第71條的結算申報規定來課徵綜合所得稅。那么,第2條第二項的規定則是除本法另有規定以外,分別就其應納稅額,就來源扣繳,也就是依照就源扣繳程序來做稅捐的稽徵。所以我們的稅捐主體,主要的區別實益,被認為不在稅捐的實體法上,而是在稽徵程序上面,以所得稅法的規定來看。

以稅捐主體這個脈絡,繼續看稅捐客體的規定。在我們的現行所得稅法裡面對「所得」是沒有定義的。我們是透過所得稅法的第14條的十類的所得去定義出,「應課所得稅的所得」的概念。所以我們有十種類型的所得。至於「其他所得」的類型,這個條文規定就是一個涵蓋性的綜合性的,一個網羅性的構成要件的規定,原則上你只要可以分類到前面這九類的話,就不會進入其他所得的第十類的所得。

那么,在第14條的規定裡面,除了第14條規定的應課稅所得,我們還有第4條的免納所得稅的所得的概念。第4條,4-1跟4-2,這一些概念上,是應稅所得,可是是免納所得稅的所得。換言之,是提供分離課稅的所得,不然就是屬於稅捐優惠的所得。也就是說,部分的所得確實是被免除掉的,比如說4-1、4-2的證券交易、期貨交易的所得,被認為並不構成所得稅的所得,而是另外以期交證交的交易稅來替代,就是用交易稅來替代所得稅的課稅。當然這個部分,老師個人是有很強烈的反對的意見。這個地方,依照目前實務的看法,4-1跟4-2是以其證交期交替代證所期所。老師個人看法是交易稅不能替代所得稅,雖然立法者自己這樣講。以立法者的邏輯來講,他認為這個就是一個分離課稅,因為他就是用證交期交來替代了證所跟期所,這個是4-1、4-2。

4-3條的規定。公益信託的免稅要件的規定信託課稅的部分,我們會留待到所得稅法三,關於遺產贈與稅的時候,我們再來去進一步去談。

信託課稅,原則上信託的財產移動並不是所得的概念,是一個設定信託行為的財產上移動,所以這個地方4-3條的條文規定他本質上也並不是一個市場經濟活動所獲得的所得。

4-4跟4-5條的規定,是房地合一稅,我們留待到所得稅法四的時候,再跟各位去談。他當然是所得,但問題就是我們有第4條的第一項,第十六款的免納所得稅的所得,所以他是第4條的第一項十六款之外,我們有另外4-4跟4-5,又是4條第一項十六款的例外。

從民國105年開始,2016年開始,我們又在4條一項十六款之外,另外加徵了房地合一稅。

所以我們這個地方的法條結構是,所得稅對於不動產土地的移轉,免納所得稅,改課土地增值稅的稅負,然後我們在105年開始,又再把這部分的稅負又再把它拉回來,改課房地合一稅。這個地方有立法上的價值的轉變。

根據第4條第一項第十六款規定,免納所得稅所得。但同時間我們考慮到有土地增值稅的課稅,所以它是一個分離課稅的概念,但從民國105年開始,我們把這部分的分離課稅的土地增值稅的這個部分,把不動產交易裡面的所有的剩餘所得,又再以4-4、4-5條的規定,再把它拉回來去課徵房地合一稅。所以這裡面有一個立法者的價值轉變。

第5條的第一項的規定,是免稅額的物價指數調整規定,理論上應該是放到17條,因為17條才是免稅額規定的所在。我們的立法者的立法技術,是把這個部分拉到前面來做規定。5條一項本應該要放到跟17條第一項第一款講免稅額規定的地方。他跟你講,這個地方要按照物價指數去做調整。

這個地方我跟各位簡單說明一下,免稅額不是只按照物價指數調整。根據生存權保障的憲法保障的這樣一個要求,立法者應該要與時俱進地調整免稅額,讓他可以符合當代維繫物理生存的最低需求。因為我們的免稅額立法上是6萬,雖然每一年每一年調整,基底太低。如果基底太低,每年物價指數的調整基本上就不太容易去反映出納稅人維繫物理生存需求,也就是衣食溫飽這個部分的物理生存需求。

立法者透過生存權保障以及我們在105年所增訂的納稅者權利保護法的規定,基本生活費用不課稅原則,其實也是在反映維繫物理生存需求跟社會生存需求的這個全部額度。這個額度應該回過頭來,變成是第5條第一項,在法律解釋適用上的一個重要的,立法者的義務,應該要對於數額做與時俱進的調整。這個是5條第一項的規定。

5條的第二項規定是綜所稅稅率的規定。從體系上,他應該是在應稅所得的計算之後,就是,稅基之後,是稅率規定,所以法條規範位置會是在17條之後。稅率的位置應該是在17條之後。

稅率放到前面,只是告訴你一件事情,我們的立法者基本上都是財稅背景,不是法律背景。就這麼簡單。

法條規範的位置,其實按構成要件去區別,是主體、客體、稅基、稅率,稅基乘稅率得到稅額,構成要件這樣一字排開來,這個才是法律邏輯的思考。

那因為我們的財稅背景主導了稅捐立法,所以他們認為稅率最重要。你去看我們的所有的稅法規定,稅率都放在稅基之前。充分表彰出我們的稅法,從財政部體系一直到送進立法院裡面,基本上都是財稅思考,他們認為稅率最重要。

但從法律觀點來講,稅基比稅率更重要。因為稅基不對,稅率乘稅基,只會得到更不對的稅額,尤其稅率是累進稅率,會扭曲稅捐負擔的分配。

第5條第二項的位置,這個是一個累進稅率,我們稱之為叫超額累進,我們是五個級距的超額累進稅率的設計。五個級距分別是5\%、12\%、20\%、30\%跟40\%。

我們在民國97年之前,稅率級距是6\%、13\%、21\%、30\%、40\%。早期的我們綜所稅的累進稅率,更體現出所得越高稅率越高的概念。各級距之間的差異,分別是7\%、8\%、9\%、10\%。

我們綜合所得稅的最高邊際稅率是40\%,共是五個級距。我們這一種叫階梯式的超額累進稅率。

德國的累進稅率,邊際稅率是線性的,比如說你第一個級距是6\%,那第二個級距是6.1\%,第二個級距是6.2\%,就一直不斷地疊上去。平均來看會形成一種弧形的平均稅率。簡單來講就是你賺100塊跟賺101塊,第101塊那一塊的稅負比前面的100塊稍微要更重一點,充分體現出累進稅率,是一種社會國底下的財富重分配思想。

如果比例稅率的稅制會是這樣?比例稅率,比如說20\%,這樣也是量能課稅原則,因為你的所得越高,基本上你的稅負比例雖然一樣,但數額會比較高。即使是固定比例稅率,也是符合最佳負擔平等原則的量能課稅原則。

超額纍進稅率,我們的是階梯狀的,或是德國的線性的邊際稅率,是來自於社會國原則的財富重分配思想,實現量能課稅的平等原則。

我們採用的是超額累進稅率,因為各國,美國,在剛開始在做所得稅課稅的時候,就是以這種超額累進的方式。所以如果你調成單一比例的稅率,會對誰有利?對誰不利?對低所得的人不利,對高所得的有利。

我們現行的超額累進稅率,級距分別是5\%、12\%、20\%、30\%跟40\%。假設我在稅制委員會裡面,或是在立法院,我主張,認為,我們這樣不好,我們來改平等課稅原則,我們要單一的稅率,一律都用20\%或28\%。

我們的股利所得,現在改28\%,那個28\%就是一個比例稅率。好,對誰有利?對賺高所得的有利,對吧?因為如果綜合所得稅,他本來最高是到40\%呢。

對誰不利?你是低所得者,本來只需要繳納5\%,你用28\%,當然就不利啊。所以他告訴你,不要去投資股票,如果你沒那個能力、沒那個資格的話,你不要去投資股票。人家投資股票是節稅啊,他們繳40\%人,有股利所得用28\%課稅,那這樣就對他有利。但是如果你是繳5\%,你用這種,變28\%,那就對你比較不利。這樣清楚嗎?

我們稅制改革方向,是對有錢人有利。我們不用講太多意識形態,就光這件事情,請問有沒有去立法院質詢,為什麼是這樣啊?

我們股利所得課稅是二擇一啊,要麼28\%,要麼一個家戶所得課8萬元的可扣抵稅額,所以你就一定要選擇8萬元的可扣抵稅額,你說不然的話,你本來是5\%甚至20\%的人,你去投資股票,你去買臺積電,臺積電發給你股利,股利所得,結果你反而被高課。

所以可以選擇,用全部的所得用可扣抵稅額一戶8萬塊錢的方式,意思是告訴你說不要賺太多股利所得。你賺太多的話扣不掉。你會被加重稅負。

這是一個我們稅制裡面的反財富分配現象。

所以回過頭來就是告訴各位一件事情,研究稅法絕對對各位腦筋清楚有很大的幫助。你才知道這個社會的財富分配的方向,究竟是從貧到富,還是富到貧。你可以大致上去看觀察這整個,我們的社會的財富分配,究竟在稅制裡面是如何地被扭曲分配。就這樣一件事。信不信由你,因為這個本來就是我們現在目前的情況。28\%的股利所得稅率基本上是對高所得的人是有利的。

好,如果你今天是拿債券,10\%。去投資債券的人,越高所得人投資債券去取得債券的利息所得,他更有利。但如果你只是適用5\%邊際稅率,債券這個地方二選一都沒有,就是10\%。對誰有利?高所得的人。

越高所得,適用40\%的邊際稅率的人有利,因為他節省了邊際稅率30\%。

但對越低所得,適用5\%邊際稅率的人,反而不利。

這是我們在稅基稅率的關係上面的,反向的財富重分配的狀態。

其實本來累進稅率的設計,應該是,多賺錢越多的他要繳納越高比例的稅率,這是財富重分配,本來在社會國原則底下,想要的那個圖像。不過現實上並不是全然是如此。我們第5條第二項的規定,我們就跟各位談到這裡。

第5條第三項,基本上就是反映第5條第二項,課稅級距的金額,要固定調整。那這一個部分,我要跟各位說一下,立法形成空間比較大。不像第5條第一項免稅額是反映物理生存需求,所以立法者不僅沒有太大的立法形成空間,立法者有依據當代社會維繫物理生存需求的調整的行為上的誡命。他有那個誡命上的義務。

相對5條一項的生存權保障規定,5條二項跟第三項同樣都是在調整稅率的形成。對於級距跟邊際稅率,立法者享有比較高度的立法形成空間。也就是根據國家的財政上的需要,立法者享有比較高度立法形成空間。只有對免稅額跟扣除額所反映出來的那個數額,是生存權保障的範圍,立法者的立法形成空間,相對應該要比較少。

第四項的規定則是關於消費者物價指數的規定,每一年,你的級距、你的免稅額,你的扣除額,每一年都是財政部在一定時間以前公布。像這第四項規定,那個就是技術性作業性規定。什麼時間公布,財政部自己的人,內部作業規則規定公布就可以,基本上那個不需要在所得稅法裡面公布。

第5條第五項,這個是營利事業所得稅的稅率規定,我們在綜所稅裡面我們就按下不表啊。

5-1條是關於扣除額跟特別扣除額,也是按物價指數調整的規定,像這個就直接放到那個扣除額,那個地方一併去做規定。

第6條,貨幣的規定,是以新臺幣為單位。不然是要什麼?金圓券嗎?不用新臺幣你要用什麼作爲單位啊?

第6-1條的捐贈規定,這個是放到17條的那個關於捐贈那個地方去做合併的規定就可以。

所以我們大致上把前面的那個總則的幾個比較條文性的規定跟各位說到這裡。我們先休息一下,待會我們繼續回顧,把所有綜所稅的條文,實體法的部分,我們全部都拉出來,全部完整的回顧一次。

我們今天要進入到關於申報程序裡面的結算申報,跟家庭的成員的合併申報程序。所以我們再下一次才會去提到就源扣繳程序。

我們先休息一下。

\hypertarget{section-22}{%
\chapter{20231120\_02}\label{section-22}}

\begin{longtable}[]{@{}l@{}}
\toprule()
\endhead
課程:1121所得稅法一 \\
日期:2023/11/20 \\
周次:12 \\
節次:2 \\
\bottomrule()
\end{longtable}

\hypertarget{ux7e7cux7e8cux689dux6587ux68b3ux7406}{%
\section{【繼續條文梳理】}\label{ux7e7cux7e8cux689dux6587ux68b3ux7406}}

\hypertarget{ux7b2cux4e00ux7ae0ux7b2c7ux689dux8d77}{%
\subsection{【第一章,第7條起】}\label{ux7b2cux4e00ux7ae0ux7b2c7ux689dux8d77}}

所得稅法第7條,第一項是關於個人自然人。第7條的第一項的第二句的規定,連結的就是第7條第二項的規定。這個就是稅籍居民的規定,也就是境內居住者個人的規定,也是剛剛我講的期中考試第一題的第一小題。就是指中華民國境內有住所,並經常居住在中華民國境內;中華民國境內無住所,那在歷年當中停留在中華民國境內,居留停留超過183天。但我們實務上面其實還有一個「境內居住者的認定標準」,有設戶籍,你只要停留超過31天以上,那這個時候就是屬於境內居住者。有設戶籍,停留1到30天之間的話,經濟生活重心在中華民國境內,則也構成稅籍居民。

第7條第三項:「本法稱非中華民國境內居住之個人,係指前項規定以外之個人。」

這一句話,我個人認為是廢話,因為,他不是稅籍居民,反面就是非稅籍居民,而且還要規定嗎?沒有太大的用意。

第7條第四項跟第五項的規定,我們其實在講到納稅義務人跟扣繳義務人,尤其在就源扣繳程序的時候,是有意義的。我們先暫且放下,到就源扣繳程序的時候再跟各位去談。

接下來第8條的規定,我們跟各位花了很長一段時間跟各位講。

第2條第一項跟第2條第二項都是以中華民國來源來作為區別,凡是中華民國來源所得的,稅籍居民的話,就是要課所得稅的所得。非稅籍居民,也是一樣,取得中華民國來源所得,那這個時候,就會要依就源扣繳程序去做扣繳。

相反的,非中華民國來源所得,理論上就不課我們的所得稅的所得。理論上,非稅籍居民如果取得非中華民國來源所得,從國際稅法的角度來看,我們跟這一個課稅事件是沒有任何關聯性的,所以我們也當然不可能對他去做課稅。這個是第8條的,中華民國來源所得判斷的依據。

採稅籍居民就全球來源所得課稅,最主要的意義是在判斷,這個是不是該外國的可扣抵稅額,因為非中華民國來源所得,相反的是外國的可扣抵稅額。是根據這個條款規定,去判斷是不是中華民國來源所得,再來去決定那個是不是可以適用外國的稅額扣抵。

我們的8條的十一款規定,相對於我們的第14條規定,看起來就是加了一個第九款規定,第九款規定裡面是營利事業裡面比較會用得著的,關於經營的地的,所得來源地的判斷標準的規定。也因此,這第8條的規定在實務上幾乎不太容易發揮作用,因為基本上他只有將第14條的所得的分類,然後加一個「中華民國來源所得」。

立法上有這樣的問題,因此我們實務上變成了按另外一個長達5頁以上的「所得稅法第八條規定中華民國來源所得認定原則」,來做操作。

因此實務上面僅觀看第8條的規定,其實不太容易去看得出來中華民國來源所得的判斷標準。法律上面,這個第8條的規定的功能,因此基本上全面被架空。

我們在講課的時候,跟各位去提到,第8條的規定,其實要定義什麼叫中華民國。當然不是憲法上那個定義,而是你現在課稅權限所及,屬地主義底下所及的,就是臺澎金馬,這個是我們的領土的範圍。

因此在這個地方,所謂的中華民國,基本上是指中華民國現在的臺澎金馬地區的領土,以及領海跟領空,這個才是「中華民國來源所得」中的「中華民國''。就是我國現在現行,在我們的這個統治高權有效行使範圍內的臺澎金馬這些領土領空跟領海的範圍。領海,原則上是以國際法裡面所提到的以我們的陸地往外推的三海浬。因為我們剛好臺灣周邊,沒有大陸棚的那個連結,基本上就不會再往外推。

第二項的規定是關於我國的這個境外被視為是境內領土的這個部分基本上包括了,我國船籍的船舶跟航空器以及駐外的使領館。這一些駐外使領館,這也是在第8條第二項,從國際法的角度裡面,被認為他是屬於中華民國的境內的領土。相反的,也有在中華民國境內,也就是第一項裡面所講的臺澎金馬的領土範圍內,但是是依照國際法,被認為是一個境外的領土。特別是各外國在在臺灣所設立的這些使領館。那當然因為我們存在著外交關係上的困境,我們並沒有辦法正式締結的一個所謂的維也納公約裡面,以國與國之間的身份,所締結的外交的條約。從而因此在這個地方也是在第二項裡面去做相關聯的,關於中華民國這種視為境外跟視為境內的領土的規制。

這樣才能夠去清楚地去呈現出來所謂中華民國來源,中華民國的意義。

第三項的規定,我們也有簡單談過。國際稅法裡面去分配,是不是我國來源所得,有4個原則。

第一個叫屬地的所在地原則,凡是跟不動產有關的,包括不動產的租金所得,包括不動產的財產交易所得,包括跟不動產有密切關聯性,屬於大自然界的第一次產出的農林漁牧礦,都是跟你的管轄的領域範圍有密切關聯。這些都是屬於中華民國來源所得的這個所在地原則。

這是第一個。判斷標準

第二個是經營地標準,經營地標準,特別是在獨立的勞務的付出,以及營利事業,因為都放在所得稅法。營利事業的經營的標準,這個也是。獨立勞務的勞務的付出,像執行業務者,就是屬於獨立勞務的付出的類型,他的經營地點。以及如果營利事業也放在所得稅法裡面的話,那就是營利事業的經營地。這個地方的經營地的標準,以他實際上的實際營運地點作為一個判斷標準,也就是我們在實務上面講的,所謂的經營的營運的行為,是否在前面所提到的中華民國的境內的領土領空和領海的範圍內去作為一個判斷標準,這是經營地為標準。

第三個,就除了所在地,除了經營地以外,債務的部分。尤其是債務人的所在地、債務人的稅籍地,也會是一個非常重要的判斷上的標準。特別是利息。因爲金錢的給付本身,是在任何地方都可以被給付,特別是在以通貨為標的利息給付裡面,以債務人的所在地,或者我們講債務人的稅籍國,來做為所得來源地國的一個判斷的標準。這個是在實務上面也經常被採用的判斷中華民國來源所得的標準。因此只要是利息的給付者是我國的稅籍居民,這個時候就有可能是作為中華民國來源所得。

第四個,就回到了非獨立性勞務。因為非獨立勞務的被指示而去做相關的勞務的履行,在判斷上面來講,是以現實上的履行地點來做為判斷上的標準。

這四大標準原則上是在應該在我們的所得稅法第8條裡面分別去做相關的規定,然後才會去對應第14條的各項類型的所得來去判斷,比如說是第四類的利息所得,就是根據在這裡面的關於債務人的所在地,來做判斷,這樣才叫中華民國來源所得。

第14條是客體的判斷,第8條是所得實現的的判斷,而兩個是不同的規範的意志,也不應該把它混在一起,應該要去做區別喔。

這裡面,第8條的第三款的規定,在我們的實務上面,關於境內提供勞務之報酬,這個條款規定,在我們實務上,適用的包括獨立非獨立勞務跟營利事業經營勞務,都是適用第8條的第三款規定。營利事業經營的勞務,以及獨立跟非獨立勞務,因為第8條的第三款規定,就提到中華民國境內提供勞務。

所以勞務的提供地點,適用在營利事業提供勞務,適用在獨立勞務提供跟非獨立勞務,也就是薪資所得的勞務。勞務的提供地點是在中華民國境內,那就是用第8條第三款的規定,這個也是透過來源所得認定原則裡面所判斷的標準。

那在這裡面,特別是只要有境內的個人或者是營利事業的協助,或者是輔助去做提供的話,這個會被認為是境內提供勞務的中華民國來源。

依據來源所得認定原則標準,我們在實務操作上,勞務,有境內的個人或是營利事業的協助或輔助的行為,這個報酬都被認為是中華民國來源的所得。因此我們第8條第三款規定,基本上適用的範圍是極大的。

第8條第三款,其實本來應該是只限於非獨立勞務,他本來應該是非獨立勞務,也就是受僱而提供勞務,適用第8點第三款,可是如果你是獨立勞務的話,是經營的話,原則上是第8條第九款規定,因為那個是用經營地點去做判斷,就是在中華民國境內經營工商農林漁牧礦冶,這一種基本上排除掉那個屬地的農林漁牧礦冶以外,基本上就是獨立勞務跟營利事業經營勞務的類型,其實都是第8條第九款規定適用的範圍才對。

我們司法實務的看法比較偏向於,第8條第九款,只要你是獨立跟繼續經營的勞務,特別是營利事業的勞務,是適用第8條第九款規定。老師個人看法是獨立勞務,跟營利事業的勞務基本上都是適用第8條第九款規定,只有非獨立勞務是第8條的第三款規定。但我們稽徵實務的看法是,第8條第三款是適用在獨立、非獨立,以及營利事業經營勞務,因為這是來源所得認定原則的標準所提到的。

那這裡面當然在判斷的區別實益主要就是,只要是涉及到非中華民國來源所得,那就涉及到外國可扣抵稅額的計算的問題。

屬地的所在地原則、經營地、勞務履行地,跟債務人所在地(稅籍),這四大原則,理論上應該在我們的所得來源地的規範上面來講,要清楚地呈現。

第9條的規定,其實是財產交易所得,原則上這個定義條文的規定也應該會放在第14條的第七類,財產交易所得裡面。

財產交易所得,我個人認為它定義並不是很清楚,基本上是一時性的財產交易,而排除屬於一時貿易盈餘跟經常性的營業行為所產生出來的所得。

第9條規定,從而在立法定義上是極為缺漏的,就是說他的條款規定哦,要排除沒有排除。而且要排除掉的標的是屬於第4條到第4-2條所適用的這些客題。財產交易所得,財產,應該排除掉第4條,4-1、4-2裡面所稱的各項客體,比如說土地,土地就不在這裡面。第9條的規定是排除掉繼續性的經常性的交易行為,排除掉客體,包括在4條跟4-1、4-2裡面所提到的這些客體。所以第9條的規定本身有規範上的漏洞。

第10條的規定是營利事業所得稅裡面的「常設機構」規定,這個我們留待所得稅法二的時候,再跟各位去做說明。

第11條的第一項規定,執行業務者的規定,跟第14條的第一項第二類執行業務者的所得範圍對應。

執行業務者,第11條的第一項規定,本身,沒有很清楚地呈現有哪些類型的人,不過我們在上課的時候跟各位去說明了,主要分三大類型。

一個是傳統被認為是四大師的,這一類型,是經由一定的學院的教育,跟經過國家考試及格而取得資格的,執行業務者的。

第二類我們稱之為叫,跟人類的精神創作有關的,智慧財產創作有關的這一類的執行業務者,包括了表演者作曲家、音樂家。

第三個是獨立的技藝,工匠、經紀人,有獨立的技藝。

這三類型的所得,其實本身就是獨立勞務的類型。也因此,第11條的第一項的獨立勞務類型,跟11條的第二項營利事業,會有一定程度上的重疊,因為營利事業包含了經營工商業。所以我們在11條第一項,這裡面這三大類型其實要排斥掉,屬於營利事業以外的。

也就是執行業務者是特殊的營利事業。他是這三種不同類型的營利事業。如果不是執行業務者,原則上就會進入營利事業的獨資合夥,或者是經營工商農林漁牧礦這些事業裡面。

營利事業的提供的服務的內容,是一種網羅性的構成要件。也就是,不是執行業務者,不是農林漁牧礦冶,其他這一些業者,全面進入營利事業所得稅的營利事業所得的概念範圍。

所以我在類型分類的時候,在那個地方有獨立勞務,我把執行業務者放在前面,然後接下來是農林漁牧礦冶。

「冶」,其實不太是,因為「冶」是精煉金屬的意思喔。

農林漁牧礦,這五業,基本上都是大自然界第一次產出。農,農業,播種耕作的收入,大自然的第一次產出,林,一樣。漁,其實包括遠洋養殖都是,也算自然界的產出啊。牧、礦,都是大自然界,跟屬地有關。所以我們剛剛講那個農林漁牧礦是跟屬地有關。

「冶」就不是。「冶」也應該是冶金、鍛鍊鍛造的意思,這個已經是開採之後的第二次提煉,所以他是第二次產出。他是大自然界第一次產出之後的磨練鍛造的行為,因此「冶」不是大自然界的第一次產出。「冶」其實是工商的形態的一種類型。你直接採礦出來的這個是「礦」。但把它鍛鍊鍛造這個是「冶」。

「冶」,也許是一種獨立技藝,執行業務者的獨立技藝,但如果不是執行業務者的話,那原則上它就會變成進入所謂的工商業的營利事業這個類型。

執行業務者、農林漁牧礦,這兩業拉出去,下面有一個兜底的網羅性的構成要件,這個叫做營利事業。你現在法律上面有區別啦,但如果沒有區別,他們都叫做獨立性勞務。

我突然想到,這樣去挖比特幣,這樣算不算?用電這種方式去自己去挖礦,這樣算不算大自然界第一次產出?我要好好想一想,挖比特幣的挖礦,這算不算大自然界第一次產出?不過我們現在的比特幣算虛擬資產,不是通貨,不是代替現金支付的工具。我要再好好想一下。

但無論如何,他會是屬於營利事業的一種形態,營利事業的話,我們在稅捐主體裡面都會開始進入我們所得稅法二裡面所講的,營利事業所得稅課稅的範圍。即使是穿透,基本上還是會先在所得稅法二裡面,我們再來去講營利事業的概念。

第11條的三四五項,都是營利事業所得稅的部分。

第11條的第六項規定,我們的課稅年度是用歷年。我們的財務會計不一定用歷年。像我們學校、政府會計,會計法的會計,是從每一年8月1號到隔一年的7月31號。所以財務會計當然跟稅務會計不一樣,因為稅務會計就是根據第11條第六項規定,我們是用歷年為主。

有些國家的政府會計年度是從4月份開始。像是日本的政府財務會計是從4月開始,所以他們預算他基本上編列是一直編到明年的3月底。

我們學校,我們國立臺灣大學,我們的學年度是從8月1號到隔一年的7月31號。這個都是財務會計裡面所做出來分別

但課稅的年度,期間稅原則,一律都是歷年。這個幾乎在各國都是不變的,所以財跟稅不是等號的關係。課稅一定是要根據財務報表,不管你是政府會計或是財務會計,根據那個會計再做一定的稅務上的調整。根據稅法的規定,我們是用歷年作為基準,所以我們要去對財務會計裡面所稱的盈餘跟虧損,做切割,去做帳外的調整。

帳外調整的概念,意味著原則上,營利事業要做財務帳,要再根據財務帳去做帳外調整,改製作稅務報表給國稅局查核,讓國稅局知道你究竟在這個歷年裡面賺了多少錢。這是營利事業所得稅課稅的根本。稅務會計跟財務會計去個別切割,個別去觀察。當然這兩個有很高的密切的關聯性,我們不可否認,但稅務會計真的跟財務會計不一樣。

\hypertarget{ux7b2cux4e8cux7ae0ux689dux6587}{%
\subsection{【第二章條文】}\label{ux7b2cux4e8cux7ae0ux689dux6587}}

第2章開始。

第13條綜合所得稅,第13條的規定,基本上是課稅的結構的規定,也就是個人之綜合所得稅是就個人綜合所得總額,也就是依第14條規定計算之個人綜合所得總額減除依17條規定之免稅額及扣除額後之綜合所得淨額。

第13條的規定,規範結構上顯現出來我們的應稅所得額的計算。這裡面有一個關鍵字沒寫出來,應稅所得額的計算是就14條規定的綜合所得總額,減除17條規定的免稅額及扣除額,之後的綜合所得淨額,為應稅的所得額,乘上所得稅法第5條第二項所規定的稅率,為應納稅額。

這個法條規範結構,呈現出不完全構成要件的這些要素的規定,就是,沒有把每個構成要件都寫清楚喔。

第14條是,從綜合所得總額,計算綜合所得淨額,裡面的第一個階段程序。我們把它稱之為叫客觀淨值原則的反映。第14條的規定,除了各類所得以外,是客觀凈所得原則的計算規定。

第17條則是客觀淨所得之後的主觀凈所得,也就是主觀淨值原則的反映。這樣各位可以對應出來那個法條規範結構的關係嗎?

第14條裡面,每一個條文規範結構,原則上應該要有一個對應的收入減成本費用,客觀凈所得的概念存在。但是他的對應的方式常常都不是很清楚。比如說第一類,營利所得看不太出來,有收入減成本費用。你看第二類就可以看得出來收入減成本費用。

第二類,執行業務所得的第一段規定:「執行業務所得:凡執行業務者之業務或演技收入,減除業務所房租或折舊、業務上使用器材設備之折舊及修理費,或收取代價提供顧客使用之藥品、材料等之成本、業務上雇用人員之薪資、執行業務之旅費及其他直接必要費用後之餘額為所得額。」

收入,減成本,減必要費用,之後的淨餘額為所得額。各位自己劃起來哦。這叫客觀淨值原則的計算。

第二類的第二段規定,這個是協力義務規定,這其實本來應該是要放到申報程序,協力義務那個地方去一併做規定,但是呢就先放在這裡。我們的所得稅法規定是,外在體系很亂。實體法沒有按構成要件的順序,然後實體法跟程序法就混在一起規定。這個是我們的立法特色,混亂的特色。

第二段的規定是在講執行業務者最低限度要設置日記帳。這是他的協力義務規定。我有跟各位提到過,執行業務者,假如他是採用跟營利事業的方式同樣記帳,要讓他有機會可以改用權責發生制。就是你有比照營利事業同樣標準的方式記帳,可以改用權責發生制。不然你本來只有日記帳,就用收付實現。因為日記帳就是流水帳的意思收多少支多少,跟業務有關的,就只要記載收支就好。不懂會計的沒關係,你就只要記載,業務收多少,業務支出多少。你不會分成本費用,沒關係,因為在這裡面,就是一個現金收付的概念。

那什麼是假的?有收,沒寫進去;不是為了業務支出,你把它寫進去,這個就是假的。這樣各位可以辨識嗎?什麼叫虛偽記載?你因為這個業務你有收入,你沒有寫進去,這個就叫虛偽記載。不是為了這個業務支出,是私人費用支出的,記上去,這個就叫虛偽記載。清楚嗎?

稅法反映出來的一個原理,就是,真的是業務目的的支出,跟真的是業務目的的收入的話,就是記載上去。這樣應該很簡單吧。這個就叫日記帳。

日記帳,不是商業會計法裡面所講的T字帳,資產負債表。在資產負債表裡面,我們同樣一個營業活動,他會同時表現在他的資產負債跟損益費損表裡面去,同時間去對應。這個對一般沒有學會計的人來講,是個困難。但所有的行業的職業的人,如果不是要去對外募集資本的話,只要求一件事情就是計日記帳。你不需要商業會計法,多麼高深的財務會計的概念,你只要一件事情就是乖乖記帳,那當然意味著要把那個真實的交易的內容記載上去,給相關國稅局查核。只是這樣一個稅捐上的稽徵協力義務。

保存憑證,5年,保存帳簿10年。憑證對帳,就這樣而已,這個就是執行業務者的協力義務。

第二類第三段的規定:「執行業務者為執行業務而使用之房屋及器材、設備之折舊,依固定資產耐用年數表之規定。執行業務費用之列支,準用本法有關營利事業所得稅之規定;其帳簿、憑證之查核、收入與費用之認列及其他應遵行事項之辦法,由財政部定之。」

關於列支憑證,由財政部來定之。在我來看,他其實應該是要讓執行業務者,如果比照營利事業來去做相關的商業會計的記帳,那就要容許納稅義務人有選擇權,可以從收付實現換成權責發生制。因為對一部分的執行業務者來講,權責發生更能反映出他真實的這個盈虧的狀態,特別是對有大量設備要求的,像醫生。醫生,有一些設備上的要求。診所,比如說牙醫診所,或者是這個外科的診所,有些特殊的機器設備。

當然啦,因為各種不同的執行業務者有會有一些差異啦,像律師跟會計師,一般來講腦力作業比較多,但也不是說他們不需要一些什麼機器設備。律師,最低限度也是要有個書狀的管理系統啊。這個本來就是看個別的業別的差異。

14條第一項,第二類,執行業務者,第一段是收入減成本費用的客觀凈所得額的規定,第二段是協力義務的規定,第三段是關於授權規定,由財政部來準用相關的營利事業的課稅的規定。

14條第一項,第三類,薪資所得的規定,也應該是客觀凈所得的規定,第一句:「薪資所得之計算,以在職務上或工作上取得之各種薪資收入,減除第十七條第一項第二款第三目之 2 薪資所得特別扣除額後之餘額為所得額,餘額為負數者,以零計算。」

17條第一項第二款第三目之二的薪資所得特別扣除額是,是一個概括扣除額的概念,是一個概括成本費用的概況扣除額的規定,因此這個薪資所得特別扣除額跟本款規定以下的一二三目,構成二擇一的狀態。第二款的規定就是告訴你,薪資所得特別扣除額跟薪資所得列舉扣除額,二擇一。

一二三目是列舉必要費用。薪資所得的特別扣除則是概括成本費用。它是二擇一的狀態。這個是我們的薪資所得特別扣除額跟薪資所得的列舉必要費用扣除額,這兩個條文規範的狀態。

這個條文的規定的狀態,是釋字745好之後,才讓薪資所得者有機會有這個列舉的必要費用啊,但我上次有跟各位講過,因為有3\%的額度,所以粗估計算大概你要薪資所得總數超過600萬以上,你才用得到這一種薪資所得的列舉扣除我的額度,達到跟概括扣除額等同的效果。

第三款規定,則是要檢具相關列舉必要費用的單據。

第四款規定,各類收入。第四款規定,是對薪資所得獲得的收入做一個概括性的規定,包括了薪金、俸給、工資、津貼、歲費、獎金、紅利及各種補助費。一次性給的跟繼續性給的全部都算在內了。一次性給的獎金、津貼,或者是繼續性給的薪金、俸給、工資,這個都是屬於薪資所得的範圍。

但書規定:「但為雇主之目的,執行職務而支領之差旅費、日支費及加班費不超過規定標準者,及依第四條規定免稅之項目,不在此限。」

不超過法定標準的出差日支跟加班費,不適用薪資所得的規定。

第五款規定是勞保條例:「依勞工退休金條例規定自願提繳之退休金或年金保險費,合計在每月工資百分之六範圍內,不計入提繳年度薪資收入課稅;年金保險費部分,不適用第十七條有關保險費扣除之規定。」

也就是說,你自己自願提繳的退休金或年金保險費,不計入薪資所得的範圍。那年金保險費的部分也不適用17條有關保險費扣除的規定,依勞工保險提撥的,這個不算是員工的薪資所得。

第四類的利息所得規定,我們現在目前在實務上面幾乎都是用那個10\%分離課稅的規定,因為我們的條文規定是短期票券跟各類利息所得,原則上就是用固定比例的方式,10\%分離課稅。分離課稅的規定是在14-1。從民國96年,公債、公司債及金融債券之的部分利息所得,分開計算,10\%。所以現在買公債就很有利喔。為什麼很有利?因為他也不課證券交易所得,就是直接就是用這個10\%。就是我國現在現行,在我們的這個統治高權有效行使範圍內的臺澎金馬這些領土領空跟領海的範圍。領海,原則上是以國際法裡面所提到的以我們的陸地往外推的三海浬。因為我們剛好臺灣周邊,沒有大陸棚的那個連結,基本上就不會再往外推。

我剛剛跟各位比較過,如果你是高所得的人,你是適用邊際稅率40\%的,買債券跟買公債都是10\%,相對是很有利的,被認為是稅捐節稅規劃的手段。

第五類,租賃。租賃包括不動產跟動產租賃的所得,以及權利金。權利金所得是指非買斷所有權,而是只有單純的使用權的授予,所以這一類的租賃所得跟權利金的所得,各位可以去看一下,在第一款的規定就有客觀凈所得的規定,也就是用全年租賃收入減除必要的費用跟損耗後的餘額。必要費用就是折舊(因為財產會有折舊哦),必要的損耗,以及費用,之後的餘額。這是客觀凈所得。

但第五款規定,請各位特別留意。關於租金收入,如果低於一般租金行情的話,則以一般租金行情為計算租賃收入的標準。

第六類,自力耕作,有收入減成本費用的客觀净值原則的適用,你看條文規定就可以看得到。

第七類一樣,有客觀凈所得的適用規定。你要把交易時的成交價額,這個就是收入,減除原始取得之成本,以及改良持有期間裡面的一切改良費用,之後的餘額為所得額。這個也是一樣,收入減成本費用后的凈所得,這個是客觀淨值原則的適用。

第八類,競技競賽的中獎,原則上是分離課稅啊。你看第三款的規定。政府舉辦之獎券,像運動彩券,這個都不計入綜合所得總額的範圍。不在這個範圍的是一般由民間企業團體所舉辦的競技競賽。民間企業或者是基金會所辦的活動,才會有併計所得的問題。

第九類,退職所得,是以法定的保險給付範圍內的所得為限。屬於保險給付範圍內的保險金給付養老金,或是退休金,依照工資下去去做計算。

第十類的其他所得,原則上有收入減成本費用的客觀凈所得。

因此第14條的規定就是一個客觀淨值原則的表現。

第14條第二項是實現原則。實現原則,也就是以實現時所取得的這個當地的市價。14條第二項的規定是我們實現原則的規範。

第14條的第三項規定,被稱之為叫變動所得。變動所得,四種類型,以半數所得課稅,半數免稅。

14條的第四項的規定,這個一樣是物價指數的調整,跟前面基本上是一樣的一個意思。其實這個都是只要在前面做一個總體的條文規定就可以。

14-1我們剛剛講過。

14-3是稅捐規避的條文規定。

14-4到14-8,基本上是房地合一稅的條文規定,我們就在所得稅法四再跟各位去談。

\hypertarget{ux689dux6587ux68b3ux7406ux5c0fux7d50}{%
\section{【條文梳理,小結】}\label{ux689dux6587ux68b3ux7406ux5c0fux7d50}}

好,我為什麼要講這麼多?因為所得稅法的國考,就是法律條文而已。你好好把法律條文看好,其實有一個規範上邏輯上的結構。

對各位同學來講,綜所說,基本上看所得稅法還可以有大致上的一個方向。

營利事業所得稅就比較不容易是這樣。營利事業所得稅,因為在整個的規範結構上面來講,他的條文規定往往在實務上,被營利事業所得稅查核準則給替代掉。所以在營利事業所得稅,我們的條文解說上面,反而在條文的解釋適用上面來講,看所得稅法的不多,大部分都是看所得稅查核準則。然後只要涉及到關係企業的,是看營利事業所得稅的不合營業常規的移轉訂價查核準則。全面性的查核準則的話,也就是變成是透過法規命令的這一條規定來去做相關的法律適用。所以反而,所得稅在營利事業所得稅裡面,看所得稅法不太容易呈現出稅捐負擔的客觀上的樣貌。

綜所稅還可以。因此也請各位務必熟悉法條的規範結構。我剛剛念到的,基本上就是綜所稅裡面的客觀淨所得。

第15條,我們上一次跟這一次來不及講的,就是合併申報,其實包括了合併所得的計算。所以放到這裡面,他也是客觀淨所得,只是現在是兩個主體。一個是納稅人,一個是配偶,合在一起。所以他有實體法規定。但這個條款規定其實應該是放到結算申報那個地方,因為他是把兩個主體合併申報,所以是一個程序法兼具實體法的規定。他本來是一個結算申報程序的規定,但是因為兩個主體合併計算所得,從而放在第15條的規定。

第16條則是第15條規定的延伸,因為他是同一個家庭申報裡面的納稅人跟配偶,他的營利事業所得的這個營利跟虧損會併計。

第17條,則是關於這個主觀凈所得的規定。所以我們的整個稅捐實體法的規定,理論上到17條規定的時候,是整個課稅構成要件的相關規定。

從17-2,這個之後是另外又出現一個稅捐優惠的規定。

因此我透過這一次的課程,也跟各位再一次去講到,我們的所得稅法的規範體系,相對其他發展比較完整完整成熟的民法跟刑法,我們確實是比較亂。但所得稅法已經是所有的稅法規範條文裡面,比較好一點得了。就是他的條文規範結構上,主體、客體、稅基、稅率的規定,基本上還可以找得出對應的那個條文的關係,而且至少在自己母法規範的層次裡面,大致上可以看得出來。雖然也有例外的,所得稅法第八條規定中華民國來源所得認定原則,或者是稅籍居民的條文規定,其實都還是要去看,財政部所公布的那些解釋函令,才能夠進一步去具體化。但這已經是算相對比較好一點了。

稅法難學的原因也是這樣,因為體系比較紊亂,常常要看解釋行政機關的行政規則,甚至是這一些沒有形諸於文字的行政規則的行政上的慣例。

綜所稅裡面算是相對比較清楚一點,請各位把那個整個規範結構稍微做一下調整,放在自己的腦海裡面。

我們先到這裡。下個禮拜,我們開始去談第71條規定跟第15條的規定,也就是家戶所得課稅制。夫妻之間的是,納稅人跟配偶的合併申報,還有其他的家庭成員,就是納稅人跟其他的受扶養親屬的合併申報。我們統稱叫家戶所得課稅制,或家庭課稅制度,那這個其實是數個主體合在一起的合併申報制。我們在下個禮拜跟各位繼續做這個部分的說明。

\hypertarget{section-23}{%
\chapter{20231127\_01}\label{section-23}}

\begin{longtable}[]{@{}l@{}}
\toprule()
\endhead
課程:1121所得稅法一 \\
日期:2023/11/27 \\
周次:13 \\
節次:1 \\
\bottomrule()
\end{longtable}

\hypertarget{ux7a05ux6350ux7a3dux5fb5ux7a0bux5e8fux7d50ux7b97ux7533ux5831ux7e73ux7d0d}{%
\section{【稅捐稽徵程序,結算申報繳納】}\label{ux7a05ux6350ux7a3dux5fb5ux7a0bux5e8fux7d50ux7b97ux7533ux5831ux7e73ux7d0d}}

進入稅捐稽徵程序裡面的申報。

我們的所得稅法第71條的規定底下,是關於納稅義務人的結算申報的義務。

71條第一項的規定:「納稅義務人應於每年五月一日起至五月三十一日止,填具結算申報書,向該管稽徵機關,申報其上一年度內構成綜合所得總額或營利事業收入總額之項目及數額,以及有關減免、扣除之事實,並應依其全年應納稅額減除暫繳稅額、尚未抵繳之扣繳稅額及依第十五條第四項規定計算之可抵減稅額,計算其應納之結算稅額,於申報前自行繳納。但依法不併計課稅之所得之扣繳稅款,不得減除。」

條文規定是用納稅義務人,其實並不是所有的納稅人,應該是要比照71-1第一項的主體的規定,「中華民國境內居住之個人」。因為,71條的「納稅義務人」並沒有包括非境內居住者。非境內居住者之所得,是用第88條跟89條的就源扣繳的規定,去做所得稅的稽徵。

因此在第71條第一項規定裡面,境內居住之個人的納稅人,應該在每年5月1號起至5月31號止,他的協力義務是填具申報書向該管稽徵機關申報,他上一個年度內構成綜合所得總額的項目跟數額,以及有關減免扣除之事實,並應依其全年應納稅額減除掉,尚未抵繳之扣繳稅額,及15條第四項規定之可抵減稅額,也就是兩稅合一裡面,容許納稅人可以去用可扣抵稅額8萬塊錢的這個部分,計算其應納之結算稅額,於申報前自行繳納。

因為我們的法條規定,他是把綜合所跟營所放在一起,所以我剛剛在念的時候,特意把跟營所有關的,排除掉。

我再念一次。境內居住之個人之納稅義務人,應於每年5月1日起至5月31日止,填具結算申報書,向該管稅捐稽徵機關申報,上一年度構成綜合所得總額之項目及數額,以及有關減免扣除之事實,並應依其全年應納稅額減除掉扣繳稅額,及15條第四項規定計算之可抵減稅額,也就是兩稅合一的一戶8萬塊錢的可抵減稅額,計算其應納之結算稅額,於申報前自行繳納。

這個是我們的稅捐申報義務。

釋字531號認為,結算申報義務是合憲的。

71條的第一項第一句的規定,體現了,應納稅額減掉扣繳稅額跟可扣抵稅額,之後得到結算稅額,這個規範結構。

應納稅額,其實是要連接在前面的第17條。我們17條的規定是,依第14條所規定所計算出來的應納稅額的總額,減掉第17條的主觀淨值之後的應納稅額的淨額。這個是應納稅額金額的規定。透過這個條款規定之後,應納所得總額的淨額,還要再回過頭,乘以第5條第二項所規定的綜所稅的稅率,才會得到應納稅額的這個課稅結構。然後之後再依據71條第一項第一句之規定,減掉扣繳稅額跟扣抵稅額,得到結算稅額。

這個部分,雖然體系上放在第71條第一項,但到這個地方為止,還是屬於稅捐實體法,也就是所謂的法律效果的規定。也就是說,我們在課稅構成要件裡面屬於稅捐實體法的部分,包括主體、客體、稅基、稅率。這個都是屬於課稅構成要件的規定。稅基乘稅率,會得到應納稅額,再減掉兩項。性質上,第一項,扣繳稅額,是預先繳納的稅額,是納稅義務人透過88條跟89條規定做就源扣繳程序,所以是事先繳納的稅額。而第二項,可抵減稅額的部分是某種程度上,稅捐優惠的性質的可扣抵稅額。

如果假設我們的第2條第一項是採全球所得來源都課稅的話,這個部分也會包含外國可扣抵稅額,只是我們的綜所稅沒有這樣一個外國可扣抵稅額的制度設計,因為我們本來就是採屬人主義。

到下個學期,我們下一次上到營利事業所得稅的時候,就會有這個部分。而所得營利事業所得稅,會有3個可扣抵稅額,包括了暫繳稅額,包括了被就源扣繳的扣繳稅額,以及可扣抵的稅額。而可扣抵稅額,裡面包括了外國可扣抵稅額,以及稅捐優惠提供給他的可扣抵稅額。最後算出來的結算稅額,才是實際上應該要繳納的稅額。

老師個人是認為說,在那個地方規範的結構上面,不太看得出來實體法跟程序法的區別。基本上實體法規定應該是要一直到結算稅額出來。出來以後,所得稅法第71條的規定,才將前面的結算稅額,再依照我們今天所剛剛所念到的規定,你應該在每一年5月1號到5月31號這一段時間裡面去做結算申報的程序,這就是你的協力義務的規定。

這個是我們跟各位去談到關於,課稅法條規範的結構的關係。從稅捐稽徵法這個部分的規定,如果我們的稅捐稽徵法能真的擔負起總則性規定的話,原則上這個部分他就是全部放到稅捐稽徵法裡面去規定。

德國稅捐通則就是這樣一個規範結構,把所有的稅捐稽徵程序,跟制裁規定全部都放到他們的稅法通則裡面去。所以他們的所得稅法的規定,就只限於是稅捐債務法,也就是構成要件跟法律效果的規定。課稅規範的結構,如果體系能夠比較清楚呈現,會是這個樣子。

當然,如果一個國家可以制定出一部統一的稅法典,統一稅法典的課稅規範結構的規定,也會是類似這樣一個情況。有總則法的結構。

總則法,從稅法的基本原則,到計徵程序,制裁的規範,然後接下來各稅,有點像我們的刑分,有稅法的各論,從所得稅法,一直到財產稅交易稅跟營業稅。這樣的那個規範結構,就會變成是一部統一的稅法典。

不過統一稅法典不太容易做得成哈。至少到目前為止,如果從各國的比較立法來講的話,是美國有做到,因為他有一部IRC(Internal Revenue Code)這樣子的一個聯邦稅法。因為他是一個聯邦制的國家,所以他把他們所有的聯邦課徵的稅捐,基本上都是放在聯邦IRC裡面去。我聽黃源浩老師說,法國也有,法國也可以做到這個。因為我不會讀法文,所以我不太能夠有這個智慧去講到這個。

我曾經有對我德國的指導老師,跟來臺灣的德國教授問過,說德國有沒有可能做一部統一的稅法典。因為這也是稅法學者的,共同努力的目標啦。那我問到的教授是跟我說,這個以德國的政治環境不太容易啊。德國,基本上也就是稅總,把所有的稅捐稽徵程序、制裁,全部都放到德國稅捐通則。各稅基本上也是分散式的立法。

那我們臺灣的稅制規範的結構是,各稅裡面有稅捐稽徵程序。稅捐稽徵法這個名目,並不是真的是總則法,毋寧是把各稅的稽徵程序又放到我們稅法的各稅法裡面去。所以你看,我們現在在所得稅法71條底下,有結算申報程序,而且這個結算申報的「結算稅額」,這個概念在所得稅裡面才有。其他稅沒有。其他稅,只有「應納稅額」的概念。其他稅裡面,算出稅基乘稅率,之後就是得到應納稅額了,因為沒有預先繳納的問題。

在所得稅裡面才會有「結算稅額」的概念。當然,我們沒有上到那個財產稅。財產稅裡面,地價稅跟土地增值稅裡面,其實也有一種調整土地地價稅的稅額,可以去扣抵土地增值稅的稅額。那就是說,誒,你把地價調高,多增加繳納出來的那個地價稅,可以拿去做扣抵土地增值稅的稅額。但那個地方我自己個人認為說,那個土地稅法規定是一個不對的規定,是一個錯誤的規定。因為地價稅是財產稅,跟那個土地增值稅,他是資本利得稅,他是所得稅,性質是不一樣。

你不能拿財產稅來去扣抵所得稅額。只有同類稅額才可以扣抵。財產稅是所得稅的所得經濟成果的前提,所以財產稅可以當成本可以當費用,但是就不能扣抵稅額。我們沒有要講財產稅。我只是說,前提要件的這個課稅,要做稅額扣抵,必須是同類型的所得。所以我們到遺產贈與稅法裡面,也會提到,贈與稅可以去拿去扣抵遺產稅,同類才可以扣抵。同一個類型的稅額,才會具有互相計算扣抵稅額的問題。那這個我們到所得稅法三的時候會再另外去談到,被繼承人死亡,然後繼承人繼承,然後繼承人又很快又死亡。前面繳的那個遺產稅,可以扣抵後面的遺產稅。

視為遺產,裡面也會有把前面的人繳過的贈與稅,拿來後面在遺產稅裡面去做扣抵。同類的,可以扣抵,不同類的,沒有扣抵的問題。不同類,原則上是要看,可能是成本,可能是費用。

其實到這個地方都還是屬於稅捐債務法的規範的內容。

這個之後就是,納稅人應該在每年5月1號到5月31號結算申報繳納。在此之前,國家沒有稅捐債權的期前請求哦。在此之前,即使課稅構成要件該當,國家沒有期前債務的請求權。除非他另外構成該當稅稽法第25條的規定,期前請求。

因為這個履行期限是法定的,提前請求必須要有稅稽法25條所規定的納稅人隱匿或者是脫逸財產執行的情況,國家才可以去做期前請求。

不過這個期前請求,目前為止沒有看過稅捐稽徵實務上用過。稅稽法24條的稅捐保全手段,常用,包括假扣押、未供擔保假扣押,或者是那個限制出境,這個還蠻常用的。可是25條的期間請求,我嘗試查過實務上的判決,基本上沒有找到。

期前請求的概念就是說,在稅捐構成要件該當之後,到稅捐債權債務以法律規定來請求之前,也就是在申報期限之前,那如果有產生了那個納稅人隱匿,或者是脫產啊,就是規避這個稅捐執行的跡象的時候,稅捐稽徵機關,可以在期前的時候,來去做稅款的保全動作。

因此稅稽法25條也是一種稅款保全程序的一環。

回到我們這個地方來。在5月1號到5月31號之前,這一段時間就是申報期間,這也是納稅人自行申報繳納的期限,因為我們是報繳合一。申報跟繳納,報繳合一。那這個報繳合一制度,根據立法者的說明,認為說,如果採取申報,待核定之後再繳納,就是,申報、核定,然後再繳納,會因為稅捐稽徵機關核定時間的長短,產生了繳納期間差異的問題,被認為不公平。所以我們的制度把繳納制度移動到前面來,也就是申報跟繳納合一的制度。這就是71條第一項第一句的報繳合一制度。報繳合一制度,因此構成了我們稅捐稽徵程序裡面,我國主要稅捐徵收的一種最主要的型態。

在稅法總論裡面,我們按照稅捐稽徵程序的方式裡面的申報、繳納、核定,或者是申報、核定、繳納,而去做一些區別。

那這個地方我們的法律規定,確實是比較紊亂的。

德國,稅制基本上是,申報,之後待核定,之後做繳納。所以德國人的解決問題的方式是這樣,你申報之後,我就我也怕剛剛講的那個情形。就是在德國來講,也可能會有這種,比如說柯格鐘納稅的時候,你就等了一年才給他核定,那柯格鐘就可以合法有效地一年之後再來繳稅。那如果是柯小鐘的,就6個月就給他核定,那這個時候是不是很快就要去做繳納?

所以德國人的做法是一個,先按照申報數去做核定,然後做一個「事後查核權保留」的附款。簡單來講,核定,為了要保持原則上先後不差超過6個月以上的前提底下,稅捐稽徵機關會先依照你申報的數額,來做一個附事後查核權保留之行政處分。核定當然是一個行政處分,所以各位去看行政法的教科書,裡面,行政處分,有附條件、附期限、附負擔,還有一個叫附事後查核權保留。這一類型專門就用在稅制裡面,因為稅捐本身是大量行政,你申報過來的,我需要好好地調查,那我不可能每一個都6個月內都核定完成呀。這個時候,他做這個程序,就是讓稅捐機關,在適度地有這個時間去調查的前提底下,事後查核權保留,也讓法律行為本身構成了明確性。

你這個案子,我還需要再進一步調查,那我就跟你講一個事後查核權保留。行政實務上,德國是會直接講明,我還需要多久,比如說我還需要6個月,那我就寫,事後查核權保留6個月的時間。好,到了時間到了,如果稅捐稽徵機關仍然還沒有調查完成,特別是涉及到境外課稅事實的調查,那稅捐稽徵機關,可以再説明,基本上是不附理由的說明,他還可以因為這樣一個調查的因素,還可以再延長,可以繼續延長。而這個繼續延長,都是為了讓納稅人明確地知道啊,你這個案子還需要更長的查核時間,而這個地方構成了一個信賴基礎。

也就是說,在這裡面,一方面是行政機關保留事後查核權,是一個期限的保留,而這個期限的保留,納稅人可以信賴說,好,原則上6個月後,這個案子應該就會確定。可是如果6個月之後,稅捐稽徵機關仍然還是有待證事實未查核。比如說境外課稅事實,因為特別要經過稅捐資訊的交換,需要一段比較長的時間。稅捐稽徵機關還可以再做一個延期,而這個延期最後不能超過整個的核課的期間。因為這個是國家稅法上債權行使的核課期間,也就是請求權的期間。

因此整個德國的稅捐法制的結構跟精神,就會彰顯在法律行為的明確性,也就是行政處分的明確性、法律上的明確性,以及提供行為人足以作為信賴基礎的這些明確性的行政處分。

在臺灣,用的結構不太一樣。就是我們把繳納放到前面來,後面的這個核定,早期的所得稅法是連核定都沒有。也就是說申報繳納,如果稅捐稽徵機關認為沒有錯,沒有核定的處分。一直到民國98年,我們才以公告代核定,用公告的方式代替實際上的核定。也就是說,因為我們一開始我們採取的立法的體系,是國家先收到稅款,那假如他收到稅款以後,認為沒問題的話,那麼早期是沒有核定,就沒有任何事情發生。

如果你被查到應繳納稅款的事實,但你沒有申報,那這個時候他當然就會做一個補稅單,補稅,就有一個補稅的行政處分。補稅當然是一個行政處分。就算稅稽法第16條跟你講,補稅通知單,用通知這個概念,他仍然是一個具有規制效力的對外的行政處分。補稅,當然是一個行政處分。

\hypertarget{ux6ea2ux7e73ux7a05ux6b3eux8fd4ux9084-vs-ux9000ux7d66ux7a05ux6b3e}{%
\section{【溢繳稅款返還 vs 退給稅款】}\label{ux6ea2ux7e73ux7a05ux6b3eux8fd4ux9084-vs-ux9000ux7d66ux7a05ux6b3e}}

即使是退給稅款好OK,因為有可能你自己自行計算錯誤,你算錯了,稅捐稽徵機關主動發現,他就退給你,這個退給稅款。理論上來講,應該是不構成行政處分,應該是不構成。但如果說稅捐稽徵機關沒有去做這個返還稅款,這個時候,納稅人可以根據當時候的稅捐稽徵法28條的規定,可以去申請,他自行計算錯誤,因為是我自己報繳的,那我算多了,算錯了。因為以前特別是用手算、人工算,沒有電腦的時代啊,容易會算錯。在這種情況底下,稅稽法28條提供一個機制。理論上,本來稅捐稽徵機關要主動發現計算上的錯誤。總而言之,你報繳稅捐都是納稅人自己自行申報的繳納的,然後他繳納了100塊錢,結果他最後面,因為他算錯了,其實他只需要繳50塊。本來稅捐機關應該要主動退還給他稅款50塊,那個理論上是個事實行為而已,因為國家的課稅債權只有50塊錢要滿足。那你自己繳納100塊,是不是多了50塊,那我就還給你50塊。可是如果稅捐稽徵機關沒有還,那這個時候28條給納稅人一個公法上的請求權,讓他可以去請求稅捐稽徵機關,因為我自己計算錯誤,從而我請求你退還給我溢繳稅款,有一個溢繳稅款返還請求權。這個時候就產生了我們當時的稅稽法28條的規定。所以當時98年以前的稅稽法28條就是納稅義務人因自行計算錯誤,可以向該管稅捐稽徵機關申請為溢繳稅款返還,當然他法條文字都還是寫退給稅款。

那我在稅法總論裡面,我都會講,這個叫返還這個叫溢繳稅款返還。就是國家稅捐債權只有50塊,納稅人自己算,算錯了,算100塊,那當然那另外50塊是無法律上原因啊,國家不能保有,所以當然是一個公法上不當得利的返還。本來應該是要稅捐稽徵機關主動發現,要還給人家的,但沒還,給納稅人一個公法上請求權,讓他可以去請求,所以請求該管稅捐接機關去返還他。

所以我個人一直講,那個是「溢繳稅款返還」,返還,不叫「退給稅款」。因為退給稅款是有公法上的原因,有法律原因的退給。這種退給必須要有法律的明文規定作為依據。

一個叫有法律原因的退給,一個叫無法律原因的溢繳稅款返還,這兩個不能分辨清楚,一直都是我們稅捐稽徵實務,乃至於學界,很大的問題所在。因為兩個的舉證責任分配是不一樣的。

溢繳稅款返還,是稅捐課稅構成要件的該當與否的問題,原則上是稅捐稽徵機關要負舉證責任。

相反的,退給稅款,因為是依法律規定的退給,從而是要有主張有退給權的請求權人,也就是基本上是納稅人,這一方要來負擔退給稅款的舉證責任,證明各項退給稅款的構成要件該當性。

這個是完全不一樣的概念,到目前為止,還是很多人不太清楚這個依法退給,跟不當得利的溢繳稅款返還的概念。

\hypertarget{ux5831ux7e73ux5408ux4e00}{%
\section{【報繳合一】}\label{ux5831ux7e73ux5408ux4e00}}

回到我們這個地方的報繳合一制度。在報繳合一制度,我們的早期的稽徵實務上面,除非是補稅,不然不一定會出現一個核定稅捐的行政處分。

從民國98年之後,有一個公告代替核定的處分。因此公告就變成是代替個別核定,到目前為止沒有特別去說明這一個性質到底是什麽。因為公告,顯然不是法規命令,不是一般的抽象性的命令。他是可以被具體化的,對特定人產生效果,至少是宣示了一件事情,說,欸,你申報的稅款跟繳納稅款,我收到了,目前為止沒問題。因此他不會是一般性的抽象的命令。這個公告的性質還是要請各位特別留意哦,因為在不同的法律規範裡面,比如說公告地價公告現值的公告,在財產稅裡面的那個公告地價跟公告現值,目前實務上是把他當作法規命令喔,他是不能救濟的。地價的公告是非具體的行政處分。

好,但接下來我們講的,如果是處分,具有個別具體規制的。這個是對物對人所做出來的一般處分,或是個別具體的行政處分?一般處分,或者是行政處分?欸,這個是沒有特別被探討過的喔。我個人比較傾向,他不是一個對物,或是對事件的公告,公告的方式,上面有講你的稅捐申報的單位,是有編號的。所以他的公告是一個具體的,集合在一起做成的,對數個可茲辨識的納稅義務人跟申報戶的一個行政處分,所以他是一種集合性行政處分。

比如說,今年大家所有的同學都申報稅捐了。假設今天A同學,我看到他漏報所得,我就給他一張補稅單,這個當然是個別具體的行政處分,沒有問題。好,其他同學,沒問題,我就貼一個公告,就像我們考試院的榜單公告一樣。那個是具體的吧?但他不是一般處分哦,他是集合在一張榜單上所對外所做的一個具體的行政處分。

從而他在公告的性質上面來講,比較是一個集合性的行政處分,也就是數個行政處分,在一張公告上去做完成。

我要跟各位去說,公告這件事情,在不同的稅法裡面,同樣的公告有往往有各式各樣不同的可能性。我剛剛講到公告地價、公告現值的公告,他本身是抽象性的法律命令。房屋稅的房屋評定標準價格的公告則又構成具體的行政處分。開徵的公告,每一年地價稅開徵時期裡面所做的開徵公告,這個又不構成具體的行政手段,因為這個公告他是一個法規範遵循的宣示,也就是宣示,請大家注意,請大家注意,什麼時間開始,我們要開始開徵地價稅了哦,有要申請稅捐優惠,要申請稅率上優惠或是什麼優惠的,請趕快在什麼時間以前,備齊你的資料。這種公告不具有個別規制效力,不能作為行政救濟程序的標的,那也不是一個所謂的可以被提起行政程序的一個標的。

也因此,公告的這個法律性質要做個別的判斷。當然了,回到我們所得稅法裡面,由於是公告代核定的方式,因此公告可以被理解為是一個行政處分,但,問題是,現在目前實務上沒有人就這個公告核定來要求,如果我有請求要做溢繳稅款返還,不需要對全體的公告來去做行政救濟程序的處分的救濟。簡單來講,就是也沒有去對這個公告來去做行政處分的救濟。相反的,納稅義務人如果應繳未繳,沒有繳納足夠的稅額,稅捐稽徵機關也一樣會透過補稅單的方式,無視於這個公告的存在。也就是不管他有沒有信賴基礎可言,無視於這個公告的存在。

當然啦,你可以講一件事情,就是說,因為納稅人申報的數額,沒有把他應繳納所有的稅款的事實,全面性地去做申報,從而稅捐接機關在不變動前面公告的前提底下,另外,再就他稽徵所查到的漏報繳所得稅的事實,再來去做補稅的行政處分。

所以我們實務上就會構成了,你先前所做成的申報核定,構成一個行政處分,後面我再查到你有臺大所得,你有成大所得,但你只報臺大所得,成大所得沒報,那我就再根據成大所得查到的事實,再給你一張補稅單,構成並存式的兩個行政處分,並存式的行政處分。

德國,會是撤掉前面的行政處分,以新代舊,也就稱之為叫取代說,變更說。就是以新代舊。你成大臺大所得全部都加起來算出一個所得額,這個就是總額主義的總額。

所以,永遠,一個納稅義務人的一年,當年度的一個稅捐事件,只有一個課稅處分上面所載的數額,永遠都只有一個行政處分,救濟程序的標的就是在上面所載的稅額。這個就叫總額主義。透過這樣子的總額主義,原則上一個稅捐救濟事件可以在一次的紛爭解決程序裡面把他解決完。

德國的稅捐法律體系,比較體現出一個稅捐債權債務關係,一個課稅的行政處分。一個課稅的行政處分,提起一個稅捐救濟程序,在一個程序裡面就把他全部解決掉。

臺灣的稅捐實務,基本上,法律行為明確性不足,信賴基礎沒有什麼可主張保護的餘地。我們又構成兩個行政處分,因此開啟兩個救濟程序,毋寧是常態,不會一次就把紛爭解決。我要有可能,我每一次短漏一個稅款,短漏一個稅款的補稅跟裁罰,我可以再繼續拆分。所以一個年度的課稅事件可以有數個救濟程序的進行,在我們是每一天的日常,每一天都有可能發生。

好,這個是我們的所得稅的稅捐申報程序。

\hypertarget{ux5a5aux59fbux8ddfux5bb6ux5ead}{%
\section{【婚姻跟家庭】}\label{ux5a5aux59fbux8ddfux5bb6ux5ead}}

儘管談到這裡,我們接下來還是要回過頭來,因為當納稅人是一個群體的時候,也就是當他有婚姻跟家庭存在的時候,我們這個時候,本來所得稅法第71條,這個地方才開始了申報稅的申報協力義務的規定,釋字531號說這個是合憲的。

我們先講婚姻的概念。夫妻,是婚姻共同體。我們現在在釋字748號施行法之後,把異性婚姻也放進來好,所以我先按接下來講了,我們就基本上都用婚姻共同體,來包攝,主要是由兩個人,異性的就是民法上的婚姻,同性的則是釋字748號施行法裡面的同性婚姻。這個是婚姻的概念。他是兩個自然人,以永久共同生活為目的,所組建而成的共同生活、消費共同體。這個是婚姻。

至於家庭,以婚姻為基礎,但不一定完全以婚姻作為限制。因為家庭的成員除了可能是從婚姻而產生出來的,所從出的這些直系親屬以外,也包含了韓旁,也可能包含了沒有血緣關係,只是單純同住在一家,但以永久共同生活為目的的,這我們稱之爲叫家庭。

婚姻跟家庭的概念,因此是稅法上密切連結到民法,乃至於釋字748號施行法的同性婚的這個概念,全部把他搬過來。在這裡面,基於制度的連結,原則上,稅法根據民法的觀念,來建構自己的婚姻跟家庭課稅制度。

因此我們接下來跟各位討論到了所得稅法第15條的規定,裡面所講的婚姻跟受扶養親屬沒有特別規定,原則上就是回到民法,跟後來我們增加制訂出來的釋字748號施行法裡面的同性婚好。就這樣一個課稅上的稅法跟民法的連結。

除了婚姻跟家庭制度有這個連結以外,我要跟各位講,其他沒有所謂的稅法一定遵民法的問題,沒有這個問題。我們國內很多學者一直認為稅法必須要遵照民法。稅法跟民法是同位階的,沒有誰一定要遵誰的問題。民法,你訂立為僱傭承攬委任,稅法就一定要照你的方式去解釋嗎?那個不是這樣子的,因為各自的法律,本身就有各自的規範目的。

私法是私法自治,這一點沒人否認。我們從來就承認私法的私法自治原則,但私法自治原則裡面,私法自治,自己本身也講到,不能「以名害意」啊。你的名目,解釋契約的時候不能拘泥於其文字,而忽略其本質上的爭議嘛。所以在這個地方有所謂的法律上的形式跟法律上的實質啊。這個是民法的概念。

稅法,除了婚姻跟家庭這個概念,是直接聯繫到民法上的規定以外,基本上沒有用同一個名詞必然會去拘束稅法名詞的問題,因為這個本來就依據個別的規範目的而去做解釋。

在稅法裡面的基本原則是量能課稅原則。簡單來講,這個「能」是指一個人的能。當一個自然人有所得、有銷售,有這個能力的時候,作為一個「能」的一個主體來講,這個人並不應該任意地被忽略跟消滅。因此我們進入到婚姻跟家庭課稅制度的時候,首先婚姻是由兩個自然人,不管你是同性婚還是異性,就是兩個自然人,不會更多,也不會更少。所以基於所謂對同性結合為共同體,說會多夫多妻,根本就是無稽之談。因為再怎麼講,婚姻一定是兩個人,只是同性或異性的差別,還是以兩個人結合為共同體作為前提,所以不會搞成一夫多妻,一妻多夫或多夫多妻,這個是不存在的。

即使是用德國的婚姻家庭婚姻制度保障概念,他還是講自然人兩個。所以不會一人一獸,你再怎麼愛你的毛小孩,你也不會是婚姻共同體。

德國現在在講制度保障這些概念的時候,制度保障是對於人。在這裡面,婚姻裡面的人只有兩個自然人,只是差在是同性或異性。德國因為受到德國基本法第6條第一項規定的關係,Ehe und Familie,Ehe,就是講異性婚,就專指講異性婚,可是在德國法制上的建構,是盡量讓同性的可以進入跟異性婚Ehr這個概念,可以受到相同的規制跟受到相同的保護。講白一點,就是只要進入這個婚姻共同體,原則上不管你是同性跟異性,德國民法確實只適用異性婚,但德國也為了同性的成立一個同性伴侶法。

當然,在臺灣,有些人認為只要名詞區別就有歧視。其實也會有對於Separate but equal的概念的爭議。美國南方的這一種,不允許黑人去白人學校。白人都上這個學校,那只要黑人也辦一個學校,跟白人一模一樣的學校,但是就是不讓黑人進入我們白人專屬的學校。這就提起了這一種,分離但平等,不是真平等的概念的時候,大家當然會覺得,哇,有道理。

在我們釋字748號通過之後,在我們國內確實掀起一個非常大的爭議。我無意在這個地方更進一步去講,如果講實質上的平等的話,同性跟異性,當然在婚姻共同體這個概念底下來講,他們是不分區別的,彼此有同樣的權利義務的關係,但他就是兩個自然人組成的一個生活、消費的共同體。原則上連結到民法上的婚姻的概念。

家庭則是以婚姻為基礎,不只是婚姻而已。因為家庭的概念,仍然還是必須要有一個永久共同生活而目的的,數個自然人。

家庭不一定來自於婚姻關係。所以單親家庭式的家庭就是一個爸爸或者一個媽媽,帶著一個兒子,女兒,這樣當然是一個家庭啊。

我們都有知道,所謂的收養制度,我們也知道,家庭中,家長家屬居住在一起,只要以永久共同生活為目的的話,就是一個家庭的概念。因此我們民法上的家庭概念基本上會決定了在我們稅法裡面的婚姻跟家庭的概念。

但如此而已,稅法跟民法的關係基本上是個別依據個別的規範目的,而來去做獨立的解釋。

這也是我們今天要開始去談婚姻跟家庭的合併申報、合併計算稅捐、合併繳納稅捐的基礎。

71條是結算申報的規定。理論上第15條的夫妻合併結算申報規定,他是一個進入個人綜合所得課稅到結算申報制度的時候,把兩個以上的主體合在一起,不管是婚姻或是家庭,都是兩個自然人的結合,合併去做申報。

合併申報,沒有必然要合併計算所得的問題。因為一個是程序法,一個是實體法。學過法律的各位,一定要有基本的概念。共同起訴,不必然叫做必要共同訴訟。這不是常識嗎?連這個常識都沒有,這就是我們的所得稅法第15條所有一切問題的來源。

其實15條的位置,本來應該是在71條這個地方,也就是結算申報,他是因為合併申報,然後又回過頭來說,你們的所得要合併計算。這個從程序法回過頭來變成實體法裡面的所得額的計算。所以早期的合併計算是對婚姻跟家庭極為不利的,因為他把兩個人所得算成一個人的所得。這個就是釋字318號的背景,釋字318號還仍然認為那是合憲的制度。這個我們待會跟各位去談,所得稅法第15條、第16條、釋字318、釋字696,乃至於家庭合併申報制的釋字692號、694號、696號。那時候一年出了三個跟婚姻跟家庭制度有關的大法官解釋,但這個地方當然有許多的問題值得進一步去加以說明。

我們先休息一下,大概回過頭來再來跟各位談婚姻家庭的課稅制度。

\hypertarget{section-24}{%
\chapter{20231127\_02}\label{section-24}}

\begin{longtable}[]{@{}l@{}}
\toprule()
\endhead
課程:1121所得稅法一 \\
日期:2023/11/27 \\
周次:13 \\
節次:2 \\
\bottomrule()
\end{longtable}

關於受扶養親屬,在我們的課稅制度裡面,婚姻跟家庭就是結合在一起。我們還是要跟各位做一下區別。

對於受扶養親屬,會為了要說明方便起見,我還要再繼續分成直系血親的尊親屬跟卑親屬,那么當然也有旁系,甚至是根據我們的法律規定,你只要以永久共同生活的目的在一起的家長跟家屬,這個都會是。

直系血親卑親屬裡面,我們再把他分成了一親等的子女,跟二親等孫子女以上。子女這個部分,我再把他分成成年,跟未成年。這個分類,待會有說明的必要。

\hypertarget{ux5a5aux59fbux8ddfux5bb6ux5eadux7684ux5408ux4f75ux7a05ux5236}{%
\section{【婚姻跟家庭的合併稅制】}\label{ux5a5aux59fbux8ddfux5bb6ux5eadux7684ux5408ux4f75ux7a05ux5236}}

我們的合併申報制度,根據我們所得稅法第15條。在釋字318號解釋之前,他基本上都一直延續著這個精神,就是合併計算、合併繳納。

合併計算,這個是實體法的規定。從申報繳納開始應該是程序法上的規定。

根據所得稅法第15條的規定,納稅義務人在做結算申報的時候,當構成數個主體合在一起,共同組建婚姻的時候,我們的法律制度基本上是強制性的,合併申報。也就是納稅義務人跟配偶,原則上必須要強制合併。只有在例外的情況底下。法條本身並沒有規定例外的情況,釋字318號甚至696號之前,我們法條並沒有任何規定,可以分開申報。但是當時候透過財政部的74年的解釋函令,裡面提到,如果你是分居的話,則無庸合併申報,分居的話就不用合併,可以分開個別申報。但稅額還是要合併計算喔。我只是讓你可以分開個別報,但你的所得額原則上還是要合併計算喔,那當然繳納還是合併繳納囉。只是讓你在74年的解釋函令裡面可以去做分開申報。因為你是分居的情況,我沒辦法強制你要合併結算申報。從而這個74年的解釋令,讓納稅義務人可以跟配偶可以去做分開來的申報,但是所得額還是要合併計算。

就因為所得合併計算,在合併計算底下,由於他有可能稅負會提高,因為這個基本上是常識啊,因為我們綜所稅是累進稅率,是超額累進稅率。兩個人,只要都有所得,加起來一定比一個人所得多啊,這還用講嗎?除非你的級距夠寬大,不會跳一個級距,不然的話,兩個人所得加起來一定比一個人所得更高,這還用講嗎?除非另外一個人完全都沒有任何所得。或者加起來以後,他們還是在同一個級距內。

我們在此之前跟各位講過,我們的所得稅法是五個級距。這個寬幅度,這個叫級距。假設這個是50萬,這個是100萬,這個150,這個是200,這個就是級距。兩個人的所得,100+100,跳一個級距,你的稅負自然就會比較高啊。所以這個是你可以算得出來的。當兩個人所得被強制合併計算的時候,當時我們的所得稅法是容許只有薪資所得分開計算。如果你們夫妻兩個,其中有薪資所得,那薪資所得可以分開。只有薪資所得可以分開。因此配偶可以就薪資所得拉出去,個別計算,但還是要合併報繳。

74年的解釋函令就是以這個作為基礎,因為你合在一起算,適用累進稅率一定會比較高,所以這個時候你們兩個要根據你們的所得額,比例分攤那個高出來的稅負。

釋字318號當時面對我們的所得稅法第15條規定,納稅義務人只要結婚之後,原則上當年度開始就要強制合併申報、合併計算,只有薪資所得可以分開來計算。配偶,如果有薪資所得,可以就個人的免稅額及薪資所得扣除額,個別計算,其他還是歸納稅義務人歸戶,由納稅義務人計算所得稅,然後來繳納。誰繳?納稅義務人繳。分開是只是計算而已,回來還是納稅義務人繳納?這個就是我們釋字318號當年的所得稅法,是長這個樣子。當年釋字318號說,合憲,哦,我警告一下哦,這樣可能有增加稅負之虞,請主管機關與時俱進檢討一下。警告性的宣示,但沒有說他違憲,或與憲法的平等原則的負擔不符合。

合併繳納,沒有講。合併申報,爲了減少申報單位,合憲。合併計算啊,這個可能會增加稅負,這個請主管機關檢討,與時俱進改進。好,後來當然沒改,因為後來實務,稅捐稽徵機關說,連318都說我們合憲啊,他也沒有說違憲,如果違憲,他就會宣告違憲啦。

\hypertarget{ux5a5aux59fb}{%
\section{【婚姻】}\label{ux5a5aux59fb}}

就這樣,釋字318號就一直維持這個制度。所以我現在舉個例子來講,假設各位同學,兩個畢業之後,你們在學校的時候相認相識之後步入婚姻的禮堂,就結婚了,兩個人都做執業律師。一個賺200,一個賺150,加起來,350,適用到30\%邊際稅率。如果你們個別分開來算,你們可能個別都是適用最高邊際稅率20\%。我這個地方只是為了簡便去反映出稅負的差異,不考慮前面的級距,實際還要考慮累進差額。

甲乙結成婚姻共同體以前,他們總共要繳的稅負是70萬。可是照我們當時的所得稅法的規定,由於兩個都是執行業務所得,所以他是200+150,執行業務所得適用邊際稅率30\%,因此他會變成是350×30\%,105萬。這個中間的差額,105-70多出來的35萬,這個怎麼會不是增加婚姻共同體的稅捐負擔?

之前兩個人個別賺來的錢,他們兩個就算同居在一起,也只需要繳70萬。結果有一天他們想不開了,進入婚姻禮堂以後,竟然國家說,你們要繳105哦。為什麼?因為你們強制合併。你沒分居,趕快分居啊。請問這個制度怎麼會對?你還敢說婚姻有制度保障?連平等對待都沒有,婚前婚後平等對待都沒有的制度。釋字318,還認為合憲。我還真輸給你,這樣叫合憲。這不叫合憲,這個是對婚姻,對進入婚姻的人的處罰。不僅是對進入婚姻的財產權的違反平等原則課稅,也是同樣對我們大法官向來一直宣示的婚姻應受憲法制度保障的背離。

進入婚姻的人沒有受到保障,反而是更不利。其實如果是兩個都是執業律師,我教各位一個方法很簡單,就可以規避這個條款規定。我當時在學校上課的話,我都會教這一招。你們兩個都執業對不對?你就請他就好,他受僱於你。反正事務所都是你們在經營的。各位聽的懂嗎?兩個律師,你們幹嘛合夥?一個受僱於另外一個,另外一個就叫薪資所得。這麼簡單就可以規避。所以當時的制度是在對不懂稅法人的懲罰。因為你們不懂稅法,那你就很自然被懲罰。但這個制度,如果兩個都是不同的執行業務者,就沒這個可能性。比如說律師跟會計師結婚,會計師跟醫師結合,就沒有辦法。兩個都是律師,兩個都會計師可以,兩個都醫生互相僱來僱去,這個都可以。但是兩個人一個律師一個醫生就是不能這樣,因為非律師不能聘僱律師執行律師業務。

這完全不是立法形成空間的問題。因為從量能課稅、個人課稅原則,就知道這個制度是在對婚姻進入婚姻的個人的歧視。一直到696號,我們才終於說這個制度,這樣不可以。

我回到這裡面講,對於分居,其實沒有立法,是財政部自己做出來的解釋。到我們現在的法律裡面才加進來分居。696號解釋之前的分居一直只存在財政部的解釋函令裡面,他沒有一個條款規定說夫妻在結婚之後,如果你因為分居,你可以分開申報。夫妻,照我們的法律15條第一項規定,一直都是強制合併,所謂的強制,就是你沒有選擇不合併的理由。

不合併的理由,是到696號解釋之後,大法官做成解釋以後,我們之後才把分居這個要件放到我們現在目前的所得稅法裡面,如果你有分居則可以分開。那我現在就問各位一件事情,在臺北市共同執業的夫妻兩個人,他們兩個同床異夢,完全就是沒有辦法跟對方分享,我們兩個報稅的時候,為什麼不可以分開來報稅?我們沒有分居,我們還是同住在一起,因為我們沒錢買另外一棟房子。你叫我搬,那你搬呢,為什麼不是你搬?

請問為什麼要設定分居這個要件?為什麼一定強制夫妻得要合併申報?強制夫妻合併申報,意味著我在申報出去的時候,我必須要告訴另外一半,我的所有的所得的資訊,我要告訴你,才能告訴國稅局。這個是一個稅捐資訊自主權的干預。

我協力義務確實是要求跟國稅局去做申報。但沒有必然要透過另外一半才跟國稅局申報的道理,這個轉彎,繞道是不必要的過度干預。因為我這個申報,我就必須要告訴我另外一半,我哪裡有所得。沒錯,我是生活共同體,但我不想跟你分享我的所得,不可以嗎?我出去上課的時候,幾乎都會問在座的每一個人,請問一下你會跟你太太或是先生報告你有賺多少錢嗎?有的請舉手。

我們現在夫妻早就分別財產制很久了,有一定要跟對方講多少所得嗎?那為什麼強制合併申報呢?請問你的道理在哪?696號仍然說強制合併是合憲。強制本身就違憲。

資訊的告知是要得其同意,我今天不跟你合併申報,就代表著我不想告訴你另外一半,即使是生活共同體,我也沒有必要非得要告訴你。所以德國的婚姻合併申報制,是任意性的。任意性的意思就是,你可以個人申報,無需要理由。你個人申報就跟你婚前的狀態一模一樣。因為你們婚前本來就是個別申報。結婚之後,你也可以選擇個別申報,也可以選擇合併申報。因為申報只是程序法,只是共同把你們的所得寫在一張結算申報單上,如此而已,沒有更多,也沒有更少,就這樣。

就像共同起訴執行訴訟的當事人,他沒有必要非要必要共同訴訟,訴訟標的要合一確定。一樣的道理。這不是很難的概念,我也不認為是很難。合併申報就只是兩個人所得的寫在這一張單子上。各自對自己的申報的所得額項目、扣除額項目,負擔誠實申報義務,這樣很難嗎?

當合併申報以後,我們現制是這樣,合併申報的納稅義務人要對配偶申報的這些所得全部都由納稅義務人合併計算、合併繳納。

所以我現在跟各位講一件實際上一定會發生的,你們那一年的時候,你們是配偶,但是你本來應該要繳納稅款的,都是納稅義務人繳,但是之後你們兩個離婚了,當年的所得稅還是要你繳。啊,不然要叫誰繳?因為當年你是納稅義務人。所以我現在上完這一堂課以後,大概就跟各位同學,以後報稅,千萬不要自己當納稅義務,你應該鼓勵對方成為納稅義務人。因為你要準備要有自己私房錢的時候,你就準備好,讓對方當納稅義務人啊,離婚之後再去檢舉他。讓他被處罰,誰叫你要離婚?你不覺得很荒謬嗎?

請問這裡面有幾個主體?兩個人啊。婚姻裡面是兩個主體。所以德國不僅是任意合併申報,德國的制度裡面是容許納稅義務人,因為他們是生活共同體,所以他也可以把所得額透過因為是合併申報,所以他們可以「折半乘二」。這個制度被稱之為有利於婚姻的家庭課稅制。

因為當我一個人賺200,一個賺150,雖然都是用20\%的稅率,可是如果我能夠加總以後除二,也就是加總以後折半,我就有可能適用比較低的累進稅率。簡單來講,就是我一個人賺200,另外一個人假設賺50萬。250+50變成250,250÷2,125,他就有可能是用比較低的邊際稅率10\%。這個制度在德國因此就被認為是對婚姻制度的促進。對同性伴侶也同樣適用。這個制度會讓進入婚姻的人,如果一個人賺200,一個人沒有賺錢,或者只有賺很低的數額,因為折半乘二會可能讓他稅負計算的稅額降低喔。

比如說200+0,因為另外一個沒有賺錢,折半是100,100去適用的邊際稅率只有10\%。所以只需要繳納10萬塊,再成二,因為是兩個人,所以10+10,所以他們家庭總稅負是20。如果各自分別計算,200適用20\%稅率,稅負是40。

德國制度的操作,第一方面是任意合併申報,你可以選擇跟另外一半要和要不要合併,但如果你選擇合併,你可以勾選合併計算所得額,折半乘二計算。這個時候的另外一半合併申報就有一個誘因,讓他可以在這個制度底下降低稅負。

但這個制度,在德國被左派認為是對於進入婚姻的人有利,但是對單身者不利。單身者沒這個機會跟可能,因為德國現在很多人不願意進入婚姻。所以當他們是事實上同居狀態,就沒辦法享受這個制度。所以當然,沒有辦法每一個人都滿足啊,像我剛剛講的這個德國的婚姻制度,即使是對同性婚的人,也同樣適用,可問題是,對沒進入婚姻的,不管你是男男女女,或者是男女,事實上同居,一樣還是沒有適用餘地。

這就是德國制度一直以來有他的爭議性所在。

但反過頭來臺灣的制度是從釋字318號開始,就是一個對進入婚姻的人不利的制度。到了696號,我們仍然強制合併申報,但對於合併計算所得,就宣告違憲。所以696號之後我們就改成了幾種可能性,第一種就是原則上全部合併,第二種,就是以前的薪資所得可以分開計算,之後我們因為696宣告違憲,所以我們現在還有其他所得可以分開計算。

這就是我們現在目前的所得稅法第15條現在的樣子。15條的第二項規定。

第一種就是全部所得合併計算。

第二種叫薪資所得的分開計算。

第三種計算方式,叫做各類所得,就是其他所得的分開計算稅額。

我們的法條規定就因此看起來極度地複雜。但第一種合併計算所得,只要你們夫妻double income,我才不信你們要合併計算所得。除非你們沒有變更課稅級距,不然一定很不利。第二個,薪資所得分開來計算。現在在實務上,這一類型確實很多。因為在實務上,大部分人取得薪資所得,大概在納稅義務人佔七成左右,所以大部分人都可以有薪資所得可以分開來計算。最後一個叫做,各類所得的分開計算,也就是將14條第一項各類所得分開計算。其實這一個條文規定,本質上本來就是把第二款規定的薪資所得分開計算,再把擴大化而已。其他各類所得都可以分開計算,他的極致就是個人個別計算所得額課稅。符合量能課稅原則的基本上是個別主體課稅。

雖然婚姻跟家庭會讓數個主體有一個生活共同體的概念,但是在個別主體課稅的原則下。個人基於扶養另外一個人,包括配偶及受扶養親屬,這一個法律義務範圍內,屬於當事人納稅義務人不可自由支配的所得,所以承認只要越親近的親屬,會有比較高的扶養親屬的扣除額,是現在OECD大多數國家的課稅立法。

OECD大多數國家都是個人課稅為原則,再按照配偶、親屬,直系關係旁系關係,或是家長親屬的這些關係的親疏遠近再給他不同額度的扶養親屬的扣除額。這正是大多數國家的課稅立法的現況。這一種數額的反應,原則上要反映出你們各自國家法律裡面的法定扶養義務的範圍。簡單來講,夫妻是同林鳥,一般來講,配偶的扶養扣除額的扶養費用額度是最高的。因為他們是共同的生活水準,只要他們沒有長期處在分居狀態底下,配偶之間的扶養費用扣除額是最高的。其次,接下來呢?未成年子女。因為未成年子女自己沒有扶養自己的能力,仍然必須要讓父母親去照顧他。從而大多數國家的所得課稅法制,稱之為綜合所得稅的,基本上一定是納稅義務人個人先算,再扣掉我付給配偶、未成年子女以及其他受扶養親屬,根據你的關係親疏遠近,你要劃分幾級,我沒有意見,但原則上立法原則就是反映出你的法律扶養義務的範圍。

這個制度本來,理論上也應該要反映出民法裡面的受扶養基礎的範圍的制度。在這裡面,稅法應該以這種面貌去跟民法的受扶養親屬的範圍去做結合。但很可惜的是,我們的所得稅法跟民法之間的結合關係,卻是以一個很怪異的方式去做結合。特別是婚姻制度的結合,這個部分我們強制合併申報。分居例外可以,分別申報,我們在696號之後的所得稅法修法,現在是可以分開計算申報。計算計算稅額,有三種可能性。這個當然更接近於個人所得課稅原則。但後面的繳納,這個就沒變。

合併繳納。什麼叫合併繳納?你們怎麼算怎麼算,不管,最後面就是一個人繳稅。最後面,這個人繳稅,就形成了一個問題。所有的所得,全部都這個人繳稅,那萬一這個當事人的婚姻狀態有變更,那怎麼辦?如果事隔多年之後,國稅局發現配偶有應報稅所得,沒有報稅,請問是命誰補繳?納稅義務人補繳。誰叫你時候要嫁他娶她。好,問題不是只有如此。裁罰呢?違反誠實申報協力義務的裁罰。我們後面還有裁罰呢。

誰漏報,理論上本來應該是處罰誰。但很抱歉,配偶不是納稅義務人。所以不是罰配偶,是罰納稅義務人。納稅義務人哦,既然被要合併申報,你就要好好地好好地仔細詢問一下,你的另外一半到底有沒有隱匿所得。沒有交代的話,將來的處罰,還是處罰你?

在德國,強制合併申報不存在,是任意合併申報。當他們兩個任意合併申報的時候,稅款的負擔,他們是一個連帶債務人,可是個別對自己的申報負擔誠實申報的義務。所以稅款確實會有連帶債務,由納稅義務人來幫另外一個納稅義務人負擔的可能性。但誰隱匿,誰就該受罰。不應該是配偶隱匿,卻讓納稅義務人受罰。這個叫個人罪責原則。這難道不是現代法律的基本精神嗎?個人罪責就是個人隱匿個人承擔責任,個人要被處罰。

所以,我們的稅捐法制完全沒有建構在以個人作為權利義務主體的基礎上。個人才是稅捐法上權利義務的歸屬主體。個人才是制裁法上的歸咎主體。要歸咎,也要歸咎隱匿課稅行為的那一方當事人。不能因為他是配偶,不是納稅義務人,就讓納稅義務人去替隱匿所得的這個配偶來負起責任。

這個是我們的整個婚姻合併申報制。到目前為止,696號之後,我們現行的法律規範狀態是,原則強制合併,例外法律規定分居的時候可以分開申報。分開申報的時候,原則上計算所得額有三種計算方式,大多數的納稅義務人基本上會選擇第三類的其他所得的計算方式。但是繳納,仍然還是所有人都歸到納稅義務人去做繳納。正因為所有都由納稅義務人去做繳納,當如果有其中配偶有漏報所得來源的時候,裁罰還是對納稅義務人去做裁罰。

我們目前的實務基本上一路都是這樣子操作。那我自己,特別是對財政部的所屬的公務員,我都跟他們一定要求,說今天合併繳納我動不了,但是裁罰,我一定要找出為什麼納稅義務人不知道這個事實,他有可歸究的事由,不可以直接因為配偶漏報就直接罰納稅義務人。不可以,你必須要有足夠的證據資料顯示,納稅義務人知悉資溪配偶有該項所得,但他卻沒有誠實申報。

像在臺中,有發生一個案例,那個案例是臺中以前一個很知名的政治人物,議長。議長在外面受賄。太太說,我不知道他在外面受賄。因為那一年報稅的時候,議長,有好好請教過稅捐顧問,叫太太當納稅義務人。根據太太自己的說明,常年在國外帶小孩,不知道先生在臺灣幹什麼事情,不知道,收多少錢,都不知道,全部都不知道。報稅的時候,太太是納稅義務人。先生是配偶。所以配偶在外面受賄。納稅義務人說,我不知道,不能罰我。不能罰你,那罰先生?也不是。因為先生不是納稅義務人。那罰納稅義務人,就是太太。太太主張,我是家庭主婦,我根本不問政治。我長年都在美國,都在照顧小孩,所以我不知道這件事。

國稅局就只好翻出來,太太幾年到幾年在臺灣待多少時間,平常在外面做哪些事情,一律調查。調查說,你怎麼會不知道收了幾千萬這些錢?賄賂,也是所得。再用這個方式,叫議長夫人,繳納、裁罰。

回過頭來,我就問各位,為什麼是議長夫人要繳納,為什麼是議長夫人要被罰?錢難道不是議長收的嗎?他們如果還維持婚姻,那就算了。如果他們已經沒有婚姻關係,你的補稅和裁罰的對象,不是錯了嗎?

所以我們的制度,我個人認為我們的婚姻合併申報制度是一個違憲法制。所以老師準備,我是有婚姻狀態的,報完稅以後,我就\ldots\ldots{}

但現在,我有一個有一個很大的心理障礙。因為我去提起訴訟啊,所以我一定會一路輸。啊,報紙就會報,臺大稅法教授啊,打稅法訴訟,一路輸!

(笑)

陳清秀老師當年就是這樣,釋字745號就這樣。國內知名稅法教授打稅法訴訟一路輸。你搞清楚了嗎?那是法律違憲,所以當然是要一路輸。而且我們還希望輸得快一點,輸得乾脆一點,不爭執,來啦,直接判我輸啦。因為我們要窮盡救濟程序,我們才能提起規範的憲法審查。所以只是這樣。

但各位同學如果要比老師快一點,我也很高興,歡迎各位同學,有這個可能性,要歷史上留名,這個機會留給大家。

這個法制,強制合併本身就違憲,因為他是一個稅捐資訊自主權的干預。

早期74年,更早之前,沒有那個分居解釋函令之前,實務上有更悲慘的案例。因為當你結婚之後,兩個人的稅捐資訊要合在一起對不對?所以你知道嗎?我現在去申報的時候,我可以直接就知道我另外一半的所得,我不用再經過他的同意哦。早期有人就透過這個資訊,來找到自己的另外一半有所得,在哪裡啊,就站到他工作的場所這邊跟他要錢。

那個個案,是一個被打到離開逃家的女生,被她先生就這個酒醉之後家暴,就逃出去,跑出去以後就到臺北橋下去打工。去打工以後,她每一年報稅,最怕的事情就是她先生。因為你不是要合併申報嗎?就會告訴國稅局,哪裡有所得來源啊。那個先生就跑去跟國稅局講,哎唷,這個,我們最近要報稅了,因為我想要誠實履行申報義務吼,我不知道我太太哪裡有所得,我怕漏了。所以他就用納稅義務人的名義去申請,他太太哪裡有所得,賺多少。下班後,他就跑去等太太,跟女生要錢。不給,就打。每一年報稅,等於是把這些你在哪裡工作,你賺多少錢,全部都主動告知給另外一半。所以我現在報稅的時候,我一用自然人憑證插進去,我都可以立刻拉出來,哦,哪裡有所得,一一對帳。而且不用再經過同意哦,就是去年有所得的資料,只要合併申報,用自然人憑證,立刻拉出來全部的資料。

請問這個怎麼沒有個資法違反問題。今年跟去年一定一樣嗎?受扶養親屬推定一樣?

這就是我們現在婚姻合併申報制的現況。

好,釋字696號以後,目前的現況,合併計算這個問題解決掉了一大半。合併申報這個部分還是存在著,分居這個情況才可以。沒有分居的話,基本上沒辦法分開申報。所以稅捐資訊仍然要洩漏給相對人。

合併繳納跟裁罰,這個情況問題依然沒有解決,因為我們的實務基本上都是對納稅義務人去做補繳跟裁罰。

我雖然可以盡力地要求稅捐稽徵機關必須要查明,納稅義務人確實對合併申報的配偶的所得,應該知悉,但往往還是會存在著個案裡面,如果其實沒有其他更多的證據資料知悉的話,我們其實都是不太能確定究竟納稅義務人是否明確知悉這一筆所得。這個,尤其是配偶,如果有不法行為的所得的話,我們不太能夠去確認納稅義務人是否知悉這個狀況。

\hypertarget{ux5bb6ux5eadux8ab2ux7a05ux5236}{%
\section{【家庭課稅制】}\label{ux5bb6ux5eadux8ab2ux7a05ux5236}}

我們進入家庭課稅制。在家庭課稅制裡面,各位都看得到,我們家庭成員可以有許多。這裡面我必須要先特別強調的,就是未成年子女。未成年子女,在法律上他是受納稅義務人扶養,因此他取得的所得,雖然有取得所得,可是只要他還不足以維繫生存的需求,一般而言,他一定必然會跟納稅義務人去做合併申報。

我們現行的制度,所得稅法第15條第一項的規定,這個部分也是強制合併的。未成年人的所得基本上是強制合併。除此以外的其他成年的直系血親尊親屬,成年的直系血親卑親屬,也就是成年的子女、旁系跟家長家屬,這一些,原則上,你只要成年,都可以選擇分開申報。

我們的法條規定雖然看起來是強制,但是事實上,成年人可以任意選擇分開申報。所以假設各位將來賺到錢,你的爸爸媽媽要不要跟你合併申報,你的爸爸媽媽可以自由決定。因為我們的法律規定,雖然看起來是應合併申報,但實際操作則是成年人可以選擇分開申報。所以你的合併申報這個事情是任意性的,在成年的這個部分。直系,成年了都可以分開來,更不用講旁系或是家長家屬之間的關係,一樣的道理,只要成年就可以分開來。

目前為止,只有未成年的子女被強制合併申報。當然,你如果是孫子女未成年,原則上你的未成年孫子女是歸屬給他的父母親一起合併申報,不一定會跟祖父。原則上是未成年子女會跟他的父母親合併申報。

在這一個強制合併的前提底下,我們現在來看,這個有沒有分開來計算的規定呢?沒有。因為我們的15條的規定,各類所得分開計算,只限於納稅義務人之配偶所取得的所得。因此未成年子女,假如有所得,理論上薪資所得還是可以分開來,但如果不是薪資所得的話,我們的未成年子女所取得的所得,原則上他沒有各類所得分開計算的可能。各位,你去看一下我們的各類所得分開計算,是就納稅義務人本人跟配偶。

所得稅法第15條,第二項第三款第一目,只有本人跟配偶可以各類所得分開計算稅額。所以第一目裡面只有本人跟配偶,然後第二目納稅義務人就前目跟他方受扶養親屬的各類所得,其他全部都是由納稅義務人合併計算稅額好。所以第二目規定,基本上未成年子女的,就回到由納稅義務人合併申報,因為分開來做各類所得計算的是第三款第一目的配偶的其他各類所得,並沒有受扶養親屬的其他各類所得。

薪資所得的部分,原則上是本人、配偶的薪資所得分開計算。也一樣,他就沒有其他的受扶養親屬的薪資所得,未成年子女的也沒有分開計算。

未成年子女的所得原則上就是全部合併計算,由納稅義務人一個人去做負擔。所以未成年子女的所得,全部都進入父或母的納稅義務人申報的所得裡面去。這個只要改變其課稅的級距,增加其累進稅率,就必然造成對家庭的稅捐負擔的加重問題。因為這個是邏輯上的必然。

納稅義務人跟未成年子女的所得,因為未成年子女的所得沒有分開計算的適用,是全部計入納稅義務人的所得裡面去,所以除非是同一個級距不變稅率,只要加進來變更級距,成為上面一個課稅的稅率的話,那這個時候一定會帶來對家庭的加重稅負的效果,因為這個是邏輯上的必然。在這個情況底下,目前696號解釋之後,如果未來還有面臨這個課稅制度要釋憲,或者是規範憲法審查申請的話,未成年子女的所得,加進納稅義務人所得裡面去做合併計算,因為要合併申報、合併計算、合併繳納、裁罰,這一路都是對納稅義務人作為家庭的家長,產生出來的加重稅負的效果。

這不僅影響個人的稅捐負擔的財產權,同時也是對家庭作為制度保障的一種侵害。這種侵害應該有違憲上的疑慮。

這個是我們對婚姻跟家庭制度課稅的制度。大致上,有比較高度違憲可能的是強制合併的部分。未成年子女跟父母之間的所得也是強制合併申報,這個部分涉及到稅捐資訊自主權的干預的問題。未成年子女的所得一定要合併計算,因此會對納稅義務人構成財產權的過度干預的問題。那么,未成年子女的所得併計之後,要求納稅義務人要去做繳納,甚至是做裁罰的部分。未成年子女如果有隱匿課稅所得,當然啦,實務上很難看到個案啊,但不乏這個個案可能。如果將來未成年子女,因為比如說在數位世界裡面有開放,甚至是參加比賽,有獲得很高的所得,這個時候加進父母親的所得裡面去,就會形成對家庭稅捐負擔加重的問題。

透過婚姻跟家庭的分別說明,大致上跟各位提到,在我們實務,這個法規範本身在適用上可能存在的違憲上的疑慮。至於不被強制合併申報的這些受扶養親屬,在實務上則產生另外一種怪異的現象。實際上並沒有受扶養,但是我如果在申報的時候,把其他受扶養親屬申報進來,那就可以實際上降低自己的稅負。在實務上因此產生一種怪異的現象,就是我實際上根本沒有扶養,但報稅的時候把你報為受扶養親屬。這種情形,在我們實務上,尤其是彼此之間不負法律上扶養義務的人,都有可能發生,因為在我們的所得稅法第17條裡面的受扶養親屬的範圍跟民法是不一樣的。

比如說所得稅法第17條的第一項第一款第三目的規定,納稅義務人及其配偶之同胞兄弟姐妹。我舉例而言,納稅義務人對配偶的同胞兄弟姐妹。納稅義務人對自己的兄弟姐妹,跟納稅義務人對配偶的同胞兄弟姐妹。納稅義務人對配偶的同胞兄弟姐妹,其實在法律上並沒有扶養義務。你會扶養你太太或者是你老公的兄弟姐妹嗎?你會扶養嗎?如果有,還真的給你鼓鼓掌。沒有法律扶養義務,可是我們報稅可以結合在一起。

沒有法律扶養義務,卻可以在申報稅捐的時候,可以合併申報成一個家戶。因此在實務上產生許多很怪異的現象。

怪異的現象是這樣,報完稅以後,那一些沒有實際受扶養的人,他可能是會去申請社會救助。然後去申請社會救助的時候,社會局拉出來資料就是,啊,你有受扶養啦,所以我不給你。因為你的有一個申報扶養的親屬,他的申報非常清楚,是適用20\%。這個時候依照社會救助法規定,既然有人扶養你,我就不給你。所以他那個時候就趕快去跟國稅局說,撤銷被扶養。

前面我說我被我的同胞兄弟姐妹們扶養,或是我作爲納稅義務人的配偶的兄弟姐妹受扶養,沒有,我實際上根本沒有被他扶養。本來先前同意的哦,同意合併申報,但之後呢,就說我沒有,因為我拿不到社會救助金。所以在實務上就常常會發生這種情況,那請問這種情況底下你怎麼去做查核?過去的一段時間裡面有沒有受扶養的事實,怎麼去查核?基本上也無從查核起,所以通常都是看那個所謂被扶養的受扶養親屬,他自己有沒有來申請撤銷,沒有申請撤銷,那我們就繼續讓他們可以去做受扶養親屬,一直到有一天,終於被撤銷或者是有檢舉。但是這個實際上都甚難查核。實務的操作,基本上一直都是在現實跟法律之間差異很大的狀態底下。

我們今天大致上跟各位介紹我們的婚姻跟家庭的合併申報制度。有強制合併,也有任意合併。成年的受扶養親屬,其實是任意合併。雖然法條規定你再看一次,寫因合併申報,但實際上的操作是任意合併申報。合併計算所得,只有配偶的部分薪資跟其他所得可以分開來。那其他的受扶養親屬原則上是合併計算所得。所以在這一些操縱組合之間,納稅義務人個別都可以去計算對自己最有利的稅負計算方式,來選擇要不要合併申報還是要分開來計算,這個都是變成納稅義務人可以自由選擇的一個範圍,這也是我們實務上的操作。越懂得稅法的人,理論上來講,就會獲得比較高的稅捐上的利益,越不懂的稅法,理論上來講,他就越不會獲得這些稅捐上的利益。但這正是一個作為一個強行法律規範稅法最大的問題所在。

懂得的人可以盡量拿去用,不懂的人卻受到稅捐負擔的不利益。這個是我想在這個地方最後面談到這個問題,給各位做一個說明。

那我們今天先講到這裡。下個禮拜,我們再來講,關於就源扣繳程序。

我們剩下最後兩個禮拜。

\hypertarget{section-25}{%
\chapter{20231204\_01}\label{section-25}}

\begin{longtable}[]{@{}l@{}}
\toprule()
\endhead
課程:1121所得稅法一 \\
日期:2023/12/04 \\
周次:14 \\
節次:1 \\
\bottomrule()
\end{longtable}

\hypertarget{ux5c31ux6e90ux6263ux7e73ux7a0bux5e8f}{%
\section{【就源扣繳程序】}\label{ux5c31ux6e90ux6263ux7e73ux7a0bux5e8f}}

在稅捐稽徵程序裡面,上個禮拜提到了關於婚姻跟家庭的合併申報制。那么我們今天進入另外一個在稅捐稽徵程序裡面,特別在個人綜合所得稅裡面扮演一個非常重要的角色的「就源扣繳程序」。

我們今天談完就源扣繳程序之後,我們再回過頭來來跟各位去談跟結算申報程序有關的相關的規定。

首先請各位翻開到所得稅法第88條跟第89條的規定。也是我們的就源扣繳程序的基礎規定所在的地方。

我們來看一下88條的第一項第一款的規定,納稅義務人有下列的這些所得的時候,原則上由扣繳義務人\ldots\ldots 這個是對一個基本權的主體,也就是他本質上本來就是一個基本權主體,他是一個第三人的角色,然後要求他作為一個行為義務人,在給付所得,有下列這些所得的時候,那麼原則上要依照法律所規定的扣繳率,或扣繳辦法來扣取稅款,並依照92條規定的方式去做繳納。

因此各位看到88條的第一項第一款,首先第一個,哪一類的所得呢?在第一款規定裡面是,當你公司分配給非中華民國境內居住之個人,以及總機構在中華民國境外營利事業之股利。簡單來講,就是你的股利所得的取得者,是非境內居住者或者他是非境內的營利事業,也就是總機構不在中華民國境內,在這裡面的股利所得。這個是第一款的第一分句。第二分句:「合作社、其他法人、合夥組織或獨資組織分配予非中華民國境內居住之社員、出資者、合夥人或獨資資本主之盈餘。」

簡單來講,第一款規定其實就對應只有一類所得,就是前面所得稅法第14條第一項第一類的營利所得。營利所得的股利盈餘的取得者,非境內居住者(自然人跟營利事業),你只要是非境內居住者取得股利盈餘的分配的話,原則上就是適用88條第一項第一款我們所規定的,由扣繳義務人在給付的時候,按照扣繳辦法的規定或扣繳率的規定,以92條的規定來去做繳納稅款的工作。

第一項第二款:「機關、團體、學校、事業、破產財團或執行業務者所給付之薪資、利息、租金、佣金、權利金、競技、競賽或機會中獎之獎金或給與、退休金、資遣費、退職金、離職金、終身俸、非屬保險給付之養老金、告發或檢舉獎金、結構型商品交易之所得、執行業務者之報酬,及給付在中華民國境內無固定營業場所或營業代理人之國外營利事業之所得。」

我們去對照第14條的各類所得。剛剛已經講過了第一類的股利盈餘。

我待會要去講,就源扣繳這個制度,存在的正當性跟必要性。但各位首先要先明白,我們現在法律規範的狀態是長什麼樣子?

各類所得,第一類的股利盈餘,第一款規定已經有談過了。接下來我們來看第二款的規定。

第一項第二款:「機關、團體、學校、事業、破產財團或執行業務者\ldots\ldots」,這是給予者的主體要件的限制,簡單來講就是,薪資所得並不是全部都要去做就源扣繳,特別是私人給付的薪資所得。舉例而言,你在家裡面請家教。這個並不是要做就源扣繳。儘管取得的人是薪資所得者,因為他是受到你的指示,在特定的時間地點去給付特定的勞務的內容,原則上按時間計算報酬。這個薪資所得,並不需要去做就源扣繳。薪資所得不是必然要就源扣繳,可是因為你只要是機關團體學校事業財團或執行業務者,你受僱於以下這些主體的話,才會適用就源扣繳。

這些主體,法條文字裡面,本身沒有上位概念去描述「機關、團體、學校、事業、破產財團或執行業務者」,每次講完都非常累。

這也是我們立法技術極差勁的一個原因。就因為他常常沒有上位概念,為了要講一句,本來應該是簡單的問題,一句話就可以交代。比如說境內居住者,境內居住者這個概念,其實在我們的所得稅法裡面並沒有很常出現。那麼德國法制裡面會用一個上位概念,就是無限制納稅義務人,或限制納稅人,就涵蓋了在法條規範技術上面所想要涵蓋的對象。可是我們的稅法的立法,立法者的技術,我不知道他們哪裡來的,為什麼每一次都要用很長的一句話去交代一個本來應該很簡單的概念。比如說總機構在中華民國境內之營利事業,哇,念完就覺得,好像要花很長的時間去講一個本來透過特定名詞去指稱即可的概念,

這個讓我也大概去聯想到我們民法的立法者,因為直接從國外的法律制度把抄過來,法律概念的用意就是透過這個名詞去定義,而且彼此之間我們就不用再去討論它的意義跟内涵。這也是法律在很大程度上不太能夠用白話文去表述一些重要意義內涵的原因。也就是他需要透過一些專有名詞去告訴你說,這個地方這個名詞代表某一個意義。透過這個意義的表述,雖然是門檻沒有錯,因為,法律本身確實是用我們的文字所堆砌起來的,可是不是白話文的用語,原因是因為在這裡面,有許多的意涵,如果沒有辦法去做約定成俗的一個表述的方式,你必須要有很長的文字去描述一個,應該要結構內容清楚的一個概念。這也是文字的表述本身,可能可以蘊含的一個價值,跟蘊含著人們在討論他的時候的一個相約成俗的概念。不然的話,我們的法律文字,尤其在現代,每一個領域都需要,如果要做高密度的立法,他本身就需要以精練的文字能夠去表述你的所想要表述的概念的內容。

雇主是一個事業組織的話,那你所獲得的薪資所得的報酬,88條會課予這個給付者的一個扣繳義務。但如果給付者是私人,薪資所得的給付,是沒有這樣子的一個扣繳義務。簡單來講就是私人給付的的薪資所得,就回到我們一般原則性的結算申報制度,由取得所得者他自己來做申報。

好,那回過頭來,「機關、團體、學校、事業、破產財團或執行業務者」,我姑且稱之為「事業組織」,這個只是老師給他一個概化的上位概念的名詞。

事業組織所給付的薪資所得,第三類所得,要適用就源扣繳。

第四類,利息。事業組織給出來的第四類利息所得。

事業組織給出來的第五類租金,不動產或動產產生出來的租金。

事業組織給出來的佣金\ldots\ldots 佣金的所得到底算哪一類啊?我們在14條裡面找不到。佣金是透過勞務的付出所取得,比較接近在民法上面委任或是行紀行為所產生出來的獨立勞務的付出的對價,所以比較接近是執行業務所得的類型,但法律並沒有在這裡面直接講執行業務所得。

簡單來講就是說,如果你是事業組織所給予的第二類執行業務所得,屬於佣金的話,他還會是一樣,要做就源扣繳。

那麼接下來是權利金,再回到我們剛剛講的分類,第五類裡面,租金跟權利金的部分。

那么接下來,競技競賽及機會中獎,第八類,一樣,要有一個主體要件的限制,他必須是事業組織所給出來的競技競賽及機會中獎。私人給予的競技競賽或機會中獎,私人自己舉辦的,並沒有就源扣繳義務的問題。

第九類啊,第九類也就是包含了這裡面非常長的一句話:「退休金、資遣費、退職金、離職金、終身俸、非屬保險給付之養老金」,對應的其實就前面的第九類的退職所得。

接下來在我們所得稅法第14條的第一項裡面第十類是其他所得。

接下來有告發或檢舉獎金,其實是第十類,是一時性的所得。雖然在實務上可能有些人就常常\ldots\ldots 但這基本上不是一個執業行為,應該是一個偶發性一次性的行為。在我們許多的行政法規範裡面,有一些檢舉獎金的設置哦,這一類的檢舉獎金的設置,原則上是一次性的所得啊,除非你以之為業哈。這個很難想像,但也不是說真的很難想像,因為在臺灣的文化裡面有「檢舉達人」,這個就可能以之為業。他就到處建築工地,他就去拍,因為只要塵土飛揚啊,水很髒的話,常常就會有人去拍拍拍,然後就一直寄寄寄,寄到環保單位、道路交通、建管、勞動主管機關。這一類的東西到處都有檢舉。財稅,也有啊。財稅也有檢舉獎金。

告發或檢舉獎金,原則上是一次性的。假設,你還真的能夠以之為業,因為這個特殊的制度,以之為業,其實不妨礙變成是一個營利事業的可能啊,有可能啦。

我以前在德國念書的時候,我們老師就講過一個非常特殊的行業,是做賭博的職業玩家。在德國,原則上,這種賭博性的收入不算所得。因為這一種十賭九輸的射悻性很高的特性的關係,德國是自始把他從所得課稅的範圍排除。

有繼續性的這個特徵,才會產生出一個所得的概念。那這種射悻性很高的賭博的收入,德國自始上稱之為叫嗜好,有個概念叫Liebhaberei,意思就是這是一個人的偏好,特殊嗜好。所以你喜歡去拉霸的,德國基本上不會把你拉霸的收入當作所得課稅。當然,你有所得,不課稅,但你有虧損就不準減除。對德國人來講,有正所得就要課稅,但有虧損也要足以減除。所以他們從概念上是自始把那一種射悻性非常高的這一種賭博性的行為,就把它排除出去,完全不當作是所得的類型。這是德國人的概念。

臺灣,我們的所得稅法並不是這樣,這種一次性的所得,還是會透過一時貿易盈餘機會中獎,第八類以及第十類,包括賭博收入,是屬於這個第十類的其他所得。從而我們的法律環境確實跟德國不太一樣,我們反而比較接近美國。因為在美國,賭博的收入,還是會當作所得來課稅,可是虧損減止零而已。

這個是,跟我們的法律的規範不太一樣的地方。

德國,對於職業玩家,因為職業玩家是可以計算他的輸贏的。所以,在西方的社會裡面有一些職業的牌手,在德國,這個可以算營利所得,他可以自己計算盈餘虧損,這個還是可以賺錢的哦。如果去賽馬或賭馬的,是賭博行為,但如果是以之為業的話,在德國還是構成了所得稅的課稅的對象。

回到我們這裡,告發跟檢舉獎金是第十類的所得。只是88條這邊把這個告發跟檢舉獎金獨立出來。

結構型商品交易之所得。結構型商品交易所得從哪來?結構型商品,意思是,他本身是一個衍生性金融商品,是一個結構型的債券,本身是財產交易所得的一種類型,也就是我們前面的第14條的第七類,財產交易所得。

接下來,執行業務者之報酬。要特別注意,並不是所有的執行業務者的報酬,是機關團體學校事業組織等,這種事業組織型態所給予的執行業務者報酬,才會是屬於要做就源扣繳。所以,換言之,如果是私人聘請執行業務者,比如說私人聘請律師,是不需要做就源扣繳的。因為在法制上認為,私人比較沒有能力去適用比較複雜的就源扣繳制度。這也是第二款規定,我們只抓事業組織,做就源扣繳義務人的這個規定的正當性所在。

因此我們大概統合一下。

第一類,要。

第二類,不一定,因為要看是不是有涉及到業務報酬。

第三類,薪資所得,只要是機關團體學校事業組織給的薪資所得,要。

第四類,利息所得。

第五類的租金權利金,以及第八類跟第九類第十類裡面的告發及檢舉獎金。

第七類的結構型商品交易之所得,以及第二類,機關團體學校事業組織給予執行業務者之報酬。

農林所得那一類。在前面8條規定,有跟各位講過農林所得,本身是限於大自然界第一次產出,所以本身不會有就源扣繳的問題。因為是第一次產出,我就是從土地山川河流海洋,第一次產出取得的,這個都是我們在所得稅法第8條裡面所謂的農林漁牧礦,不會有就有扣繳問題。

因此,這個條文規定,其實換句話講,只要是機關團體事業組織給出來的所得,基本上在臺灣是全面性地適用就源扣繳程序。老師要得出的是這個結論。當然,仔細的適用,確實還是要回到88條的第一項跟第二款的規定,但這個兩款規定其實告訴你,臺灣的就源扣繳程序基本上是全面涵蓋,各類所得,都有就源扣繳問題。你只要是事業組織給出來的各類所得,原則上就會有就源扣繳程序。

我等下會講到,這個是有歷史背景的。有歷史背景,才會這樣一個非常廣泛的就源扣繳程序。

相對於德國,德國的就源扣繳程序,其實只有兩類所得,一個是利息所得,一個是薪資所得。德國的就源扣繳程序是有限的所得類型。

在臺灣,第一款規定是股利盈餘,第一類所得。第二款規定幾乎涵蓋了所有除了農林所得以外的各種類型的所得,只要你是事業組織給出來的所得,是涵蓋範圍極為廣泛的就源扣繳程序的所得。

那么我們最後面,再來談,88條第一項第二款,「及給付在中華民國境內無固定營業場所或營業代理人之國外營利事業之所得。」

這個規定,其實可以跟前面的第一款規定結合在一起看,原則上是所得者,不管你是取得股利盈餘,還是你給付給境外營利事業的所得,當然這裡面有一個構成要件是,你在中華民國境內沒有固定營業場所或營業代理人的國外營利事業。

就是,你付錢給他,這個所得的取得者,他本身是國外營利事業,但他有一個要件是,在中華民國境內沒有固定營業場所或營業代理人。

這個條文,請各位跟41條的規定,結合在一起看一下啊,因為在這個地方,我們的法條規範的技術極為繁瑣,而且那個概念,看單一這個法條不太容易清楚。

我們來看一下所得稅法第41條的規定:「營利事業之總機構在中華民國境外,其在中華民國境內之固定營業場所或營業代理人,應單獨設立帳簿,並計算其營利事業所得額課稅。」

國外營利事業在中華民國境內有一個固定營業場所或是營業代理人的時候,

第41條的規定可以告訴你,為什麼法條在88條第一項第二款裡面這樣規定。

「營利事業之總機構在中華民國境外」,這個就是88條講的「外國營利事業」。所謂的外國營利事業,在我們的稅法裡面,一下子用「總機構在中華民國境外」,一下子用「外國營利事業」,這兩個概念,立法者自己不會覺得自己有用詞不太一樣,他不會覺得哦。

這個就是外國營利事業。外國營利事業在中華民國境內有固定營業場所或營業代理人的時候,根據41條的規定,他要單獨設立帳簿,並計算營利事業所得額課稅,簡單來講,就是一個外國營利事業,只要在中華民國境內設有,第一個是固定營業場所,第二個,或者你有營業代理人,那這一個在境內的固定營業場所或營業代理人,他就變成了,根據41條的規定,要獨立設置帳簿,申報營所稅。

所以他雖然在法人格上,他不是一個獨立的法人格,可是在稅法上,他是一個獨立的稅捐申報主體。這個外國營利事業在我國境內之固定營業場所或營業代理人。在法人格上,他不是獨立的法人格,也就是說,他還沒有到子公司這個程度,他沒有獨立法人格,他只是一個分公司、固定營業場所、辦公場所,或者他在境內設有一個代理人,代理人沒有非得必然要有一個固定辦公的地方。

第10條的第一項是固定場所的規定。第10條的第二項是營業代理人的規定。那這兩個條文,我們到營利事業所得稅的時候,會再跟各位去詳細談,在稅捐法上的權利義務。這兩個是獨立的具有稅捐負擔能力的營利事業。也就是外國營利事業,設在中華民國境內的,只要你構成了固定營業場所,跟營業代理人,符合第10條第一項或第二項的這個要件的時候,他本身就是一個獨立的稅捐申報主體。

因此,稅法並沒有全然遵守民法上或者是公司法上的要件的規定。也就是說,稅法他自己本身認為,在這個地方他是一個獨立經濟主體,他已經有辦法自己去做一個稅捐申報。

所以,回過頭來,我們的再來看一下剛剛我們念的88條的第一項第二款的後段規定。當你付錢給一個外國營利事業,如果他在中華民國境內有固定營業場所或營業代理人,因為你付錢給他,拿錢的這個主體,雖然他在法人格上,不是獨立的法人格的主體,可是根據所得稅法的規定,他是一個獨立的經濟主體。他是可以有稅捐申報能力的,根據41條的規定,我們是要求他要獨立設立帳簿來做營所稅申報。

所以,為什么剛剛老師會講說,其實第二款後段規定跟第一款規定的背後的規範意旨都是相同的,就是,當你付錢給一個境外所得者,沒辦法,我只能做就源扣繳。

第一款的規定是,你付給境外所得者股利盈餘,第二款的最後那一個半段的那個規定,他是付錢給境外營利事業。只要你付錢給境外營利事業,他沒辦法去做獨立的結算申報的話,你要付錢給他,你就必須要就源扣繳,不管你是不是事業組織。

也就是說在這裡面,我們對第88條的第一項第二款的規定,特別是在你給付給境外的非境內居住者,他是一個所得者,他是取得中華民國來源之所得的時候,第一款規定股利盈餘,第二款的後段的這一個對外所做的給付,原則上必須要用就源扣繳的程序,不然我們在現實上沒有辦法再去對一個境外所得者,去要求他再回過頭來,去做所得稅的結算申報。

就源扣繳程序,特別是在對境外所得者的給付裡面,體現了 pay as you earn 的想法。賺到錢的時候,那個給錢的人就直接幫你扣繳下來,這個就源扣繳,在非境內居住者的課稅裡面,是一個在國際之間基本上被承認的一種課稅手段。因為你沒辦法回過頭來期待非境內居住者要去做結算申報制度。

這個第一款跟第二款後半段背後的法律上的規範的正當性依據在於,對境外所得者的這個課程程序,原則上你沒辦法讓所得者自己去做結算申報,必須要仰賴給錢的人站在所得的源頭,pay as you earn,PAYE。就源扣繳程序,一般來講就是所謂的PAYE制度,你在賺到錢的時候付錢。另外一個是PAYG,就是pay as you go。就是當你離開的時候,你要棄籍,要到國外去的時候,離境的時候,也會要求一個結算申報的這個制度

我們待會去講結算申報制度的71條的相關規定的時候,我們再跟各位去做說明。

回到88條的第一項第二款的後段,跟第一項第一款規定,基本上,取得所得者是非境內居住者,取得中華民國來源所得的這種情形底下,適用所謂的就源扣繳程序的規定。

88條的第一項,第三款跟第四款規定,都是對應營利事業所得稅裡面的第25條跟第26條。

第25條是,跨國際活動的海運跟空運,這種運輸事業;26條則是跨國際活動的影片事業,像好萊塢的影片業,這種國外的營利事業,在臺灣所取得的所得課稅的方式。

25跟26條的規定,我們一樣在營利事業所得稅裡面再去跟各位去講啊。本質上,這兩類的所得都不太容易去拆分他的收入跟成本費用,由於他本身是一個國外的營利事業或國外的海空運輸事業,那基本上都會比較偏向由船籍國,來去做管轄,因此原則上只有船籍國才會取得相關的財務申報的資料。那也因此大多數的非船籍國的國家,通常都是用比例的方式去做課稅。這也是25條跟26條,本身有一些營利事業所得裡面的特殊背景。這個我們留待到營利事業所得稅裡面去做進一步說明。

我們,回過頭來,在個人綜合所得的這個規範裡面,88條的第一款第一項第一款規定跟88條第一項第二款的規定,各類型所得,十種類型的所得,其實都被涵蓋在內,有適用就源扣繳程序。這是一個非常廣泛的就源扣繳程序的規定。原則上是你只要是事業組織所給出來的所得的話,不管是執行業務者所獲所得或是薪資所得,基本上都會要適用就源扣繳程序。

88條的第二項也還是涉及到獨資跟合夥組織的營利事業:「有應分配予非中華民國境內居住之獨資資本主或合夥組織合夥人之盈餘者」。

獨資合夥的營利事業在你分配,就是辦理結算申報或清算申報有應分配予非中華民國境內居住之獨資資本主或合夥事業之合夥人的盈餘的時候,「應於該年度結算申報或決算、清算申報法定截止日前,由扣繳義務人依規定之扣繳率扣取稅款,並依第九十二條規定繳納」。

第二項規定其實也是88條第一項第一款規定的延伸,只是他是在結算申報或決算或清算申報裡面,再另外獨立出來去做一個規定而已。其實他的重點仍然是在,你取得股利盈餘的所得者是非境內居住者,那取得的所得,我們在88條第一項第一款的規定裡面就有規定,這個本來就是要做就源扣繳。第二項規定,因此主要是適用在這個清算決算,因為公司你這個營利事業組織,不管是獨資或合夥,你要做清算,大家要拆夥了。第二項規定是第一項第一款規定的延伸而已。總而言之,你只要是對於境外居住者給錢,在給錢的時候就要適用就源扣繳程序的規定。

第三項的規定,仍然是針對前面的獨資跟合夥商號,而更進一步去做,營利事業所得稅申報核定增加的所得,這條規定其實是為了配合71條第二項。

我們在民國107年,兩稅合一制度開始分拆以後,特別是針對獨資跟合夥,我們設了這個營利事業組織的型態,在課稅的形態上面,實體法嘗試穿透課稅,可是申報制度裡面我們卻做了蠻複雜的規定,相當複雜的規定。這個地方為了配合第71條對獨資跟合夥組織之營利事業,做對應的規範。

獨資跟合夥,原則上他在營所稅的結算申報,他做了申報,但他是不用再報繳所得稅的,所以他沒有稅款的繳納這個程序。因此我們在第三項的規定裡面去做了一個進一步的規定。直接就是,當你分配給不是境內居住者這個人的時候,原則上就是在分配給他錢的時候,一樣還是要去做就源扣繳程序的適用。這個是我們的第88條的第三項,一樣延伸了前面的第一項的第一款跟第二項的規定,重點仍然還是在獨資跟合夥的資本主,或者是合夥事業的合夥人,只要他不是我們的境內的個人居住者或是境內的營利事業的時候,都是要做就扣繳程序的適用。

第88條第四項是關於授權規定的條文,我們這個地方就暫且打住。

第88條,不看法條,是常常會對類型不太容易去做仔細分辨的一個條文。我自己也是。基本上,我們的就源扣繳程序是各類所得都有適用,但具體哪些類型裡面的所得,我也是要看法條。我也沒辦法用背的,因為我認為這種條文不需要背,一點邏輯都沒有了,不需要背。要背的話,不如背樂透,哪個號碼常開出來,那個我覺得還比較有意思一點。像這種條文,不需要背,各位就是看法條多看幾遍就好。真的碰到實務案例,就是法條多看幾遍。

我們國家考試也是這個樣子。國家考試雖然考所得稅法,但基本上都是法條的運用,所以。對各位來講就是法條熟悉,就這樣好了,不需要背,一個條文都不需要背,這個我可以明確告訴各位。

我跟各位也說過,以前有聽我上課的同學都會聽過,我跟各位講過,我民法,我也不會背呀,有人在背民法嗎?我倒是有背過一條:因故意或過失,不法侵害他人之權利者,負損害賠償責任。我會背這一條,是因為我在當兵的時候,我要讓人家相信我是法律系的呀。因為我們資歷不夠深啦,我們只是考上預官當個預官,人家誰理你啊。

人家説,哎,那個排長,你們都是法律系畢業的,哇,那法條很會背哦。

我說,對呀,我六法全書都背下來。

我就法典丟給他(丟小六法):你隨便考,哪一條我都背給你聼。

哇,真的還假的啊。

不然我背一條給你看,你翻翻看啊。那個,民法184條,因故意或不實不法侵害他的權利者,應負損害賠償責任。

他不會,我就翻給他看,讓他看到,哇,你真的都背下來。

不信的話,我再背另外一條,那個不當得利那個條。我為了保險,我還再把無因管理背下來,就這樣,我就只會這三條。

就這樣,不用一點方式,那個部隊裡面,因為我們都是義務役的士兵啊,那他為什麼憑什麼要服從呢。就是說,你看你不要做什麼壞事,我法條都可以背,你幹甚麼壞事我都會記得的。就是這樣。好,這個只是純粹花招而已。

法律從來不是在背,法律是在考運用、組合、排列、邏輯分析、邏輯歸納。我們法律,因此是一個歸納分析運用之學,透過你學到的這一些法條規範告訴你的邏輯,告訴你價值上的判斷,在個案裡面去做出一個最適合的一個判斷。

所以88條,背後有一個邏輯。第一個邏輯,境外居住者拿到中華民國來源所得,就源扣繳。第二個,在臺灣的所得稅裡面,原則上只要是機關團體學校事業執行業務者給出去的所得,原則上各類所得全部都要做就源扣繳。

就這個部分,待會兒我們會去跟各位去講,這個有他的歷史背景,我們大法官也做過解釋啊,對他的看法如何,他在法律上的地位,我們待會再進一步去看。因為我對各位的要求就是,不光只是知道 why,為什麼,你要先知道法律長這個樣子。

我順帶跟各位提到,以往老師在改,不管是哪一種類型的考試,我們法律系的同學們常常有一個毛病,就是法條都沒看清楚,就在那邊講這個法條規定違憲。我每次看到這個地方的時候,我心裡面都一個 OS,就是,那你知道現在法條這個樣子,到底怎麼解釋操作,能夠在不違憲的前提底下,能夠盡量讓他做合憲性的解釋的適用,你知道什么樣子叫合憲性的解釋適用嗎?

我們很多同學動不動就說,一個法條規定違憲,所以行政處分違法。老師每次看到這裡面,我都會覺得,行政處分的違法性,你要先看法條本身的解釋適用,他如果不符合法律規範,這個行政處分固然是違法。可是如果行政處分確實是根據法律的規定,你認為他違憲,沒錯,但他是根據這個法律規定作成的話,這一件事情,由於他本身符合法律的規定,而且這裡也正是立法者所做規定的問題。因此,我們接下來其實在行政救濟程序裡面,你沒有辦法去變更這個法律違憲的狀態。

在一個法治國家,你必須要透過,要不是受理事件的行政法院法官,他去提起憲法上法律規階法規範的憲法審查,不然的話,就是人民要窮盡救濟程序之後,來提起憲法審查,你怎麼那麼快就跳到違憲層次的問題?每一次在問行政處分有沒有違法,都還沒談到符合法律規定的解釋的狀態,行政處分是否違法的時候,就一下子就跳到規範的違憲。老師都覺得非常的憂心,因為這樣你去打訴訟,難怪會一直打輸。不是這樣,真的不是這樣。有些訴訟,你是必敗的,你真的會敗,因為問題不是出在行政處分,問題是出在法律。你是必敗的,因為在這種法律維持現狀的情況下,你沒有去做規範的違憲審查,那個法律就是如此的不義。

當然啦,我今天跟各位講的例子是,連大法官都說他合憲,那我還真的有點沒輒。

因此我們在接下來的這個說明裡面,各位你先了解現況長的樣子。

根據我們第88條的規定,第88條第一項第一款第二項第三項規定,跟第二款的後半段規定,基本上歸納出來的一個類型的所得就是非境內居住者取得中華民國來源所得,股利所得,股利盈餘,第一款規定;其他的所得在第二款後段規定。二項三項的規定,這個地方延伸著非境內居住的,取得我國來源所得。這個是第一個要做PAYE,也就是這個就源扣繳程序的一個類型。

然後另外第二類就是機關團體學校事業,所給予的這一種各類型的所得,我們剛剛做了一次的總覽。當然各類所得裡面並不是全部,特別是比如說第十類其他所得,裡面只有告發檢舉獎金而已,沒有其他,比如說賭博收入。賭博的收入,不會這樣叫莊家去做就源扣繳程序吧,這個不太可能啊。像職業賽鴿,你不會叫舉辦單位,辦職業賽鴿的那個人,說,欸,我付錢給你的時候,我直接做就源扣繳。都是做不法行為了,他怎麼還會去做這個就源扣繳程序呢?如果有就源扣繳一定是詐騙啊,他們自己騙自己,就是說,哎,這個根據法律規定,合法非法都是要報繳所得稅,所以我也來做就源扣繳程序啊。絕對不可能有這樣一個情況。

第二種就是只要機關團體學校事業給的各類所得,基本上都要做就源扣繳規定。

接下來我們來看第89條的規定。承接這個88條是關於客體,也就是適用PAYE這一種就源扣繳程序的稅捐客體,所得的類型。

\hypertarget{ux6263ux7e73ux7fa9ux52d9ux4eba}{%
\section{【扣繳義務人】}\label{ux6263ux7e73ux7fa9ux52d9ux4eba}}

之後,就是扣繳的主體,扣繳行為義務的主體,在第89條的規定裡面。

好,我們來看89條的第一項第一款規定,很長,每次看到這裡,我就有點昏了。

「公司分配予非中華民國境內居住之個人及總機構在中華民國境外之營利事業之股利;合作社分配予非中華民國境內居住之社員之盈餘;其他法人分配予非中華民國境內居住之出資者之盈餘;獨資、合夥組織之營利事業分配或應分配予非中華民國境內居住之獨資資本主或合夥組織合夥人之盈餘\ldots\ldots」

我不太知道為什麼你不就直接規定,前條第一款規定之所得,其扣繳義務人是誰,這樣就好了。我不太知道為什麼當初他們不這樣規定啊。

我念了那麼多,89條第一項第一款就對應88條第一項第一款。88條是客體的規定,89條是主體的規定,也就是扣繳義務的主體,行為主體的規定。

我們來看一下扣繳主體是誰。扣繳義務人,也就是負責人。第一分句,「其扣繳義務人為公司、合作社、其他法人、獨資組織或合夥組織負責人」。

我請各位同學把「組織負責人」畫起來,他就是以這個自然人作為扣繳義務人。不管你今天分配股利,你是什麼樣一個組織型態,是公司合作社有限合夥,或者是獨資或合夥,總而言之就是組織負責人。

第二分句,「納稅義務人為非中華民國境內居住之個人股東、總機構在中華民國境外之營利事業股東、非中華民國境內居住之社員、出資者、合夥組織合夥人或獨資資本主。」

總而言之,就是你的股東非境內居住者,這個是納稅義務人。扣繳義務人,則是給錢的組織的負責人。89條是以自然人作為扣繳義務人,那麼所得的取得者當然不一定是自然人,因為也有可能是非境內的營利事業。

第二款規定一樣,直接講看:「其扣繳義務人為機關、團體、學校之責應扣繳單位主管、事業負責人、破產財團之破產管理人及執行業務者」。

機關、團體、學校之「責應扣繳單位主管」,請各位畫起來。事業一樣,就負責人。破產財團則是破產管理人。執行業務者,執行業務者給錢的時候一樣,也有可能要去適用做就源扣繳,執行業務者在這個地方也是扣繳義務人。

納稅義務人為取得所得者。

第三款跟第四款規定,當然對應的也是前面的88條的第三款跟第四款。所以原則上我們在申報稅捐的時候,納稅義務人則是外國營利事業,或者是第四款規定,講得國外影片事業,扣繳義務人則是營業代理人,或給付所得之給付人。這個是對應的關係。

因此這個89條設定了扣繳行為義務的主體,原則上並不是直接針對營利事業本身,也就是付錢的這個營利事業本身,而是事業組織私底下,我們分兩種類型,一個是機關團體學校,用一個叫「責應扣繳單位主管」的一個名字,至少在法條規範層次裡面,不太知道到底是在講誰。

但可以很明確的,你知道他並不是要找機關團體學校的負責人,他並沒有這個意思。就是說這個法條規定,其實你看的時候你不太清楚他想要找誰。因為這個地方要看法條的立法背景,有一個立法過程,才會產生這個「責應扣繳單位主管」,因為在88年之前,他叫做「會計單位主管」。

89條的第一項第二款規定,88年以前他叫這些機關團體學校的「會計單位主管」。那88年的時候發生了一個個案事件以後,就改成叫「責應扣繳單位主管」。對於事業,還是負責人。扣繳義務人是這一些自然人,納稅義務人則是取得所得者。這個就是我們的PAYE,也就是就源扣繳程序裡面,我們現在法規範的狀態。

扣繳義務人,分成兩個,一個是機關、團體、學校之「責應扣繳單位主管」,那另外如果是事業的話,是負責人。因此在這個地方,事業付錢給境外的非境內居住者的這個外國營利事業,或是境外的自然人的話,你給錢的時候,就是由事業負責人來負責做就源扣繳程序。

好,這個就是我們現況的規定。

接下來我們再來跟各位去談這樣一個條文規定,我們現行的法制的狀態,我們大法官怎樣看待,以及對這樣一個法法治的狀態,我自己個人認為到底有沒有正當性的問題,這個就真的是在講合憲性與否的問題。

因為我們法條是長這個樣子,所以我們現在實務上有大量的行政訴訟的事件,是被認為是事業負責人,也就是我開一家公司,我付錢給國外的營利事業,比如說我付錢給臉書給微軟。我付錢給國外,誰是扣繳的義務人啊,是那個付錢的營利事業的負責人。不是營利事業本身哦,是負責人,因為根據第88條的第一項第二款的規定,你付錢給境外的營利事業啊,原則上就要做就源扣繳程序。扣繳義務人呢?根據89條的第一項第二款規定,扣繳義務人是事業負責人。那你要給股利給國外的股東,假設臺積電的投資有來自於美國的這個艾斯摩爾,假設這樣,美國艾斯摩爾投資了臺灣的臺積電。當他每一年發放股利盈餘的時候,一樣,也是要做就源扣繳程序,因為根據第88條的第一項第一款規定,這個股利盈餘只要給付給境外的營利事業,那么就必須要去做就源扣繳。那扣繳義務人是誰呢?是89條的第一項裡面所講的事業負責人,也就是現在臺積電的董事長劉德音,或者是以前的張忠謀先生,不是臺積電本身。

這就是我們現在的法律規定的狀態。那當然在臺灣基本上不太會有人去挑戰這樣一個法秩序,都認為這個法律就長這個樣子,是因為我們背後曾經有過大法官解釋,認為這個是一個合憲的一個制度。

我們先休息一下,待會再來跟各位去談這樣子的一個條文規定,到底合憲與否,大法官的看法又是如何。

\hypertarget{section-26}{%
\chapter{20231204\_02}\label{section-26}}

\begin{longtable}[]{@{}l@{}}
\toprule()
\endhead
課程:1121所得稅法一 \\
日期:2023/12/04 \\
周次:14 \\
節次:2 \\
\bottomrule()
\end{longtable}

剛剛看到助教太忙了,非常謝謝助教,因為待會在隔壁那個1301的教室裡面,有一個「稅捐法中的基本權保護」的研討會,那如果各位同學時間上允許的話,也歡迎各位參加。主要是我們臺灣財稅法中心辦的活動,想說年底之前,趕快寫個議題去做討論。那個比較偏向,就是剛剛我跟各位談到的所謂的憲法審查的部分。

老師,這些年的研究,都會發現,其實很早就有這個想法了,就是說,通常問題都是出在立法的問題。立法問題,在法治國家,我們只有一個憲法審查的機關。其他機關其實不是不能做憲法審查,其實其他機關可以,特別是行政法院一定可以去做憲法審查,只是他不能得出違憲的結論,拒絕適用。也就是說,行政法院在解釋上,其實要盡可能把那個法規範在文義可能的範圍內,往合憲的方向去做解釋。這是法院的職責所在。所以,我在那個《稅捐法秩序》裡面就談到說,其實行政法院不是沒有做憲法解釋,每一個行政法院案件都是在做憲法的具體解釋,只是由於行政法院在這個權力分立的結構之下,至少在法院組織裡面,本身並沒有違憲宣告的權利,他不能拒絕一個法規範的適用,他必須要停下來,讓憲法法庭去做違憲的宣告。因此才會說,行政法院,當他對所適用的法律位階法規範具有違憲的確信的時候,那原則上他就要依憲法訴訟法11條規定,去裁定停止具體個案當中正在進行的訴訟程序,然後提到憲法法庭那個地方去做法律位階法規範的憲法審查。

這一塊算是我個人認為,在稅法裡面,你要回到憲法法秩序的這樣一個非常重要的路徑。這也是我很一直期待能夠在我們法律人的這個稅捐法的課程裡面能夠做的。不是說會計師不行,而是這個本來就是法律人的角色,要特別去著重到這樣一個法規範的正當性。那會計師們的訓練本質上就不是在這個方面,這個真的不是會計的任務,這個是法律的任務。

\hypertarget{ux6263ux7e73ux7fa9ux52d9ux7684ux5167ux5bb9}{%
\section{【扣繳義務的內容】}\label{ux6263ux7e73ux7fa9ux52d9ux7684ux5167ux5bb9}}

除了88條跟89條規定以外,還有第94條的規定是蠻重要的,是扣繳行為義務的內容。也就是說,剛剛的這個PAYE的扣繳程序,有適用的客體對象的設定,也有適用的行為義務的主體的設定,在第88、89條的規定。那我們來看一下第94條,這樣我們才能夠對於PAYE的制度,做比較能夠全面性的了解。

第94條的規定:「扣繳義務人於扣繳稅款時,應隨時通知納稅義務人,並依第九十二條之規定,填具扣繳憑單,發給納稅義務人。如原扣稅額與稽徵機關核定稅額不符時,扣繳義務人於繳納稅款後,應將溢扣之款,退還納稅義務人。不足之數,由扣繳義務人補繳,但扣繳義務人得向納稅義務人追償之。」

第94條規定的內容,其實規定了,就源扣繳的行為義務的內容。根據94條的規定,行為義務的內容有這一些

我們又回過頭來看92條,但在這裡面,我們把扣繳義務的內容做一個比較細的切割。

我們來看一下92條的第一項規定:「第八十八條各類所得稅款之扣繳義務人,應於每月十日前將上一月內所扣稅款向國庫繳清,並於每年一月底前將上一年內扣繳各納稅義務人之稅款數額,開具扣繳憑單,彙報該管稽徵機關查核;並應於二月十日前將扣繳憑單填發納稅義務人。每年一月遇連續三日以上國定假日者,扣繳憑單彙報期間延長至二月五日止,扣繳憑單填發期間延長至二月十五日止。但營利事業有解散、廢止、合併或轉讓,或機關、團體裁撤、變更時,扣繳義務人應隨時就已扣繳稅款數額,填發扣繳憑單,並於十日內向該管稽徵機關辦理申報。」

「每月十日前」,我們扣繳義務是一個月一次喔,蠻頻繁,密集的。

92條的第一項規定還在講扣繳憑單,但扣繳行為義務的內容,其實嚴格來講,可以切分成以下這幾個階段:扣取、繳納、申報、填發扣繳憑單

第一個叫扣取。就是付錢給納稅義務人的時候,要做一個事實上的行為,這個叫做扣取行為。

第二個,繳納。一樣,其實這也是一個事實行為,就是他去向該管的稅捐稽徵機關去做稅款的付。

第三個,申報行為。申報,是一個具有對外發生法律效力的一個法律上的行為。

第四個,第92條跟第94條所共同建構起來的,叫做填發扣繳憑單。填發扣繳憑單有兩個對象主體,一個是納稅義務人,一個是該管稅捐稽徵機關。那個扣繳憑單,不是單純只有給該管機關而已,還要給扣繳義務人。因為根據94條的規定,其實他是要隨時通知納稅義務人。

所以,我們的扣繳義務的內容,包括了扣取、繳納、申報、填發。填發,有兩個對象,受領這個扣獎憑單的對象主體,有兩個,該管稅捐機關跟納稅義務人。這個是我們從92條跟94條我們共同建構起來扣繳行為義務的內容是如此,包括了一些事實上的行為,包括了具有告知相關稅款以及讓權利義務之歸屬主體的納稅人跟稅捐接機關都能夠去知悉,究竟這個錢是算誰的錢,我是幫誰來繳納這一筆稅款。該管稅捐稽徵機關,在你申報出你的納稅人之後,他才有辦法去做相互勾稽。因為接下來就是國稅局他們內部裡面去做作業,讓你這個納稅人在他年底之後5月份去做結算申報程序的時候,能夠知道這筆錢歸誰,誰享有可扣抵稅額的扣抵權。

這是我們第92條跟94條所建構起來的扣繳行為義務的內容。

\hypertarget{ux6263ux7e73ux7fa9ux52d9ux4ebaux7684ux88dcux7e73}{%
\section{【扣繳義務人的補繳】}\label{ux6263ux7e73ux7fa9ux52d9ux4ebaux7684ux88dcux7e73}}

回到第94條的規定,我剛剛念了一段話,就是當你扣太多的時候,你要把溢扣之稅款還給納稅人。但重點在後面那一句話,如果你扣不足額的時候,是由扣繳義務人去「補繳」稅款,再由扣繳義務人向納稅義務人去做追償。我們在稅法總論裡面,我們談到稅捐債務法,有講責任債務人。這個就責任債務的一種表現。扣繳義務人本身並不是納稅義務人,其實不是依法律規定所產生稅法上權利義務之歸屬主體,他是以第三者的身份進來,變成是在這裡面第94條規定的補繳的主體,從而他是一個德文稱之的叫Haftung,中文其實叫責任。但是我們沒有在制定法裡面說,這個是責任,而是直接用補繳這個概念。

我再講一次,責任這個概念是學理上引用德國法去做延伸的闡明或解釋。這是一個責任,所謂責任的意思,他跟納稅人是處於一個兩個債權債務關係,都為了滿足一個國家的稅捐債權。所以我們把這個扣繳義務人的補繳責任,稱之為叫責任債務。因為主債務人是納稅義務人。取得所得者有稅捐負擔能力的是納稅義務人,不是扣繳義務人。

扣繳義務人,毋寧是以第三人的身分,依照所得稅法88條跟89條的規定,特別是89條的規定納入稽徵程序裡面,成為這裡面的第三人,被課予一個公法上的義務,而讓他去承擔起92條跟94條所規定的扣取、繳納、申報、填發,這些義務。這些行為義務固然不直接讓他變成是稅捐債務的主體,但是當他扣繳不足額的時候,根據94條的規定,他必須要去補繳。

法律條文規定是說補繳,學說則是稱之為叫責任債務。其實如果是講「補繳」,我毋寧會比較偏向認為,納稅義務人才是補繳的主體,因為納稅人在稅捐債權債務沒有被滿足的時候,他依然對國家負擔起繳納稅款的補繳責任。這個地方比較正確的名詞比較接近責任債務,應該用「賠繳」的概念。「賠」,就是你沒有做好你法律上的行為義務,那我讓你去賠。其實本來你不是義務人,但你沒有做好92、94條告訴你的扣取、繳納、申報、填發,所以我要你賠。

國家,特別是在他的稅捐債權沒有滿足的時候,他要求這個責任債務人去賠繳,那這個會比較符合責任債務的本質,只是問題是我們的法律,第一方面並沒有去講責任債務的概念,那他的法條規範的名稱也是寫扣繳義務人是補繳。但無論如何,在學理上並不認為補繳義務人是原先的納稅義務人,因為這個地方很清楚地透過後面的「追償」的規定,可以看到,即使我們的立法者也認為真正該要負責任的是納稅義務人,因為他沒有繳到該繳納的稅款,從而要讓納稅義務人去負起最後的補償稅款的這個責任。

這也是第94條規定。我們這整體去看到我們的扣繳行為義務。扣繳程序適用的客體,在88條的規定。扣繳行為義務的主體在89條的規定。扣繳行為義務的內容在94+92條的規定。扣繳,如果沒有在法律規定裡面去做如期如額的申報的時候,94條是補繳的責任,但容許扣繳義務人在事後再去向納稅義務人去做追償。

\hypertarget{ux6263ux7e73ux7fa9ux52d9ux4ebaux7684ux5236ux88c1}{%
\section{【扣繳義務人的制裁】}\label{ux6263ux7e73ux7fa9ux52d9ux4ebaux7684ux5236ux88c1}}

我們請同時翻開第114條的規定。扣繳義務人還會在沒有履行義務的時候受到一個制裁,因為他有一個協力義務的違反,因此還有一個114條的未依扣繳程序去做的時候的處罰的規定。我們來看一下114條。

114條一樣非常囉嗦而又複雜,用各款的規定。我們分開來好。來看一下114條的第一款規定。「扣繳義務人未依88條規定扣繳稅款,除限期責令補繳應扣未扣或短扣之稅款以外」,一樣,跟94條一樣,這邊用「補繳」。還要補報扣繳憑單以外,並按應扣未扣或短扣之稅額處一倍以下之罰鍰。

第二分句:「其未於限期內補繳應扣未扣或短扣之稅款,或不按實補報扣繳憑單者,應按應扣未扣或短扣之稅額處三倍以下之罰鍰。」

簡單來講,扣繳行為義務,當你短扣或者是應扣未扣,沒有去繳納稅款,國家認為他會有一筆稅款短收之風險存在,這個時候根據你的態樣,而去做三倍或一倍以下的,性質屬於漏稅罰的處罰。

第二款規定:「扣繳義務人已依本法扣繳稅款,而未依第九十二條規定之期限按實填報或填發扣繳憑單者,除限期責令補報或填發外,應按扣繳稅額處百分之二十之罰鍰。但最高不得超過二萬元,最低不得少於一千五百元;\ldots\ldots」

這個全然針對的是剛剛我們所提到的扣取、申報、繳納,之後的第四個義務,叫做填發扣繳憑單的義務。你雖然已經有扣取申報繳納,國家其實理論上沒有短收稅款的風險,但仍然還要做最後一個讓國家可以對帳的義務。因為我剛剛有講過,填發扣繳憑單是同時對納稅義務人和該管稅捐稽徵機關。如果沒有這一個填發扣繳憑單,從國家的角度就是,他收錢,不知道到底收了誰的錢。有這樣一個意味存在。透過這個規定,他說,你還是要來做一個填發扣繳憑單,我可以對得起帳。

所以第二款規定是一個行為罰的概念。因為國家其實已經收到錢了,理論上沒有短收稅款的問題。但因為國家對不起帳,他收到錢,他不知道誰是納稅義務人,所以在這裡面有一個第二款規定,應按扣繳稅額處20\%的罰鍰。這一種行為罰,但按所謂的扣繳稅款,用一個比例的方式,雖然不是像前面的一款的一倍或三倍,是20\%。那正是因為這個是行為罰,從而他曾經被大法官做過解釋說,這一種沒有限度的處罰一律按20\%是過度的過重的責罰,從而後來才加進來一個我們這裡面講的不得超過2萬塊錢,但最低也不得少於1500。

這是一個行為罰,但以所謂的扣繳憑單上所載的那個數額,扣繳稅額,來做為量罰時的唯一基準。這個本質上應該是行為罰,可是這一種處罰的方式又非常接近是漏稅罰的制裁的方式,因為只有漏稅罰才會按應扣未扣短扣稅款的額度來做為處罰的標準。但正因為用這一種比較怪異的規定的方式,所以我們經過大法官解釋以後,他就設定了一個上下限的一個處罰規定。

第二款的第一句的第2分句的規定:「逾期自動申報或填發者,減半處罰。」

然後第二句開始:「經稽徵機關限期責令補報或填發扣繳憑單,扣繳義務人未依限按實補報或填發者,應按扣繳稅額處三倍以下之罰鍰。但最高不得超過四萬五千元,最低不得少於三千元。」

一樣,他還是按扣繳稅額處以一個倍數的裁罰,但是有一個上下限的規定。本質上應該是填發扣繳憑單,沒有辦法對帳的問題而已,國家沒有短漏稅款的問題,但在這裡面的處罰規定,法律規定性質應該是行為罰,但這個處罰規定有看起來非常像漏稅罰。後來因為被大法官解釋過以後,就設了一個上下限規定。

這樣一個罰鍰的性質,仍然具有相當大的疑問,他到底算哪一種啊?我們剛剛所提到的,本質上,他有扣取申報繳納,其實國家已經取得稅款,但沒辦法讓國家對帳的,沒有填發扣繳憑單申報給該管稅捐稽徵機關,讓他知道這筆錢是誰的錢,從而在這裡面,他就存在著這樣一個違反義務的裁罰。

114條第三款規定還要加徵滯納金。每2天加徵1\%的滯納金。這個是所得稅法第114條規定,對扣繳義務人所做違反義務的懲罰,違反義務的非難。但實際上不只如此而已,其實還有一個條文規定,不在所得稅法,而在稅捐稽徵法。

在我們稅捐稽徵法第42條的規定,也針對扣繳義務人做了一個跟納稅義務人幾乎同樣內容的裁罰。

稅稽法(0790116 修正)41條:「納稅義務人以詐術或其他不正當方法逃漏稅捐者,處五年以下有期徒刑、拘役或科或併科新臺幣六萬元以下罰金。」

稅稽法(1101130 修正)41條第一項:「納稅義務人以詐術或其他不正當方法逃漏稅捐者,處五年以下有期徒刑,併科新臺幣一千萬元以下罰金。」

稅稽法的第42條針對扣繳義務人也做了是同樣內容的裁罰。在稅稽法第42條規定:「代徵人或扣繳義務人以詐術或其他不正當方法匿報、短報、短徵或不為代徵或扣繳稅捐者,處五年以下有期徒刑、拘役或科或併科新臺幣六萬元以下罰金。」

這個刑度,基本上跟41條在110年稅稽法修正之前,納稅人的違反自己的稅款繳納義務是一模一樣,就是當時的法律條文規定,基本上都是對扣繳義務人跟納稅義務人是同樣對待。

從國家的角度來講,他覺得我收不到錢,就是客觀事實,所以不管今天我要處罰你納稅義務人,跟處罰你扣繳義務人,都一樣。在我們的稅捐法制裡面對扣繳義務人是如此地看待。這樣一個扣繳的義務,內容跟制裁有沒有違憲呢?

\hypertarget{ux91cbux5b57673ux865f}{%
\section{【釋字673號】}\label{ux91cbux5b57673ux865f}}

大法官在民國99年做出釋字673號。這一號解釋基本上決定了我們的扣繳義務的合憲性與否的說明,雖然當時的這個釋憲的程序標的的法律規範其實不全然是現在的現行的所得稅法第89條的規定。不過我們來看一下,在這裡面的解釋文第一段,正是我們現在目前的法條規定的原型:「中華民國七十八年十二月三十日修正公布之所得稅法第八十九條第一項第二款前段,有關以機關、團體之主辦會計人員為扣繳義務人部分,及八十八年二月九日修正公布與九十五年五月三十日修正公布之同條款前段,關於以事業負責人為扣繳義務人部分,與憲法第二十三條比例原則尚無牴觸。」

這個地方就看到,以前的法條在「責應扣繳單位主管」之前,是以「主辦會計人員」為扣繳義務人。

我們的釋字673號解釋認為,不管是你之前的主辦會計人員,自然人作為義務人,或者是現在法律裡面的事業負責人,其實都是合憲的

接下來,解釋文第二段:「七十八年十二月三十日修正公布及九十年一月三日修正公布之所得稅法第一百十四條第一款,有關限期責令扣繳義務人補繳應扣未扣或短扣之稅款及補報扣繳憑單,暨就已於限期內補繳應扣未扣或短扣之稅款及補報扣繳憑單,按應扣未扣或短扣之稅額處一倍之罰鍰部分;就未於限期內補繳應扣未扣或短扣之稅款,按應扣未扣或短扣之稅額處三倍之罰鍰部分,尚未牴觸憲法第二十三條比例原則,與憲法第十五條保障人民財產權之意旨無違。」

也就是第二段針對所得稅法114條第一款規定,我們剛剛念到的那個第一款規定,因為那個地方存在著國家收不到稅款的風險,因此從漏稅罰的角度去理解第一款規定的情形,他認為,這個地方用三倍以下裁罰並沒有過度責罰非難的問題,這是第二段。

剛剛我們講了,這個看起來應該是行為罰,但是又用漏稅罰的這一種按扣繳憑單上所載的扣繳數額來做為裁罰標準。這個是接下來第三段要講的。

「上開所得稅法第一百十四條第一款後段,有關扣繳義務人不按實補報扣繳憑單者,\ldots\ldots」

跟第二段不同的地方是,第二段是你沒有及時報繳稅款,就是你扣繳義務人,你沒有及時把稅款送到國家的手裡面去,國家覺得他有短漏稅款、短收稅款的這個危險存在。

扣繳義務,分成兩個階段。一個是扣取申報繳納,這是第一個階段。還有第二個階段,你要在每一年1月的時候去告訴國稅局說,你這筆錢是扣誰的,讓稅捐稽徵機關可以去對起帳來。

所以這個部分在第三段裡面說,「\ldots\ldots,應按應扣未扣或短扣之稅額處三倍之罰鍰部分,未賦予稅捐稽徵機關得參酌具體違章狀況,按情節輕重裁量罰鍰之數額,其處罰顯已逾越必要程度,就此範圍內,不符憲法第二十三條之比例原則,與憲法第十五條保障人民財產權之意旨有違,應自本解釋公布之日起停止適用。有關機關對未於限期內按實補報扣繳憑單,而處罰尚未確定之案件,應斟酌個案情節輕重,並參酌稅捐稽徵法第四十八條之三之規定,另為符合比例原則之適當處置,併予指明。」

所以他告訴你說,你應該要去停止適用,這個是違憲的,因為可能會有所謂的過度責罰非難的問題。正是因為673號解釋的這個第三段宣告了剛剛我們念到的這個第二款的這種情形。所以後來立法者就根據財政部的建議,就上去做一個修法,有一個上下限的規定,但是他的規範結構仍然用一個蠻怪異的狀態,就是本來是行為罰,但是按一個所謂百分比的方式,也就是按扣繳憑單上所載的扣繳的數額來作為一個量罰基準的因素。

透過第三段的解釋,大法官認為,只有這一個部分才會違反比例原則。所以我們673號解釋原則上肯認了扣繳PAYE制度,對扣繳義務人的設定、扣繳義務的內容,基本上都是合憲的。只有對扣繳義務內容裡面的違反,特別是在後半段的扣繳憑單的填發,關於他的責罰,有過度非難的情況,違反比例原則。我們大法官在673號解釋裡面,原則上肯認了我們的就源扣繳制度本身的合憲。但正是這個肯認,讓我個人自己寫的德國博士論文,就是就源扣繳程序,完全是一個違反比例原則的狀態。

比例原則,在審查自由權的干預侵犯裡面,是一個在德國法制上非常重要的違憲審查的基礎。也就是當一個人他的營業自由,他的工作權因為法律的規定而受到限制,在這種情況底下,那我們在往往在規範的憲法審查上面,會用比例原則。這也是德國法治的特色。

其實在美式的違憲審查裡面,不太會用比例原則這個概念,美國比較偏向這個是讓立法者可以去做形成的。他們的審查往往一直都是根據個別不同的基本權內容,去發展出一套獨特的標準。那這一些比較重要的基本權裡面可能會用平等原則來作為一個審查的標準,不會用比例原則。

德國式的規範審查往往在自由權裡面都比較用比例原則。

我們回過頭來看這一個問題,我們來看一下,我們剛剛從88,89,92,94條,扣繳義務人的設定。673號解釋說這個是合理比例原則的。

首先第一個我要跟各位說,這個制度的本身,我們的就源扣繳,各種所得類型幾乎全部適用。因為我們這一套制度是來自於分類所得稅時代的概念。也就是我們這一個綜合所得稅制度,其實是民國46年,我們的所得稅法才全面地從分類所得改成現在的綜合所得。在此之前,分類所得稅,跟綜合所得稅最大的差別就在,分類所得稅,因為是就不同的類型,個別做稅捐稽徵程序,所以分類所得稅的最重要的稅捐稽徵程序是就源扣繳。

分類所得就是不同的所得來源,就直接,在給的時候就直接就源扣繳,報給稅捐稽徵機關。所以分類所得稅的稅制,本質上他會用就源扣繳制度來確保國家稅捐債權的實現。分類所得稅的稅制,原則上,並沒有綜合在一起去做結算申報。

結算申報自是綜合所得稅的類型特徵,也是他做稅捐稽徵程序的主要手段。一般而言,綜合所得稅制的國家為了要去對一個人的稅捐負擔能力去做正確的考量,通常是課予取得所得者一個,在特定期間裡面自己來報告自己來說我有多少所得的義務,把你的各類型所得全部都放到同一個稅基底下,這個就叫綜合所得稅加結算申報制。你自己告訴國家稅捐稽徵機關說,欸,我有哪些類的所得,原則上全部所得都放在一起,這個就叫綜合所得稅。

綜合,所以是分類的對立名詞。綜合所得稅要如何去課稅?他可以借用就源扣繳,但一般來講,因為就源扣繳不足以反映最後真實的所得,所以一定會有一個納稅義務人自己,在一定的時間裡面去做結算申報的制度。在綜合所得稅制中,爲了正確計算稅捐負擔,原則上都會有納稅義務人自行在一段時間內去做結算申報的制度。因此採綜合所得稅制的國家,例如美國、德國、日本,包括民國46年1957年之後我們的綜合所得稅制引進的時候,我們就慘結算申報。

這種結算申報制,只有納稅人自己最清楚自己有所得,但是如果當納稅人本身就不是結算申報的納稅義務人的時候,也就是他是一個境外所得者,請問這個時候我要怎麼讓他報稅?因為他不會來做結算申報,他不會呆到每一年的5月才來做結算申報,他拿了錢就走人了,從而他才會用就源扣繳的程序。

所以綜合所得稅裡面,通常會讓境內居住者去做結算申報,非境內居住之納稅人則按就源扣繳程序,也就是PAYE,就是你賺錢的時候,就直接讓所得的給付者直接在給你錢的時候,就要去做就源扣繳。那如果是境內居住者,對境內居住者,我給錢的時候基本上是沒有扣繳的義務。

因此有些國家,為了要避免他對各種所得來源,沒有辦法去查核,所以通常他們會對第三人,也就是所得的給付者,給予一個通報義務,課予一個協力的義務。那個義務叫做通報。我付多少錢給柯格鐘,這個時候我不是去做就源扣繳,而是我付錢的時候,我就告訴稅局,說,欸,我付多少錢給柯格鐘。這個叫通報義務。像比如說在荷蘭、比利時、盧森堡這三個國家低地國家,他們基本上並沒有要求所得的給付者要去做就源扣繳,他就是付錢出去以後,告訴國稅局說,哦,我給柯格鐘多少錢。那這個人在做結算申報的時候,原則上因為國稅局也取得了通報者所通報的資料,這個時候我就可以去核對,稽核柯格鐘有沒有在結算申報制度裡面去做正確數額的申報。這個是最小程度範圍的干預吧?對吧?對於一個第三人的身份,納入稅捐稽徵程序,他既不是稅捐債權人,也不是稅捐債務人。

從自由權的干預的角度來看,扣繳義務遠比通報義務是更高程度的干預,我想各位學過自由權,應該都不會否認吧。通報義務本身就是個干預了,只是算比較低的程度。當你存在著有通報義務跟扣繳義務之同時,如果你真要用比例原則裡面所稱之為叫必要性審查,我看應該通不過。因為有通報義務存在,就意味著,其實扣繳行為義務是蠻高度的行為。扣繳義務人,我要做扣取、申報、繳納,還要做填發,然後沒有做好,我還要賠,我還要被罰,還被罰到跟納稅義務人的本人一樣的程度。哪一個國家是這樣苛責第三人的?這個叫不違反比例原則?法律規定就長這個樣子。

好,如果法律不改,我就跟各位講,千萬不要當事業負責人,要找也要找個人頭哦。我們法律是這樣規定。扣繳義務人,在事業,是事業負責人。在機關、團體、學校,88年以前叫「會計單位主管」,主辦會計人員。那時候就是因為發生中科院的案子。

中科院裡面,民國80到85年之間,長達5年的時間,有一群本來是軍職人員,因為當時的所得稅軍人不用繳,可是他轉民職,就退伍之後,他變成是民職的人,還是一樣在中科裡面任職。他的薪水並不是免稅所得了,因為他不再具有軍人身分,這個時候,其實要做就源扣繳。可是這一些主辦會計的人,他們以為這些人員仍然是為國家貢獻,所以就不做就源扣繳,在5年以上的這個時間裡面,多達上萬個人都沒有做薪資所得的就源扣繳。

當時的法律規定,就是對這些主辦會計人員說,對不起,你沒有扣繳,根據94條規定你要做補繳。而且補繳不打緊,再加上114條的規定,你還要再罰三倍以下,所以這9個會計單位主管總共被命補稅9億元再加罰18億,加起來是27億,請你們這9個人補繳連帶處罰。那個時候才產生這個案子,釋字673號解釋。

主辦會計人員,他們認為說為什麼要叫我們負責?第一段的解釋裡面,他說這個沒有違憲,所以並沒有找你們找錯。可是立法者後來覺得我找會計弄錯了,所以後來才改一個條文,就是「責應扣繳單位主管」,我剛剛有跟各位講過。當時他改的背景是,有一位被被拜託陳情的立法委員就出來開會,開了研討會開了好多喔,就説,會計師管帳的,他們不管錢,所以責應扣繳單位主管,他說,欸,弄錯了,應該要找出納才對,他又不好直接寫出納,所以就用一個很抽象的名詞叫責應扣繳單位主管,因為你也不知道責應扣繳單位在講誰。

但那個修法,從「主辦會計人員」變成是現在的「責應扣繳單位主管」,其實基本上都是在找自然人,只是不是會計,而是當時認為應該是出納。結果這個抽象名詞哦,這個「責應扣繳單位主管」在我們現在機關團體裡面,原則上還是找出納主管,可是在我們學校,我有注意到,我們學校早期都不是找校長,可是近幾年來責應扣繳單位主管就變校長。

所以「責應扣繳單位主管」,這個概念本身還是一個很抽象的名詞,在機關團體裡面不是指首長,而是通常是指,裡面的那個出納組的組長,跟從事出納業務的人,也因為他們認為這個是付錢的人,你在付錢給他的時候,你馬上就要就源扣繳下來。但到學校的時候,本來其實以前早期,因為學校裡面辦行政的大部分都是老師兼差,覺得找老師實在是很不公平,老師自己平常就業務很忙,又要教學生又要跟學生做課業輔導,好忙好忙,做行政,已經大家都不願意了,結果去做了這一個出納的時候,哇,我又要負起這個責任,所以後來,我們財政部就改成找學校的校長。所以現在是校長變「責應扣繳單位主管」。

好,我們現在就變成,機關團體是出納,學校是校長,事業就是負責人。根據釋字673號解釋,這個都全部合憲。

\hypertarget{ux6263ux7e73ux7fa9ux52d9ux4e3bux9ad4ux7684ux932fux8aa4ux8a2dux5b9a}{%
\section{【扣繳義務主體的錯誤設定】}\label{ux6263ux7e73ux7fa9ux52d9ux4e3bux9ad4ux7684ux932fux8aa4ux8a2dux5b9a}}

我們再回過頭來,我剛剛講的一件事情,這個扣繳義務人,他本質上是第三人。扣繳義務其實已經比通報義務是更嚴重的干預了,好,我們就姑且不講這一件事情。這個義務主體的設定也是錯誤的,因為我剛剛跟各位講過,我自己在德國就寫這個論文。德國雖然也有扣繳義務,但扣繳義務人不是那個自然人。扣繳義務人是組織團體學校事業本身。他們的名詞叫雇主。我剛剛有跟各位說過,在德國,因為他們的原則上是結算申報制,他們的就源扣繳是有限度的就源扣繳,因為他主要是納稅人自己的結算申報義務,自己稅捐自己負責申報繳納,有問題嗎?你要找人,你總是要做一個最低限度最小範圍的干預吧。在德國就只有兩個。一個是薪資所得,找誰,找雇主。另外一個叫利息所得,找金融機構,因為金融機構會付利息給存款的人。就只有這兩類所得需要做就源扣繳,其他所得全部都回去讓納稅人自己去做結算申報。

根據臺灣的法律的規定,我們的納稅義務人有結算申報義務,釋字537號說這個是合憲,這個沒有問題。因為這個不管在德國,臺灣都是合憲的制度,申報制度是為了要去確實履行憲法上的依法繳納稅款的義務。因為沒有你來申報,其實要職權調查無從查核起,大海撈針。從而結算申報制度是為了要履行依法課稅、量能課稅原則的目的的必要手段。

但我們回過頭來再看,當結算申報制度,這個存在的時候,我們有對極為廣泛的所得類型,設定以這一些機關團體學校事業裡面的不同的主體。機關團體以前是主辦會計,後來改成責應扣繳單位主管,實際是找出納組長。學校則是一樣,是找責則應扣繳單位主管,但是是找校長。事業,則是找負責人,也就是臺積電,鴻海,你找他們的董事長。我們的大法官解釋說,這個全部都合憲。

機關團體的首長不用付機關執政成效的成敗嗎?你憑什麼認為?因為大部分都是在行政機關嘛,不管你今天是國家級的或地方級的行政機關,執政的政策其實也是一樣,是首長來負責啊。為什麼機關你就講責應扣繳單位主管,結果事業,你就講負責人。請問就算從平等原則來看,你告訴我這樣叫平等嗎?

找錯主體了,因為你不是用這個事業機關團體組織,不是用這些法人本身,而是用他們底下這一些自然人。而自然人裡面你又對機關團體學校跟事業做區別對待。這樣的一個就源扣繳制度,673號解釋,第一段,合憲。不僅違反比例原則,在我來看,也是違反平等原則。因為他在機關團體學校跟事業當中做區別對待,無理由的區別對待。

根本不應該是什麼事業負責人跟這個所謂的機關團體責應扣繳單位或主辦會計人員,因為這些自然人的勞務,其實都是歸屬給法人本身。機關團體學校事業的這一些實際執行扣繳或給付所得款項業務的人,其實他的勞務都是歸屬給事業組織本身吧。各位都學過民法的規定啊。自然人的勞務是歸屬給法律本身。連這樣一個基本的概念都分不清楚嗎?所以臺積電誰要付扣繳義務,不是劉德音,也不是張忠謀,是臺積電本身。行政院跟臺北市政府,誰要付扣繳義務?不是蔣萬安,也不是現在的陳建仁,都不是,更不是他們底下的所謂責應扣繳單位主管,出納組長,而是臺北市本身,行政院機關本身。

這才是正確的,所謂的雇主,你付出薪資所得,應該是以雇主本身作為納稅義務人。我們的釋字673號不僅錯誤地設定行為義務主體,而且課予過重的責任,他的責任就在於他一路過來,第一順位的補繳責任。因為在責任債務裡面,理論上應該要由納稅義務人先付第一順位,也就是國家自己要去對納稅人去追償,國家自己去跟納稅人要錢。第三人,他是無償來幫你國家完成這個任務,結果做好沒賞,打破要賠,而且還賠到底,你跟納稅人一樣的責任,這一點,還說沒有違反比例原則?

你只要從比較法制來看,就可以完全看出我們現行法制之對扣繳義務人的責任之苛刻之重。我都錢繳進來了,只是扣繳憑單沒給,讓你沒辦法對帳。國家當然用很最低的成本,因為國家是坐著等收錢,所以他有最低的成本費用,可是我們的扣繳義務人是無償為國家完成公法上的任務欸。在這種情況底下,完全是一個違反比例原則的侵害扣繳義務人的職業上的工作權,跟他的營業上的自由權。我完全不能贊同673號的結論,我只贊同第三段,因爲那是過度責罰。第三段是對的,但第一段完全是錯誤的。很可惜的是,673號解釋作成之後,立法者當然不會覺得自己有什麼不對。

我們的現制就是一直到目前為止。未來沒有新的立法之前,我也只能跟各位講,你千萬不要去當機關團體學校裡面的出納。千萬不要當出納,要趕快轉職,從那個地方移開,申請轉去任何業務都好,就是不要當出納,因為你不知道在你任內,少扣繳稅款,找你負責。

我是出納,我也是個薪資所得者,為什麼中科院這些主管跟出納要為全體的人的扣繳義務,要補繳稅款要負責。我不能明白,這怎麼會不違反比例原則。

我們先跟各位談到這裡。下個禮拜還有一點點關於扣繳義務的這個部分,再做補充上的說明。

我們最後面還進入結算申報跟協力義務違反的制裁,也是我們最後一次的所得稅法上課。

\hypertarget{section-27}{%
\chapter{20231211\_01}\label{section-27}}

\begin{longtable}[]{@{}l@{}}
\toprule()
\endhead
課程:1121所得稅法一 \\
日期:2023/12/11 \\
周次:15 \\
節次:1 \\
\bottomrule()
\end{longtable}

今天是所得稅法一最後一次上課。下個禮拜,請各位能夠參加所得稅法的期末考試。我們的學期分數的總成績,就是期中再加期末,然後老師會看一下大概同學們的分數的分布,再做一個整體性的調整。因為老師並不太希望在考試過程裡面考一個對每位同學都很難的。如果每位同學都答得很好,其實也沒必要調整。就是萬一有一個是大多數同學都不太會答的話,那我會做一個整體性的調整。那大致上就這樣。因為我們沒有其他的點名或者是出席參與討論的這個成績可以供參考。老師原則上就是期中跟期末加起來,然後去做一個整體的調整。那所謂整體調整的意思就是說,如果不及格的人太多的話,老師會看一下那個考卷的內容,深自檢討,是老師沒有把題目出好,鑑別度太低,所以會做一個整體,所謂整體調整就是大概整體的一個移動的方式啦。

\hypertarget{ux5c31ux6e90ux6263ux7e73ux7a0bux5e8f-1}{%
\section{【就源扣繳程序】}\label{ux5c31ux6e90ux6263ux7e73ux7a0bux5e8f-1}}

我們今天是所得稅法一,綜合所得稅的最終的部分。

我們上個禮拜跟各位大概談到綜合所得稅的稽徵程序裡面非常重要的一個程序叫就源扣繳程序,或者是PAYE,是pay as you earn的這個概念啊,就是說你的稅負,理論上來講,應該是到隔一年的5月,也就是說,你的繳納期限屆至應該是隔一年5月的時候,那這一種等於是稅捐立法者透過條文的規定,把你的本來可以享受到明年5月份的期限利益,依法課稅原則底下,你那個時候才需要去繳稅,把那個期限利益給取消掉。

簡單來講就是,就源扣繳相對於結算申報,其實是有一個期限利益被取消的問題,也就是說,他其實本來應該到隔一年5月,如果最長以01年1月份來算,他等於是有一個一年4個月左右的,相當於是國家期前去做徵收。因為照道理來講,本來應該是到結算申報的5月份。那也因此,就有扣繳程序,是一個在法制上面來講,應該要謹慎以對的制度。就不用特別再去講其他的事項,就光一個綜合所得稅制的國家,用了所謂的就源扣繳制度。就有扣繳制度,等於是讓納稅義務人本來依法可以享有的,依法課稅原則底下,有一個延扣繳納的這個期限上的利益,會因為就源扣繳程序,在他賺到錢的時候,立刻就被所得的給付則給依照法律規定的扣繳率,扣除掉。

這個問題的思考,老師給各位的建議,就是要依序地去討論以下幾個問題。

第一個是就源扣繳制度,這個本身的合憲性與否。對我們,特別是法律系背景的,我們不會因為只是法律規範有扣繳義務,所以我們就毫不質疑他的正當性與否。任何法律背景的同學們都必須要對既有的法律規範存有一個某種程度上,為什麼,為什麼,你為什麼要這樣做,這樣的思考。我相信各位作為法律背景,相對只是把現況當作現況,就是,啊,法律長什麼樣子,就這樣接受來講,我想在這個地方,如果各位背後要有一個思考哦。當然,在這個地方,你必須要分兩個層次,就是說現況就是長這個樣子,你必須要很清楚了解現在的規定是長什么樣子,然後其次才是所謂的正當性的基礎。

特別是在行政處分的適法性,這個階段,原則上就是行政機關,他必須要依照現行法律規範的現況去做出他的行政處分,不管他是不利益的行政處分,或者是裁罰性的行政處分,都是一樣的,你必須依照法律的規定。

但在此同時,行政機關在依法作成行政處分,乃至於當徵納雙方有爭議的時候進入法院,法院在做審理裁判的時候,他也必須要依法裁判。但作為法官,往往在這個地方,他也必須要去考慮這一個系爭行政機關,據以適用為行政處分基礎的法律規範,本身是否具有合憲性。

這個是,各位,特別是法律人,我會期待也會要求各位做這樣一個考慮。因為各位將來有一天,你可能會當法官,那你當法官的時候,你其實你必須要去看,

行政機關沒有釋憲的聲請權,我再講一次,行政機關的公務員,你再怎麼懷疑法律,你沒有辦法去背離法律,你沒辦法違背上級機關所公布的解釋函令,因為這個在行政一體底下,你就沒有這個權限,你不能自己恣意地去脫離法規範,跟上級機關依法所公佈的各項解釋函令或者是事實認定的行政規則。不要說法規命令,行政規則你也不能違背。行政機關的公務員沒有太大的可能性去挑戰法律本身。但作為法院的法官是有這個可能的。因為我們在憲法訴訟法裡面第55條規定給受理各該行政訴訟事件的法院法官聲請釋憲的權利,當他認為法律位階法規範具有違憲的確信的時候,他可以裁定停止訴訟程序去做釋憲的聲請。當然,在我們國家並沒有讓一般的法院直接有違憲宣告的權利。

我再講一次,一般的法院沒有違憲宣告的權利,可是他可以在法規範文義許可的範圍內,盡量去做合憲的解釋。這也是一種憲法意旨的表現。不是行政法院的法官就不需要有憲法意識,這個是完全錯誤的概念。行政法院的法官,他必須要依法裁判,他必須要有法規範,這個無論如何,他是去必須具備有這樣一個意識。因為他畢竟他並不是一個民選的機構。對法官來講,他是,依照法律,跟良知良能去做判斷而已。當然,在這個同時間,他必須要透過他的法律解釋的方式,透過個別行政法去體現憲法上的意旨,這個也就是我們稱之為叫合憲性解釋的一種方法。當法官認為無論怎麼去做解釋,我都得不到合憲的結論,這個時候必須要停下來,去聲請釋憲,去向我們國家唯一有權限宣告法律位階法規違憲的司法機關,也就是司法院大法官,去聲請釋憲。

這個層次要分開來。行政法院不能做違憲宣告,可是他可以盡可能在法律規範文義裡面去盡可能去做符合憲法規範意旨的解釋,這個是每一個行政法院的法官都可以做的。那行政法院法官做,他發現文義拘束了他,讓他怎麼解釋都逃不過那個文義可能解釋的範圍,這個時候,法治國家賦予給這個行政法院法官一個法律位階憲法審查的聲請的權利,也就是讓你可以去做這件事情。當然你不能直接宣告他違憲,因為在我們國家只有一個機關,可以宣告法律位階的法規範的違憲,讓他產生一般性的失其效力的這樣一個效力。

當然,作為司法院大法官,作為整個國家體制裡面唯一能夠去宣告法律位階法規範的違憲的這個司法機關,他當然必須要非常謹慎地去行使他的權限。

那么當行政法院法官不能夠做這樣合憲性解釋的時候,那么最後面,當然體制上面來講,我們只能夠仰賴司法院大法官對系爭法規範,在做具體的解釋的時候,能夠去體現憲法規範裡面的意旨。

我們在上個禮拜談到673號解釋。那673號解釋,其實有些他講的地方,並不完全是正確,那我現在把673號解釋裡面大概所提到的幾個問題點整理一下。針對就源扣繳程序,各位同學可以以下幾個層次去考慮這個問題。

第一個就是就源扣繳制度本身的合憲性。

第二個就是就扣繳制度制度是透過裡面的義務主體跟義務的內容去設定的,所以第二個就是你要看就源扣繳義務人的主體設定,是否是正確的,也就是說,是不是符合於事務本質,他找到對的人,在對的時間點去做對的事情,也就是第二個,義務內容的設定。

第三個,就源扣繳制度裡面的行為義務的內容。

673號,我在上個禮拜花了很多時間跟各位講,我主要是再跟各位講一件事情,他找錯人了。你不應該是找機關團體學校裡面的,不管你是負責人或責應扣繳單位主管,不管你叫做出納或是叫會計,你找錯人了,這個本身就違反比例原則。因為他找錯人,沒辦法達到目的啊。這個是我對673號最大的批評,也就是說,673號裡面他基本上是跳過第一個層次的問題,叫做就源扣繳制度本身到底合憲與否,沒有很直接正面去回答這個問題。

那我接下來今天也是跟各位去討論這個問題,就就源扣繳制度到底本身合不合憲。這個是第一個層次的問題,第一個過了才會有第二個。當然其實一二三毋寧是相互交錯作用的喔。就源扣繳制度本身、就源扣繳制度裡面的行為義務主體、就源扣繳制度裡面的行為義務的內容。

我們的行為義務內容對扣繳義務人課予扣取、申報、繳納、填發扣繳憑單,總共4個扣繳義務的內容。

接下來我們再來看,再下一個層次的問題是賠繳,也就是你沒有做好,我要苛責於你一個不具有非難,但是是一個不利益,因為賠錢本身就是個不利益。我們稱之為叫責任債務人,這個概念,其實我們在稅法總論裡面說了很多次,但我們現行法一個字都沒出現過。我們現行法完全沒有那個不同層次的納稅義務人的概念。

我舉個例子來講,我們到所得稅法三的時候,我們會去講到遺產稅的納稅義務人。遺產稅的納稅義務人,理論上他本來應該就是遺產的權利義務歸屬者,也就是拿到錢的人,他才叫做納稅義務人。也就是繼承人跟受遺贈人。那遺產的管理者跟遺囑執行人又是什麼身份地位?遺囑執行人,他是負責執行被繼承人立下來的遺囑,他是一個業務執行者,他本質上不是權利義務的歸屬者,各位讀過法律應該對這個算基本認識,在我來看,這個是基本認識。遺囑執行人跟遺產管理人根本不是權利義務之歸屬者。但他卻經由我們的遺產贈與稅法的規定,被納為權利義務之歸屬者,把他同視作為繼承人受遺贈人一樣的納稅義務人地位,這個本質上就是一個完全搞不清楚狀況的立法。

你絕對不能因為這個立法,所以你就認為,這個立法就長這個樣子,我還能怎樣。不對!各位,如果將來有一天當法官的時候,我是希望各位能夠勇敢地鼓舞起,聲請釋憲,因為行政機關是沒有辦法聲請釋憲的,只有行政法院的法官,他有這個機會跟可能。當然啦,其實人民窮盡救濟程序也可以去提起,但你看這麼多年來,之所以沒有人去挑戰他,就是因為人民打完救濟程序以後,他大概也累了,他大概也不知道原來問題出在法律本身,而不是稅官,也不是行政法院法官。

也因此,我請各位也當然要特別去注意,問題往往出在法律本身,而不在於為什麼他要這麼苛刻,為什麼要找我一個遺囑執行人?我常常以前上課的時候,有一些會計師或律師跟我講說,柯老師為什么要被限制出境,我幫忙做事情,我為什么要被限制出境?我說,對你限制,這個就法律規定的不義,法律規定不義,就是你要勇敢去提出釋憲啦,因為你只有你才是權利人,你才是一個被限制出境的當事人。結果你這個事情後來去財政部,財政部說,哦,對不起,我們弄錯了,你這樣是形式納稅義務人,那個拿到錢的繼承人跟受遺贈人,他叫實質那受益人。哦,突然出現一種分類,叫做形式跟實質納稅義務人。好奇怪。

但我不是非難財政部這個分類,因為其實他也想要嘗試著在立法的這個地方裡面,自己想要去找出一個出路,後來就形成了一個叫做形式納稅義務人跟實質納稅義務人。實質納稅義務人,才要真正被限制出境,形式納稅義務人不需要。我們姑且不講這一種分類到底從何而來,財政部想要去解決的,就其實他是職務管理人,職務管理人本來不是權利義務的歸屬者,真正被權利義務歸屬的人,其實是拿到錢的繼承人跟受遺贈者。你這樣一清楚去思考,你就知道問題本身,問題出在法律規範的本身。稅法這一類的例子還算不少。好,那也因此,回到我們的就源扣繳制度。

第一個層次是就源扣繳制度本身的合違憲性的問題。

第二個,扣繳義務人的設定。

第三個,扣繳義務的內容,是否過重的負擔。

第四個,賠繳義務人,也就是責任債務人的設定。賠繳責任就是,他其實是無償來幫國家完成扣繳義務。無償,這件事情,我做不好,拿不到錢就算了,我還要賠。這個概念會不會有過重的問題?

最後第五個,就是扣繳義務人的處罰受罰。一個扣繳義務人,沒有依法去做扣繳,他有違反的行為義務,這一點毋庸置疑。但他受的處罰,基本上跟納稅義務人相當,這個道理在哪?要考慮責罰的相當性。

這個是我們在扣繳義務裡面大概一路過來,可能會思考到了幾個問題點。

\hypertarget{ux5c31ux6e90ux6263ux7e73ux5236ux5ea6ux4e4bux5408ux61b2ux6027}{%
\section{【就源扣繳制度之合憲性】}\label{ux5c31ux6e90ux6263ux7e73ux5236ux5ea6ux4e4bux5408ux61b2ux6027}}

首先,第一個,就源扣繳制度本身的合違憲性?

\hypertarget{ux5883ux5916ux6240ux5f97ux8005}{%
\subsection{【境外所得者】}\label{ux5883ux5916ux6240ux5f97ux8005}}

其實這個部分,我們最早就源扣繳制度,有一個釋字116號,就國外廠商因分期付款所生之利息所得應如何扣繳所得稅,解釋文裡面提到:「支付國外廠商分期付款,訂有利息者,其利息所得,仍應由扣繳義務人於給付時扣繳之(稅款)。」

這個解釋當然是依循著當時的所得稅法第86條第三款的扣繳義務人的規定,後來就是變成了第89條的規定。納稅義務人,是取得利息的這個所得者,然後扣繳義務人則是89條所規定的。我們這時候的法律規定跟當時或法律規定不全然相同。我之所以會提116號,主要是,他在沒有直接提到這個概念前提底下,認為付給國外廠商的利息所得,用扣繳制度。他在這個地方,沒有討論跟辯證,他肯認的這個制度,本身的正當性所在。

也就是說,對於取得所得者,依照PAYE就源扣繳制度,我們要求給付所得的人在給付的時候直接做就源扣繳。釋字116號,沒有直接正面去談到正當性。但你可以認為,他是在這個前提底下去討論,關於付給國外廠商的利息所得的情形。

我個人也認為,如果取得一個屬於本地有課稅權限的所得,而所得的取得者,是一個國外的,非境內的居住者,那麼確實在課稅的手段裡面,你沒有辦法期待他在離境的時候就自然去做申報。因此我個人也贊同,境外所得者,也就是,取得我國來源所得的非境內居住者,適用PAYE制度,這個是有正當性的。也就是說,這個是為了掌握稅源,為了實現有所得就應課稅的基本原則,我個人認為他是合憲的一個手段。

\hypertarget{ux5883ux5167ux6240ux5f97ux8005}{%
\subsection{【境內所得者】}\label{ux5883ux5167ux6240ux5f97ux8005}}

那接下來我們來談,如果是境內的所得者取得我國來源所得,那這個時候是不是還有這一個就源扣繳制度的正當性,這個才是接下來的問題所在。

我們在此之前也跟各位去提到,在臺灣的所得稅法裡面88條,綜合所得稅納稅義務人的各類所得,基本上都會有就源扣繳制度的適用可能。我們的就源扣繳制度,是各種類型所得都有適用的可能。這其實是源自於我們在民國46年以前,是分類所得稅制。

所謂的分類所得,就是在課稅的時候,對於各類所得來源,直接做就源扣繳。分類所得稅,他沒有綜合在一起去做結算申報的制度。分類所得稅制的國家,對於所得課稅,原則上都是按就源扣繳PAYE制度。所以你可以看得到,大概是英國為主的,包含英國前殖民地國家或地區的新加坡香港,就是屬於這一類的。根據分類所得,就是你各類不同所得,比如說在香港,薪資所得直接就源扣繳12\%結束。

英國是最早1799年引進全世界第一個所得稅,那時候為了要對抗法國的拿破崙,所以他需要這一些錢來支應,因為打仗就是錢啊。那在這種情況底下,英國雖然是引進了全世界第一個概念上針對你的所得課徵的稅制的一個國家,但英國到目前為止,他基本上都還是所謂的分類所得,也就是直接各類所得來源,直接在源頭那個地方按照一個固定比例的方式去做課稅。這個就是分類所得,他的稅捐稽徵手段基本上就是用就源扣繳。

可是採取綜合所得稅制的國家,意味著各類所得不分類,至少原則上不分類。什麼叫綜合所得?綜合所得,意思就是不分類嘛?大家放在同一個稅基。因此綜合所得稅制的國家,一方面受到19世紀的社會國思想的影響,一方面會在綜合所得稅基裡面搭配超額累進稅率。另外一方面,他在稅捐的稽徵上面,也會採取結算申報制度,因為只有你自己最清楚自己的所得。所謂的綜合所得的概念,就是不管你是租金,你是薪資,你是財產交易,全部的錢都放在一起,這個時候才有辦法去適用超額累計稅率,所以通常都會用結算申報,就是你自己來報。那我根據你申報的結果,我核定你一張綜合所得稅的稅單。臺灣只是把報繳程序合在一起而已,基本上也是你自己來申報,然後國家再根據你的申報的內容給你一張核定的稅額通知單,或者我們用公告的方式,我們待會會去講,會用公告的方式去代替核定的稅額通知單。大致上就是這個差別而已。但結算申報制度,一般而言就是搭配所謂的綜合所得稅制。

那我們民國46年從分類所得稅分類所得稅改制變成綜合所得稅制的時候,沒有放棄就源扣繳的制度,因為這對稅捐稽徵機關是一個最小成本費用,就可以拿到錢的方式。這個制度本質上是對國家極端有利,國家以最小的費用,如果這樣講最小費用,他應該講費用為零,因為國家把這些費用都轉嫁出去,讓誰負擔?讓扣繳義務人負擔。就這麼一個簡單的一個稅制轉換過程,我們從分類所得稅走向綜合所得稅,我們並沒有放棄在分類所得稅制底下的就源扣繳。我們要就源扣繳,我們還要做什麼?還要做結算申報。等於是為了一次稅捐的徵收,我們既要給錢的人要負責就源扣繳,我們還要拿錢的人也要自己來告訴我,你賺多少錢。

這正是這個制度本身,我們在第一個層次去講,就源扣繳制度對境內所得者究竟有沒有過度干預。是誰的問題?是誰是納稅義務人的問題,還是是扣繳義務人的問題?你覺得是誰的問題呢?我們先講納稅義務人。納稅義務人,你是取得所得者,所以要求你自己來申報自己的稅捐,你有問題嗎?釋字537號說,這個是合憲的。所以納稅義務人自己有所得,自己來申報自己稅捐,確實是最小干預手段。簡單來講就是,你自己告訴我你賺多少錢。稅捐稽徵機關根據你申報的事實,再去做查核。不然大海撈針我去哪裡查,我怎麼查?

你自己告訴我你在臺大有所得嘛,我再去發函問問看臺大究竟是不是給了你自己講的那麼那麼多所得。這樣他會比較清楚,因為他可以透過勾稽的方式。我說我是從臺大賺到的所得好,那稅捐稽徵機關就去問臺大,看臺大是不是在申報稅的時候,這裡面有一筆費用就是付給的柯格鐘,那這個時候我可以做交互查核,透過這種方式來去確認柯格鐘在申報所得稅的時候,是否做誠實的申報。

\hypertarget{ux6263ux7e73ux7fa9ux52d9ux4eba-1}{%
\section{【扣繳義務人】}\label{ux6263ux7e73ux7fa9ux52d9ux4eba-1}}

在這樣一個前提底下,我們還要再額外對一個第三人,賦予一個公法上的義務。

當然在這個地方,扣繳義務人的法律上地位,這裡面學說有不同看法。

扣繳義務人法律上地位,例如陳敏老師,他就認為這個叫公權力授予私人行使。也就是國家把課稅的公權力授予給扣繳義務人來對納稅義務人行使,換言之,納稅義務人不能拒絕扣繳義務人的扣繳。

葛克昌老師認為,他是一個行政助手,行政助手的概念就是他是一個沒有自主意識的手臂延伸而已。扣繳義務人沒有自主的意思,他只是受到稅捐稽徵機關的指示,然後去做扣繳。

公權力委託私人行使跟行政助手這兩個概念的差別,只是差在被委託的這個私人,扣繳義務人,他本身是否具有自主的意識,而決定要不要去做成該項公法上的行政處分的行為。差別只是差在這裡而已。

我個人則認為,這個地方不存在著公權力委託私人行使,也不存在著一個所謂的行政助手,因為,行政助手的概念,還是把扣繳義務人當作是公權力機關的手腳延伸,這是差在有沒有自主的意思形成而已。

其實不然。扣繳義務人毋寧是幫納稅義務人去完成公法上的繳納稅款的義務。各位能夠聽出來這幾個概念之間的差別嗎?雖然這純粹只是一個概念上的說明,但最大的區別實益就在扣繳義務人,如果沒有完成扣繳義務,請問稅捐債務是否已經完成繳納義務,也就是稅法上的稅捐債務關係是否消滅?

我舉例而言,扣繳義務人,扣繳稅款以後,他跑了。不管是陳敏老師的說法,或者是葛老師的說法,他最後面承擔損失的必然是國家。因為不管是行政助手或是公權力委託私人,那是你們內部的事情,你只要從我納稅義務人那個地方扣了一部分,你只要扣了,你沒有交給國家,那是你們內部之間的公法上委任契約的債務不履行。跟我無關。跟誰無關?跟納稅義務人無關。

相反的,如果你是採我後面說的這個,叫做服公法勞務的私人。所謂的「服公法勞務的私人」,這個私人本身不具有高權地位,他本身是為另外一個人去服這個人本來應自行申報繳納稅款的義務。「服公法勞務的私人」,跟本來應服公法勞務的「本人」之間,存在著一個類似公法上的委任的關係。基本上委任人也沒有辦法,直接指示受任人去做如何如何的扣繳,比如說不要做扣講,比如說以比法律所規定更低的比例去做扣繳。納稅義務人並沒有辦法去指示扣繳義務人,扣繳義務人全然還是要依照法律的規定而去做法定給付時的一個依照比例的扣繳。

從而這個制度本身,確實在概念上來講,是對國家比較有利的。但正因為我們的所得稅法裡面對扣繳義務人如果未履行扣繳義務,法律仍然規定國家可以向納稅義務人去請求繳納他未完成的稅款繳納義務,所以我國的現行法是採取「服公法勞務的私人」的規範,而不是前述兩位學者所講的,授權與私人行使。因為最明顯的差別,就在只有在「服公法勞務的私人」的這個概念以下,沒有扣繳稅款,誰承擔那個損失?國家,原則上他不承擔,他認為,我還是可以跟你納稅義務人,因此他會變成是,可以對納稅義務人要,也可能去對扣繳義務人要,而形成了扣繳義務人跟納稅義務人之間,某種程度上是一個未經法律明文規定的,不真正連帶債務的關係。

不真正連帶,因為真正要負起最扣繳納稅款義務的,應該是納稅義務人。扣繳義務人,他自己扣繳義務沒有履行,他確實要賠繳,他賠繳繳完以後,他是回過頭去,去跟納稅義務人,要求追償。這個概念正是各位在的學民法裡面的不真正連帶。不真正連帶,也沒有法律明文規定。

在德國,他確實是把扣繳義務人跟納稅義務人規定成連帶債務人,有法律的明文規定,國家更有依法課稅,依法去對扣繳義務人去追償的正當性的基礎,因為法律有明定。扣繳義務是一個加諸在第三人的行為上的義務。先是行為義務,之後在你沒有履行扣繳義務的時候,而產生對你的財產權利的剝奪。賠繳本身就是財產權的干預。從而,是工作權在前,財產權在後。

扣繳制度,原則上是要求,你給東西的時候要先做扣繳,那由於我們是一個非常廣泛的就源扣繳義務,不直接連結特定的工作領域,從而他是一般行為義務的干預,然後之後才是財產權的干預。也就是說,因為扣繳義務,並沒有限定在特定的所得類型。我們的扣繳義務是牽涉到所有的所得類型都有,從而他是從一般行為義務這個角度,在你沒有依法去做扣繳的時候,國家一樣可以對扣繳義務人去請求,作賠繳的主張,這個就是責任債務的請求。這個責任債務請求,當然涉及到財產權的干預。

也因此,我們在第一個層次裡面,我們要討論的就是,就源扣繳義務,對境外所得者做就源扣繳,我個人認為是合憲的,因為也沒有更好的方法,你不太能夠期待一個境外所得者,他離境或者在每一年結算申報的時候,他就立刻來做所得的結算申報程序。因此,對境外的有所得稅捐義務的人,我們要求付錢的人,給付所得者,直接在源頭之處,做就源扣繳所得。

這個也是在世界各國範圍內大致上承認的一種稅捐稽徵的手段,特別是對境外所得者,也就是取得所得者,是非境內居住人或是非稅籍居民的事業。這一種類型,只要這個是我們課稅權所及的話,會要求給付所得者去直接做PAYE這個制度。

但相反的,如果本身取得所得者,他本身就是一個稅籍居民,在分類所得稅制的國家,確實還是會用扣繳制度。可是一個綜合所得稅制的國家,繼續採用如此廣泛的就源扣繳制度,特別是在有結算申報制度的前提底下,畢竟還是納稅義務人,自己取得所得自己來做結算申報,那何以要對第三人,課予這麼高的行為義務的內容?至少立法者要在比例原則底下,要好好地告訴我們,說明為什要做這件事情。

相對於就源扣繳,比較輕的干預手段,毋寧是通報,就是你給錢,那你告訴國家,給他多少?就好像我在臺大授課,我賺錢,國家怎麼知道我賺多少錢?請國立臺灣大學通報給國家就可以。而不需要做進一步的就源扣繳的相關義務。你只要比過干預手段,所謂最低限度,你就可以了解,哪一種是比較嚴重的干預。這一種比較嚴重的干預,至少國家在立法上要說出一個正當性。

因此釋字116號,是對境外所得,這個我個人沒有意見。但對境內所得所涉及的釋字673號解釋,我就有很多意見,理由在於,就源扣繳制度本身的合違憲性。特別是,我們是一個綜合所得稅制的國家。理論上,綜合所得者,你自己就去做結算申報,應該要自己的稅捐自己申報,自己負擔繳納行為義務。這個我沒有意見,釋字537號說,這個是對課稅的金錢給付義務,你自己來報,必然必要的一種最小干預手段。

但在此同時,為什麼我們國家有存在著第二軌的制度?這個第二軌制度一樣讓國家因此取得所得,而且是提前取得所得。提前取得所得,因為本來是結算申報啊,你應該是隔一年的5月份的時候,你才取得。然後這也是比通報義務是比較高程度的干預,基本上因為他要做扣取、申報、繳納、填發,這四種義務的內容,因此這是一個一般行為自由的干預。

我剛剛提到,這並不是針對特定的工作權的內容,對特定職業的干預,這是對所有的,只要符合88條法律構成要件規定的所得類型,符合89條所規定的扣繳行為義務人。

臺灣的所得稅法89條又針對特定的職業類別,所謂的職業類別,包括負責人,包括責應扣繳單位主管,包括從事會計的主辦人員,這個就會是對工作權的干預。這個是一個很典型的對工作權的干預。也就是釋字673號當時釋憲的程序標的,是以民國88年以前的主辦會計人員,會計。就源扣繳,當然涉及到對會計從業人員的,不是職業工作規則,而是稅法加諸給會計人員在職業上面,他們在工作上面而負擔的一個行為上的義務。

如果以目前的法律來看,大致上可以說這是一個對工作權的干預,毋寧大概是在事業負責人這個部分,雖然其實看不太出來一個具體特定的工作權的內容。越是看不出來,會越往一般行為自由的干預上面去評價,但越能夠被特定,針對特定的工作領域範圍,就會往我們憲法第15條的工作權這個部分去做行為自由的干預的審查。因為工作權是特定領域裡面的行為自由。

這個是我們在談到的第一個程度的問題,也就是就源扣繳制度本身的合憲性,以及,第二個問題,就源扣繳義務人之設定。

\hypertarget{ux7fa9ux52d9ux5167ux5bb9}{%
\section{【義務內容】}\label{ux7fa9ux52d9ux5167ux5bb9}}

那第三個就是就源扣繳的義務。就源扣繳的義務,其實我自己早期我一直都不太能夠明白為什麼要做四個階段的區隔,扣取、申報、繳納、填發。我後來到財政部訴願會才知道,哦,原來填發是這樣,因為他們在扣取申報繳納的時候,基本上扣繳義務人並沒有指明,這個錢是誰的錢,是要到隔一年1月份的時候填發扣繳憑單的時候,他才具體去告知稅捐稽徵機關跟納稅義務人,哦,我把你的錢給國家了,國家才知道說哦,原來我收了錢是柯格鐘的不是柯小鐘的,不是陳衍任老師的。

填發,是對帳用的。

每個月你拿到錢的時候,扣繳義務人就是逐次給、逐次扣,扣取、申報、繳納,他一次就要把那個單位或是機關內部的所有申報扣取的稅款,他就要一次繳給國家。給國家的時候,其實國家不知道他收到錢,國家也不那麼care,反正我收到錢。

所以我們釋字673號講說,這是國庫調度,我想說,以我自己對673號解釋的想法,就國庫調度,請問國庫調度干納稅有什麼事情?干扣繳義務人什麼事?這個怎麼能夠經過目的正當性的審查?説是增加國庫調度的可能性,國庫調度你可以去跟銀行借錢呢,可以發行公債啊。

納稅義務人的申報義務,在綜合所得稅就是隔一年的5月份的結算申報義務啊,你今天你為了國庫調度,你切割成在前一年的每一個月份的時候就要給錢,讓國家比較好收支。請問這個國庫調度,是加重第三人扣繳義務人行為負擔義務的正當理由嗎?你要做比例原則的第一個階段檢驗就是,這個規範的目的正當性。我個人認為一點正當性都沒有。所以對673號我當然有很多意見。到目前為止,除了境外所得者就有扣繳制度以外,我們境內所得的取得的所得,最重要關鍵的673號解釋說他合憲。

在673號裡面,大致上就把所有的就源扣繳制度的相關問題,全面的,不管你設定誰,都符合,為了要讓國家掌握稅收,這個是正當目的,為了讓國家掌握稅收,其實是為了實踐量能課稅原則。但為了國庫調度,則沒有正當性。因為國庫調度這一件事情,在你的立法理由裡面,不管你立法者有沒有寫這個東西,你寫這個東西就代表著你是基於不當的考量而引進了就源扣繳制度。

對於扣繳義務的內容,我最初的看法比較認為,就源扣繳裡面的「填發」是沒有必要的,因為如果你在申報的時候,你就已經告知,我扣繳義務人扣繳的稅款,納義務稅人是誰,那其實不需要後面那個「填發」。但我們現在實務的作法確實是沒有直接告知國家,所以就變成他後面還是要有一個對帳的動作,不然繳那個錢,到底最後面要算誰的可扣抵的所得稅額,這個就會不清楚,從而會有後面那一個扣繳憑單的「填發」。

一方面要給國稅局,一方面要給納稅義務人,讓納稅義務人也知道哦,原來我你幫我扣取了多少的稅款。這個是我們制度上因此導出來的一個,讓他變成要做這一種兩道的行為義務。

\hypertarget{ux8cacux4efbux50b5ux52d9}{%
\section{【責任債務】}\label{ux8cacux4efbux50b5ux52d9}}

賠繳這個部分,我個人認為,由於行為義務的違反,讓他承擔這個不利益,我認為是合於比例的。扣繳義務人應扣未扣,讓他負擔賠繳義務,主要是為了讓扣繳義務制度,在前面第一個合憲的前提底下,因為我去對納稅義務人,我基本上要不到,特別是境外所得者,我要不到那個錢,所以要扣繳義務人賠繳這筆稅款。我個人認為,扣繳義務制度本身的合憲性,也會證立了扣繳義務人作為第二順位的責任債務人的賠繳的義務。

\hypertarget{ux88c1ux7f70}{%
\section{【裁罰】}\label{ux88c1ux7f70}}

最後一個叫做,責罰非難,也就是裁罰。正因為在這個整個程序裡面,我們的扣繳義務可能涉及到你是境內所得者跟非境內所得者,只要國家沒有稅款因此短收的風險,原則上不能對扣繳義務人做過度的責罰非難。如果這個扣繳義務人所要扣繳的這個稅款,依照我國法律規定,根本不需要做就源扣繳,比如說我國沒有課稅管轄權,那這個時候你去責罰非難扣繳義務人,是沒有意義的。這一種扣繳的申報填發的義務建立在,我們國家對這筆所得有課稅權限,這一筆所得是屬於中華民國來源所得,納稅義務人本來應該要報繳我們的所得稅,但他沒有這個行為申報的義務,從而必須要仰賴透過就源扣繳制度。那麼也因此最後這一個裁罰的層次,要看現實上是不是會產生國家的稅捐短收的風險存在。

如果一路過來,扣繳義務人其實都有扣取申報繳納,那最後面只是沒有填發扣繳憑單,是對帳的問題,那理論上來講,這個就應該是行爲罰,從而不得責予與過度的非難。行爲罰往往會在接下來的責難裡面會被認為,如果你沒有設一個上限的限制的規定,就會有過度責罰非難的問題。

那我們除了673號解釋第三段有這個文字以外,各位還可以去看們一路以來對扣繳義務有涉及到責罰跟非難的是釋字317、釋字327、釋字673、釋字713號解釋,在這四號解釋裡面,統合背後的意旨是,只要國家沒有因為扣繳稅款而產生應收未收稅款的這一種風險存在,原則上不得過度責難非難扣繳義務人。

唯一留下的問題就是扣繳義務人在我國法制上幾乎是跟納稅義務人是同樣的對待。也就是我們的稅稽法第42條的規定,也給扣繳義務人一個等同於納稅義務人的逃漏稅捐的制裁。這麼高度的責任目前為止沒有任何釋憲。這是不是過度的苛責?原則上他確實有行為義務的違反,補繳責任,在很大程度上已經把國家短收的這個危險透過賠繳,讓國家取得他該取得的稅款,那是不是還要用這樣的一個方式去過度的責難跟非難,本質上是無償為國家完成這個公法上任務的第三人,我個人比較抱持著高度懷疑他的合憲性的看法啊。

這個地方,通過對各個層次的討論,講到所涉及的大法官解釋,跟各位說明扣繳制度。

到這裡我們先休息一下。接下來我們要談關於所得稅的結算申報的最終章,也就是稅捐稽徵程序。

\hypertarget{section-28}{%
\chapter{20231211\_02}\label{section-28}}

\begin{longtable}[]{@{}l@{}}
\toprule()
\endhead
課程:1121所得稅法一 \\
日期:2023/12/11 \\
周次:15 \\
節次:2 \\
\bottomrule()
\end{longtable}

\hypertarget{ux7e8cux8ac7ux5c31ux6e90ux6263ux7e73}{%
\section{【續談就源扣繳】}\label{ux7e8cux8ac7ux5c31ux6e90ux6263ux7e73}}

補充一下,我們的納稅者權利保護法施行細則,第2條第二項:「本法所定納稅者,包含各稅法規定之納稅義務人、扣繳義務人、代徵人、代繳人及其他依法負繳納稅捐義務之人。」

所以,我們現行法制裡面,私人被納入稅法規範裡面的地位,其實有各種不同的類型,直接因為公法上的權利義務關係,公法上的稅捐債權債務關係而發生的繳納稅款的行為義務,這個叫繳納義務人。我在稅總的課程裡面,我通常會區別成說,因為你是依稅法規定而產生的稅捐債務,所以其實是先是稅捐債務人,然後因此自己的稅捐債務自己去申報繳納,所以他是納稅義務人。依照法律規定而產生的稅捐債權債務關係,其實是首先是稅捐債務人。

稅捐債權人是誰,稅捐債權人就是依照財政收支劃分法裡面劃分的成國稅,或是直轄市及縣市稅裡面的稅捐債權人。

我們在稅法總論裡面,大致上跟各位提到過,稅捐債權人跟債務人的概念。債務人是實體法,納稅義務人則是程序法,但我們的現行法是直接跳過實體法規定,直接就用程序法的這個概念,涵蓋涵攝了稅捐債務人,同時成為納稅義務人。

但稅捐債權債務關係在我們的現行法裡面還有一些,並不是自己申報繳納的,像是扣繳義務人。扣繳義務人,其實是程序法概念。

納稅義務人,是實體法概念,在實體法依法課稅原則底下,稅捐債權債務關係在構成要件該當的時候,就因此該當了。取得所得者,同時是納稅義務人,也是納保法施行細則第2條第二項所講納稅者。

接下來,還有代徵人。代徵人,這個制度是臺灣有的,德國沒有。但正是這個制度,我可以跟很明確跟各位講,代徵人就真的是公權力委託私人行使,因為那個名詞就告訴你,代為徵收啊,徵收權限給代徵人代而為之啊。證交稅、期交稅、娛樂稅,就是這種代徵人概念,所以這就是很典型的公權力委託私人行使。代徵人,有代徵的獎金

代徵人,是公權力委託私人行使。那扣繳義務人是什麼人?怎麼會一樣呢?這個地方你只要明白臺灣跟德國的差異,你很快就可以分辨,這個人,這個代徵人,真的就叫做公權力委託私人行使。扣繳義務人,叫做,服公法勞務的私人,而且,他服公法勞務是不用錢的。代徵人還可以拿錢你。你這樣一比較下去,回到前面的第一個命題,我就可以跟各位講,扣繳制度的不義或者是不正當性更明顯。

代徵人,有代徵的獎金,名詞不太一樣,有些會講代徵獎金。就是國家公權力委託私人行使,國家給你一筆錢。

扣繳義務人呢?是這樣,沒錢。做不好,要賠。而且罰也罰到你身上,罰到你身上的時候,跟納稅義務人是同樣的處罰。你真的好倒楣喔。扣繳義務人的這樣一個制度,你認為合憲,我也是輸了。真的完全不能理解釋字673號。

也許各位聽我的課的時候,或者是有時候,像在不同的領域裡面,大法官解釋往往被認為是圭臬,被認為是準則。在稅法領域,往往這個要稍微打一點折扣。我倒不是否定大法官的貢獻,因為他確實是在我們國家法制裡面,唯一能夠現行現行法律規範違憲的一個機制。可是大法官在行使相關權限的時候,我個人還是強烈的建議,好好地審視,在先進國家已經發展出來的理論。我也不是首先提出的,我沒那麼偉大。你只要好好的去參考,比較法制。比較法制最大的功能作用,就是告訴你自己哪裡不足,就這麼簡單。就看各位願不願意去把你的眼光看到比較法。

我為什麼會知道扣繳的「通報」?因為很簡單,荷、比、盧,這幾個國家都是通報而已,不需要就源扣繳。德國有就源扣繳,但他的就源扣繳是比較有限的。我上一次有跟各位講過,薪資所得、利息所得,就這兩種,結束。一個綜合所得稅是國家原則上是不需要就源扣繳制度的。我們不僅有結算申報,我們還要有廣泛的就源扣繳制度。光這個制度一比較以後,你立刻就會產生很大的懷疑,爲什麽爲什麽,為什麼我們國家要找這麼多人來當扣繳義務人。還不給錢哦。做不好,要賠,那就算了,還要罰。罰的程度,以前是一點,限度都沒有。彷彿國家在這個地方把你規定進去就具有正當性。

在我來看,公法勞務的私人,在某種程度上被拉進來,他是一個犧牲。我們的特別犧牲理論從來都不會去講到這一種服公法勞務的私人。如果不過度,那就算了。

就源扣繳制度,確實在德國也引起爭議,德國也在討論,究竟扣繳的義務是否要有償。但目前的主流意見還是認為這個是無償的。正是因為無償,所以他會盡量減輕對扣繳義務人的非難,因為基本上我們只是讓他賠而已。德國的所得稅法,雖然也規定扣繳義務人要負連帶責任,但基本上也大致上就是連帶而已。就是國家,如果有可能的話,還是要優先去向納稅義務人去做,請他補繳稅款的主張,因為本來就是納稅義務人的責任,不是扣繳義務人的責任。

扣繳義務人是用他自己的錢,用他自己的時間跟精力來幫國家做這件事情,犧牲沒拿到對價,那就算了,還要負擔這麼高的行為義務。

沒辦法,現在就長這個樣子。所以如果可以,要趕快換工作。沒辦法,就法律就長這個樣子。

上一堂提到的,那幾號解釋,317、327、673號跟713號,裡面都有一些責罰非難的規定,那些規定你大致上可以歸納出,就是說單純,如果是應扣未扣,因為這個時候有產生國家短收稅款的風險,這一種短收稅款的風險,原則上他會用漏稅罰,這種處罰,比較沒有所謂的上限額度的問題。可是單純的未填發扣繳憑單,其實主要是風險,是在國家無法對帳,沒辦法確切地清楚知道說,這個錢到底是誰的。那原則上是行爲罰。也就是說,這一種情形應該是一個行爲罰的概念。而這種行爲罰,我們從一路317、327,基本上當時都是警告一下,說,這個要注意到有比例原則的問題。到673號,才比較清楚地在第二段跟第三段,特別是在第三段裡面去講這個責罰的過當的問題。釋字713號亦如是。

好,這個是就源扣繳。就跟各位談到這裡。

\hypertarget{ux7d50ux7b97ux7533ux5831ux5236ux5ea6}{%
\section{【結算申報制度】}\label{ux7d50ux7b97ux7533ux5831ux5236ux5ea6}}

我們最後面來談到所得稅的結算申報制度。

所得稅的結算申報在所得稅法第71條的第一項規定。當然,這個條文規定本身是綜所稅營所稅都適用喔。

根據我們的報繳合一的制度的規定,結算申報繳納是同時間在這裡完成,所以我們的綜所跟營所,申報跟繳納是報繳合一制度。我們在上稅總的時候,有提到報繳制度的合一,這個主要是考慮到避免稅捐稽徵機關核定時間差異太大,導致納稅義務人繳納稅款時間不公平而做的一個考量。那麼在你申報繳納之後,稅捐稽徵機關理論上會做調查,之後會做核定。核定的話,可能在實務上面不一定是用具體的行政處分來送達給相對人,而是用公告的方式去代替核定。相關的規定,調查,第80條的規定,公告代核定是81條第一項跟81條第三項。

這個是整個綜所稅裡面最主要的協力義務。你自己的所得結算申報的時候,你自己來做結算申報。綜所稅最重要的,往往也是許多所得類型裡面,唯一的協力義務,通常就是你有所得,你自己來報。

往前推進啊,其實在綜所稅裡面,也不乏有其他的協力義務之可能。其他協力義務,譬如說記帳,跟更早之前的登記。

申報繳納是稅捐構成要件該當的時候,要求納稅義務人在申報期限內,依法而為的一定作為。

在此之前,登記跟繳納。在綜合所得稅裡面,由於課稅的稅捐主體就是自然人。對於自然人課予登記跟記帳的義務,這一個性質類別主要是在如果你是獨立付出勞務,國家才會課予你一個登記或記帳的義務。如果不是獨立付出勞務,一般而言,不管你是資本投入或是勞動力投入,尤其是非獨立的勞務,要求薪資所得者去記帳,那是沒有太大道理。

薪資所得者,一般而言沒有記帳的能力。你要記帳只是讓自己清楚自己收入多少支出多少。

各位同學可以自己記帳看看。老師自己在學生時代的時候也想過記帳,但記不到一天就立刻做不下去,因為太瑣碎啊,做不來。當然,後來我在研究生的時候,我有信用卡,開始嘗試著想說,哦,我透過信用卡來記帳,就是刷什麼東西,原則上都用那一張卡刷。但你也知道,有些地方以前電子支付沒那麼發達,所以還是要用現金。所以當然我有想過說,那我可不可以用提領一個固定定額現金的方式,來了解一下,除了刷卡以外,我現金支付要多少。但我每次都會忘記。所以我就告訴各位一件事情。記帳很難。就這樣,我只是爲了告訴各位,記帳有點難。對一般人而言,我們沒有辦法期待他記帳。

可是我們對一個獨立執業的人,我們是有一個記帳的義務。請各位看一下所得稅法第14條第一項的第二類,第二段。

就我我因為我們所得稅法有點羅裡吧嗦的慘吼,這第二段第一句第一句因為你看,所得稅法第14條第一項第二類第二段:「執行業務者至少應設置日記帳一種,詳細記載其業務收支項目;業務支出,應取得確實憑證。帳簿及憑證最少應保存五年;帳簿、憑證之設置、取得、保管及其他應遵行事項之辦法,由財政部定之。」

所以,執行業務者的協力義務,要記帳、保持憑證。好在這個地方,根據14條的第一項第二類,第二段的第一分句,至少要設置日記帳。第二句的規定,要設置帳簿跟憑證啊,要依財政部規定的帳目跟憑證來設置好。

這個就是協力義務。協力義務不是只有71條的申報,包括在此之前的記帳跟憑證。

如果是獨資跟合夥,基本上是拿到第一類所得。第一類所得,他的記帳跟保持憑證義務是在第三章營利事業所得稅,21條、22條的第一項規定。他的登記義務,則是18到20條,已經廢止了。

所以我們現在就用這個圖形來跟各位說一下。原則上綜所稅的納稅義務人,除非你是獨資合夥、執行業務者,我們沒有特別針對其他的所得類型,課予記帳憑證的義務。

就是這一些獨立付出勞務去取得所得的人,我們課予他一個記帳、保持憑證的義務。以前有登記,現在沒登記的義務。

這個是我們現在除了申報義務,往前,大致上有協力義務的相關規定。

執行業務者要記日記帳。日記帳,其實就是流水帳。我剛剛做的那個說明,其實就是流水帳,就是我把我個人,一天會花多少,或者一個月會花多少跟收多少,去做一個收支的單純的記錄,這個就叫日記帳。現在是12月11號,昨天是12月10號,就這樣每天記帳就好。跟業務有關的收支都記進去,這個就叫日記帳。

商業會計法裡面,記的帳是T字帳,有資產負債表,看起來就像一個T。T字裡面,有資產這一邊,跟負債這一邊。負債裡面,有自己去跟人家借來的,資本的負債,跟我自己除了資本的自有資本的負債。這就是商業會計法裡面的T字帳。這個是適用商業會計「權責發生制」的基礎。

簡單來講就是,你的記帳決定你用什麼方式去呈現你的經營成果。日記帳,原則上沒有相互用收入對成本費用的問題,因為他就是一個流水帳而已,原則上就是用現金收付。如果你是用商業會計法用T字帳去記帳,因為有資產負債表、有損益表,這個時候他就有辦法去相互勾稽對應,這個就叫做權責發生制。

所以釋字722號講,今天你是執行業務者,你如果用日記帳,基本上就是現金收付制而已。但如果你是執行業務者,假設你自己比照商業會計法的規定去做T字帳的話,那這個時候,在德國就讓你執行業務者可以選擇改用權責發生制。所以,你的所得怎麼去計算,會根據你的帳簿記載的方式去做呈現。

用商業會計法,權責發生制,讓稅跟財就合一,原則上稅遵財,再做稅的帳外調整就可以。因為這樣讓遵財的人可以不用做兩本帳。兩本帳不是指內帳外帳的意思,是指稅帳跟財務帳,不要做分離。但沒有做財務帳的人,沒有資格去用權責發生制。你沒有做財務帳,對不起來,沒有辦法這樣勾稽。記流水帳,從而就退而求其次,只能用收付實現。因為只有收付實現,他才能真正表現出你的稅捐負擔能力。

因此,回過頭來,我們談到,在綜所稅裡面,凡是採用商業會計法的規定來用T字帳,去表現出他的經營成果的,理論上要給予一個選擇的可能性,讓他可以改用權責發生制。這一個就是對獨立經營業務的執行業務者,跟獨資合夥,只要他用商業會計法方式記帳,理論上就應該給予,比照營利事業所得稅裡面的方式去記帳,去做權責發生制的適用。

相反的,對一般的綜合所得稅的自然人而言,他不太會去做記帳,或者是簡單的流水帳,這種流水帳無法去對應,因為他只是單純地去記,哎,我這個時候收多少支多少,但這個支出可能對應的他前面的收多少,這件事情無法去做勾稽。在不能做勾稽的前提下,日記帳,因此他的申報稅捐的時候,原則上就是以收付實現,也就是Cash Basis Accounting,來作為一個計算稅捐負擔能力的基礎。如此而已。

進入綜所稅的申報,原則上沒有前面的這些記帳義務的干擾的話,在所得稅裡面,他就是自己的稅捐自己申報,現金收付制。

當然,這一種現金收付,往往由於我們現在實務上的給付現金的態樣也比較複雜一點,尤其是私人的營利事業,往往他給付給他的受僱人不是當月給付。可能會隔一個月,5月份才給。甚至拖欠款項,一直沒給。我們在公立學校服務最大的好處就是月初就給,我還沒服勞務,國立臺灣大學就會先給我錢。我如果換到私立東吳大學,不知道會不會這樣,我不知道。我只是說,一般私營的事業通常不會當月就給你,他也不會在那一個月的月底給你,他是隔一個月才支付,上個月你付出的勞務的報酬。

而至於在這種情況底下,比如說啊,我領到錢,我是隔一個月啊,那你可不可以把12月的薪資所得算成1月份的薪資所得?換言之,因為你是1月份才實際領到啊,那你這個時候12月就提前課,那這樣會不會提早一個月課?對納稅義務人來講,他會有那個期限利益上的損失。

這個地方的回覆是這樣。假如僱傭關係連續,這個差別也就沒那麼大。因為你等於是隔一年,你一定會領到,一樣有薪水。確實會有時間差,也因此實現原則這一個原則,確實產生實現時間點的差。但如果這個時間點差異差太遠的話,確實就會存在著量能課稅原則過早實現所得,去計算課稅的稅捐負擔的問題。尤其是累積多年所得,如果一次大量實現的話,這個時候會對綜合所得稅的納稅義務人,產生一個稅捐負擔上急劇(集聚)的效應,就是累計稅率底下,他稅負會提高許多。

所以在我們的所得稅法第14條第二項,有一個變動所得的概念。變動所得,是累積多年所得一次大量實現。可是因為14條第二項只限於四種類型所得。我們的累積多年所得一次大量實現,當你時間差,差太遠,比如說非法解雇受僱人。雇主非法解僱受僱人,受僱人對雇主提起的勞動關係確認之訴,跟請求給付基於勞動關係的薪資報酬。這個當他打勝的時候,雇主才會給吧?他很可能會是,累積多年所得,一次大量實現。

這一種情況底下在我們現行實務裡面並沒有適切地去反映出他可能是累積多年所得,而一次大量實現在,累計稅率底下會增加稅負的問題。我們只有在變動所得這一個類型裡面,才特別去顧慮到累積多年所得一次大量實現。

正因為如此,所以這裡面確實,實現原則,是量能課稅原則上講的現金收付制,並沒有必然要用權責發生制,可是因為我們現行法律上規範的缺陷,往往即使是現金收付,也仍然會跟實際上的稅捐負擔能力有差異甚大的情形。依據個案的情況,由於立法者沒有做適度的規範,從而也會讓納稅義務人不當地增加不少的稅捐上的負擔。

申報跟繳納之後,做調查,在我們的所得稅法第80條的規定。第81條的第三項核定跟公告,那這大致上是我們整個的稅捐申報程序的規定。

我們接下來要先去談,這個申報義務有被免除的情況。我們的所得稅法第71條,請大家看71條的第三項規定:「中華民國境內居住之個人全年綜合所得總額不超過當年度規定之免稅額及標準扣除額之合計數者,得免辦理結算申報。但申請退還扣繳稅款及第十五條第四項規定之可抵減稅額,或依第十五條第五項規定課稅者,仍應辦理結算申報。」

71條的第一項,原則上所有的納稅義務人都應該做結算申報。

結算申報的義務的免除是71條的第三項第一句的規定,當你的所得額太低,低於一個門檻,就是免稅額加標準扣除額的合計數,這時候免辦理結算申報的義務。

我們處理兩種情況,第一個是納稅義務人,如果死亡的情形。有所得的,納稅義務人在年度還沒有結算申報之前,他過世了。這個是在71-1條的第一項規定:「中華民國境內居住之個人於年度中死亡,其死亡及以前年度依本法規定應申報課稅之所得,除依第七十一條規定免辦結算申報者外,應由遺囑執行人、繼承人或遺產管理人於死亡人死亡之日起三個月內,依本法之規定辦理結算申報,並就其遺產範圍內代負一切有關申報納稅之義務。但遺有配偶為中華民國境內居住之個人者,仍應由其配偶依第七十一條之規定,合併辦理結算申報納稅。」

第71-1條的第一項跟第一項的第一句的規定,納稅義務人取得所得,但他在年度中死亡,由其他的第三人來負責做結算申報程序。納稅義務人以外之第三人,包括執行人繼承人或遺產管理人。他是以一個第三人的身分,為取得所得者之納稅義務人來履行結算申報的義務。所以他不是稅捐債務人,他不是納稅義務人,因為他不是取得所得的人。也不是扣繳義務人。他是什麼人?他是第三人,就是這樣。

他是被法律賦予給他一個行為義務,這就是納保法施行細則第2條第二項所講的其他行為義務。法律沒有給他一個正式名稱,他是一個第三義務人的概念,這個第三義務人要負責來去做納稅義務人的結算申報義務的申報,而且要就該項稅款來負責繳納。

如果他有配偶的話,則根據71-1的第一項但書規定,由配偶來做結算申報。所以納稅義務人死亡有配偶,原則上有配偶做結算申報。

同樣的情形,在第二種情況叫離境。納稅義務人在年度內取得所得,但結算申報之前離境,根據71-1的第二項規定:「中華民國境內居住之個人,於年度中廢止中華民國境內之住所或居所離境者,應於離境前就該年度之所得辦理結算申報納稅。但其配偶如為中華民國境內居住之個人,仍繼續居住中華民國境內者,應由其配偶依第七十一條規定,合併辦理結算申報納稅。」

離境的時候,要由納稅義務人他自己來做結算申報。但如果配偶是中華民國境內居住個人,而且繼續居住在中華民國境內,則由配偶來做所得稅法的結算申報。

所以主要兩種情況,第一個是死亡,第二個是離境,就是境內居住者的納稅義務人在結算申報之前離境,這個時候有配偶是我國的稅籍居民繼續居住在我國境內,原則上由配偶去做結算申報。因為有兩個人嘛,就是其中一個人死掉,另外一個是境內居住者的身份,他就繼續依結算申報的規定。那如果沒有配偶,他是單身或配偶非境內居住者。這時候就會回過頭來來去適用71-1的第一項規定,由這個遺囑執行人繼承人或遺產管理人於被繼承人死亡時三個月內依本法規定來做結算申報。離境的時候則是71-1的第二項,是在離境時做結算申報做結算申報。

你離境的時候,如果你沒有去做離境時的結算申報,根據72條第二項的規定,可以在委託中華民國境內居住之個人負責代理申報。

這個條文規定有點凌亂,不過我們大致上整理。

第一個納稅義務人的自己稅捐,應該在71條第一項規定裡面讓自己去做結算申報。

納稅義務人如果在結算申報前死亡,或納稅義務人在結算申報前離境,就是去中華民國再也不回來哦,這種情況底下有配偶的話,原則上由配偶來當結算申報的行為義務人。

沒有配偶啊,或者配偶非境內居住者的身份,那這個時候死亡的情形則由遺囑執行人、遺產管理人或者是繼承人來為納稅義務人做結算申報程序。這是一個行為義務的負擔,所以他是一個第三義務人的概念。

那麼如果是離境,就是他在所得取得之後,在結算申報之前就離境,他沒有配偶或是境內居住人的配偶的話,這個時候離境的時候去做結算申報,或者他再委託本國人來為他去做結算申報程序,這個就是72條的規定。

我們的現行法規範,有點複雜,但我還是期待各位同學,整理過後,至少現況的情形你一定要熟悉。

再講一次,結算申報是自己本人申報。如果配偶是境內居住者,當他死亡或離境,則由配偶來做結算申報程序。如果這個配偶的要件不該當,那麼就是死亡的時候,由遺囑執行人、繼承人或遺產管理人,來做所得稅的結算申報。如果是離境,則由本人離境時做結算申報。沒有做結算申報,可以委託第三人來做結算申報。第三人仍然是第三義務人,他並不是一個在法律上,我們規定他作為有結算申報義務的行為義務人。

這個是我們現在的對關於這個納稅義務人的結算申報行為義務的規範。

\hypertarget{ux7fa9ux52d9ux9055ux53cdux4e4bux5236ux88c1}{%
\section{【義務違反之制裁】}\label{ux7fa9ux52d9ux9055ux53cdux4e4bux5236ux88c1}}

有結算申報的義務,才會有在應履行而不履行的情況,產生漏稅罰的問題。換言之,到這個地方為止,是申報繳納的義務。我們是報繳合一制度,所以在申報的時候,也要將相關的稅款繳納完成。當他沒有繳納國家認為該收取的稅捐債權的時候,對申報協力義務之違反,就因此產生了所得稅法第110條第一項跟第二項所規定的漏稅罰的處罰。

我們來看第110條第一項跟第二項。有申報義務之違反,如果納稅義務人申報的時候將足額的的所得,已依本法規定辦理結算決算或清算申報,而對依本法規定應申報課稅之所得額有漏報或短報的話,處以所漏稅額二倍以下之罰鍰。

應申報未申報,這個時候,除了補徵稅而以外,還要處以三倍以下的罰鍰。110條的第二項規定,他認為,應申報完全沒有去做申報,這個可責罰性非難性是比,已申報、短漏所得額申報的可責罰性非難性更高。

簡單來講,依照我國立法者的邏輯,你自己來報了,好歹我收到一部分錢。可能我沒有收足額,但你有繳一部分,但連來報都不來報,那他就認為你很過分很重大地違反義務,讓國家連錢都沒收到,也無從查核起。因為不申報,他也無法去分辨你,到底是所得額為報繳的情況,還是是未達申報門檻的情形。

我們的所得稅法第71條第三項有一個申報所得額的門檻規定,就是標準扣除額加免稅額。你只要所得沒有超過那個門檻,那就免除你稅捐申報的義務。這很明顯的是,認為既然你有所得,但你沒有超過免稅額加扣除額,計算的結果,你仍然不用報繳任何稅捐,從而就免掉稅捐申報上的義務。這是基於實用性原則,而給予納稅義務人免申報的義務。

我個人認為這個條款規定並不妥當,不妥當的原因是在,有所得理論上就應該要去做申報。因為他變成了讓納稅義務人主張,如果主張我的所得,雖然有所得,但沒有高過免稅額再加標準扣除額,因此依照法律規定,他是可以免申報的義務。那至於他的所得,他就會變成了,那我究竟我的所得是不是所得、是不是高過免稅額加扣除額,額度上面會產生法律見解跟稅捐稽徵機關的歧義。

原則上,如果依照法律規定不超過那個門檻,他確實不用申報。可是究竟他是不是所得,究竟他是不是高過免稅額跟標準扣除額的門檻,是不是因此免納稅捐,這個仍然要由稅捐稽徵機關在你申報之後,才有辦法去做判斷。

從而老師個人認為71條的第三項,免申報義務是不太適宜的規定,不太妥當,他會讓納稅義務人對這一類的所得,由於他要自行去判斷他到底是不是所得,他要自己去判斷他到底是應稅所得或是免稅所得。因為如果他應稅所得不超過免稅額加標準扣除額,理論上他是不用申報。他會構成了讓納稅義務人自己去做判斷,只要他未達門檻,所以他就不用申報。

這就會對申報義務究竟有沒有違反,可能是來自於納稅義務人跟稅捐稽徵機關對系爭所得,採取不同法律見解,而產生出來的歧義。理論上應該是請他盡量去做申報,所有只要該當於所得,不管是應稅或免稅的所得,都讓他去做申報,再由稅捐接機關來判斷是應稅或免稅所得。因為稅法本身,有非常複雜的解釋函令,對納稅義務人,他並不知道究竟這個是應稅或免稅所得,或者是所得還是非所得。讓納稅義務人自己去判斷,這個風險是非常高的。他自己如果沒有把各類所得全部都寫上去,就很可能會因為他只是存在跟稅捐稽徵機關不同的法律見解,那為什麼他會存在不同法律見解?因為他不知道解釋函令是這樣操作。

原則上應該是讓納稅義務人盡可能申報,把所有他認為可能該他所得的事實全面性的申報。當他申報以後,再由稅捐稽徵機關來為他去決定。因為稅捐稽徵機關有自己常用的解釋函令,好,這個所得,是所得還是非所得,這個是應稅或免稅所得,我幫你判斷。所以你給我事實,我就給你稅額。而不是納稅義務人自己要判斷,他到底要適用哪一條法律,到底是應稅所得還是免稅所得,或是非所得或是所得。

在這種情況底下,稅捐法制如此複雜,對非專業的納稅義務人而言,他等於是自己要去判斷。所以,有這個法條規定,我還是會建議所有的納稅義務人盡量都申報。你自己不要自以為是,說,啊,我所得沒有超過免稅額,加扣除額,所以我就不要申報。因為你不申報,很可能就會110條第二項。

你完全不申報,這個時候他查到的時候,稅捐稽徵機關會主張,可能跟你當初認知的法律見解不同。這時候他認為這是所得,而且應稅所得,你沒有報,三倍以下罰鍰?那為什麼會加重處罰?因為他認為你完全不申報。但可能納稅義務人只是認為說,這個是在我的法律概念來講,應該是非所得的概念。

所以法律條文規定,這個71條第三項的免申報義務,我個人認為是一個很不好的規定。在法律沒有變更之前,原則上納稅義務人應該是盡可能把認為有可能構成所得的事實,盡量申報。因為對這個所得的性質,他到底是所得還是非所得,你納稅義務人沒有辦法比稅捐稽徵機關做更好的判斷。

就以損害賠償為例。那一種是所得,哪一種是非所得?我們花了一段時間跟各位去講,何謂損害賠償,損害賠償為什麼是非所得,或者為什麼他是替代經濟成果的履行利益的損害賠償,就應該要報所得稅。所以有可疑,盡量揭露,盡量寫。

在現行法制裡面,雖然有告訴你一個,如果你的應稅所得不超過免稅額跟扣除額,那這個時候就不要申報。你可以免申報義務。但因為這個地方第110條第二項跟第一項的規定,他有稅負裁罰上的差異,簡單來講,就是應申報未申報這種情形底下對納稅義務人構成極大的一個風險。

也因此,回過頭來,我們講在這裡面短漏稅的裁罰分成了應申報未申報三倍以下的裁罰,如果以已申報,短報跟漏報的情形不太一樣。

短報是指所得來源申報,但對於他的數額有短少。短報所得來源是我從臺大拿到100,但我只報80,這個叫短報。

漏報則是數個所得來源,只申報了一部分的所得來源,另外一部分所得來源沒有申報。比如說,我在臺大有所得,我在成大有所得,我只報臺大的所得,不報成大所得,這個就叫漏報。

什麽是應申報不申報,就是我有臺大成大所得,但我從來不申報。

這個就叫做第110條的短報申報,應申報未申報。

那其實從理論上來講,不管是短報漏報或應申報未申報,都讓國家無法即時取得應收的稅款,從而產生短收稅款的風險。理論上這幾種情況,包括,應報未報、應報已報但短報、應報已報但漏報、應報已報但做虛偽不實的成本費用的增加,其實都是讓國家無法即時收取全部的稅捐債權債務的內容,從而存在著短收稅款的風險。這個就被認為是一個應受處罰,漏稅罰的態樣。

實務上面,漏稅罰,如果是過高的倍數,往往司法機關也會遵照立法機關的判斷,但一般而言,漏稅罰按倍數裁罰,不被認為是一個過度的責罰。行爲罰,如果有按倍數裁罰,才會有過度責罰的問題。從而這也是我們剛剛在提到扣繳義務的時候,扣繳義務為什麼會去區別,他到底是屬於哪種態樣。如果是應收未收到稅款,這個時候漏稅罰,確實會以倍數裁罰,而倍數其實是沒有數額的絕對上限的問題。可是用行爲罰概念的話,他則必須要有上限的概念,否則會有違反責罰相當性的問題。

目前為止,跟各位提到結算申報前置的協力義務,申報義務的違反,因此產生出來的處罰。雖然有申報義務的免除規定,但現行實務上面,如果納稅義務人採取跟稅捐稽徵機關不同的看法,而納稅義務人也確實沒有去做申報的話,這個時候可能會有比較高額的漏稅罰,在110條第二項的三倍以下的罰鍰。補繳稅款不是罰鍰。再講一次補繳稅款不是罰鍰,只是履行稅法規範的納稅的義務。裁罰才是適用責罰相當性的部分,而漏稅罰目前為止,大法官解釋沒有出現過,認為責罰過度,過度非難的問題。漏稅罰沒有,但行爲罰是有的,行爲罰有過度非難的可能,在我們前面講扣繳義務的漏稅跟行爲罰的時候,也有同樣的適用。

那我們就把所得稅法,大致上從所得稅法的應適用的基本原則,到主體、客體、稅基、稅率,稅基乘稅率得到稅額,進入稽徵程序裡面的結算申報、就源扣繳程序,協力義務違反的制裁,包括了行爲罰跟漏稅罰這兩種可能性,這就是一整個所得稅法裡面規範結構,大致上會討論到的相關的法律規範。

理論上來講,關於協力義務跟義務的違反,在德國是都放到稅捐通則,所以在德國的所得稅法裡面,原則上就只有稅捐債權債務的構成要件跟法律效果規定。稽徵程序則一律到稅捐通則那個地方,去做統一性的規定。那我們的稅捐稽徵法並不全然是稅捐稽徵法。我們自己的所得稅法還是有稅捐稽徵程序的規定。

所以我們所得稅法的期末考試,是全部,從實體法道程序法跟稅捐制裁法,因為這是所得稅法的全部。我們並不是只有考稅捐債務法實體法而已。請各位好自為之。評分的方式,我們一開始就跟各位講過。

我們今天就到這裡。

\bibliography{book.bib,packages.bib}

\backmatter
\printindex

\end{document}
